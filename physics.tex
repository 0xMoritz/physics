\documentclass[11pt]{article}
%\usepackage[ngerman]{babel}	 % deutsche Silbentrennung
\usepackage[utf8]{inputenc}	 % Unicode support (Umlaute etc.)
\usepackage{a4}
\usepackage{array}
\usepackage{amsmath}
\usepackage{amssymb}
\usepackage{graphicx}
\usepackage{listings} 		% Für Code-Segmente
\usepackage{color}
\usepackage{caption}
\usepackage{dblfloatfix}	% To enable figures at the bottom of page
\usepackage{subcaption}
\usepackage{stackengine}
\usepackage{braket}
\usepackage{float}
\usepackage{placeins}
\usepackage{mathtools}
\usepackage{svg}
\usepackage{siunitx}	% SI units \SI{1+-2}{\micro e V}
\usepackage{slashed}	% Feynman slash notation \slashed{}
\usepackage{ulem} 		% strike through text horizontally with \sout{}
\usepackage[margin=2cm,footskip=1cm]{geometry}
\usepackage{imakeidx}	% For indexing keywords
\usepackage{hyperref}
%\usepackage{tabularx}
\makeindex[intoc, columns=2, title=German Index]


%\definecolor{darkblue}{rgb}{0,0,0.45}
% \hypersetup{
%	 colorlinks,
%	 citecolor=darkblue,
%	 filecolor=darkblue,
%	 linkcolor=darkblue,
%	 urlcolor=darkblue
% }

\definecolor{dkgreen}{rgb}{0,0.6,0}
\definecolor{gray}{rgb}{0.5,0.5,0.5}
\definecolor{mauve}{rgb}{0.58,0,0.82}

\newcolumntype{L}[1]{>{\raggedright\let\newline\\\arraybackslash\hspace{0pt}}m{#1}}
\newcolumntype{C}[1]{>{\centering\let\newline\\\arraybackslash\hspace{0pt}}m{#1}}
\newcolumntype{R}[1]{>{\raggedleft\let\newline\\\arraybackslash\hspace{0pt}}m{#1}}

\lstset{frame=tb,
  language=C,
  aboveskip=3mm,
  belowskip=3mm,
  showstringspaces=false,
  columns=flexible,
  basicstyle={\small\ttfamily},
  numbers=none,
  numberstyle=\tiny\color{gray},
  keywordstyle=\color{blue},
  commentstyle=\color{dkgreen},
  stringstyle=\color{mauve},
  breaklines=true,
  breakatwhitespace=true,
  tabsize=3
}

\author{Moritz Geßner}

\DeclareMathOperator*{\SumInt}{%
\mathchoice%
  {\ooalign{$\displaystyle\sum$\cr\hidewidth$\displaystyle\int$\hidewidth\cr}}
  {\ooalign{\raisebox{.14\height}{\scalebox{.7}{$\textstyle\sum$}}\cr\hidewidth$\textstyle\int$\hidewidth\cr}}
  {\ooalign{\raisebox{.2\height}{\scalebox{.6}{$\scriptstyle\sum$}}\cr$\scriptstyle\int$\cr}}
  {\ooalign{\raisebox{.2\height}{\scalebox{.6}{$\scriptstyle\sum$}}\cr$\scriptstyle\int$\cr}}
}

\numberwithin{equation}{section}

\begin{document}
	\renewcommand{\i}			{\mathrm{i}} %\imath
	\renewcommand{\emph}		{\textit} %\imath
	\newcommand{\xrowht}[2][0]	{\addstackgap[.5\dimexpr#2\relax]{\vphantom{#1}}}
	\newcommand{\diff}			{\mathrm{d}}
	\newcommand{\ii}			{\imath}
	\newcommand{\binomkoeff}[2]	{\left(\begin{matrix}#1\\#2\end{matrix}\right)}
	\newcommand{\pvec}[1]		{\vec{#1}\mkern2mu\vphantom{#1}} % primed Vector
	\newcommand{\tder}[2] 		{\frac{\diff #1}{\diff #2}}% total derivative
	\newcommand{\pder}[2]		{\frac{\partial #1}{\partial #2}} % partial derivative
	\newcommand{\dpder}[2]		{\dfrac{\partial #1}{\partial #2}} % default size partial derivative
	\newcommand{\Nabla}			{ \vec{\nabla} }
	\newcommand{\unitvec}[1]	{ \vec{e}_{#1} }
	\newcommand{\com}[2]		{ \left[#1,#2\right] }% Commutator
	\newcommand{\anticom}[2]	{ \left[#1,#2\right]_{+} } % Anticommutator
	\newcommand{\const}			{ \mathrm{const.} }
	\newcommand{\customEq}[1]	{ \stackrel{#1}{=} }
	\newcommand*{\rom}[1]		{ \uppercase\expandafter{\romannumeral #1\relax} }
	\newcommand{\Br}[1]			{ \left( #1 \right) } % normal brackets
	\newcommand{\cBr}[1]		{ \left\lbrace #1 \right\rbrace } % curly brackets
	\newcommand{\cbr}[1]		{ \lbrace #1 \rbrace } % curly brackets
	\newcommand{\rBr}[1]		{ \left[ #1 \right] } % rectangudlar brackets
	\newcommand{\rbr}[1]		{ [ #1 ] } % rectangudlar brackets
	\newcommand{\aBr}[1]		{ \left\langle #1 \right\rangle } % angled brackets
	\newcommand{\abr}[1]		{ \langle #1 \rangle } % angled brackets
	\newcommand{\vBr}[1]		{ \left| #1 \right| } % vertical brackets
	\newcommand{\vbr}[1]		{ | #1 | } % vertical brackets
	\newcommand{\tr}			{ \mathrm{tr} }
	\newcommand{\vsp}			{ \vspace{10pt} }
	\newcommand{\nl}			{ \vsp \newline }
	\newcommand{\rmmu}			{ \SI{}{\micro\nothing} }
	\newcommand{\exact}			{ \hfill(exact) }
	\newcommand{\M}				{ \phantom{-}}
	%\newcommand{\-}				{ \phantom{-}}
	% \Br{\frac{x}{2}} \rbr{\frac{x}{2}} \abr{\frac{x}{2}} \vbr{\frac{x}{2}} \cbr{\frac{x}{2}}

	\begin{center} % Title
		\Large\textbf{Theoretical Physics Formulary} \\
		\large\textit{Moritz Geßner} \\
		\vsp
		\begin{minipage}{10cm}
			\textit{\center{I would rather have questions that can't be \\answered than answers that can't be questioned.}} \\
			\raggedleft{---\,Richard Feynman}
		\end{minipage}
	\end{center}

	\tableofcontents

	\newpage
	% !TEX root = ../physics.tex
\section{Relativity}
	\subsection{Basics}
		\subsubsection{Symmetries of Flat Spacetime}
			\begin{itemize}
				\item Homogeneity of spacetime
				\item Isotropy of spacetime
				\item Lorentz invariance
			\end{itemize}

			\noindent
			Inertial frame of reference\index{Inertialsystem} \\
				\indent \textit{Frame of reference in which spacetime is homogeneous and isotropic.}

		\subsubsection{Einstein Postulates}
			\textbf{First Postulate}: Principle of Relativity\index{Relativitätsprinzip} \newline
				\indent \textit{The laws of nature are forminvariant in all inertial frames of reference.}  \nl
			\textbf{Second Postulate}: Constancy of the Speed of Light\index{Konstanz der Lichtgeschwindigkeit} \newline
				\indent \textit{The maximum propagation speed of information is equal to $c$ in all frames of reference.\index{Lichtgeschwindigkeit}}
				\label{post:c}\nl
			\textbf{Third Postulate}: Strong Equivalence Principle\index{Starkes Äquivalenzprinzip} \newline
				\indent \textit{Spacetime is locally flat, i.e. the laws of special relativity hold in local inertial frames of reference.}  \vsp

	% \begin{description}
	%   \item[Relativitätsprinzip]\hfill \\
	%	 Die Gesetze der Physik sind in allen Inertialsystemen gleich.
	%   \item[Konstanz der Lichtgeschwindigkeit]\hfill \\
	%	 Die maximale Ausbreitungsgeschwindigkeit von Informationen ist in allen Bezugssystemen gleich $c$.
	%   \item[Starkes Äquivalenzprinzip]\hfill \\
	%	 Die Raumzeit ist lokal flach, d.h. in lokalen Inertialsystemen gelten die Gesetze der Speziellen Relativitätstheorie.
	% \end{description}

	\subsubsection{Metric Tensor\index{Metrischer Tensor}}
		Metric Tensor and Lorentz invariant spacetime interval:
		\begin{equation}
			\dd s^2 = c^2 \dd \tau^2 = \dd x^\mu \dd x_\mu = g_{\mu\nu} \dd x^\mu \dd x^\nu
		\end{equation}

	\subsection{General Relativity\index{Allgemeine Relativitätstheorie}}
		\subsubsection{Definitions}
			\noindent
			Partial derivatives:
			\begin{equation}
				\pder{\phi}{x^\mu} = \partial_\mu \phi = \phi_{,\mu}
			\end{equation}

			\noindent
			Christoffel symbols\index{Christoffel!Symbole} (local inertial frame with coordinates $\xi^\alpha$):
			\begin{equation}
				\Gamma_{\mu\nu}^{\kappa} := \frac{\partial x^\kappa}{\partial \xi^\alpha}\frac{\partial^2 \xi^\alpha}{\partial x^\mu\partial x^\nu}
				=\frac{1}{2}g^{\kappa\lambda}\left(\frac{\partial g_{\nu\lambda}}{\partial x^\mu}+\frac{\partial g_{\mu\lambda}}{\partial x^\nu}-\frac{\partial g_{\nu\mu}}{\partial x^\lambda}\right)
			\end{equation}
			The Christoffel symbols are symmetric in the lower indices $\Gamma_{\mu\nu}^{\kappa} = \Gamma_{\nu\mu}^{\kappa}$. They are zero in all free falling local frames of reference.

			\noindent
			Proper time\index{Eigenzeit}:
			\begin{equation}
				\Delta\tau := \frac{1}{c}\int_A^B\sqrt{g_{\mu\nu}\dd x^\mu \dd x^\nu}
			\end{equation}

			\noindent
			Covariant derivative\index{Kovariante Ableitung}:
			\begin{equation}
				D_\lambda T^{\alpha ...}_{\beta...} =
				T^{\alpha ...}_{\beta...;\lambda} = T^{\alpha...}_{\beta ...,\lambda}
				+ \Gamma^\alpha_{\lambda\alpha'} T^{\alpha'...}_{\beta\phantom{\prime}...} + ...
				-\Gamma^{\beta'}_{\lambda\beta} T^{\alpha\phantom{\prime}...}_{\beta'...} - ...
			\end{equation}
			The second term vanishes in all free falling frames of reference and the covariant derivative becomes a partial derivative, %In frei fallenden Bezugssystemen verschwindet der zweite Term und die kovariante Ableitung wird zur partiellen Ableitung.

			\noindent
			Riemann curvature tensor\index{Riemann!Krümmungstensor}:
			\begin{equation}
				\begin{aligned}
					R^{\lambda}_{\phantom{\lambda}\sigma\mu\nu} &= \Gamma^{\lambda}_{\phantom{\lambda}\nu\sigma,\mu}
					+ \Gamma^{\lambda}_{\phantom{\lambda}\mu\kappa}\Gamma^{\kappa}_{\phantom{\kappa}\nu\sigma}
					-
					\Gamma^{\lambda}_{\phantom{\lambda}\mu\sigma,\nu}
					+ \Gamma^{\lambda}_{\phantom{\lambda}\nu\kappa}\Gamma^{\kappa}_{\phantom{\kappa}\mu\sigma} \\
					R_{\lambda\sigma\mu\nu} &= \frac{1}{2}\left(
					g_{\lambda\nu,\sigma,\mu} - g_{\lambda\mu,\sigma,\nu} + g_{\sigma\mu,\lambda,\nu} -	 g_{\sigma\nu,\lambda,\mu}
					\right)
					+ g_{\alpha\beta} \left(
					\Gamma^{\alpha}_{\mu\sigma} \Gamma^{\beta}_{\nu\lambda} - \Gamma^{\alpha}_{\nu\sigma} \Gamma^{\beta}_{\mu\lambda}
					\right)
					\\
					R^{\lambda}_{\phantom{\lambda}\sigma\mu\nu}v^\sigma &= \left(D_\mu D_\nu - D_\nu D_\mu \right) v^\lambda
				\end{aligned}
			\end{equation}

			\noindent
			Symmetries of the curvature tensor:%Symmetrien des Krümmungstensors:
			\begin{equation}
				\begin{aligned}
					R_{\lambda\sigma\mu\nu} &= - R_{\lambda\sigma\nu\mu} \\
					R_{\lambda\sigma\mu\nu} &= \phantom{-} R_{\mu\nu\lambda\sigma} \\
					R_{\lambda\sigma\mu\nu} &= - R_{\sigma\lambda\mu\nu} \\
				\end{aligned}
			\end{equation}

			\noindent
			Bianchi Identities\index{Bianchi!Identitäten}:
			\begin{equation}
				\begin{aligned}
					R^{\lambda}_{\phantom{\lambda}\alpha\beta\gamma} + R^{\lambda}_{\phantom{\lambda}\beta\gamma\alpha} + R^{\lambda}_{\phantom{\lambda}\gamma\alpha\beta} &= 0 \\
					R^{\lambda}_{\phantom{\lambda}\mu\alpha\beta;\gamma} + R^{\lambda}_{\phantom{\lambda}\mu\beta\gamma;\alpha} + R^{\lambda}_{\phantom{\lambda}\mu\gamma\alpha;\beta} &= 0 \\
				\end{aligned}
			\end{equation}

			\noindent
			Ricci-Tensor\index{Ricci!Tensor}:
			\begin{equation}
				R_{\mu\nu} = R^\lambda_{\phantom{\lambda}\mu\lambda\nu}
			\end{equation}

			\noindent
			Curvature scalar\index{Krümmungsskalar}:
			\begin{equation}
				R = R^\mu_{\phantom{\mu}\mu}
			\end{equation}

		\subsubsection{Field Equations\index{Feldgleichungen}}
			\noindent
			Einstein-Hilbert Action\index{Einstein!-Hilbert Wirkung}:
			\begin{equation}
				\mathcal{S} = \frac{c^4}{16\pi G} \int \sqrt{\left|\det{g_{\mu\nu}(x)}\right|} R(g_{\mu\nu}(x))\;\dd^4 x
			\end{equation}

			\noindent
			Einstein field equations\index{Einstein!Feldgleichungen}:
			\begin{equation}
				R_{\mu\nu} - \frac{1}{2} R g_{\mu\nu} = \frac{8\pi G}{c^4} T_{\mu\nu}
			\end{equation}

			\noindent
			Alternative Formulation of the field equations:
			\begin{equation}
				R_{\mu\nu} = \frac{8\pi G}{c^4} \left( T_{\mu\nu} - \frac{1}{2} T g_{\mu\nu} \right)
			\end{equation}


		\subsubsection{Geodesics\index{Geodäten}}
			\noindent
			Action of a free particle%Wirkung eines freien Teilchens:
			\begin{equation}
				\mathcal{S} = -\int_A^B mc^2\;\dd \tau = -\int_A^B mc\sqrt{g_{\mu\nu} \tder{x^\mu}{\tau} \tder{x^\nu}{\tau}} \;\dd \tau
			\end{equation}

			\noindent
			Equations of motion for a free particle (time-like geodesic\index{Zeitartige Geodäte}):%Bewegungsgleichungen eines freien Teilchens (Zeitartige Geodäte):
			\begin{equation}
				\frac{\mathrm{d}^2 x^\kappa}{\mathrm{d}\tau^2}=-\Gamma_{\mu\nu}^{\kappa}\frac{\mathrm{d}x^\mu}{\mathrm{d}\tau}\frac{\mathrm{d}x^\nu}{\mathrm{d}\tau}
			\end{equation}

	\subsection{Special Relativity\index{Spezielle Relativitätstheorie}}
		\noindent
		Special Relativity\index{Spezielle Relativitätstheorie} is a limiting case of General Relativity for flat spacetime, it can be realized by $G \rightarrow 0$. In Special Relativity the Metric\index{Metrischer Tensor}is constant.%Die Spezielle Relativität ist der Spezialfall der flachen bzw. ungekrümmten Raumzeit $G\rightarrow 0$, in ihr wird der metrische Tensor konstant.

		\noindent
		Minkowski-Metric\index{Minkowski!Metrik} (Sign is dependent on convention):
		\begin{equation}
			g_{\mu\nu} = \eta_{\mu\nu}
			= \left( \begin{matrix}
				\pm1 & 0		& 0		& 0		\\
				0		& \mp1 & 0		& 0		\\
				0		& 0		& \mp1 & 0		\\
				0		& 0		& 0		& \mp1 \\
			\end{matrix} \right)
		\end{equation}

		\noindent
		Classical Limit\index{Klassischer Grenzfall}:
		\begin{equation}
			c \rightarrow \infty
		\end{equation}

		\noindent
		Ultrarelativistic limit\index{Ultrarelativistischer Grenzfall}:
		\begin{equation}
			v\rightarrow c
		\end{equation}

		\subsubsection{Definitions}
			\noindent
			Lorentz-factor\index{Lorentz!Faktor} ($\beta = \frac{v}{c}$):
			\begin{equation}
				\gamma = \frac{1}{\sqrt{1-\beta^2}}
			\end{equation}

			\noindent
			Four-momentum\index{Viererimpuls}:
			\begin{equation}
				P^\mu =
				\left(\begin{matrix}
					E \\ \vec{p}
				\end{matrix}\right)
				= \left(\begin{matrix}
					m\gamma c \\ m\gamma\vec{v}
				\end{matrix}\right)
				= \left(\begin{matrix}
					\frac{mc}{\sqrt{1-\beta^2}} \\ \frac{m\vec{v}}{\sqrt{1-\beta^2}}
				\end{matrix}\right)
				= m u^\mu
			\end{equation}

		\subsubsection{Lorentz Transformations}
			\noindent
			Homogeneous Lorentz transformation\index{Lorentz!Transformation} and back-transformation of tensors (defining property of Lorentz Tensors\index{Lorentz!Tensor}\index{Tensor}):%Homogene Lorentztransformation und Rücktransformation von Tensoren (definierende Eigenschaft eines Tensors):
			\begin{equation}
				\begin{aligned}
					x'^\mu &= \Lambda^{\mu}_{\nu}(\vec{v}) x^\mu \\
					x'_\mu &= \overline{\Lambda}_\mu^{\nu}(\vec{v}) x_\nu \\
					\Lambda^{\mu}_{\nu}(\vec{v}) = \Lambda^{\mu}_{\phantom{\mu}\nu}(\vec{v}) &= \Lambda^{\phantom{\nu}\mu}_{\nu}(\vec{v}) = \overline{\Lambda}^{\mu}_{\nu}(-\vec{v})
				\end{aligned}
			\end{equation}

			\noindent
			General (inhomogeneous) Lorentz transformation\index{Lorentz!Transformation} (Poincaré group\index{Poincaré!Gruppe} $\mathcal{L}$%Allgemeine (inhomogene) Lorentz-transformation (Mit Poincaré Gruppe $\mathcal{L}$):
			\begin{equation}
				\begin{aligned}
					x'^\mu &= \Lambda^\mu_{\nu} x^\nu + b^\mu \\
					\Lambda \in \mathcal{L} &= \mathcal{L}^\uparrow_+ \cup \mathcal{L}^\uparrow_- \cup \mathcal{L}^\downarrow_+ \cup \mathcal{L}^\downarrow_-
				\end{aligned}
			\end{equation}

			\noindent
			Condition for the Lorentz transformation\index{Lorentz!Transformation} following from the constancy of the speed of light \ref{post:c}:%Bedingung an die Lorentztransformation als Folge der Konstanz der Lichtgeschwindigkeit:
			\begin{equation}
				\Lambda^{\alpha}_{\mu} \eta_{\alpha\beta} \Lambda^{\beta}_{\nu} = \eta_{\mu\nu}
				 \;\Leftrightarrow\; \Lambda^T \eta \Lambda = \eta
			\end{equation}

			\noindent
			Classification of the homogeneous Lorentz transformations\index{Lorentz!Transformation}:%Klassifikation der homogenen Lorentztransformation:
			\begin{itemize}
				\item Chronus Lorentz Transformation $\mathcal{L}^\uparrow$: $\Lambda^0_0 > 1$
				\item Antichronus Lorentz Transformation $\mathcal{L}^\downarrow$: $\Lambda^0_0 < 1$
				\item Proper Lorentz Transformation $\mathcal{L}_+$: $\det\Lambda = 1$
				\item Improper Lorentz Transformation $\mathcal{L}_-$: $\det\Lambda = -1$
			\end{itemize}

			\noindent
			Lorentz Boost\index{Lorentz!Boost} ($\Lambda\in\mathcal{L}^\uparrow_+$) between two Inertial frames of reference with parallel coordinate axes ($\vec{v}=c\vec{\beta}$):%Lorentz Boost ($\Lambda\in\mathcal{L}^\uparrow_+$) zwischen zwei Inertialsystemen mit parallelen Koordinatenachsen ($\vec{v}=c\vec{\beta}$):
			\begin{equation}
				\Lambda(\vec{v}) = \left( \begin{matrix}
					\gamma & -\gamma\dfrac{\pvec{v}^T}{c} \\[6pt]
					-\gamma\dfrac{\vec{v}}{c} & \delta_{ij}+\dfrac{v_i v_j(\gamma-1)}{v^2}
					\end{matrix} \right)
					=
					\left(\begin{matrix}
						\gamma & -\gamma \beta_1 & -\gamma \beta_2 & -\gamma \beta_3 \\
						-\gamma \beta_1 & 1+(\gamma -1){\dfrac {\beta_1^{2}}{\beta^{2}}} & (\gamma -1){\dfrac {\beta_1 \beta_2}{\beta^{2}}}&(\gamma -1){\dfrac {\beta_1\beta_3}{\beta^{2}}} \\
						-\gamma \beta_2 & (\gamma -1){\dfrac {\beta_2\beta_1}{\beta^{2}}} & 1+(\gamma -1){\dfrac {\beta_2^{2}}{\beta^{2}}}&(\gamma -1){\dfrac {\beta_2 \beta_3}{\beta^{2}}} \\
						-\gamma \beta_3 &(\gamma -1){\dfrac {\beta_3\beta_1}{\beta^{2}}}&(\gamma -1){\dfrac {\beta_3\beta_2}{\beta^{2}}}&1+(\gamma -1){\dfrac {\beta_3^{2}}{\beta^{2}}}
					\end{matrix}\right)
			\end{equation}

		\subsubsection{Implications of Special Relativity}
			\noindent
			Proper time\index{Eigenzeit} in special Relativity:%Eigenzeit in der speziellen Relativitätstheorie:
			\begin{equation}
				\Delta\tau = \int_A^B \frac{\dd t}{\gamma}
			\end{equation}

			\noindent
			Relativistic Doppler effect\index{Doppler!Effekt} (Signals are sent with angle $\vartheta$ as seen from the emitter):%Relativistischer Doppler-effekt (Signale werden von Betrachter aus gemessen mit Winkel $\theta$ ausgesendet):
			\begin{equation}
				\omega = \omega_0\frac{\sqrt{1-\beta^2}}{1+\beta\cos\vartheta}
			\end{equation}

			\noindent
			Addition of velocites\index{Geschwindigkeitsaddition} (Rapidity\index{Rapidität} $\psi = \mathrm{artanh}\left(\frac{v}{c}\right)$):
			\begin{equation}
				\begin{aligned}
					\psi_{tot} &= \psi_1+\psi_2 \\
					\vec{v}_{tot} &= \frac{\vec{v}_1+\vec{v}_{2\parallel}+\vec{v}_{2\perp}\sqrt{1-\dfrac{\vec{v}_1^2}{c^2}}}{1+\dfrac{\vec{v}_1\cdot\vec{v}_2}{c^2}} \\
					\vec{v}_1\parallel\vec{v}_2 \;\Rightarrow\; v_{tot} &= \frac{v_1+v_2}{1+\dfrac{v_1 v_2}{c^2}}
				\end{aligned}
			\end{equation}

			\noindent
			Relativistic aberration\index{Relativistische Aberration} (Observed Angle $\vartheta'$ for a relative velocity $\beta$ and an inclination of $\vartheta$ as measured in the observer's reference frame; Both formulas are equivalent):
			\begin{equation}
				\begin{aligned}
					\tan\left(\frac{\theta}{2}\right) = \sqrt{\frac{1-\beta}{1+\beta}}\tan\left(\frac{\theta'}{2}\right)\\
					\cos\vartheta' = \frac{\cos\vartheta+\beta}{1+\beta\cos\vartheta}
				\end{aligned}
			\end{equation}

			\noindent
			Energy Momentum Relation\index{Energie-Impuls-Relation}:
			\begin{equation}
				\begin{aligned}
					P^\mu P_\mu &= m^2 c^2\\
					E^2 &= p^2 c^2 + m^2 c^4 \\
				\end{aligned}
			\end{equation}
			\newpage

	\newpage
	% !TEX root = ../physics.tex
\section{Classical Mechanics\index{Klassische Mechanik}}
	\subsection{Newton's Laws of Motion\index{Newton!Kraftgesetze}}
		\textbf{First Axiom} \newline
		\indent \textit{Every body continues in its state of rest, or of uniform motion in a straight line, unless it is compelled to change that state by forces impressed upon it.} \nl
		\textbf{Second Axiom} \newline
		\indent \textit{The change of motion of an object is proportional to the force impressed and is made in the direction of the straight line in which the force is impressed.}
		\begin{equation}
			\dv{\vec{p}}{t} = \vec{F}
		\end{equation}\nl
		\textbf{Third Axiom} \newline
		\indent \textit{To every action there is always opposed an equal reaction.} \nl
		\textbf{Additional}: Principle of Superposition \newline
		\indent \textit{Forces are additive.}
		\begin{equation}
			\vec{F}_{\text{tot.}} = \sum_{i=1}^{N} \vec{F}_i
		\end{equation}

	\subsection{Basics}
		\noindent
		Virial Theorem (for the average kinetic Energy $\langle T \rangle$ and the virial $\sum_j \vec{x}_j\cdot\dot{\vec{p}}_j$):
		\begin{equation}
			\langle T \rangle = - \frac{1}{2} \sum_j \left\langle \vec{x}_j\cdot\dot{\vec{p}}_j \right\rangle
		\end{equation}
		for a potential of the form $V(\vec{r}_1,\vec{r}_2) \propto |\vec{r}_1-\vec{r}_2|^n$ this implies:
		\begin{equation}
			2\Avg{T} = n\Avg{V}
		\end{equation}

		\noindent
		Lyapunov exponent\index{Ljapunov!Exponent} $\lambda$ (for a dynamical system $\vec{Z}(t)$ with two trajectories initially separated by $\delta\vec{Z}(0)$ in phase space, Lyapunov time\index{Ljapunov!Zeit} $\tau=\frac{1}{\lambda}$):
		\begin{equation}
			\begin{aligned}
				\norm{\delta\vec{Z}(t)} &\sim \ex^{\lambda t} \norm{\delta\vec{Z}(0)}\\
				\lambda &= \lim_{t\to\infty} \lim_{\norm{\delta\vec{Z}(0)}\to 0} \frac{1}{t} \ln\qty(\frac{\norm{\delta \vec{Z}(t)}}{\norm{\delta\vec{Z}(0)}}) \\
			\end{aligned}
		\end{equation}

	\subsection{Rotation}
		\noindent
		Definition of the Inertia Tensor\index{Trägheitstensor}:
		\begin{equation}
			\Theta_{ij}=\int_{\mathcal{V}} \rho \left[r^2\delta_{ij}-r_i r_j\right] \;\dd^3\vec{r}
		\end{equation}

		\noindent
		Parallel Axis Theorem\index{Steiner!Satz} ($\Theta_{ij}$ is the inertia tensor for rotation around the center of mass.
		For rotation around a point displaced by $\vec{a}$, the inertia tensor transforms to $\Theta'_{ij}$, where $M$ is the total mass.):
		\begin{equation}
			\Theta'_{ij} = \Theta_{ij} + M\qty(\vec{a}^2\delta_{ij} - a_i a_j)
		\end{equation}

		\noindent
		Angular momentum and energy in relation to the inertia tensor:
		\begin{equation}
			\begin{aligned}
				L^i &= \Theta^i_j \omega^j \\
				T_{\text{rot.}} &= \frac{1}{2}\vec{\omega}^T \Theta \vec{\omega}
			\end{aligned}
		\end{equation}

	\subsection{Moving Coordinate Systems}
		\noindent
		Accelerated system $K'$ and inertial system $K$
		\begin{equation}
			\begin{aligned}
				\vec{x}'(t) &= \vec{x}(t)-\vec{x}_0(t) &\hsp
				m\dv{\vec{p}}{t} &= \vec{F} \\
				\dv{\vec{x}'}{t} &= \dv{\vec{x}}{t}-\vec{\omega}\times\vec{x} &\hsp
				m\dv{\vec{p}'}{t} &= \vec{F} + \vec{F}_\text{T} + \vec{F}_\text{Z} + \vec{F}_\text{L} + \vec{F}_\text{C} \\
			\end{aligned}
		\end{equation}

		\noindent
		Fictitious forces\index{Scheinkraft}: displacement force\index{Scheinkraft!Translationskraft} $\vec{F}_\text{T}$, centrifugal force\index{Scheinkraft!Zentrifugalkraft} $\vec{F}_\text{Z}$, Euler force\index{Scheinkraft!Linearkraft} $\vec{F}_\text{L}$, Coriolis force\index{Scheinkraft!Corioliskraft} $\vec{F}_\text{C}$
		\begin{equation}
			\begin{aligned}
				\vec{F}_\text{T} &= -m\dv[2]{\vec{x}_0}{t} &\hsp
				\vec{F}_\text{Z} &= -m\vec{\omega}\times\qty(\vec{\omega}\times\vec{x}') \\
				\vec{F}_\text{L} &= -m\dv{\vec{\omega}}{t}\times\vec{x}' &\hsp
				\vec{F}_\text{C} &= -2m\vec{\omega}\times\dv{\vec{x}'}{t} \\
			\end{aligned}
		\end{equation}


	\subsection{Newtonian Gravitation}
		\noindent
		Newton's law of universal gravitation\index{Newton!Gravitationskraft} (Acting on mass 1):
		\begin{equation}
			\vec{F}_1 = - G m_1 m_2 \frac{\vec{r}_1-\vec{r}_2}{\abs{\vec{r}_1-\vec{r}_2}^3}
		\end{equation}

		\noindent
		Equations of motion for Newton's law of universal gravitation:
		\begin{equation}
			\vec{F}=-m\Nabla\phi
		\end{equation}

		\noindent
		Poisson Equation for gravity\index{Poisson!Gleichung-Gravitation} / field equations for the Newtonian gravitational potential\index{Newton!Gravitationspotential} $\phi$ :
		\begin{equation}
			\Nabla^2\phi = 4\pi G\rho
		\end{equation}


	\subsection{Kepler Problem\index{Kepler!Problem}}
		\noindent
		Lagrangian (Gravitational parameter\index{Gravitationsparameter} $\mu$, for central body approximation $\mu=GM$, see \ref{Sec:AstronomicalConstants}):
		\begin{equation}
			\mathcal{L}(\vec{r},\dot{\vec{r}}) = \frac{m}{2} \dot{\vec{r}}^2 + \frac{\mu}{r}
		\end{equation}

		\noindent
		Kepler's second law\index{Flächensatz} (For a particle with a central force\index{Zentralpotential}):
		\begin{equation}
			\dv{A}{t} = \frac{\abs*{\vec{L}}}{2 m} = \const
		\end{equation}

		\noindent
		Gravitational parameter \index{Gravitationsparameter} $\mu$, eccentricity\index{Exzentrizität} $\epsilon$ and $\rho_0$:
		\begin{equation}
			\begin{aligned}
				\mu &= G m &\hsp
				\epsilon &= \sqrt{1+\frac{2 E \vec{L}^2}{m \mu^2}} &\hsp
				\rho_0 &= \frac{\vec{L}^2}{m\mu} \\
			\end{aligned}
		\end{equation}

		\noindent
		Conserved Laplace-Runge-Lenz vector:
		\begin{equation}
			\vec{A} = \dot{\vec{r}}\times\vec{L} - \frac{\mu \vec{r}}{r}
		\end{equation}

		\noindent
		Orbit parametrization (Cylindrical coordinates $\rho, \varphi$):
		\begin{equation}
			\rho = \frac{\rho_0}{1+\epsilon \cos\varphi}
		\end{equation}

		\noindent
		Vis-Viva\index{Vis-Viva Gleichung} Equation
		\begin{equation}
			v^2 = \mu\qty(\frac{2}{r} - \frac{1}{a})
		\end{equation}

		\noindent
		Rocket Equation\index{Raketengleichung} (final mass $m_\text{f}$, initial mass $m_\text{i}$, effective exhaust speed $u$, specific Impulse $I_{\text{sp.}}$, standard earth acceleration $g_0$):
		\begin{equation}
			\frac{m_\text{f}}{m_\text{i}} = \exp(-\frac{\Delta v}{u}) = \exp(-\frac{\Delta v}{g_0 I_{\text{sp.}}})
		\end{equation}

		\noindent
		Kepler's third law\index{Kepler!Drittes Gesetz}:
		\begin{equation}
			\frac{T^2}{a^3} = \frac{4\pi^2}{G(M+m)}
		\end{equation}

	\subsection{Lagrange-Formalism}
		\noindent
		Action\index{Wirkung}:
		\begin{equation}
			\mathcal{S}=\int_{t_0}^{t_1}\mathcal{L}(q_i, \dot{q_i},t)\;\mathrm{d} t
		\end{equation}

		\noindent
		Stationary-action principle / Principle of least action \index{Wirkung}:
		\begin{equation}
			\delta \mathcal{S}=0
		\end{equation}

		\noindent
		Euler-Lagrange equation\index{Euler!-Lagrange Gleichung}:
		\begin{equation}
			\frac{d}{dt} \frac{\partial \mathcal{L}(q_{i},\dot{q_{i}},t)}{\partial \dot{q_{i}}} - \frac{\partial \mathcal{L}(q_{i},\dot{q_{i}},t)}{\partial q_{i}} = 0 \quad \forall q_i
		\end{equation}

		\noindent
		Canonical momentum / conjugated momentum \index{Kanonisch-konjugierte Impulse}:
		\begin{equation}
			p_i=\frac{\partial \mathcal{L}}{\partial\dot{q_i}}
		\end{equation}

	\subsection{Hamiltonian Formalism}
		\noindent
		Hamiltonian\index{Hamilton!Funktion} (\ie the Legendre Transformation\index{Legendre!Transformation} of the Lagrangian):
		\begin{equation}
			H(q_i,p_i,t)=\sum_{j=1}^{f}p_j\dot{q_j}(p) - \mathcal{L}(q_i, \dot{q_i}(p),t)
		\end{equation}

		\noindent
		Hamilton's equations\index{Hamilton!Bewegungsgleichungen}:
		\begin{equation}
			\begin{aligned}
				\dot{q}_k &= \pdv{H}{p_k}, &&\hsp
				\dot{p}_k &= -\pdv{H}{q_k}, &&\hsp
				\pdv{H}{t} &= -\pdv{\mathcal{L}}{t}
			\end{aligned}
		\end{equation}

		\noindent
		Poisson-brackets (in Quantum Mechanics $\qty{A,B} \to -\frac{\i}{\hbar}\Avg{\comm{A}{B}}$):
		\begin{equation}
			\begin{aligned}
				\qty{A,B} &= \sum_k \left(
				\pdv{A}{q_k}\pdv{B}{p_k} - \pdv{A}{p_k}\pdv{B}{q_k}
				\right) \\
				\dv{A}{t} &= \qty{A, H} + \pdv{A}{t}
			\end{aligned}
		\end{equation}

		\noindent
		Canonical transformations\index{Kanonische Transformation} (transformations that leave the Hamilton equations invariant):
		\begin{equation}
			\begin{aligned}
				q_i &= q_i(q_i',p_i',t) &\hsp
				p_i &= p_i(q_i',p_i',t) \\
				H(q_i,p_i,t) &= H'(q_i',p_i',t) &\hsp
				\mathcal{L}(q_i,\dot{q}_i,t) &= \mathcal{L}'(q_i',\dot{q}_i',t) \\
			\end{aligned}
		\end{equation}

		Probability density function / phase space distribution function:
		\begin{equation}
			f(t,\vec{x},\vec{p}) = \frac{\dd N}{\dd^3 x \dd^3 p}
			\hsp
			n(t,\vec{x}) = \int \dd^3 p\, f(t,\vec{x},\vec{p})
			\hsp
			N(t) = \int \dd^3 x \dd^3 p\, f(t,\vec{x},\vec{p})
		\end{equation}

		\noindent
		Boltzmann Equation\index{Boltzmann!Gleichung} (where $f(\vec{q},\vec{p},t)$ is the phase space distribution function, $\mathcal{C}[f]$ is the collision term):
		\begin{equation}
			\pdv{f}{t} + \pdv{H}{\vec{p}}\cdot\pdv{f}{\vec{q}} - \pdv{H}{\vec{q}}\cdot\pdv{f}{\vec{p}} = \mathcal{C}[f]
		\end{equation}

		\noindent
		Liouville's theorem\index{Liouville!Theorem}, \ie the distribution function is constant along any trajectory in phase space. (phase space distribution function $f$, for quantum mechanical analogue see Eq.~\ref{eq:VonNeumannEquation}):
		\begin{equation}
			\label{eq:LiouvilleTheorem}
			\dv{f}{t} = \pdv{f}{t} + \qty{f, H} = 0
		\end{equation}

	\subsection{Noether's Theorem\index{Noether!Theorem}}
		\noindent
		Continuous transformation (infinitesimal $\varepsilon$):
		\begin{equation}
			\begin{aligned}
				q_i \to q_{i}^{\prime} &= q_i+\varepsilon\psi_i\left(q,\dot{q},t\right) \\
				t \to t^{\prime}\, &= t+\varepsilon\varphi\left(q,\dot{q},t\right) \\
			\end{aligned}
		\end{equation}

		\noindent
		Condition of invariance\index{Invarianzbedingung}:
		\begin{equation}
			\frac{d}{d\varepsilon}\left[\mathcal{L}\left( {\vec{q}}^{\,\prime},\frac{d {\vec{q}}^{\,\prime}}{dt'},t'\right) \frac{dt'}{dt}\,\right]_{\varepsilon=0}=\frac{df(\vec{q}, t)}{dt} \implies S \sim S'
		\end{equation}

		\noindent
		Resulting conserved quantity\index{Erhaltungsgröße}:
		\begin{equation}
			\begin{aligned}
				S &\sim S' \implies
				\dv{t} Q\left(\vec{q},\dot{\vec{q}},t\right) = 0 \\
				Q\left(\vec{q},\dot{\vec{q}},t\right) &= \sum_{i=1}^{n}\left(\frac{\partial\mathcal{L}}{\partial{\dot{q}}_i}\psi_i\right)+\left(\mathcal{L}-\sum_{i=1}^{n}{\frac{\partial\mathcal{L}}{\partial{\dot{q}}_i}{\dot{q}}_i}\right)\varphi - f\left(\vec{q},t\right).
			\end{aligned}
		\end{equation}
	\newpage
	% !TEX root = ../physics.tex
\section{Classical field theory\index{Klassische Feldtheorie}}
	\subsection{Lagrange Formalism}
		\noindent
		Action for a scalar field $\phi(x)$ (\ie $\phi(x)\to P(\phi(x))=\phi(x)$, where $P\in\mathcal{P}$ is a Poincaré Transform, $\mathcal{L}$ is the Lagrange density):
		\begin{equation}
			\mathcal{S}\qty[\phi(x)] = \int_{\mathbb{R}} L \;\dd t = \int_{\mathbb{R}^4} \mathcal{L}(\phi(x),\partial_\mu \phi(x)) \;\dd^4 x
		\end{equation}

		\noindent
		Euler--Lagrange equation\index{Euler!--Lagrange Gleichung}\index{Lagrange!Gleichung}\index{Lagrange!Euler--Lagrange Gleichung}:
		\begin{equation}
			\delta\mathcal{S} = 0
			\implies \partial_\mu \qty(\pdv{\mathcal{L}}{\left(\partial_\mu \phi\right)}) - \pdv{\mathcal{L}}{\phi} = 0
		\end{equation}

		\noindent
		Classical Klein--Gordon Equation\index{Klein!--Gordon Gleichung}\index{Gordon!Klein--Gordon Gleichung} for a free Field $\mathcal{L}(\phi(x)) = \frac{1}{2} \partial_\mu \phi \partial^\mu \phi - \frac{1}{2} m^2 \phi^2$:
		\begin{equation}
			\delta \mathcal{S} = 0 \implies \qty(\partial_\mu \partial^\mu + m^2)\phi = 0
		\end{equation}


	\subsection{Hamilton Formalism}
		\noindent
		Transformation to Hamilton function $H$ and Hamiltonian density $\mathcal{H}$:
		\begin{equation}
			H = \int_{\mathbb{R}} \mathcal{H}(\phi(x),\pi(x)) \;\dd^3 x = \int_{\mathbb{R}^3} \qty[\pi(x)\dot{\phi}(x) - \mathcal{L}] \;\dd x^3
		\end{equation}

	\subsection{Noether's Theorem\index{Noether!Theorem}}
		\noindent
		Continuous transformation (with a continuous infinitesimal variation $\delta_\varepsilon \phi$, characterized by an infinitesimal variable $\varepsilon$):
		\begin{equation}
			\phi(x)\to\phi'(x) = \phi(x) + \delta_\varepsilon \phi(x)
		\end{equation}

		\noindent
		Condition of invariance\index{Invarianzbedingung}:
		\begin{equation}
			\delta\mathcal{L} = \partial_\mu F^\mu(\phi)
			\implies \mathcal{S} \sim \mathcal{S}'
		\end{equation}

		\noindent
		Resulting conserved current / Noether current\index{Noether!-Strom} $j^{\mu}$ and conserved charge $Q$ (Where $\Delta \phi = \eval{\pdv{\delta_\varepsilon \phi}{\varepsilon}}_{\varepsilon=0}$ is the generator of that transformation):
		\begin{equation}
			\begin{aligned}
				\mathcal{S} \sim \mathcal{S}' \implies
				\partial_\mu j^\mu &= 0;\quad
				j^\mu \qty( \phi(x), \partial_\nu\phi(x) ) = \pdv{\mathcal{L}}{\left(\partial_\mu \phi\right)}\Delta\phi - F^\mu(\phi) \\
				\dv{Q}{t} &= 0; \quad Q = \int_{\mathbb{R}^3} j^0\;\dd^3\vec{x} \\
			\end{aligned}
		\end{equation}
		Note: The conserved current is only defined up to divergence $j^\mu \sim j^\mu + K^\mu$ for $\partial_\mu K^\mu = 0$.
	\newpage
	% !TEX root = ../physics.tex
\section{Electromagnetism\index{Elektromagnetismus}}
	\subsection{Equations of motion}
		\noindent
		Action of a free particle in an electromagnetic field:%Wirkung eines freien Teilchens in einem Elektromagnetischen Feld:
		\begin{equation}
			\mathcal{S}=-\int_{a}^{b}\left(mc\sqrt{g_{\mu\nu}\frac{\diff x^\mu}{\diff \tau}\frac{\diff x^\nu}{\diff \tau}}
	+ q\frac{\diff x^\mu}{\diff \tau}A_\mu(x)\right)\;\diff\tau
		\end{equation}

		\noindent
		Lagrangian with parametrization $\tau$ (e.g. proper time):%Lagrange Funktion mit Parametrisierung $\tau$ (Zum Beispiel durch die Eigenzeit):
		\begin{equation}
			\mathcal{L} (\tau,x,u) =-\left(mc\sqrt{g_{\mu\nu}\frac{\diff x^\mu}{\diff \tau}\frac{\diff x^\nu}{\diff \tau}}
			+ q\frac{\diff x^\mu}{\diff \tau}A_\mu(x)\right)
		\end{equation}

		\noindent
		Non-relativistic Lagrangian:%Nicht-relativistische Lagrange Funktion:
		\begin{equation}
			\mathcal{L}(t,\vec{x},\dot{\vec{x}}) = \frac{1}{2}m\dot{\vec{x}}^2 - q\phi(t,\vec{x}) - \dot{\vec{x}}\cdot\vec{A}(t,\vec{x})
		\end{equation}

		\noindent
		Lorentz force\index{Lorentz!Kraft} (relativistic Minkowski force\index{Minkowski!Kraft} and Newtonian force):
		\begin{equation}
			\begin{aligned}
				\frac{\diff P^\mu}{\diff \tau} &= q F^{\mu\nu}\frac{\diff x_\nu}{\diff \tau} \\
				\vec{F} &= q\left(\vec{E}+\vec{v}\times\vec{B}\right) \\
			\end{aligned}
		\end{equation}

	\subsection{Field equations}
		\noindent
		Lagrange density:
		\begin{equation}
			\mathcal{L} = -\frac{1}{4\mu_0}F^{\mu\nu} F_{\mu\nu} - A_\mu J^\mu
		\end{equation}

		\noindent
		General field equations without gauge \index{Entwicklungsgleichung}:
		\begin{equation}
			\Box A^\mu-\partial^\mu\left(\partial_\nu A^\nu\right) = \partial_\nu F^{\nu\mu} =  \mu_0 J^\mu
		\end{equation}

		\noindent
		Maxwell's equations\index{Maxwell!Gleichungen} in a vacuum:
		\begin{equation}
		\begin{array}{rl}
			\vec{\nabla}\times \vec{E} + \cfrac{\partial\vec{B}}{\partial t} = 0 \phantom{\mu_0}
			&\hspace{30pt} \vec{\nabla}\cdot\vec{E} = \cfrac{\rho}{\varepsilon_0} \\ \xrowht{40pt}
			\vec{\nabla}\times\vec{B} - \mu_0 \epsilon_0 \cfrac{\partial \vec{E}}{\partial t} = \mu_o\vec{j}
			&\hspace{30pt} \vec{\nabla}\cdot\vec{B} = 0 \\
		\end{array}
		\end{equation}

		\noindent
		Maxwell's equations\index{Maxwell!Gleichungen} in matter:
		\begin{equation}
		\begin{array}{rl}
			\vec{\nabla}\times \vec{E} + \cfrac{\partial\vec{B}}{\partial t} = 0 \phantom{_f}
			&\hspace{20pt} \vec{\nabla}\cdot\vec{D} = \rho_f\\ \xrowht{40pt}
			\vec{\nabla}\times\vec{H} - \cfrac{\partial \vec{D}}{\partial t} = \vec{j}_f
			&\hspace{20pt} \vec{\nabla}\cdot\vec{B} = 0 \\
		\end{array}
		\end{equation}

		\noindent
		Electric flux density and magnetic field strength:%Elektrische Flussdichte und Magnetische Feldstärke:
		\begin{equation} \label{Eq:FluxDensityFieldStrength}
		\begin{array}{cc}
			\vec{H} := \cfrac{1}{\mu_0}\Vec{B} - \vec{M}\left(\vec{B}\right)
			&\hspace{20pt} \vec{D} := \varepsilon_0\vec{E} + \vec{P}\left(\vec{E}\right)
		\end{array}
		\end{equation}


		\noindent
		Maxwell's equations\index{Maxwell!Gleichungen} in their integrated form (Electric voltage $U(\Gamma)=\int_\Gamma \vec{E}\cdot\diff\vec{l}$, electric flux $\Psi(\mathcal{A})=\int_\mathcal{A}\vec{D}\cdot\vec{n}\;\diff A$, magnetic voltage $V(\Gamma)=\int_\Gamma \vec{H}\cdot\diff\vec{l}$ and magnetic flux $\Phi(\mathcal{A})=\int_\mathcal{A}\vec{B}\cdot\vec{n}\;\diff A$):
		\begin{equation}
		\begin{array}{rl}
			U(\partial \mathcal{A}) + \cfrac{\diff\Phi}{\diff t}(\mathcal{A}) = 0\phantom{(\mathcal{A})}
			&\hspace{20pt} \Psi(\partial\mathcal{V}) = Q_f(\mathcal{V}) \\ \xrowht{40pt}
			V(\partial \mathcal{A}) - \cfrac{\diff \Psi}{\diff t}(\mathcal{A}) = I_f(\mathcal{A})
			&\hspace{20pt} \Phi(\partial\mathcal{V}) = 0 \\
		\end{array}
		\end{equation}

		\subsubsection{Discontinuity Equations\index{Diskontinuitätsgleichungen}}
			\noindent
			Boundary conditions in a vacuum (Surface charge density $\sigma$, surface charge current density $\vec{K}$):
		\begin{equation}
		\begin{array}{cc}
			\vec{E}^2 - \vec{E}^1 = \cfrac{\sigma}{\varepsilon_0}\Vec{n}^2
			&\hspace{20pt} \vec{B}^2 - \vec{B}^1 = \mu_0\vec{K}\times\Vec{n}^2
		\end{array}
		\end{equation}

			\noindent
			Boundary conditions in matter:
		\begin{equation}
		\begin{array}{cc}
			D_\perp^2 - D_\perp^1 = \sigma_f
			&\hspace{20pt} \vec{H}_\parallel^2 - \vec{H}_\parallel^1 = \vec{K}_f\times\Vec{n}^2
		\end{array}
		\end{equation}

	\subsection{Electrodynamics\index{Elektrodynamik}}
		\noindent
		Continuity equation\index{Kontinuitätsgleichung!Elektromagnetismus}:
		\begin{equation}
			\partial_\mu J^\mu = \pder{\rho}{t} + \Nabla\cdot\vec{j} = 0
		\end{equation}

		\noindent
		Integrated form:
		\begin{equation}
			\dot{Q}(\mathcal{V}) = I(\partial\mathcal{V})
		\end{equation}

		\noindent
		Lorentz invariants of the electromagnetic field:
		\begin{equation}
			\begin{aligned}
				F_{\mu\nu} F^{\mu\nu} = \mathrm{invariant} &\Rightarrow \vec{E}^2 - \vec{B}^2 = \mathrm{invariant} \\
				F_{\mu\nu} F^{\nu}_{\;\gamma} F^{\gamma}_{\;\rho} F^{\rho\mu} = \mathrm{invariant} &\Rightarrow \vec{E}\cdot \vec{B} = \mathrm{invariant} \\
			\end{aligned}
		\end{equation}

		\subsubsection{Electromagnetic Energy and Momentum}
			\noindent
			Energy density:
			\begin{equation}
				u_{EM}=\frac{1}{2}\epsilon_0 \vec{E}^2+\frac{1}{2\mu_0}\vec{B}^2
			\end{equation}

			\noindent
			Poynting vector\index{Poynting!Vektor} / Energy flux density:
			\begin{equation}
				\vec{S} = \frac{1}{\mu_0}\vec{E}\times\vec{B}
			\end{equation}

			\noindent
			Poynting theorem\index{Poynting!Theorem} / Energy continuity:
			\begin{equation}
				-\Nabla\cdot\vec{S}
				= \pder{u_{\mathrm{mech}}}{t} + \pder{u_{EM}}{t}
				= \vec{j}\cdot\vec{E} + \frac{1}{2}\pder{}{t}\Br{\varepsilon_0\vec{E}^2 + \frac{1}{\mu_0}\vec{B}^2}
			\end{equation}

			\noindent
			Momentum density:
			\begin{equation}
				\vec{\pi} = \tder{\vec{p}_{EM}}{V} = \frac{1}{c^2}\vec{S}
			\end{equation}

			\noindent
			Conservation of Momentum (Maxwell stress tensor\index{Maxwell!Spannungstensor} $T$):
			\begin{equation}
				\pder{\vec{p}_{EM}}{t} - \Nabla\cdot T + \rho \vec{E} + \vec{J}\times\vec{B} = 0
			\end{equation}

			\noindent
			Conservation of energy and momentum (Force $F$, power $P$, Poynting vector $S_i$, stress tensor $T_{ij}$, momentum density $\pi_i$):
			\begin{equation}
				\begin{aligned}
					\pder{u_{EM}}{t} &= \tder{P}{V} - \pder{S_j}{x_j} \\
					\pder{\pi_i}{t} &=	\tder{F_i}{V} - \pder{T_{ij}}{x_j}
				\end{aligned}
			\end{equation}

		\subsubsection{Electromagnetic Energy-Momentum Tensor\index{Energie-Impuls-Tensor}}
			\noindent
			Energy-momentum tensor (Maxwell stress tensor $T_{ij}$):
			\begin{equation}
				T^{\mu\nu} = \frac{1}{\mu_0}\left(g^{\mu\alpha} F_{\alpha\beta} F^{\beta\nu} +\frac{1}{4}g^{\mu\nu} F_{\alpha\beta} F^{\alpha\beta} \right)
				= \left( \begin{matrix}
					u_{EM} & S_1/c & S_2/c & S_3/c \\
					S_1/c & -T_{11} & -T_{12} & -T_{13} \\
					S_2/c	& -T_{21} & -T_{22} & -T_{23} \\
					S_3/c & -T_{31} & -T_{32} & -T_{33}
				\end{matrix} \right)
			\end{equation}

			\noindent
			Maxwell stress tensor\index{Maxwell!Spannungstensor} and properties
			\begin{equation}
				T_{ij} = \varepsilon_0 \Br{ E_i E_j + c^2 B_i B_j - \frac{\delta_{ij}}{2}\Br{ \vec{E}^2 + c^2 \vec{B}^2 } }
			\end{equation}
			\begin{equation}
				\begin{aligned}
					T^{\mu\nu} &= T^{\nu\mu} &\hspace{20pt}
					T^\mu_{\phantom{\mu}\mu} &= 0 \\
				\end{aligned}
			\end{equation}

			\noindent
			Conservation of energy and momentum:
			\begin{equation}
				\begin{aligned}
					\partial_\mu T^{\mu\nu} + F^{\nu\lambda} J_\lambda &= 0
					%\partial_\mu T^{\mu\nu} &= 0
				\end{aligned}
			\end{equation}


		\subsubsection{Electromagnetic Waves in a Vacuum}
			\noindent
			Wave equation in a vacuum\index{Wellengleichung}:
			\begin{equation}
				\Box\psi = \frac{1}{c^2} \frac{\partial^2 \psi}{\partial t^2} - \Nabla^2 \psi = 0
			\end{equation}

			\noindent
			General vector wave (e.g. $\vec{\psi}=\vec{E}$ or $ \vec{\psi}=\vec{B}$):
			\begin{equation}
				\vec{\psi}(t,\vec{r}) = \int_{\mathbb{R}^3} \tilde{\vec{\psi}} (\vec{k}) e^{\i\left(\vec{k}\cdot\vec{r} - \omega(\vec{k})t \right)}\; \diff^3\vec{k}
			\end{equation}

			\noindent
			Orthogonaliy of the wave vector (Electromagnetic waves in a vacuum are always transversal):
			\begin{equation}
				\tilde{\vec{B}} = \frac{1}{c}\hat{\vec{k}}\times\tilde{\vec{E}} = \frac{1}{\omega}\vec{k}\times\tilde{\vec{E}}
			\end{equation}

			\noindent
			Energy flux density of an electromagnetic wave:
			\begin{equation}
				\vec{S} = u_{EM}c\hat{\vec{k}}
			\end{equation}

			\noindent
			Energy density and intensity of a linearly polarized wave:
			\begin{equation}
				\begin{aligned}
					u_{EM} &= \frac{1}{2}\epsilon_0\vec{E}^2 &\hspace{20pt}
					I &= \langle|\vec{S}|\rangle = \frac{1}{2}c\epsilon_0\vec{E}^2
				\end{aligned}
			\end{equation}

			\noindent
			Energy density and intensity of a circularly polarized wave:
			\begin{equation}
				\begin{aligned}
					u_{EM} &= \epsilon_0\vec{E}^2 &\hspace{20pt}
					I &= \langle|\vec{S}|\rangle = c\epsilon_0\vec{E}^2
				\end{aligned}
			\end{equation}

			\noindent
			Emitted wave of an oscillating dipole\index{Dipol!Strahlung} $\vec{p} = \vec{p}_0 e^{-\i\omega t}$ (Retarded time $\tilde{t} = t-\frac{\left|\vec{r}-\pvec{r}'\right|}{c}$):
			\begin{equation}
				\begin{aligned}
					\vec{E}(t,\vec{r}) &= -\frac{\mu_0}{4\pi r c}	\left(\vec{n}\times\ddot{\vec{p}}(\tilde{t})\right) \times \vec{n} \\
					\vec{B}(t,\vec{r}) &= -\frac{\mu_0}{4\pi r c} \phantom{\Big(}\vec{n}\times\ddot{\vec{p}}(\tilde{t}) \\
				\end{aligned}
			\end{equation}

			\noindent
			Larmor formula\index{Larmor!Formel} (Emitted energy of a non-relativistic particle, retarded acceleration $a(\tilde{t}$):
			\begin{equation}
				P(t) = \frac{q^2 \mu_0}{6\pi c}a^2(\tilde{t})
			\end{equation}

		\subsubsection{Electromagnetic Waves in Matter}
			\noindent
			Definition index of refraction\index{Brechungsindex} (phase velocity $v_{PH}$):
			\begin{equation}
				n(k) := \frac{c}{v_{PH}(k)} = \sqrt{\frac{\epsilon\mu}{\epsilon_0\mu_0}}
			\end{equation}

			\noindent
			Snell's law\index{Snellius!Brechungsgesetz}:
			\begin{equation}
				\frac{\sin\alpha_1}{\sin\alpha_2} = \frac{n_2}{n_1}
			\end{equation}



		\subsubsection{Electrical Engineering}
			\noindent
			Ohm's law\index{Ohm!Gesetz} (Specific conductivity $\sigma$, resistance $R$, both assumed linear):
			\begin{equation}
				\begin{aligned}
					\vec{j} &= \sigma\vec{E} \\
					U &= R I
				\end{aligned}
			\end{equation}

			\noindent
			Kirchhoff's circuit laws\index{Kirchhoff!Regeln}:
			\begin{description}
				\item[1. Kirchhoff's current law\index{Kirchhoff!Knotenregel}] \hfill \\
					{\textit{The algebraic sum of currents in a network of conductors meeting at a point is zero.} \\(special case of continuity equation)}
				%\item[1. Knotenregel]\hfill \\
				%	An jedem Knoten ist die Summe der einfließenden elektrischen und ausfließenden elektrischen Ströme gleich null (Kontinuitätsgleichung).
				\item[2. Kirchhoff's voltage law\index{Kirchhoff!Maschenregel}] \hfill \\
					{\textit{The directed sum of the potential differences (voltages) around any closed loop is zero.} \\(electrostatics)}
				%\item[2. Maschenregel]\hfill \\
				%	Alle Teilspannungen eines Umlaufs bzw. einer Masche addieren sich zu null (Elektrostatik)
			\end{description}

			\noindent
			Joule heating\index{Joule!Wärmegesetz} (Heat generation in an ohmic conductor):
			\begin{equation}
				P = UI = RI^2 = \frac{U^2}{R}
			\end{equation}



	\subsection{Definitions}
		\noindent
		Speed of light\index{Lichtgeschwindigkeit}:
		\begin{equation}
			c=\frac{1}{\sqrt{\epsilon_0 \mu_0}}
		\end{equation}

		\noindent
		D'Alembert operator\index{d'Alembert!Operator}:
		\begin{equation}
			\Box = \partial^\mu \partial_\mu = \frac{1}{c^2}\frac{\partial^2}{\partial t^2} - \vec{\nabla}^2
		\end{equation}

		\noindent
		Four-current\index{Viererstromdichte}:
		\begin{equation}
			J^\mu = \binomkoeff{c\rho}{\vec{j}} = \binomkoeff{c\tder{q}{V}}{\vec{v}\tder{q}{V}}
		\end{equation}

		\noindent
		Four-potential\index{Viererpotential}:
		\begin{equation}
			A^\mu = \binomkoeff{\phi/c}{\vec{A}}
		\end{equation}

		\noindent
		Faraday tensor\index{Faraday!tensor} / electromagnetic tensor / field strength tensor\index{Faraday!Feldstärkentensor}:
		\begin{equation}
			F^{\mu\nu} = \partial^\mu A^\nu - \partial^\nu A^\mu
			= \left( \begin{matrix}
				0 & -E_x/c & -E_y/c & -E_z/c \\
				E_x/c & 0 & -B_z & B_y \\
				E_y/c	& B_z & 0 & -B_x \\
				E_z/c & -B_y & B_x & 0
			\end{matrix} \right)
		\end{equation}

		\noindent
		Hodge Dual\index{Faraday!Dualer Feldstärkentensor}:
		\begin{equation}
			\tilde{F}^{\mu\nu} = \frac{1}{2}\varepsilon^{\mu\nu\alpha\beta}F_{\alpha\beta}
		\end{equation}

		\noindent
		Electric field strength:
		\begin{equation}
			\vec{E} = -\Nabla\phi-\pder{\vec{A}}{t}
		\end{equation}

		\noindent
		Magnetic flux density:
		\begin{equation}
			\vec{B} = \Nabla\times\vec{A}
		\end{equation}

	\subsection{Gauge Transformation\index{Eichtransformation}}
		\noindent
		Gauge transformation\index{Eichtransformation}:
		\begin{equation}
			A^\mu \rightarrow A^\mu-\partial^\mu \lambda
		\end{equation}

		\subsubsection{Lorenz Gauge\index{Lorenz!Eichung}}
			\noindent
			Lorenz gauge\index{Lorenz!Eichung}:
			\begin{equation}
				\partial_\mu A^\mu = \frac{1}{c}\frac{\partial \phi}{\partial t} + \vec{\nabla}\cdot\vec{A} = 0
			\end{equation}

			\noindent
			Maxwell equations using Lorenz gauge \index{Maxwell!Gleichungen}:
			\begin{equation}
				\Box A^\mu = \mu_0 J^\mu \;\Leftrightarrow\;
				\Box \phi = \dfrac{\rho}{\epsilon_0} \;\wedge\;
				\Box \vec{A} = \mu_0 \vec{j}
			\end{equation}

		\subsubsection{Coulomb Gauge \index{Coulomb!Eichung}}
			\noindent
			Coulomb gauge (not unique):
			\begin{equation}
				\Nabla\cdot\vec{A}(t,\vec{r})=0
			\end{equation}

	\subsection{Special Solutions of the Field Equations}
		\subsubsection{Retarded Potentials\index{Retardierte Potentiale}}
			\label{Sec:RetardedPotentials}
			\noindent
			Retarded potentials (general solutions of the non-static Maxwell's equations with Lorenz gauge for known charge and current distributions for all times):
			%Retardierte Potentiale (Allgemeine Lösung der nicht-statischen Maxwellgleichungen mit Lorenz-Eichung für Strom- und Ladungsverteilung die zu jedem Zeitpunkt bekannt sind):
			\begin{equation}
				\begin{aligned}
					\phi\left(t,\vec{r}\right)
					= \frac{1}{4\pi\epsilon_0} \int_{\mathbb{R}^3} \frac{\rho(\tilde{t},\pvec{r}')}{\left|\vec{r}-\pvec{r}'\right|}\;\diff^3 \pvec{r}'
					&=	\frac{1}{4\pi\epsilon_0} \int_{\mathbb{R}^4} \frac{\rho(t',\pvec{r}')}{\left|\vec{r}-\pvec{r}'\right|}\delta\left( c(t-t')-\left|\vec{r}-\pvec{r}'\right|\right)\,\diff^3 \pvec{r}'\diff t' \\
					\vec{A}\left(t,\vec{r}\right)
					= \phantom{\epsilon_0} \frac{\mu_0}{4\pi} \int_{\mathbb{R}^3} \frac{\vec{j}(\tilde{t},\pvec{r}')}{\left|\vec{r}-\pvec{r}'\right|}\,\diff^3 \pvec{r}'
					&=	\phantom{\epsilon_0} \frac{\mu_0}{4\pi} \int_{\mathbb{R}^4} \frac{\vec{j}(t',\pvec{r}')}{\left|\vec{r}-\pvec{r}'\right|}\delta\left( c(t-t')-\left|\vec{r}-\pvec{r}'\right|\right)\;\diff^3 \pvec{r}'\diff t' \\
					\tilde{t}&=t-\frac{\left|\vec{r}-\pvec{r}'\right|}{c}
				\end{aligned}
			\end{equation}

		\subsubsection{Liénard-Wiechert Potentials\index{Liénard!-Wiechert Potentiale}}
			\noindent
			Liénard-Wiechert potentials\index{Liénard!-Wiechert Potentiale} (Solutions of the field equations for a moving point charge at $\vec{u}(\tilde{t})$ and velocity $\vec{v}(\tilde{t})$; retarded time $\tilde{t}$):
			%Lösung der Feldgleichungen für eine bewegte Punktladung mit Ort $\vec{u}(\tilde{t})$ und Geschwindigkeit $\vec{v}(\tilde{t})$ mit der retardierten Zeit $\tilde{t}$. Liénard-Wiechert-Potentiale:
			\begin{equation}
				\begin{aligned}
					\phi(t,\vec{r}) & =\frac{q c}{4\pi \epsilon_0}\frac{1}{c \left|\vec{r}-\vec{u}(\tilde{t})\right|-\vec{v}(\tilde{t})\cdot\left(\vec{r}-\vec{u}(\tilde{t})\right)} \\
					\vec{A}(t,\vec{r}) &= \frac{1}{c^2}\vec{v}(\tilde{t})\phi(t,\vec{r}) \\
				\end{aligned}
			\end{equation}

	\subsection{Electrostatics\index{Elektrostatik}}
		\subsubsection{Basics}
			\noindent
			Coulomb force\index{Coulomb!Kraft} (Acting on charge 1):
			\begin{equation}
				\vec{F}_1 = \frac{q_1 q_2}{4\pi\varepsilon_0}\frac{\vec{r}_1-\vec{r}_2}{\left|\vec{r}_1-\vec{r}_2\right|^3}
			\end{equation}

			\noindent
			Poisson equation\index{Poisson!Gleichung} of electrostatics:
			\begin{equation}
				\Nabla^2\phi = -\frac{\rho}{\epsilon_0}
				\label{Eq:Poisson}
			\end{equation}

			\noindent
			Electrical potential for static systems (Resulting from the retarded potentials Sec.~\ref{Sec:RetardedPotentials}):
			\begin{equation}
				\phi\left(t,\vec{r}\right)
				= \frac{1}{4\pi\epsilon_0} \int_{\mathbb{R}^3} \frac{\rho(\pvec{r}')}{\left|\vec{r}-\pvec{r}'\right|}\;\diff^3 \pvec{r}'
			\end{equation}

		\subsubsection{Solving the Poisson Equation\index{Poisson!Gleichung}}
			\noindent
			Dirichlet-Green function\index{Dirichlet!-Green Function} $G_D$:
			\begin{equation}
				\forall\pvec{r}\in\mathcal{V}': \left\{\begin{array}{ll}
						\forall\vec{r}\in\mathcal{V}\phantom\partial:
						\Nabla^2 G_D(\vec{r},\pvec{r}') = \delta(\vec{r}-\pvec{r}') \\
						\forall\vec{r}\in\partial\mathcal{V}:
						\phantom{\Nabla^2}G_D(\vec{r},\pvec{r}') = 0
					\end{array}\right.
			\end{equation}

		\subsubsection{Multipole Expansion\index{Multipolentwicklung}}
			\noindent
			Multipole expansion\index{Multipolentwicklung} in cartesian coordinates (expansion of the general solution in $\frac{r'}{r}=0$):
			\begin{equation}
				\begin{aligned}
					\phi(\pvec{r}) = \frac{1}{4\pi\epsilon_0}\int\frac{\rho(\pvec{r}')}{\left|\vec{r}-\pvec{r}'\right|}\;\diff \pvec{r}'
					&= \frac{1}{4\pi\epsilon_0}\left(\frac{Q}{r} + \frac{\vec{r}\cdot\vec{p}}{r^3} + \frac{1}{2}\frac{r_i r_j Q_{ij}}{r^5} + \mathcal{O} \left(\frac{1}{r^4}\right)\right) \\
					&= \frac{1}{4\pi\epsilon_0}\frac{1}{r}\int\rho(\pvec{r}')\;\diff \pvec{r}' \\
					&+ \frac{1}{4\pi\epsilon_0}\frac{1}{r}\int\rho(\pvec{r}')\left(\frac{r'}{r}\right)\cos\theta\;\diff \pvec{r}' \\
					&+ \frac{1}{4\pi\epsilon_0}\frac{1}{r}\int\rho(\pvec{r}')\left(\frac{r'}{r}\right)^2\frac{3\cos^2\theta-1}{2}\;\diff \pvec{r}' \\
					&+ \mathcal{O}\left(\;\;\frac{1}{r}\int\rho(\pvec{r}')\left(\frac{r'}{r}\right)^3 \;\diff \pvec{r}'\right) \\
				\end{aligned}
			\end{equation}
			Monopole $Q$, dipole $\vec{p}$, Quadrupole $Q_{ij}$:
			\begin{equation}
				\begin{aligned}
					Q &= \int \rho(\pvec{r}') \;\diff^3 \pvec{r}' \\
					\vec{p} &= \int \rho(\pvec{r}')\pvec{r}' \;\diff^3 \pvec{r}' \\
					Q_{ij} &= \int \rho(\pvec{r}')\left(3r'_i r'_j - \delta_{ij} r'^2 \right)	\;\diff^3 \pvec{r}' \\
				\end{aligned}
			\end{equation}

			\noindent
			Expansion of the distance in spherical harmonics\index{Kugelflächenfunktionen}:
			\begin{equation}
				\frac{1}{\left|\vec{r}-\pvec{r}'\right|}=\sum_{l=0}^{\infty}\sum_{m=-l}^{l} \frac{4\pi}{2l+1}\frac{r'^l}{r^{l+1}} Y^{*}_{lm}(\theta',\varphi')Y_{lm}(\theta,\varphi)
			\end{equation}

			\noindent
			Multipole expansion in spherical harmonics:
			\begin{equation}
				\begin{aligned}
					\phi(r,\theta,\varphi) &= \sum_{l=0}^{\infty}\sum_{m=-l}^{l} \frac{b_{lm}}{r^{l+1}}Y_{lm}(\theta,\varphi) \\
					b_{lm} &= \frac{1}{(2l+1)\epsilon_0}\int\rho(\pvec{r}')r'^lY^{*}_{lm}(\theta',\varphi')\;\diff^3\pvec{r}'
				\end{aligned}
			\end{equation}

			\noindent
			Electric dipole moment\index{Elektrisches Dipolmoment} of two point charges with a distance vector $\vec{d}$ spanning from the negative to the positive charge:%zweier Punktladung mit Abstandsvektor $\vec{d}$ von der negativen zur positiven Ladung:
			\begin{equation}
				\vec{p}=q\vec{d}
			\end{equation}

			\noindent
			Force, torque (with respect to the center of mass) and potential energy of an electric dipole:%Kraft, Drehmoment und potentielle Energie eines Dipols (Drehmoment in Bezug auf den Massenschwerpunkt $\vec{r}_m$):
			\begin{equation}
				\begin{aligned}
					\vec{F} &= \Nabla\left(\vec{p}\cdot\vec{E}\right) \\
					\vec{D}(\vec{r}_m) &= \vec{p}\times\vec{E} \\
					\mathcal{E}_{pot} &= -\vec{p}\cdot\vec{E} \\
				\end{aligned}
			\end{equation}

		\subsubsection{Solving the Laplace Equation}
			\noindent
			Laplace equation\index{Laplace!Gleichung} (harmonic Poisson equation\index{Poisson!Gleichung} \ref{Eq:Poisson})
			\begin{equation}
				\Nabla^2 \phi = 0
			\end{equation}

			\noindent
			Solution of the Laplace equation\index{Laplace!Gleichung} in systems with spherical and axial symmetry (Legendre polynomials\index{Legendre!Polynome} $P_l$): \ref{Eq:LegendrePolynomials})%Lösungen der Laplace-Gleichung mit Axialsymmetrie (Mit Legendre-Polynomen $P_l$ (\ref{Eq:LegendrePolynomials})):
			\begin{equation}
				\phi(r,\theta)=\sum_{l=0}^\infty \left(A_l r^l + \frac{B_l}{r^{l+1}}\right)P_l(\cos\theta)
			\end{equation}

			\noindent
			Solutions of the Laplace equation\index{Laplace!Gleichung} with spherical but without axial symmetry (spherical harmonics\index{Kugelflächenfunktionen} $Y_{lm}$ (\ref{Eq:SphericalHarmonics})):
			\begin{equation}
				\phi(r,\theta,\varphi) = \sum_{l=0}^{\infty}\sum_{m=-l}^{m=l} \left(a_{lm} r^l + \frac{b_{lm}}{r^{l+1}}\right) Y_{lm}(\theta, \varphi)
			\end{equation}

		\subsubsection{Capacitance\index{Kapazität}}
			\noindent
			Capacitance\index{Kapazität}:
			\begin{equation}
				C=\frac{Q}{U}
			\end{equation}

			\noindent
			Capacity\index{Kapazität} between two plates with surface area $A$, distance $d$ and dielectric\index{Dielektrikum} with permittivity \index{Permittivität} $\epsilon$:
			\begin{equation}
				C=\epsilon \frac{A}{d}
			\end{equation}

			\noindent
			Energy of a capacitor:
			\begin{equation}
				\mathcal{E} = \frac{1}{2}\frac{Q^2}{C} = \frac{1}{2}CU^2
			\end{equation}

		\subsubsection{Dielectric\index{Dielektrikum}}
			\noindent
			Generally, Eq.~\ref{Eq:FluxDensityFieldStrength} holds, where $\vec{P}$ depends on $\vec{E}$, $\vec{r}$, $T$, ... \vspace{10pt}

			\noindent
			Bound charge density and bound surface charge density:
			\begin{equation}
				\begin{aligned}
					\rho_P(\vec{r}) &= -\Nabla \cdot\vec{P}(\vec{r}) \\
					\sigma_P(\vec{r}) &= \vec{n}(\vec{r})\cdot P(\vec{r})
				\end{aligned}
			\end{equation}

			\noindent
			Polarization density in homogeneous, isotropic, linear dielectrics (Electric susceptibility\index{Suszeptibilität} $\chi_e$):
			%In homogenen, isotropen, linearen Dielektrika (mit elektrischer Suszeptibilität $\chi_e$) gilt für die Polarisationsdichte $P$:
			\begin{equation}
				\begin{aligned}
					\vec{P} &= \epsilon_0 \chi_e \vec{E} \\
					\epsilon &= \epsilon_0 \epsilon_r = \epsilon_0(1+\chi_e) \\
					\vec{D} &= \epsilon \vec{E} \\
				\end{aligned}
			\end{equation}


	\subsection{Magnetostatics\index{Magnetostatik}}
		\subsubsection{Basics}
			\noindent
			Biot-Savart law\index{Biot!-Savart Gesetz}:
			\begin{equation}
				\vec{B}(\vec{r}) = \frac{\mu_0}{4\pi}\int_{\mathcal{V}} \vec{j}(\pvec{r}')\times\frac{\vec{r}-\pvec{r}'}{\left|\vec{r}-\pvec{r}'\right|^3}\;\diff^3\pvec{r}'
			\end{equation}

			\noindent
			Static potential equation
			\begin{equation}
				\Nabla^2 \vec{A} = - \mu_0 \vec{j}
			\end{equation}

			\noindent
			Vector potential for static systems (Resulting from the retarded potentials Sec.~\ref{Sec:RetardedPotentials}):
			\begin{equation}
				\vec{A}(\vec{r}) = \frac{\mu_0}{4\pi} \int_{\mathcal{V}} \frac{\vec{j}(\pvec{r}')}{\left|\vec{r}-\pvec{r}'\right|} \;\diff^3\pvec{r}'
			\end{equation}

			\noindent
			Charged particles in a homogeneous magnetic field move on circles with Larmor radius\index{Larmor!Radius} $r_L$ and gyration frequency\index{Zyklotronfrequenz} $\omega$:
			%Geladene Teilchen im homogenen magnetischen Feld bewegen sich auf Kreisen mit dem Larmor-Radius $r_L$ und der Zyklotronfrequenz / Gyrationsfrequenz $\omega$:
			\begin{equation}
				\begin{aligned}
					r_L = \frac{p}{qB} &&\hspace{30pt} % horizontal space
					\omega = \frac{|q| B}{m}
				\end{aligned}
			\end{equation}

		\subsubsection{Multipole Expansion}
			\noindent
			Multipole expansion\index{Multipolentwicklung} (Expansion of the general solution in $\frac{r'}{r}=0$):
			\begin{equation}
				\begin{aligned}
					\vec{A}(\vec{r}) = \int_{\mathcal{V}} \frac{\vec{j}(\pvec{r}')}{\left|\vec{r}-\pvec{r}'\right|} \;\diff^3\pvec{r}'
					=& \frac{\mu_0}{4\pi} \left(\frac{\vec{m}\times\vec{r}}{r^3} + \mathcal{O}\left(\frac{1}{r^3}\right)\right) \\
					=& \frac{\mu_0}{4\pi}\frac{1}{r}\int\vec{j}(\pvec{r}')\;\diff \pvec{r}' \\
					&+ \frac{\mu_0}{4\pi}\frac{1}{r}\int\vec{j}(\pvec{r}')\left(\frac{r'}{r}\right)\cos\theta\;\diff \pvec{r}' \\
					&+ \frac{\mu_0}{4\pi}\frac{1}{r}\int\vec{j}(\pvec{r}')\left(\frac{r'}{r}\right)^2\frac{3\cos^2\theta-1}{2}\;\diff \pvec{r}' \\
					&+ \mathcal{O}\left(\;\frac{1}{r}\int\vec{j}(\pvec{r}')\left(\frac{r'}{r}\right)^3 \;\diff \pvec{r}'\right) \\
				\end{aligned}
			\end{equation}

			\noindent
			Magnetic dipole moment\index{Magnetisches Dipolmoment} $\vec{m}$:
			\begin{equation}
				\begin{aligned}
					\vec{m} = \frac{1}{2}\int \pvec{r}'\times\vec{j}(\pvec{r}')\;\diff^3\pvec{r}'
				\end{aligned}
			\end{equation}

			\noindent
			Magnetic dipole moment\index{Magnetisches Dipolmoment} of a current $I$ encirceling a surface $A$:
			%Magnetisches Dipolmoment einer Stromschleife mit Flächeninhalt $A$ in der ein Strom $I$ fließt:
			\begin{equation}
				\vec{m} = IA\vec{n}
			\end{equation}

			\noindent
			Force, dipole moment (with respect to the center of mass) and potential energy of a magnetic dipole\index{Magnetisches Dipolmoment}:
			%Kraft, Drehmoment und potentielle Energie eines Dipols (Drehmoment in Bezug auf den Massenschwerpunkt $\vec{r}_m$):
			\begin{equation}
				\begin{aligned}
					\vec{F} &= \Nabla\left(\vec{m}\cdot\vec{B}\right) \\
					\vec{D}(\vec{r}_m) &= \vec{m}\times\vec{B} \\
					\mathcal{E}_{pot} &= -\vec{m}\cdot\vec{B} \\
				\end{aligned}
			\end{equation}

		\subsubsection{Magnets}
			\noindent
			Bound current density and bound surface current density:
			%Gebunde Stromdichte und gebundene Oberflächenstromdichte:
			\begin{equation}
				\begin{aligned}
					\vec{j}_m(\vec{r}) &= \Nabla\times\vec{M}(\vec{r}) \\
					\vec{K}_m(\vec{r}) &= \vec{M}(\vec{r})\times \vec{n}(\vec{r}) \\
				\end{aligned}
			\end{equation}

		\subsubsection{Linear Systems}
			\noindent
			Magnetization density $M$ in linear, homogeneous, isotropic, magnetic materials (Magnetic susceptibility\index{Magnetische Suszeptibilität} $\chi_m$, permeability $\mu$):
			%In linearen, homogenen, isotropen, magnetischen Medien (Mit magnetischer Suszeptibilität $\chi_m$, Permabilität $\mu$) gilt für die magnetisierungsdichte $M$:
			\begin{equation}
				\begin{aligned}
					\vec{M} &= \chi_m\vec{H} \\
					\mu &= \mu_0 \mu_r = \mu_0(1+\chi_m) \\
					\vec{B} &= \mu \vec{H} \\
				\end{aligned}
			\end{equation}

	\newpage
	% !TEX root = ../physics.tex
\section{Quantum Mechanics\index{Quantenmechanik}}
	\subsection{Postulates}
		\begin{description} % skript Seite 75
			\item[Postulate 1]\hfill \\
				Every closed quantum system has an associated Hilbert space\index{Hilbert!Raum} $\mathcal{H}$. The state of the system at a given time $t$ is represented by an normed element $\Ket{\psi(t)} \in \mathcal{H}$, such that $\braket{\psi(t)|\psi(t)} = 1$.
			\item[Postulate 2]\hfill \\
				Every measurable physical quantity $\mathcal{A}$ is described by a linear, self-adjoint operator $\hat{A}$ on $\mathcal{H}$.
				$\hat{A}$ has a complete system of eigenvectors, i.e. there exists a partition of unity and a spectral representation\index{Spektraldarstellung} of the operator composed by eigenvectors:
				%Jede messbare physikalische Größe $\mathcal{A}$ wird durch einen linearen selbstadjungierten Operator $\hat{A}$ auf $\mathcal{H}$ beschrieben. $\hat{A}$ hat ein vollständiges System von Eigenvektoren, d.h. es existiert eine Zerlegung der Eins und eine Spektraldarstellung des Operators aus den Eigenvektoren:
				\begin{equation}
					\begin{aligned}
						\hat{1} &= \SumInt_n \SumInt_\nu \Ket{a_n,\nu}\Bra{a_n,\nu} \\
						\hat{A} &= \SumInt_n \SumInt_\nu a_n\Ket{a_n,\nu}\Bra{a_n,\nu}. \\
					\end{aligned}
				\end{equation}
				$\hat{A}$ is called observable.
				%Man nennt $\hat{A}$ eine Observable.
			\item[Postulate 3]\hfill \\
				The possible measurement values of $\mathcal{A}$ are the eigenvalues of $\hat{A}$.
				%Die möglichen Messwerte von $\mathcal{A}$ sind die Eigenwerte von $\hat{A}$.
			\item[Postulate 4]\hfill \\
				When measuring the observable $\hat{A}$ of a system in a state $\Ket{\psi}$, the probability of measuring the eigenvalue
				%Misst man die Obersvable $\mathcal{A}$ an einem System im Zustand $\Ket{\psi}$ so ist die Wahrscheinlichkeit den Eigenwert
				\begin{itemize}
					\item[i)] $a_n$, where $a_n$ is non-degenerate and discrete eigenvalue to the eigenstate $\Ket{a_n}$, is given by
					\begin{equation}
						w_{a_n}(\Ket{\psi}) = \left| \braket{a_n|\psi} \right|^2.
					\end{equation}
					% \item[i)] $a_n$ zu messen, wenn $a_n$ ein nicht-entarteter und diskreter Eigenwert zum Eigenvektor $\Ket{a_n}$ ist, durch
					% \begin{equation}
					% 	w_{a_n}(\Ket{\psi}) = \left| \braket{a_n|\psi} \right|^2
					% \end{equation}
					% gegeben.
					\item[ii)] $a_n$, where $a_n$ is a degenerate and discrete eigenvalue to the eigenstate $\Ket{a_n,\nu}$, is given by
					\begin{equation}
						w_{a_n}(\Ket{\psi}) = \SumInt_\nu \left|\braket{a_n ,\nu|\psi}\right|^2.
					\end{equation}
					% \item[ii)] $a_n$ zu messen, wenn $a_n$ ein entarteter und diskreter Eigenwert zum Eigenvektor $\Ket{a_n,\nu}$ ist, durch
					% \begin{equation}
					% 	w_{a_n}(\Ket{\psi}) = \SumInt_\nu \left|\braket{a_n ,\nu|\psi}\right|^2
					% \end{equation}
					\item[iii)] $a$, where $a$ is a non-degenerate and continuous eigenvalue to the eigenstate $\Ket{a}$, is given by
					\begin{equation}
						\dd w_a(\Ket{\psi}) = \left| \braket{a|\psi} \right|^2 \dd a.
					\end{equation}
					% \item[iii)] $a$ zu messen, wenn $a$ ein nicht-entarteter und kontinuierlicher Eigenwert zum Eigenvektor $\Ket{a}$ ist, durch
					% \begin{equation}
					% 	\dd w_a(\Ket{\psi}) = \left| \braket{a|\psi} \right|^2 \dd a
					% \end{equation}
					\item[iv)] $a$, where $a$ is a degenerate and continuous eigenvalue to the eigenstate $\Ket{a, \nu}$, is given by
					\begin{equation}
						\dd w_a(\Ket{\psi}) = \left( \SumInt_\nu \left| \braket{a,\nu |\psi} \right|^2 \right) \dd a.
					\end{equation}
					% \item[iv)] $a$ zu messen, wenn $a$ ein entarteter und kontinuierlicher Eigenwert zum Eigenvektor $\Ket{a, \nu}$ ist, durch
					% \begin{equation}
					% 	\dd w_a(\Ket{\psi}) = \left( \SumInt_\nu \left| \braket{a,\nu |\psi} \right|^2 \right) \dd a
					% \end{equation}
				\end{itemize}
				Either summation or integration is performed depending on whether $\nu$ is discrete or continuous.
				%Wobei Summiert bzw. integriert wird, je nachdem, ob es sich um ein diskreten oder kontinuierlichen Parameter $\nu$ handelt.
			\item[Postulate 5]\hfill \\
				If a measurement of an observable $\hat{A}$ yields the eigenvalue $a_n$, then the state of the system after the measurement will be the normed projection on the subspace to $a_n$
				%Ergibt die Messung einer Observablen $\hat{A}$ den Eigenwert $a_n$, so befindet sich das System nach der Messung in einem Zustand, der durch die normierte Projektion auf den entsprechenden Unterraum zu $a_n$ gegeben ist
				\begin{equation}
					\begin{aligned}
						\Ket{\psi} &\rightarrow \frac{\hat{P}_n\Ket{\psi}} {\sqrt{\Bra{\psi}\hat{P}_n\Ket{\psi}}} \\
						\hat{P}_n &:= \SumInt_\nu \Ket{a_n,\nu}\Bra{a_n,\nu}. \\
					\end{aligned}
				\end{equation}
			\item[Postulate 6]\hfill \\
				The time evolution of a closed quantum system is governed by the Schrödinger equation\index{Schrödinger!Gleichung}
				%Die zeitliche Entwicklung eines abgeschlossenen Quantensystems ist durch die Schrödingergleichung
				\begin{equation}
					\i\hbar\tder{}{t}\Ket{\psi} = \hat{H}\Ket{\psi},
				\end{equation}
				where, the Hamiltonian $\hat{H}$ corresponds to the observable $\mathcal{H}$ associated with the total energy of the system.
				%gegeben, wobei der Hamiltonoperator $\hat{H}$ die Observable ist, die mit der Gesamtenergie des Systems verknüpft ist.
			\item[Postulate 7]\hfill \\
				The Hilbert space of a composed quantum system is given by the tensor product of the sub-systems
				%Der Hilbertraum des Gesamtsystems ist durch das Tensorprodukt der Hilberträume der Teilsysteme gegeben
				\begin{equation}
					\mathcal{H} = \mathcal{H}_1 \otimes \mathcal{H}_2.
				\end{equation}
			\item[Postulate 8]\hfill \\
				Micro-objects with identical properties are described by either totally symmetric (bosonic) or by totally antisymmetric (fermionic) states. Fermions\index{Fermionen} have half-integer, bosons\index{Bosonen} have integer spin values.
				%Mikroobjekte mit identischen Eigenschaften werden entweder durch totalsymmetrische (Bosonen) oder durch totalantisymmetrische (Fermionen) Zustandsvektoren beschrieben. Fermionen haben halbzahligen, Bosonen ganzzahligen Spin.
	\end{description}
	% 	\begin{description} % skript Seite 75
	% 		\item[Postulat 1]\hfill \\
	% 			Jedem abgeschlossenen Quantensystem ist ein Hilbertraum $\mathcal{H}$ zugeordnet. Der Zustand des Systems zu einer festen Zeit $t$ wird durch ein Element $\Ket{\psi(t)} \in \mathcal{H}$ beschrieben, welches auf Eins normiert ist, d.h. $\braket{\psi(t)|\psi(t)} = 1$.
	% 		\item[Postulat 2]\hfill \\
	% 			Jede messbare physikalische Größe $\mathcal{A}$ wird durch einen linearen selbstadjungierten Operator $\hat{A}$ auf $\mathcal{H}$ beschrieben. $\hat{A}$ hat ein vollständiges System von Eigenvektoren, d.h. es existiert eine Zerlegung der Eins und eine Spektraldarstellung des Operators aus den Eigenvektoren:
	% 			\begin{equation}
	% 				\begin{aligned}
	% 					\hat{1} &= \SumInt_n \SumInt_\nu \Ket{a_n,\nu}\Bra{a_n,\nu} \\
	% 					\hat{A} &= \SumInt_n \SumInt_\nu a_n\Ket{a_n,\nu}\Bra{a_n,\nu}. \\
	% 				\end{aligned}
	% 			\end{equation}
	% 			Man nennt $\hat{A}$ eine Observable.
	% 		\item[Postulat 3]\hfill \\
	% 			Die möglichen Messwerte von $\mathcal{A}$ sind die Eigenwerte von $\hat{A}$.
	% 		\item[Postulat 4]\hfill \\
	% 			Misst man die Obersvable $\mathcal{A}$ an einem System im Zustand $\Ket{\psi}$ so ist die Wahrscheinlichkeit den Eigenwert
	% 			\begin{itemize}
	% 				\item[i)] $a_n$ zu messen, wenn $a_n$ ein nicht-entarteter und diskreter Eigenwert zum Eigenvektor $\Ket{a_n}$ ist, durch
	% 				\begin{equation}
	% 					w_{a_n}(\Ket{\psi}) = \left| \braket{a_n|\psi} \right|^2
	% 				\end{equation}
	% 				gegeben.
	% 				\item[ii)] $a_n$ zu messen, wenn $a_n$ ein entarteter und diskreter Eigenwert zum Eigenvektor $\Ket{a_n,\nu}$ ist, durch
	% 				\begin{equation}
	% 					w_{a_n}(\Ket{\psi}) = \SumInt_\nu \left|\braket{a_n ,\nu|\psi}\right|^2
	% 				\end{equation}
	% 				\item[iii)] $a$ zu messen, wenn $a$ ein nicht-entarteter und kontinuierlicher Eigenwert zum Eigenvektor $\Ket{a}$ ist, durch
	% 				\begin{equation}
	% 					\dd w_a(\Ket{\psi}) = \left| \braket{a|\psi} \right|^2 \dd a
	% 				\end{equation}
	% 				\item[iv)] $a$ zu messen, wenn $a$ ein entarteter und kontinuierlicher Eigenwert zum Eigenvektor $\Ket{a, \nu}$ ist, durch
	% 				\begin{equation}
	% 					\dd w_a(\Ket{\psi}) = \left( \SumInt_\nu \left| \braket{a,\nu |\psi} \right|^2 \right) \dd a
	% 				\end{equation}
	% 			\end{itemize}
	% 			gegeben. Wobei Summiert bzw. integriert wird, je nachdem, ob es sich um ein diskreten oder kontinuierlichen Parameter $\nu$ handelt.
	% 		\item[Postulat 5]\hfill \\
	% 			Ergibt die Messung einer Observablen $\hat{A}$ den Eigenwert $a_n$, so befindet sich das System nach der Messung in einem Zustand, der durch die normierte Projektion auf den entsprechenden Unterraum zu $a_n$ gegeben ist
	% 			\begin{equation}
	% 				\begin{aligned}
	% 					\Ket{\psi} &\rightarrow \frac{\hat{P}_n\Ket{\psi}} {\sqrt{\Bra{\psi}\hat{P}_n\Ket{\psi}}} \\
	% 					\hat{P}_n &:= \SumInt_\nu \Ket{a_n,\nu}\Bra{a_n,\nu} \\
	% 				\end{aligned}
	% 			\end{equation}
	% 		\item[Postulat 6]\hfill \\
	% 			Die zeitliche Entwicklung eines abgeschlossenen Quantensystems ist durch die Schrödingergleichung
	% 			\begin{equation}
	% 				\i\hbar\tder{}{t}\Ket{\psi} = \hat{H}\Ket{\psi}
	% 			\end{equation}
	% 			gegeben, wobei der Hamiltonoperator $\hat{H}$ die Observable ist, die mit der Gesamtenergie des Systems verknüpft ist.
	% 		\item[Postulat 7]\hfill \\
	% 			Der Hilbertraum des Gesamtsystems ist durch das Tensorprodukt der Hilberträume der Teilsysteme gegeben
	% 			\begin{equation}
	% 				\mathcal{H} = \mathcal{H}_1 \otimes \mathcal{H}_2.
	% 			\end{equation}
	% 		\item[Postulat 8]\hfill \\
	% 			Mikroobjekte mit identischen Eigenschaften werden entweder durch totalsymmetrische (Bosonen) oder durch totalantisymmetrische (Fermionen) Zustandsvektoren beschrieben. Fermionen haben halbzahligen, Bosonen ganzzahligen Spin.
	% 	\end{description}

	\subsection{Basics}
		\noindent
		Schrödinger equation\index{Schrödinger!Gleichung}:
		\begin{equation}
			\i\hbar\tder{}{t}\Ket{\psi} = \hat{H}\Ket{\psi}
		\end{equation}

		\noindent
		De Broglie wavelength\index{De Broglie!Wellenlänge}:
		\begin{equation}
			\begin{aligned}
				p &= \frac{h}{\lambda} \\
				\vec{p} &= \hbar \vec{k} \\
			\end{aligned}
		\end{equation}

		\noindent
		Compton wavelength\index{Compton!Wellenlänge}:
		\begin{equation}
			\lambda = \frac{h}{mc}
		\end{equation}

		\noindent
		Energy of a photon:
		\begin{equation}
			E_\gamma = \hbar\omega = h\nu
		\end{equation}

		\noindent
		Classical Schrödinger equation in position space representation:
		%Klassische Schrödingergleichung in Ortsdarstellung:
		\begin{equation}
			\i\hbar\pder{}{t}\psi(\vec{x}) = -\frac{\hbar^2}{2m}\frac{\partial^2}{\partial \vec{x}^2}\psi(\vec{x}) + V(x)\psi(\vec{x})
		\end{equation}

		\noindent
		Heisenberg uncertainty principle\index{Heisenberg!Unschärferelation} ($\Delta_{\Ket{\psi}} (\hat{A}) = \sqrt{\Bra{\psi}\hat{A}^2\Ket{\psi} - \Bra{\psi}\hat{A}\Ket{\psi}^2}$):
		\begin{equation}
			\begin{aligned}
				\Delta_{\Ket{\psi}}(\hat{A}) \Delta_{\Ket{\psi}}(\hat{B}) &\ge
				\frac{1}{2} \left|\Bra{\psi} \comm{\hat{A}}{\hat{B}} \Ket{\psi}\right| \\
				\Delta_{\Ket{\psi}}(\hat{x}) \Delta_{\Ket{\psi}}(\hat{p}) &\ge
				\frac{\hbar}{2}
			\end{aligned}
		\end{equation}

		\noindent
		Energy time uncertainty ($\Delta t$ is the required measurement time for the reduction of the energy uncertainty to $\Delta E$):
		%Energie-Zeit Unschärfe ($\Delta t$ entspricht der benötigten Messzeit für die Reduktion der Energieunsicherheit auf $\Delta E$):
		\begin{equation}
			\Delta E \Delta t \ge \frac{\hbar}{2}
		\end{equation}

		\noindent
		Unitarity:%Unitarität:
		\begin{equation}
			\begin{aligned}
				\Ket{\psi(t)} &= \hat{U}(t) \Ket{\psi(0)} \\
				\hat{1} &= \hat{U}(t)\hat{U}^\dagger(t) = \hat{U}^\dagger(t)\hat{U}(t) \\
			\end{aligned}
		\end{equation}

		\noindent
		Position space and momentum space representations\index{Ortsdarstellung}\index{Impulsdarstellung}:
		\begin{equation}
			\begin{aligned}
				\Braket{\vec{x} | \psi} &= \psi(\vec{x}) &\hspace{20pt}
				\Braket{\vec{x} | \pvec{x}'} &= \delta(\vec{x} - \pvec{x}') \\
				\Braket{\vec{p} | \psi} &= \tilde{\psi}(\vec{p}) &\hspace{20pt}
				\Braket{\vec{p} | \pvec{p}'} &= \delta(\vec{p} - \pvec{p}') \\
			\end{aligned}
		\end{equation}
		\begin{equation}
			\begin{aligned}
				\Braket{\vec{x} | \vec{p}} &= \left( \frac{1}{\sqrt{2\pi\hbar}} \right)^\mathrm{dim} e^{\i \vec{p}\cdot\vec{x} / \hbar} \\
			\end{aligned}
		\end{equation}

		\noindent
		Classical limit:%Klassischer Grenzfall:
		\begin{equation}
			\hbar \rightarrow 0
		\end{equation}

		\subsubsection{Continuity Equation}
			\noindent
			Probability density (Born rule\index{Born!Wahrscheinlichkeitsinterpretation}) and probability current density:
			%Wahrscheinlichkeitsdichte und Wahrscheinlichtkeitsstromdichte:
			\begin{equation}
				\begin{aligned}
					\rho(t,x) :=& \left|\psi(t,x)\right|^2 \\
					j(t,x) :=& \frac{\hbar}{2mi}\left(
						\psi^*(t,x)\pder{}{x}\psi(t,x) - \psi(t,x)\pder{}{x}\psi^*(t,x)
					\right) \\
					=& \frac{\hbar}{m} \mathrm{Im}\left(
						\psi^*(t,x) \pder{}{x}\psi(t,x)
					\right)\\
				\end{aligned}
			\end{equation}

			\noindent
			Continuity equation\index{Kontinuitätsgleichung!Quantenmechanik}:
			\begin{equation}
				\pder{}{t}\rho(t,x) + \pder{}{x}j(t,x) = 0
			\end{equation}

	\subsection{Operators}
		\noindent
		Position operator\index{Ortsoperator}:
		\begin{equation}
			\begin{aligned}
				\hat{\vec{x}} :=& \int_{\mathbb{R}^n} \Ket{\vec{x}} \vec{x} \Bra{\vec{x}\,}\;\dd^n\vec{x} &\hspace{20pt}
				\Bra{\vec{x}\,}\hat{\vec{x}}\Ket{\psi} =& \,\vec{x}\, \psi(\vec{x}) &\hspace{20pt}
				\Bra{\vec{p}\,}\hat{\vec{x}}\Ket{\psi} =& \,\i\hbar\pder{}{\vec{p}}\tilde{\psi}(\vec{x}) \\
			\end{aligned}
		\end{equation}

		\noindent
		Momentum operator\index{Impulsoperator}:
		\begin{equation}
			\begin{aligned}
				\hat{\vec{p}} :=& \int_{\mathbb{R}^n} \Ket{\vec{p}} \vec{p} \Bra{\vec{p}\,}\;\dd^n\vec{p} &\hspace{20pt}
				\Bra{\vec{p}\,}\hat{\vec{p}}\Ket{\psi} =& \,\vec{p}\, \tilde{\psi}(\vec{p}) &\hspace{20pt}
				\Bra{\vec{x}}\hat{\vec{p}}\Ket{\psi} =& -\i\hbar\pder{}{\vec{x}} \psi(\vec{x}) \\
			\end{aligned}
		\end{equation}

		\noindent
		Classical (non-relativistic) Hamiltonian:
		%Klassischer (nicht relativistischer) Hamiltonoperator:
		\begin{equation}
			\hat{H} = \frac{\hat{p}^2}{2m}+\hat{V}(\vec{x}) = -\frac{\hbar^2}{2m}\frac{\partial^2}{\partial \vec{x}^2} + V(\vec{x})
		\end{equation}

		\noindent
		Time evolution operator\index{Zeitentwicklungsoperator} (For $\left[\hat{H}(t),\hat{H}(t')\right] = 0\;\forall t,t'$; the second case follows for $\pder{\hat{H}}{t}=0$):
		%Zeitentwicklungsoperator (Für $\left[\hat{H}(t),\hat{H}(t')\right] = 0\;\forall t,t'$; der zweite Fall folgt aus $\pder{\hat{H}}{t}=0$):
		\begin{equation}
			\begin{aligned}
				\hat{U}(t) &= \exp{\left(-\frac{\i}{\hbar}\int_0^t \hat{H}(t')\;\dd t'\right)} \\
				\hat{U}(t) &= \exp{\left(-\frac{\i}{\hbar}\right)\hat{H} t} \\
				%&= \SumInt_E \SumInt_{\nu} e^{-\iEt/\hbar}\Ket{E,\nu}\Bra{E,\nu}\;\dd E \\
				%&= \int_\mathbb{R} e^{-\i\frac{p^2}{2m}t/\hbar}\Ket{p}\Bra{p}\;\dd p \\
			\end{aligned}
		\end{equation}

		\noindent
		Canonical commutator relations\index{Kanonische Kommutatorrelationen}:
		%Kanonische Kommutator-Relationen
		\begin{equation}
			\begin{aligned}
				\left[ \hat{x}_i, \hat{x}_j \right] &= \left[ \hat{p}_i, \hat{p}_j \right] = 0 \\
				\left[ \hat{x}_i, \hat{p}_j \right] &= \i\hbar\, \delta_{ij}\hat{1} \\
			\end{aligned}
		\end{equation}

		\noindent
		Parity operator\index{Paritätsoperator} and properties (Definition, self-adjointness, unitarity, eigenvalue, eigenstate):
		% und Eigenschaften (Definition, Selbstadjungiertheit, Unitarität, EW, EV):
		\begin{equation}
			\begin{aligned}
				\Bra{\vec{x}} \hat{\Pi} \Ket{\Psi} :=& \Braket{-\vec{x} | \psi} = \psi(-\vec{x}) \\
				\hat{\Pi} =& \hat{\Pi}^\dagger\\
				1 =& \hat{\Pi} \,\hat{\Pi}^\dagger = \hat{\Pi}^\dagger \hat{\Pi} \\
				\psi(\vec{x}) = \psi(-\vec{x}) \Rightarrow&\; \hat{\Pi}\Ket{\psi} = \Ket{\psi} \\
				\psi(\vec{x}) = -\psi(-\vec{x}) \Rightarrow&\; \hat{\Pi}\Ket{\psi} = -\Ket{\psi} \\
			\end{aligned}
		\end{equation}

		\noindent
		Displacement operator\index{Translationsoperator}:
		\begin{equation}
			\begin{aligned}
				\hat{T}_{\vec{y}}\Ket{\vec{x}} :=& \Ket{\vec{x} + \vec{y}} = e^{-\i\hat{\vec{p}}\cdot\vec{y}/\hbar}\Ket{\vec{x}} \\
			\end{aligned}
		\end{equation}

	\subsubsection{Angular Momentum Operators\index{Drehimpulsoperatoren}}
		\noindent
		Defining property (Angular momentum commutator relations):
		%Definierende Eigenschaft (Drehimpuls-Kommutatorrelationen):
		\begin{equation}
			\left[ \hat{j}_i, \hat{j}_j \right] = \i\hbar\, \epsilon_{ijk} \hat{j}_k
		\end{equation}

		\noindent
		Orbital angular momentum operator \index{Bahndrehimpulsoperator}:
		\begin{equation}
			\begin{aligned}
				\hat{\vec{L}} &= \hat{\vec{x}}\times \hat{\vec{p}} = \vec{x}\times\left(-\i\hbar\pder{}{\vec{x}}\right) \\
				\left< \hat{\vec{L}}^2 \right> &= l(l+1)\hbar^2 \\
				\left< \hat{L}_z \right> &= m_l \hbar \\
			\end{aligned}
		\end{equation}

		\noindent
		Relation to the Laplace operator\index{Laplace!Operator}:
		%Zusammenhang zum Laplace Operator:
		\begin{equation}
			\begin{aligned}
				\hat{\vec{p}}^{\,2} &= -\hbar^2 \Nabla^2 = \hat{\vec{p}}_r^{\,2} + \frac{1}{r^2} \hat{\vec{L}}^2 \\
				&= -\hbar^2 \rBr{{\displaystyle \frac{1}{r^{2}} \frac{\partial }{\partial r}\!\left(r^{2}\frac{\partial }{\partial r}\right)\!+\!\frac{1}{r^{2}\!\sin \theta } \frac{\partial }{\partial \theta }\!\left(\sin \theta \frac{\partial }{\partial \theta }\right)\!+\!\frac{1}{r^{2}\!\sin ^{2}\theta }\frac{\partial ^{2}}{\partial \varphi ^{2}}}}\\
				\hat{\vec{p}}_r &= -\i\hbar\frac{1}{r}\pder{}{r}r \\
			\end{aligned}
		\end{equation}

		\noindent
		Spin operator \index{Spinoperator} (For a magnetic field $\vec{B}\parallel\vec{e}_z$):
		\begin{equation}
			\begin{aligned}
				\left<\hat{\pvec{S}}^2\right> &= s(s+1)\hbar^2 \\
				\left<\hat{S}_z\right> &= m_s \hbar \\
			\end{aligned}
		\end{equation}

		\noindent
		Spin representation in $\vec{e}_3$-basis: $\mathcal{B} = \{ \Ket{\vec{e}_3,+}, \Ket{\vec{e}_3,-} \}$:
		\begin{equation}
			\begin{aligned}
				\hat{\vec{\sigma}} \cdot \vec{n} &\doteq
				\left( \begin{matrix}
					\cos\theta & e^{-\i\phi}\sin\theta \\
					e^{\i\phi}\sin\theta & -\cos\theta \\
				\end{matrix} \right) \\
				\exp\left( \i\alpha \; \hat{\vec{\sigma}} \cdot \vec{n} \right) &= \cos\alpha \;\hat{1} + \i\sin\alpha \; \hat{\vec{\sigma}} \cdot \vec{n} \\
			\end{aligned}
		\end{equation}

		\noindent
		Pauli spin matrices\index{Pauli!Spinmatrizen} (spin operator $\hat{\vec{s}} = \frac{\hbar}{2} \hat{\vec{\sigma}}$):
		\begin{equation}
			\begin{aligned}
				\hat{\sigma}_1 &= \phantom{-i}\Ket{\vec{e}_3,+} \Bra{\vec{e}_3,-} + \phantom{i}\Ket{\vec{e}_3,-} \Bra{\vec{e}_3,+} \\
				\hat{\sigma}_2 &= -\i\Ket{\vec{e}_3,+} \Bra{\vec{e}_3,-} + \i\Ket{\vec{e}_3,-} \Bra{\vec{e}_3,+} \\
				\hat{\sigma}_3 &= \phantom{-i}\Ket{\vec{e}_3,+} \Bra{\vec{e}_3,+} - \phantom{i}\Ket{\vec{e}_3,-} \Bra{\vec{e}_3,-} \\
			\end{aligned}
		\end{equation}

		\noindent
		Properties of the Pauli matrices\index{Pauli!Spinmatrizen}:
		%Eigenschaften der Pauli-matrizen:
		\begin{equation}
			\begin{aligned}
				\det\Br{\hat{\sigma}_j} &= -1 \\
				\tr\Br{\hat{\sigma}_j} &= 0 \\
				\Br{\hat{\sigma}_1}^2 = \Br{\hat{\sigma}_2}^2 = \Br{\hat{\sigma}_3}^2 &= -\i\hat{\sigma}_1\hat{\sigma}_2\hat{\sigma}_3 = \hat{1} \\
				\hat{\sigma}_j \hat{\sigma}_k &= \delta_{jk}\hat{1} + \i\varepsilon_{jkl}\hat{\sigma}_l
				%\comm{\hat{\sigma}__\mu,\hat{\sigma}_\nu} &= 2i\varepsilon_{\mu\nu\lambda}\hat{\sigma}_\lambda \\
				%\hat{\sigma}_\mu \hat{\sigma}_\nu \hat{\sigma}_\lambda = i\varepsilon_{\}\hat{\sigma}_\xi
			\end{aligned}
		\end{equation}

		\noindent
		Total angular momentum operator:
		%Gesamtdrehimpulsoperator
		\begin{equation}
			\begin{aligned}
				\hat{\vec{J}} &= \hat{\vec{L}} + \hat{\vec{S}} \\
				\left< \hat{\pvec{J}}^2 \right> &= j(j+1)\hbar^2 \\
				\left< \hat{J}_z \right> &= m_j\hbar \\
			\end{aligned}
		\end{equation}

		\subsubsection{Commutator}
			\noindent
			Definition of the commutator:
			\begin{equation}
				\comm{\hat{A}}{\hat{B}} = \hat{A}\hat{B} - \hat{B}\hat{A}
			\end{equation}

			\noindent
			Definition of the anticommutators:
			\begin{equation}
				\anticom{\hat{A}}{\hat{B}} = \hat{A}\hat{B} + \hat{B}\hat{A}
			\end{equation}

			\noindent
			Calculation rules:
			\begin{equation}
				\begin{aligned}
					\comm{\hat{A}}{\hat{B}\hat{C}}
					&= \hat{B}\comm{\hat{A}}{\hat{C}} + \comm{\hat{A}}{\hat{B}}\hat{C} \\
					\comm{\hat{A}}{f(\hat{B})}
					&= \comm{\hat{A}}{\hat{B}}\tder{f}{\hat{B}}(\hat{B})
				\end{aligned}
			\end{equation}

		\subsubsection{Operators in the Heisenberg Picture\index{Heisenberg!Bild}}
			\noindent
			Time dependent operators:
			\begin{equation}
				\begin{aligned}
					\hat{A}_H(t) &= \hat{U}^\dagger \hat{A} \hat{U} \\
					\langle \hat{A} \rangle_{\Ket{\psi(t)}} &= \Bra{\psi(0)}\hat{A}_H(t)\Ket{\psi(0)}
				\end{aligned}
			\end{equation}

			\noindent
			Time evolution of observables (Heisenberg equation\index{Heisenberg!Gleichung}:
			%Zeitentwicklung von Observablen (Heisenberggleichung):
			\begin{equation}
				i\hbar \tder{}{t} \hat{A}_H(t) = \left(\left[\hat{A}, \hat{H}\right] + i\hbar \pder{}{t} \hat{A}\right)_H (t)
			\end{equation}

			\noindent
			Ehrenfest theorem\index{Ehrenfest!Theorem}:
			\begin{equation}
				\begin{aligned}
					\tder{}{t} \hat{\vec{x}}_H(t) &= \frac{1}{m}\hat{\vec{p}}_H(t) \\
					\tder{}{t} \hat{\vec{p}}_H(t) &= - \pder{}{\vec{x}} V \left(\hat{\vec{x}}_H(t)\right) \\
				\end{aligned}
			\end{equation}

			\noindent
			Ehrenfest equations\index{Ehrenfest!Gleichungen}:
			\begin{equation}
				\begin{aligned}
					\tder{}{t} \langle \hat{\vec{x}} \,\rangle_{\Ket{\psi(t)}} &= \frac{1}{m} \langle \hat{\vec{p}} \,\rangle_{\Ket{\psi(t)}} \\
					\tder{}{t} \langle \hat{\vec{p}} \,\rangle_{\Ket{\psi(t)}} &= - \left\langle \pder{}{\vec{x}} V(\hat{\vec{x}}) \,\right\rangle_{\Ket{\psi(t)}} \\
				\end{aligned}
			\end{equation}

		\subsubsection{Conserved quantities}
			\noindent
			Symmetry transformation of a self-adjoint generator $\hat{G}$ ($\hat{T}$ becomes unitary):
			%Symmetrietransformation mit selbstadjungiertem Generator $\hat{G}$ ($\hat{T}$ ist unitär):
			\begin{equation}
				\hat{T}(\nu) = e^{-\i\hat{G}\nu/\hbar}
			\end{equation}

			\noindent
			Condition of invariance \index{Invarianzbedingung} and conserved quantity (Heisenberg picture\index{Heisenberg!Bild}:
			%und Erhaltunggröße (Heisenberg-Bild):
			\begin{equation}
				\hat{H} =\hat{T}^\dagger(\nu)\hat{H} \hat{T}(\nu) \Leftrightarrow \comm{\hat{H}}{\hat{G}} = 0
			\end{equation}

	\subsection{Bound States}
		\subsubsection{Basics}
			\noindent
			Bound states have negative interaction energy $E<0$ and discrete energy levels. For a time dependent Hamiltonian, the probability density is stationary.
			%Gebundene Zustände haben negative Wechselwirkungsenergie $E<0$ und diskrete Energieeigenwerte. Bei zeitunabhängigen Hamiltonoperatoren ist die Aufenthaltswahrscheinlichkeit zeitlich konstant
			 $\left|\psi(x)\right|^2=\const.$

		\subsubsection{Wave Function of an Electron in an Atom}
			\noindent
			For the hydrogen atom (Bohr radius\index{Bohr!Radius} $\rho$, normalization constant $a_0$):
			%Im Wasserstoffatom (Bohr Radius $\rho$, Normierungskonstante $a_0$):
			\begin{equation}
				\begin{aligned}
					\cBr{-\frac{\hbar^2 \Nabla^2}{2 m_e} + V(r)}\psi(\pvec{r})
					&= \cBr{-\frac{\hbar^2}{2 m_e}\frac{1}{r^2}\pder{}{r}\Br{r^2\pder{}{r}} + \frac{1}{2 m_e r^2} \hat{L}^2 - \frac{e^2}{4\pi\varepsilon_0}\frac{1}{r}}\psi(\pvec{r})
					= E \psi\Br{\pvec{r}}
					\\
					\Braket{\vec{r}|n,l,m_l,m_s} &= \psi_{n,l,m}\left(r,\theta,\phi\right)\chi(m_s)
					= R_{n,l}\left(r\right) Y_l^m\left(\theta,\phi\right)\chi(m_s) \\
					R_{n,l}\left(r\right)
					&= \left(\frac{1}{n\rho}\right)^{\frac{3}{2}}
					\sum_{j=0}^{n-l-1} a_j \left(\frac{r}{n\rho}\right)^{j+l} \exp\left({-\frac{r}{n\rho}}\right) \\
					a_j &= 2\frac{j+l-n}{j(j+2l+1)} a_{j-1} \\
				\end{aligned}
			\end{equation}

			\noindent
			Quantum numbers:
			\begin{itemize}
				\item Principal quantum number $n \in \mathbb{N}$
				\item Orbital angular momentum quantum number $l \in \mathbb{Z};\; \left|l\right| < n$
				\item Magnetic quantum number $m_l \in \mathbb{Z};\; \left|m\right| \le \left|l\right|$
				\item spin quantum number $s = \frac{1}{2}$ (For fermions)
				\item spin projection $m_s = \pm \frac{1}{2}$ (For fermions)
			\end{itemize}

			\noindent
			Energy states of a single electron in an atom ($\mathrm{H}$, $\mathrm{He^{+}}$, $\mathrm{Li^{2+}}$,...) with atomic number $Z$ (Rydberg energy\index{Rydberg!Energie} $\left.R^*=R_y\right|_{m_e\rightarrow \mu}$ and reduced mass $\mu$ of the atomic system):
			%Energiezustände eines einzelnen Elektrons im Atom ($\mathrm{H}$, $\mathrm{He^{+}}$, $\mathrm{Li^{2+}}$,...) mit Ordnungszahl $Z$ (Rydbergenergie $\left.R^*=R_y\right|_{m_e\rightarrow \mu}$ mit der reduzierten Masse $\mu$ des Atom-Systems):
			\begin{equation}
				\mathcal{E}_n = -R^* Z^2 \frac{1}{n^2} = -R^* \frac{\mu}{m_e} Z^2 \frac{1}{n^2} = - \frac{\mu e^4}{8 h^2 \varepsilon_0^2} Z^2 \frac{1}{n^2}
			\end{equation}

			\noindent
			Electron configuration:
			\begin{itemize}
				\item Aufbau principle\index{Aufbauprinzip} / Pauli principle\index{Pauli!Prinzip}: The total wave function of a system of electrons is totally antisymmetric with respect to permutation of two electrons.
				%\item Pauli-Prinzip: Die Gesamtwellenfunktion eines Systems aus mehreren Elektronen ist immer antisymmetrisch	gegen die Vertauschung zweier Elektronen.
				\item Madelung rule\index{Madelung!Energieschema}: For the ground state, orbitals with the lowest value of $n+l$ are filled first.
				%Im Grundzustand werden Orbitale mit dem kleinsten Wert für $n+l$ zuerst aufgefüllt.
				\item Hund's rules\index{Hund!Regel}:
				\begin{itemize}
					\item For a given electron configuration, the term with maximum multiplicity has the lowest energy. The multiplicity is equal to $2S+1$, where $S$ is the total spin angular momentum for all electrons. The multiplicity is also equal to the number of unpaired electrons plus one. Therefore, the term with lowest energy is also the term with maximum $S$, and maximum number of unpaired electrons.
					\item For a given multiplicity, the term with the largest value of the total orbital angular momentum quantum number  $L$, has the lowest energy.
					\item For a given term, in an atom with outermost subshell half-filled or less, the level with the lowest value of the total angular momentum quantum number  $J$, (for the operator $\vec{J} = \vec{L} + \vec{S}$) lies lowest in energy. If the outermost shell is more than half-filled, the level with the highest value of  $J$, is lowest in energy.
				\end{itemize}
				%Im Grundzustand eines Atoms hat der Gesamtspin den größtmöglichen Wert, der mit dem Pauli-Prinzip verträgtlich ist.
			\end{itemize}

			\noindent
			Selection rules\index{Auswahlregeln} for the transition of atoms by emission or absorption of a photon:
			% für Übergänge durch Emission oder Absorption eines Photons:
			\begin{itemize}
				\item $\Delta l = \pm 1$
				\item $\Delta m = 0, \pm 1$
				\item $\Delta S = 0$
				\item $\Delta J = 0, \pm 1$
				\item $\Delta L = \pm 1$
				\item $J=0 \nrightarrow J=0$
			\end{itemize}

			\noindent
			\href{https://www.nist.gov/pml/atomic-spectra-database}{NIST database for energy levels and spectral lines}

		\subsubsection{Quantum Mechanical Harmonic Oscillator}
			\noindent
			Hamiltonian and its construction from creation operator $\hat{a}^\dagger$ and annihilation operator $\hat{a}$\index{Leiteroperatoren}:
			%aus Leiteroperatoren (Aufsteigeoperator $\hat{a}^\dagger$ und Absteigeoperator $\hat{a}$):
			\begin{equation}
				\begin{aligned}
					\hat{H} =& \frac{\hat{p}^2}{2m} + \frac{1}{2}m\omega^2 \hat{x}^2 = \hbar\omega(\hat{a}^\dagger \hat{a} + \frac{1}{2}) \\
					\hat{a} :=& \sqrt{\frac{m\omega}{2\hbar}}\hat{x} + \frac{i}{\sqrt{2m\hbar\omega}}\hat{p} \\
					\hat{a}^\dagger :=& \sqrt{\frac{m\omega}{2\hbar}}\hat{x} - \frac{i}{\sqrt{2m\hbar\omega}}\hat{p} \\
				\end{aligned}
			\end{equation}

			\noindent
			Representation of position and momentum operators by the creation and annihilation operators:
			%Darstellung von Orts- und Impulsoperator aus den Leiteroperatoren:
			\begin{equation}
				\begin{aligned}
					\hat{x} &= \sqrt{\frac{\hbar}{2m\omega}}\left(\hat{a}^\dagger + \hat{a} \right) \\
					\hat{p} &= i\sqrt{\frac{m\hbar\omega}{2}}\left(\hat{a}^\dagger - \hat{a} \right) \\
				\end{aligned}
			\end{equation}

			\noindent
			Properties of the creation and annihilation operators:
			%Eigenschaften der Leiteroperatoren:
			\begin{equation}
				\begin{aligned}
					\hat{a}\Ket{n} &= \sqrt{n}\Ket{n-1} \\
					\hat{a}^\dagger\Ket{n} &= \sqrt{n+1}\Ket{n+1} \\
					\Bra{m}\hat{a}\Ket{n} &= \sqrt{n} \,\delta_{m,n-1} \\
					\Bra{m}\hat{a}^\dagger\Ket{n} &= \sqrt{n+1} \,\delta_{m,n+1} \\
				\end{aligned}
			\end{equation}

			\noindent
			Eigenstates and energy eigenvalues (With the substitution $q:=\sqrt{\frac{m\omega}{\hbar}}x$ and the Hermite polynomials\index{Hermite!Polynome} $H_n(q)$ (Eq.~\ref{Eq:HermitePolynomials})):
			%Eigenzustände und Energiewerte (Mit der Substitution $q:=\sqrt{\frac{m\omega}{\hbar}}x$ und den Hermitpolynomen $H_n(q)$ (Gl. \ref{eq:Hermitpolynome})):
			\begin{equation}
				\begin{aligned}
					\Ket{n} &= \frac{1}{\sqrt{n!}}\left(\hat{a}^\dagger\right)^n\Ket{0} \\
					\Braket{x|n} &= \sqrt[4]{\frac{m\omega}{2\hbar}} \frac{1}{\sqrt{n! 2^n}} e^{-\frac{q^2}{2}} H_n(q) \\
				\end{aligned}
			\end{equation}

	\subsection{States in the Atom}
		\subsubsection{Zeeman Effect\index{Zeeman!Effekt}}
			\noindent
			Ordinary Zeeman effect: $\vec{s} = 0$ \\
			Anomalous Zeeman effect: $\vec{s}\ne 0$ \\
			\noindent
			Magnetic dipole moment of the electron orbit and spin respectively and energy shift in case of $\vec{j}=\vec{l}$ or $\vec{j}=\vec{s}$ respectively (Sec.~\ref{Sec:ParticleConstants}):
			%Magnetisches Dipolmoment des Elektronen Orbits bzw. Spins:
			\begin{equation}
				\begin{aligned}
					\vec{\mu}_l &= -\mu_B g_l \vec{l} &\hspace{20pt} V_l &= m_l g_l \mu_B B\\
					\vec{\mu}_s &= -\mu_B g_s \vec{s} &\hspace{20pt} V_s &= m_s g_s \mu_B B\\
				\end{aligned}
			\end{equation}

			\noindent
			Interaction Energy (see Eq.~\ref{Eq:MagneticDipoleInteraction}:
			\begin{equation}
				\begin{aligned}
					V_l &= -\vec{\mu} \cdot \vec{B} \\
				\end{aligned}
			\end{equation}

			\noindent
			Landé factor\index{Landé!Faktor} / g-factor (For $g_s\approx 2$ approximation):
			\begin{equation}
				\begin{aligned}
					\vBr{\Br{\vec{\mu}_j}_j} &= \mu_B\frac{3j(j+1)+s(s+1)-l(l+1)}{2\sqrt{j(j+1)}} = \mu_B g_j \frac{\vBr{\vec{j}}}{\hbar}\\
					g_j &= 1+\frac{j(j+1) + s(s+1) - l(l+1)}{2j(j+1)} \\
				\end{aligned}
			\end{equation}

			\noindent
			Paschen-Back effect\index{Paschen!-Back Effekt}: For strong magnetic fields, spin and orbital angular momentum decouple and precess independently around the magnetic field. In this case:
			\begin{equation}
				V_{m_s,m_l} = -\Br{\vec{\mu}_s + \vec{\mu}_l}\cdot\vec{B}
				=\Br{g_s m_s + g_l m_l} \mu_B B
			\end{equation}

	\subsection{Perturbation theory\index{Störungstheorie}}
		\noindent
		Fermi's golden rule\index{Fermi!Goldene Regel} (First order approximation for the transition rate $P_{fi}$ from the initial state $\Ket{i}$ to the final state $\Ket{f}$ with a periodic perturbation of the form $\hat{H}' e^{\pm\i\omega}$):
		%für die Übergangsrate $P_{fi}$ von $\Ket{i}$ nach $\Ket{f}$ periodische Störungen der Form $\hat{H}' e^{\pm\i\omega}$ in erster Ordnung:
		\begin{equation}
			P_{fi} = \lim_{t\rightarrow\infty} \frac{W_{fi}(t)}{t} = \frac{2\pi}{\hbar} \vBr{\Bra{f^0}\hat{H}'\Ket{i^0}}^2 \delta\Br{E_f^0 - E_i^0 \pm \hbar\omega}
		\end{equation}


	\subsection{Scattering\index{Streuung}}
		\noindent
		Lippmann-Schwinger equation\index{Lippmann!-Schwinger Gleichung}
		\begin{equation}
			\Ket{\vec{p},\sigma,\mu,\pm} = \Ket{\vec{p}, \sigma, \mu} + \lim_{\eta\rightarrow 0} G_0(\epsilon_{\vec{p},\mu}\pm i\eta) V \Ket{\vec{p},\sigma,\mu,\pm}
		\end{equation}

		\noindent
		Rutherford cross section\index{Rutherford!Streuquerschnitt} (Form factor\index{Formfaktor} $F(\pvec{q}) = \int_{\mathbb{R}^3} \rho(\pvec{r}) e^{-i \vec{q}\cdot\vec{r}/\hbar}\, \dd^3 \vec{r} $):
		\begin{equation}
			\tder{\sigma}{\Omega} = \frac{1}{4} \Br{\frac{e}{4\pi\epsilon_0 m v^2}}^2
			\frac{1}{\sin^4\frac{\theta}{2}} \vbr{F}^2(\pvec{q})
		\end{equation}

	\newpage
	% !TEX root = ../physics.tex
\section{Quantum Field Theory\index{QFT}\index{Quantenfeldtheorie}} % QFt
	\subsection{Postulates}
		\subsubsection{Symmetries?}
			%Locality
			%CPT
			%...
			%TODO
	\subsection{Definitions}
		Feynman slash notation\index{Feynman!Slash Notation}
		\begin{equation}
			\slashed{A} := \gamma_\mu A^\mu
		\end{equation}

		\noindent
		Pauli matrices\index{Pauli!Matrizen}:
		\begin{equation}
			\begin{aligned}
				\sigma^0 = \left(\begin{matrix}
					1 & 0 \\
					0 & 1 \\
				\end{matrix}\right) &&\hspace{30pt}
				\sigma^1 = \left(\begin{matrix}
					0 & 1 \\
					1 & 0 \\
				\end{matrix}\right) &&\hspace{30pt}
				\sigma^2 = \left(\begin{matrix}
					0 & -i \\
						i & 0 \\
				\end{matrix}\right) &&\hspace{30pt}
				\sigma^3 = \left(\begin{matrix}
					1 & 0 \\
					0 & -1 \\
				\end{matrix}\right)
			\end{aligned}
		\end{equation}

		\subsubsection{Gamma Matrices}
			\noindent
			Defining properties:
			\begin{equation}
				\anticom{\gamma^\mu}{\gamma^\nu} = \gamma^\mu \gamma^\nu + \gamma^\nu \gamma^\nu = 2g^{\mu\nu}
			\end{equation}

			\noindent
			Further properties:
			\begin{equation}
				\begin{aligned}
					\Br{\gamma^0}^\dagger &= +\gamma^0 \\
					\Br{\gamma^j}^\dagger &= -\gamma^j \\
					\gamma^0 \gamma^0 &= 1_4 \\ % TODO: replace hat with mathbb{1}
					\gamma^i \gamma^j &= - \delta^{ij} 1_4 \\
					\Br{\gamma^\mu}^\dagger &= \gamma^0 \gamma^\mu \gamma^0 \\
				\end{aligned}
			\end{equation}

			\noindent
			Dirac $\gamma$-matrices\index{Dirac!Gamma Matrizen}:
			\begin{equation}
				\begin{aligned}
					\gamma^0
					= \left( \begin{matrix}
					\sigma^0 & \M0 \\
					0 &  -\sigma^0
					\end{matrix} \right)
					= \left( \begin{matrix}
					\M 1 & \M 0 & \M 0 & \M 0 \\
					\M 0 & \M 1 & \M 0 & \M 0 \\
					\M 0 & \M 0 &   -1 & \M 0 \\
					\M 0 & \M 0 & \M 0 &   -1 \\
					\end{matrix} \right)
					&\hspace{20pt}
					\gamma^1
					= \left( \begin{matrix}
					\M0 & \sigma^1 \\
					-\sigma^1 &  0
					\end{matrix} \right)
					= \left( \begin{matrix}
					\M 0 & \M 0 & \M 0 & \M 1 \\
					\M 0 & \M 0 & \M 1 & \M 0 \\
					\M 0 &   -1 & \M 0 & \M 0 \\
					-1 & \M 0 & \M 0 & \M 0 \\
					\end{matrix} \right) \\[8pt]
					\gamma^2
					= \left( \begin{matrix}
					\M0 & \sigma^2 \\
					-\sigma^2 &  0
					\end{matrix} \right)
					= \left( \begin{matrix}
					\M 0 & \M 0 & \M 0 &  -\i \\
					\M 0 & \M 0 & \M\i & \M 0 \\
					\M 0 & \M\i & \M 0 & \M 0 \\
					-\i & \M 0 & \M 0 & \M 0 \\
					\end{matrix} \right)
					&\hspace{20pt}
					\gamma^3
					= \left( \begin{matrix}
					\M 0 & \sigma^3 \\
					-\sigma^3 &  0
					\end{matrix} \right)
					= \left( \begin{matrix}
					\M 0 & \M 0 & \M 1 & \M 0 \\
					\M 0 & \M 0 & \M 0 &   -1 \\
					-1 & \M 0 & \M 0 & \M 0 \\
					\M 0 & \M 1 & \M 0 & \M 0 \\
					\end{matrix} \right) \\
				\end{aligned}
			\end{equation}

			\noindent
			Transformation ($U$ unitary: $U U^\dagger=1_4$):
			\begin{equation}
				\tilde{\gamma} = U \gamma^\mu U^\dagger
			\end{equation}

			\noindent
			$\gamma^5$-Matrix:
			\begin{equation}
				\gamma^5 = \gamma_5 = \gamma_5 = i\gamma^0 \gamma^1 \gamma^2 \gamma^3 = -\frac{1}{4!}\varepsilon_{\mu\nu\rho\sigma} \gamma^\mu \gamma^\nu \gamma^\rho \gamma^\sigma
			\end{equation}

			\noindent
			Properties of the $\gamma^5$-matrix:
			\begin{equation}
				\begin{aligned}
					\gamma_5^\dagger &= \gamma_5 \\
					\gamma_5 \gamma_5 &= 1_4 \\
					\anticom{\gamma^\mu}{\gamma^5} &= 0 \\
				\end{aligned}
			\end{equation}

			\noindent
			$\sigma^{\mu\nu}$ matrices:
			\begin{equation}
				\sigma^{\mu\nu} = \frac{i}{2}\comm{\gamma^\mu}{\gamma^\nu} = -\sigma^{\nu\mu}
			\end{equation}


	\subsection{Feynman Path Integral\index{Feynman!Pfadintegral}}
		%TODO

	\subsection{Bosons}
		\subsubsection{Free Klein-Gordon Equation\index{Klein!-Gordon Gleichung}}
			\noindent
			Klein-Gordon Equation (spin $s=0$)
			\begin{equation}
				\left(\partial^\mu\partial_\mu+m^2\right) \phi(x) = 0
			\end{equation}

			\noindent
			Solution (plane waves):
			\begin{equation}
				{\phi}^{ ( \pm ) }_{\vec{p}} (x) = e^{\mp p_\mu x^\mu}
			\end{equation}

	\subsection{Fermions}
		\subsubsection{Free Dirac Equation\index{Dirac!Gleichung}}
			\noindent
			Dirac Equation\index{Dirac!Equation} (Spin $s=1/2$):
			\begin{equation}
				\Br{ i\slashed{\partial} - m } \psi\Br{x} = 0
			\end{equation}

	\subsection{Interaction}
		\subsubsection{Mandelstam Variables\index{Mandelstam!Variablen}}
			\noindent
			Lorentz scalars\index{Lorentz!Skalare} in scattering processes if the form: $P_1 + P_2 \rightarrow P_3 + P_4$:
			\begin{itemize}\itemsep -0pt	% reduce space between items
				\item $s:=(p_1+p_2)^2=(p_3+p_4)^2$ \hfill{(Square of the invariant mass\index{Schwerpunktsenergie})}
				\item $t:=(p_1-p_3)^2=(p_2-p_4)^2$ \hfill{(Square of the four-momentum transfer)}
				\item $u:=(p_1-p_4)^2=(p_2-p_3)^2$ \hfill{(Four-momentum transfer of the rebound particle)}
			\end{itemize}

			\noindent
			Connection:
			\begin{equation}
				s+t+u = \sum_i m_i^2
			\end{equation}

	\newpage
	% !TEX root = ../physics.tex
\section{Statistical Mechanics}
	\subsection{General}
		\noindent
		Equipartition theorem\index{Äquipartitionstheorem} $(z_1,...,z_{2f}) = (q_1, p_1, ..., q_f, p_f)$:
		\begin{equation}
			\overline{ z_i \pdv{H}{z_j}} = k_B T \delta_{ij}
		\end{equation}

		\subsubsection{Definition of Entropy}
			\noindent
			Boltzmann Entropy\index{Boltzmann!Entropie} (number of microstates\index{Mikrokanonische Zustandssumme} $\Gamma$ and state vector $\vec{Z}=(U,V,N,...)$)
			\begin{equation}
				S_{B}(\vec{Z}) = k_B \ln\Gamma(\vec{Z})
			\end{equation}

			\noindent
			Gibbs Entropy\index{Gibbs!Entropie}:
			\begin{equation}
				S_{G} = -k_B \sum_i p_i \ln p_i
			\end{equation}

			\noindent
			Shannon Entropy\index{Shannon!Entropie}:
			\begin{equation}
				S_{Sh} = -\sum_i p_i\log_2{p_i} \ge 0
			\end{equation}

			\noindent
			Von Neumann Entropy\index{Von Neumann!Entropie}:
			\begin{equation}
				S_{vN} = -\mathrm{tr}(\hat{\rho}\log_2\hat{\rho})
			\end{equation}

			\noindent
			Equivalence of these entropies:
			\begin{equation}
				S = -k_B\,\mathrm{tr}(\hat{\rho} \ln \hat{\rho}) = S_B = S_G = k_B\ln(2) S_{Sh} = k_B\ln(2) S_{vN}
			\end{equation}

		\subsubsection{Density Matrix\index{Dichtematrix}}
			\noindent
			Definition of the density matrix / density operator (Where $p_i$ are classical, i.e. Laplacian probability / Bayesian probability interpretations\index{Bayes!Wahrscheinlichkeitsbegriff}\index{Laplace!Wahrscheinlichkeitsbegriff} for a state $\Ket{\psi_i}$):
			\begin{equation}
				\hat{\rho} = \sum_i p_i \Ket{\psi_i}\Bra{\psi_i}
			\end{equation}

			\noindent
			Axioms of the density operator:
			\begin{itemize}\itemsep -0pt	% reduce space between items
				\item $\hat{\rho} = \hat{\rho}^\dagger$ \hfill{(self-adjoint)}
				\item $\hat{\rho} \ge 0 $ \hfill{(positive semidefinite)}
				\item $\mathrm{tr} \hat{\rho} = 1$ \hfill{(normed)}
			\end{itemize}

			\noindent
			Expectation value of an observable $\mathcal{A}$:
			\begin{equation}
				\overline{A} = \sum_i p_i \Bra{\psi_i}\hat{A}\Ket{\psi_i} = \mathrm{tr} (\hat{\rho}\hat{A})
			\end{equation}

			\noindent
			Von Neumann equation\index{Von Neumann!Gleichung}:
			\begin{equation}
				\dv{t}\hat{\rho}(t) = -\frac{\i}{\hbar} \comm{\hat{H}}{\hat{\rho}(t)}
			\end{equation}

			\noindent
			Liouville equation\index{Liouville!Gleichung} (Analogous to the Von Neumann equation for classical systems):
			\begin{equation}
				\dv{\rho}{t} = \pdv{\rho}{t}+\qty{\rho,H}
			\end{equation}

	\subsection{Thermodynamic Ensembles}
		\subsubsection{Microcanonical Ensemble\index{Mikrokanonisches Ensemble}}
			\noindent
			Axioms for the microcanonical partition function $\Gamma$:
			\begin{itemize}\itemsep -0pt	% reduce space between items
				\item $\pdv{t}\Gamma(\vec{Z}) = 0$ \hfill{(stationary)}
				\item $\Gamma(\vec{Z}_1,\vec{Z}_2) = \Gamma(\vec{Z}_1)\Gamma(\vec{Z}_2) $ \hfill{(multiplicative)}
				\item $\ln\Gamma(\vec{Z}) \propto N$ \hfill{(extensive)}
			\end{itemize}
			Where the third condition is needed for thermodynamics but not statistical mechanics in general.

			\noindent
			Calculating $\Gamma$ ($\Delta$ is irrelevant in the thermodynamic limit. A stationary $\Gamma(\vec{Z})$ is guaranteed by the unitary time evolution for quantum systems and the Liouville theorem for classical systems respectively):
			\begin{equation}
				\begin{aligned}
					\Gamma(\vec{Z}) = \langle \delta_\Delta (U-H) \rangle
						&= \begin{cases}
								\SumInt_n \delta_\Delta(U-E_n) & \text{quantum mechanical} \\
								\int \dd \Gamma_N \delta_\Delta(U-E_n) & \text{classical}
							\end{cases} \\
					\delta_\Delta(x) &= \Theta(x) - \Theta(x-\Delta)\\
					\dd \Gamma_N &= \frac{\dd^{3N}q \dd^{3N}p }{h^{3N} N!} \\
					\hat{\rho} &= \frac{1}{\Gamma} \SumInt \Ket{\psi}\Bra{\psi}\,\dd\psi
				\end{aligned}
			\end{equation}
			
			\subsubsection{Canonical Ensemble}
			A canonical ensemble is the statistical ensemble that represents the possible states of a mechanical system in thermal equilibrium with a heat bath at a fixed temperature.

			Density matrix in the canonical ensemble:
			\begin{equation}
				\hat{\rho} = \frac{1}{Z_N} e^{-\beta \hat{H}} = e^{-\beta (F-\hat{H})}
			\end{equation}

			\noindent
			Canonical partition function ($\beta = \frac{1}{k_B T}$):
			\begin{equation}
				Z_N=\langle e^{-\beta H} \rangle
					= \begin{cases}
							\SumInt_n e^{-\beta E_n} & \text{quantum mechanical} \\
							\int \dd \Gamma_N e^{-\beta H} & \text{classical}
						\end{cases} \\
			\end{equation}

			\noindent
			Connection to the Helmholtz free energy\index{Helmholtz!Freie Energie} $F$:
			\begin{equation}
				F(T, V, N) = -k_B T \ln{Z_N(T)}
			\end{equation}

			\noindent
			Internal energy in the canonical ensemble:
			\begin{equation}
				U = \overline{H} = \frac{1}{Z_N}\langle He^{-\beta H}\rangle = -\frac{1}{Z_N}\pdv{Z_N}{\beta} = -\pdv{\ln Z_N}{\beta}
			\end{equation}

		\subsubsection{Grand Canonical Ensemble}
			The grand canonical ensemble is the statistical ensemble that represents the possible states of a mechanical system in thermal and chemical equilibrium with a heat bath and particle reservoir.
			
			Density matrix for the grand canonical ensemble
			\begin{equation}
				\hat{\rho} = \frac{1}{\mathcal{Z}} e^{-\beta (\hat{H} - \mu \hat{N})}
				= e^{-\beta (\Omega + \hat{H} - \mu \hat{N})}
			\end{equation}

			Grand canonical partition function (The trace has to be calculated in Fock space\index{Fock!Raum} for the quantum mechanical case):
			\begin{equation}
				\mathcal{Z} = \sum_N z^N Z_N(\beta) = \langle e^{-\beta(H-\mu N)} \rangle
			\end{equation}

			\noindent
			Connection to the grand potential $\Omega$:
			\begin{equation}
				\Omega(\beta,z) = -k_B T \ln \mathcal{Z}(\beta,z)
			\end{equation}


		\subsubsection{Gibbs Variation Principle\index{Gibbs!Variationsprinzip}}
			\noindent
			In the thermodynamic equilibria, given the side conditions $g_k(p_i) = 0$, the probabilities $p_i$ are distributed such that the entropy is maximal.
			\begin{equation}
				0 = \dd \big( S - \sum_k \lambda_k g_k(p_i) \big)
			\end{equation}

			\noindent
			This leads to a uniform distribution for the microcanonical ensemble (side condition $\sum_i p_i = 1$):
			\begin{equation}
				\begin{aligned}
					0 &= \dd \big( S - \lambda \sum_i p_i \big) \\
					p_i &= e^{-1-\lambda} = \const \\
					\hat{\rho} &= \frac{\hat{1}}{\Gamma} \\ 
				\end{aligned}
			\end{equation}

			\noindent
			This leads to the Boltzmann distribution\index{Boltzmann!Verteilung} for the canonical ensemble (side condition $\sum_i p_i = 1$; $\sum_i p_i H_i = \overline{H} = U$):
			\begin{equation}
				\begin{aligned}
					0 &= \dd \big( S - \lambda \sum_i p_i - \beta \sum_i p_i H_i \big) \\
					p_i &= e^{-1-\lambda-\beta H_i} \propto e^{-\beta H_i} \\
					\hat{\rho} &= \frac{e^{-\beta \hat{H}}}{Z_N} \\
				\end{aligned}
			\end{equation}

			\noindent
			Grand canonical ensemble (side condition $\sum_i p_i = 1$; $\sum_i p_i H_i = U$; $\sum_i p_i N_i = \overline{N} = N$, For the particle count operator $\hat{N}$ the trace must be calculated in Fock space\index{Fock!Raum}):
			\begin{equation}
				\begin{aligned}
					0 &= \dd \big( S - \lambda \sum_i p_i - \beta \sum_i p_i H_i - \beta\mu \sum_i p_i N_i \big) \\
					p_i &= e^{-1-\lambda-\beta H_i-\beta\mu N} \propto e^{-\beta H_i - \beta\mu N} \\
					\hat{\rho} &= \frac{e^{-\beta (\hat{H} - \mu \hat{N})}}{\mathcal{Z}} \\
				\end{aligned}
			\end{equation}

			\noindent
			Equivalence of the ensembles (Laplace transformation\index{Laplace!Transformation} between ensembles is equivalent to a Legendre transformation\index{Legendre!Transformation} of the corresponding potentials):
			\begin{equation}
				\int \frac{\Gamma(U)}{\Delta} e^{-\beta U} \dd U = Z_N(\beta)
			\end{equation}



	\newpage
	% !TEX root = ../physics.tex
\section{Mathematical Formulary}
	\subsection{Analysis}
		\subsubsection{Exponential Function}
			\noindent
			Definition:
			\begin{equation}
				e^x=\exp{(x)}=\lim_{n\rightarrow \infty}\left(1+\frac{x}{n}\right)^n
			\end{equation}

		\subsubsection{Trigonometric Functions}
			\noindent
			Identities:
			\begin{equation}
				\begin{split}
					e^{\i x}&=\cos x+i\sin x \\
					1&=\sin^{2}x+\cos^{2}x \\
					\cos{x}&=\frac{e^{\i x}+e^{-\i x}}{2} \\
					\sin{ x}&=\frac{e^{\i x}-e^{-\i x}}{2i} \\
				\end{split}
			\end{equation}

			\noindent

			Addition formulae\index{Additionstheoreme}:
			\begin{equation}
				\begin{split}
					\sin\left( x\pm y\right)&=\sin\left( x\right)\cos\left( y\right)\pm\cos\left( x\right)\sin\left( y\right) \\
					\cos\left( x\pm y\right)&=\cos\left( x\right)\cos\left( y\right)\mp\sin\left( x\right)\sin\left( y\right) \\
					2\cos( x)\cos( y)&=\cos( x+ y)+\cos( x- y) \\
					2\sin( x)\cos( y)&=\sin( x+ y)+\sin( x- y) \\
					2\sin( x)\sin( y)&=\cos( x- y)-\cos( x+ y) \\
				\end{split}
			\end{equation}

			\noindent
			Composition of trigonometric functions:
			\begin{center}
				\begin{tabular}{| c || c | c | c |}
					\hline\xrowht{10pt}
					$\mathrm{trig}(\mathrm{arctrig}(x))$ & $\sin$ & $\cos$ & $\tan$ \\
					\hline
					\hline\xrowht{24pt}
					$\arcsin(x)$ & $x$ & $\sqrt{1-x^2}$ & $\dfrac{x}{\sqrt{1-x^2}}$ \\
					\hline\xrowht{24pt}
					$\arccos(x)$ & $\sqrt{1-x^2}$ & $x$ & $\dfrac{\sqrt{1-x^2}}{x}$ \\
					\hline\xrowht{24pt}
					$\arctan(x)$ & $\dfrac{x}{\sqrt{x^2+1}}$ & $\dfrac{1}{\sqrt{x^2+1}}$ & $x$ \\
					\hline
				\end{tabular}
			\end{center}

			\noindent
			Standard values for trigonometric functions:
			\begin{center}
				\begin{tabular}{| c c || l l l |}
					\hline
					Radian & Degree & Sine & Cosine & Tangens \\
					\hline
					\hline\xrowht{12pt}
					$0$ & $0^\circ$ & $\sin\left(0\right)=0$ & $\cos\left(0\right)=1$ & $\tan\left(0\right)=0$ \\
					\hline\xrowht{12pt}
					$\frac{\pi}{6}$ & $30^\circ$ & $\sin\left(\frac{\pi}{6}\right)=\frac{1}{2}$ & $\cos\left(\frac{\pi}{6}\right)=\frac{\sqrt{3}}{2}$ & $\tan\left(\frac{\pi}{6}\right)=\frac{1}{\sqrt{3}}$ \\
					\hline\xrowht{12pt}
					$\frac{\pi}{4}$ & $45^\circ$ & $\sin\left(\frac{\pi}{4}\right)=\frac{1}{\sqrt{2}}$ & $\cos\left(\frac{\pi}{4}\right)=\frac{1}{\sqrt{2}}$ & $\tan\left(\frac{\pi}{4}\right)=1$ \\
					\hline\xrowht{12pt}
					$\frac{\pi}{3}$ & $60^\circ$ & $\sin\left(\frac{\pi}{3}\right)=\frac{\sqrt{3}}{2}$ & $\cos\left(\frac{\pi}{3}\right)=\frac{1}{2}$ & $\tan\left(\frac{\pi}{3}\right)=\sqrt{3}$ \\
					\hline\xrowht{12pt}
					$\frac{\pi}{2}$ & $90^\circ$ & $\sin\left(\frac{\pi}{2}\right)=1$ & $\cos\left(\frac{\pi}{2}\right)=0$ & $\tan\left(\frac{\pi}{2}\right)=\pm\infty$ \\
					\hline
				\end{tabular}
			\end{center}

		\subsubsection{Hyperbolic Functions}
			\noindent
			Identities
			\begin{equation}
				\begin{split}
					e^{ x}&=\cosh x+\sinh x \\
					1&=\cosh^2{x}-\sinh^2{x} \\
					\cosh{ x}&=\frac{e^{ x}+e^{- x}}{2} \\
					\sinh{ x}&=\frac{e^{ x}-e^{- x}}{2} \\
					\sinh( x) &= -i \sin(i x)\\
					\cosh( x) &= \cos(i x)
				\end{split}
			\end{equation}

			\noindent
			Addition formulae\index{Additionstheoreme}:
			\begin{equation}
				\begin{split}
					\sinh\left( x\pm y\right)&=\sinh\left( x\right)\cosh\left( y\right)\pm\cosh\left( x\right)\sinh\left( y\right) \\
					\cosh\left( x\pm y\right)&=\cosh\left( x\right)\cosh\left( y\right)\pm\sinh\left( x\right)\sinh\left( y\right) \\
					\sinh{2x}&=2\sinh{x}\cosh{x} \\
					 \cosh{2x}&=2\cosh^2{x}+\sinh^2{x}
				\end{split}
			\end{equation}

			\noindent
			Composition of hyperbolic functions:
			\begin{center}
				\begin{tabular}{| c || c | c | c |}
					\hline\xrowht{10pt}
					$\mathrm{trigh}(\mathrm{artrigh}(x))$ & $\sinh$ & $\cosh$ & $\tanh$ \\
					\hline
					\hline\xrowht{24pt}
					$\mathrm{arsinh}(x)$ & $x$ & $\sqrt{x^2+1}$ & $\dfrac{x}{\sqrt{x^2+1}}$ \\
					\hline\xrowht{24pt}
					$\mathrm{arcosh}(x)$ & $\sqrt{\dfrac{x-1}{x+1}}(x+1)$ & $x$ & $\sqrt{\dfrac{x-1}{x+1}}\dfrac{(x+1)}{x}$ \\
					\hline\xrowht{24pt}
					$\mathrm{artanh}(x)$ & $\dfrac{x}{\sqrt{1-x^2}}$ & $\dfrac{1}{\sqrt{1-x^2}}$ & $x$ \\
					\hline
				\end{tabular}
			\end{center}

			\noindent
			Inverse hyperbolic functions:
			\begin{equation}
				\begin{aligned}
					\mathrm{arsinh}(x) &= \ln\left(x+\sqrt{x^2+1}\right) \\
					\mathrm{arcosh}(x) &= \ln\left(x+\sqrt{x^2-1}\right) \\
					\mathrm{artanh}(x) &= \frac{1}{2}\ln\left(\frac{1+x}{1-x}\right) \\
				\end{aligned}
			\end{equation}

		\subsubsection{Differentiation}
			\noindent
			Chain rule:
			\begin{equation}
				\pder{\vec{f}(\vec{g}(\vec{x}))}{x^k} = \sum_j \pder{ \vec{f}(\vec{g}(\vec{x}))}{g^j} \pder{g^j(\vec{x})}{x^k}
			\end{equation}

			\noindent
			Schwarz's theorem\index{Schwartz!Satz} ($f(x,y)\in C^2$, i.e. $f$ is two times differentiable):
			\begin{equation}
				\frac{\partial^2 }{\partial x \partial y} f(x,y) = \frac{\partial^2 }{\partial y \partial x} f(x,y)
			\end{equation}

			\noindent
			For connected quantities $x(y,z), y(x,z), z(x,y)$ and an inevitable function $w(y,z)$:
			\begin{equation}
				\begin{aligned}
					\left( \pder{x}{y} \right)_z &= \frac{1}{\left( \pder{y}{x} \right)_z} \\
					-1 &= \left( \pder{x}{y} \right)_z \left( \pder{z}{x} \right)_y \left( \pder{y}{z} \right)_x \\
					\left( \pder{x}{w} \right)_z &= \left( \pder{x}{y} \right)_z \left( \pder{y}{w} \right)_z \\
					\left( \pder{x}{y} \right)_z &= \left( \pder{x}{y} \right)_w + \left( \pder{x}{w} \right)_y \left( \pder{w}{y} \right)_z \\
				\end{aligned}
			\end{equation}

			\noindent
			Derivatives of trigonometric and hyperbolic functions:
			\begin{equation}
				\begin{split}
					\frac{\dd}{\dd x}\arcsin x &= \frac{\pm 1}{\sqrt{1-x^2}} \\
					\frac{\dd}{\dd x}\arccos x &= \frac{\mp 1}{\sqrt{1-x^2}} \\
					\frac{\dd}{\dd x}\arctan x &= \frac{1}{x^2+1} \\
					\frac{\dd}{\dd x}\mathrm{arsinh}\,x &= \frac{1}{\sqrt{1+x^2}} \\
					\frac{\dd}{\dd x}\mathrm{arcosh}\,x &= \frac{\pm 1}{\sqrt{x^2-1}} \\
					\frac{\dd}{\dd x}\mathrm{artanh}\,x &= \frac{1}{1-x^2} \\
				\end{split}
			\end{equation}

			\noindent
			Jacobi's formula\index{Jacobi!Formel}:
			\begin{equation}
				\tder{}{t}\mathrm{det} A(t) = \mathrm{tr}\left(A^T \tder{}{t} A\right)
			\end{equation}

		\subsubsection{Integration}
			\noindent
			Leibniz integral rule\index{Leibniz!Regel für Parameterintegrale}:
			\begin{equation}
				\tder{}{y} \int_{\alpha(y)}^{\beta(y)}f(x,y)\;\dd x = \int_{\alpha(y)}^{\beta(y)} \pder{f}{y}(x,y)\;\dd x + f(\beta(y),y)\tder{\beta}{y}(y)-f(\alpha(y),y)\tder{\alpha}{y}(y)
			\end{equation}

			\noindent
			Cauchy integral theorem\index{Cauchy!Integralsatz} (for a closed curve $\partial \mathcal{A}$ and a holomorphic function $f(z)$ on a connected domain $G \subseteq \mathbb{C}$):
			%(für eine geschlossene Kurve $\partial \mathcal{A}$ und eine holomorphe Funktion $f(z)$ auf einem einfach zusammenhängenden Gebiet $G \subseteq \mathbb{C}$):
			\begin{equation}
				\int_{\partial \mathcal{A}} f(z)\;\dd z = 0
			\end{equation}

			\noindent
			Cauchy integral formula\index{Cauchy!Integralformel} (for a positively oriented circular loop $\gamma(t)=z_0+re^{\it};\;t\in[0,2\pi]$ and a holomorphic function $f(z)$ on a domain $G \subseteq \mathbb{C}$):
			%(für eine im positiven Drehsinn orientierte geschlossene Kreis-Kurve $\gamma(t)=z_0+re^{\it};\;t\in[0,2\pi]$ und eine holomorphe Funktion $f(z)$ auf einem Gebiet $G \subseteq \mathbb{C}$):
			\begin{equation}
				\frac{\dd^k f}{\dd z^k}(z_0) = \frac{k!}{2\pi \i} \int_\gamma \frac{f(z)}{(z-z_0)^{k+1}}\;\dd z
			\end{equation}

			\noindent
			Residue theorem\index{Residuensatz} (for a simple, closed piece wise continuously differentiable path (contour) $\gamma:\left[a,b\right]\rightarrow G$ and $f:G\backslash\lbrace a_1, a_2, .., a_n\rbrace \rightarrow \mathbb{C}$ holomorphic,
			with poles $a_1, a_2, .., a_n$, and corresponding residues $\mathrm{Res}(f; a) = \frac{1}{(k-1)!}\frac{\dd^{k-1} g}{\dd z^{k-1}}\Big|_{z=a}$ for $f(z)=\frac{g(z)}{(z-a)^k}$ near $a$:
			\begin{equation}
				\int_\gamma f(z)\,\dd z = 2\pi \i \sum_{k=1}^{n} \mathrm{Res}(f; a_k)
			\end{equation}

			\noindent
			Trigonometric integrals:
			\begin{equation}
				\begin{split}
					\int \sin^2 x\;\dd x &= \frac{1}{2}( x-\sin  x\cos  x) \\
					\int \cos^2 x\;\dd x &= \frac{1}{2}( x+\sin  x\cos  x) \\
					\int{\sin x\cos x\;\dd x} &= -\frac{1}{2}\cos^2( x) \\
					\int \sin^3 x\;\dd x &= -\frac{1}{3}\cos{ x}\sin^2{ x}-\frac{2}{3}\cos{ x}
					= \frac{1}{12}\cos{3 x}-\frac{3}{4}\cos{ x} \\
					\int \cos^3 x\;\dd x &= \phantom{-}\frac{1}{3}\sin x \cos^2 x+\frac{2}{3}\sin  x
					= \frac{1}{12}\sin{3 x}+\frac{3}{4}\sin{ x} \\
				\end{split}
			\end{equation}

			\noindent
			Famous integrals (Gauß integral\index{Gauß!Integral}, $\sinc$-integral\index{Sinus Cardinalis}):
			\begin{equation}
				\begin{aligned}
					\int_\mathbb{R} e^{-x^2}\;\dd x &= \sqrt{\pi} \\
					\int_\mathbb{R} e^{-\frac{x^2}{\sigma}+c}\;\dd x &= \sqrt{\pi\sigma} \;\forall c\in\mathbb{C} \\
					\int_\mathbb{R} \frac{\sin x}{x}\;\dd x &= \pi \\
				\end{aligned}
			\end{equation}

			\noindent
			Convolution\index{Faltung} ($f*g:\mathbb{R}^n \rightarrow \mathbb{C}$):
			\begin{equation}
				(f*g)(x) := \int_{\mathbb{R}^n} f(t) g(x-t)\;\dd t
			\end{equation}

		\subsubsection{Distributions}
			\noindent
			Dirac distribution / Dirac delta / Dirac function\index{Dirac!Distribution} $\delta(x)$:
			\begin{equation}
				\int_\Omega f(x)\delta(x)\;\dd x = f(0)\;\forall\, \Omega\ni 0
			\end{equation}

			\noindent
			Dirac distribution of a function $f(x)$ zeros $x_{0,i}$:
			\begin{equation}
				\delta(f(x)) = \sum_i \frac{1}{\left|\pder{f}{x}(x_{0,i})\right|}\delta(x-x_{0,i})
			\end{equation}

			\noindent
			Heaviside function\index{Heaviside!Funktion} $\Theta(x)$ and its properties:
			\begin{equation}
				\begin{aligned}
					\Theta(x) :=& \left\{\begin{array}{ll}
						1 & x\ge 0 \\
						0 & x<0 \\
						\end{array}\right. \\
						\tder{\Theta}{x}(x) =&\; \delta(x) \\
					\int_{\mathbb{R}} f(x)\Theta(x)\;\dd x =& \int_{0}^{\infty} f(x)\;\dd x
				\end{aligned}
			\end{equation}

			\begin{equation}
				\Nabla^2\frac{1}{\left|\vec{r} - \pvec{r}'\right|} = -4\pi\delta\left(\vec{r} - \pvec{r}'\right)
			\end{equation}

			\noindent
			See Eq.~\ref{Eq:FourierIdentities}

		\subsubsection{Gamma Function\index{Gamma Funktion}}
			\noindent
			Definition:
			\begin{equation}
				\Gamma(z)=\int_0^{\infty}e^{-t}t^{z-1}\,dt
			\end{equation}

			\noindent
			Properties ($\forall n\in\mathbb{N}_0$):
			\begin{equation}
				\begin{array}{cc}
					\Gamma(z+1)=z\Gamma(z);
					&\hspace{20pt} \Gamma\left(1 \right)=1; \\
					\Gamma(n+1) = n!
					&\hspace{20pt} \Gamma\left(\frac{1}{2} \right)=\sqrt{\pi} \\
				\end{array}
			\end{equation}

			\noindent
			Stirling Formula\index{Stirling!Formula} (Approximation for $n\rightarrow\infty$):
			\begin{equation}
				n! = \Gamma(n+1) \approx \sqrt{2\pi n} \left( \frac{n}{e} \right)^{n}
			\end{equation}


		\subsubsection{Function space\index{Funktionenraum}}
			\noindent
			$D$ is a definition space.
			\begin{itemize}
				\item $C^p(D)$, space of $p$-times continuously differentiable functions
				\item $L^p(D)$, space of equivalence classes (with respect to lower-dimensional exceptions) of functions which $p$-Norm is Lebesque integrable\index{Lebesque!integrierbar}
				%Raum der Äquivalenzklassen (bezügl. niederdimensionaler Ausnahmemengen) von Funktionen deren $p$-Norm Lebesque-integrierbar sind
				\item $S(\mathbb{R})$, space of functions that decrease faster than any polynomial function (Schwartz space\index{Schwartz!Raum})
				%Raum der Funktionen die schneller fallen als jede Polynomfunktion (Schwartz-Raum)
			\end{itemize}

		\subsubsection{Sequences\index{Folge} and Series\index{Reihe}}
			\noindent
			Geometric summation formula\index{Geometrische Summenformel} ($\forall q \ne 1$):
			\begin{equation}
				\sum_{k=0}^n q^k=\frac{1-q^{n+1}}{1-q}
			\end{equation}

			\noindent
			Geometric series\index{Geometrische Reihe} ($\forall \left|q\right| < 1$):
			\begin{equation}
				\sum_{k=0}^\infty q^k= \frac{1}{1-q}
			\end{equation}

			\noindent
			Binomial theorem\index{Binomischer Satz} ($n\in\mathbb{R}$):
			\begin{equation}
				(x+y)^n=\sum_{k=0}^{n}\binomkoeff{n}{k}x^k y^{n-k}
			\end{equation}

			\noindent
			Identities:
			\begin{equation}
				\lim_{n\rightarrow\infty} \sqrt[n]{n} = 1
			\end{equation}

			\noindent
			\href{https://en.wikipedia.org/wiki/Abel%27s_theorem}{Abel limit theorem}\index{Abel!Grenzwertsatz}:
			\begin{equation}
				\begin{aligned}
					\lim_{t\rightarrow+\infty} f(t) &= \lim_{\eta\rightarrow 0} \eta \int_0^\infty e^{-\eta t} f(t) \,\dd t \\
					\lim_{t\rightarrow-\infty} f(t) &= \lim_{\eta\rightarrow 0} \eta \int_{-\infty}^{0} e^{+\eta t} f(t) \,\dd t \\
				\end{aligned}
			\end{equation}

			\noindent
			\href{https://en.wikipedia.org/wiki/Euler%E2%80%93Maclaurin_formula}{Euler-Maclaurin formula}\index{Euler!-McLaurin Formel}:
			\begin{equation}
				\sum_{k=a+1}^{b} f(k) = \int_{a}^{b}f(x) \;\dd x
				+ \frac{1}{2} \left.\frac{\dd f}{\dd x} \right|_{a}^{b}
				+ \frac{1}{12}\left.\frac{\dd^2 f}{\dd x^2} \right|_{a}^{b}
				+ \frac{1}{720} \left.\frac{\dd^3 f}{\dd x^3} \right|_{a}^{b}
				+ ...
			\end{equation}

		\subsubsection{Taylor Series\index{Potenzreihe}\index{Taylor!Reihe}}
			\noindent
			Polynomial sequence:
			\begin{equation}
				f(z) = \sum_{n=0}^{\infty} a_n (z-z_0)^n
			\end{equation}

			\noindent
			Radius of convergence\index{Konvergenzradius}:
			\begin{equation}
				r = \frac{1}{\limsup\limits_{n\rightarrow\infty} \sqrt[n]{\left|a_n\right|}} = \lim_{n\rightarrow\infty}\left|\frac{a_n}{a_{n+1}}\right|
			\end{equation}

			\noindent
			Taylor expansion\index{Taylor!Entwicklung} at $x_0$:
			\begin{equation}
				f\left(x\right)=\sum_{n=0}^{\infty}\frac{1}{n!}\frac{\dd^{n}f\left(x_0\right)}{\dd x^{n}}\left(x-x_0\right)^{n}
			\end{equation}

			\noindent
			Polynomial sequence of important functions:
			\begin{equation}
				\begin{aligned}
					e^x &= \sum_{k=0}^{\infty}{\frac{x^k}{k!}} \\
					\sin(x) &= \sum_{k=0}^\infty (-1)^k \frac{x^{2k+1}}{(2k+1)!} \\
					\cos(x) &= \sum_{k=0}^\infty (-1)^k \frac{x^{2k}}{(2k)!} \\
				\end{aligned}
			\end{equation}

	\subsection{Basis Functions}
		\noindent
		Complex inner product\index{Hermite!Skalarprodukt} in $R([0,2\pi],\mathbb{C})$:
		\begin{equation}
			\left<f,g\right>:=\frac{1}{2\pi}\int_0^{2\pi} f(x)g^* (x)\;\dd x
		\end{equation}

		\subsubsection{Fourier Analysis\index{Fourier!Analyse}}
			\noindent
			Complex trigonometric basis functions:
			\begin{equation}
				\hat{e}_k(x):=e^{\i kx}
			\end{equation}

			\noindent
			Discrete complex Fourier transformation\index{Fourier!Transformation}:
			\begin{equation}
				(Tf)(x)=\sum_{k\in\mathbb{Z}}c_k e^{\i kx}
			\end{equation}

			\noindent
			Complex Fourier coefficients:
			\begin{equation}
				c_k = \left<f,\hat{e}_k\right>=\frac{1}{2\pi}\int_0^{2\pi}f(x)e^{-\i kx}\;\dd x
			\end{equation}

			\noindent
			Discrete real Fourier transformation\index{Fourier!Transformation}:
			\begin{equation}
				(Tf)(x)=\frac{a_0}{2}+\sum_{k=1}^{\infty}\left[a_k \cos(kx) + b_k \sin(kx) \right]
			\end{equation}

			\noindent
			Relation between complex and real coefficients:
			\begin{equation}
				\begin{array}{clc}
					c_0 = \dfrac{a_0}{2} & a_k = c_k+c_{-k} & \phantom{_{-}}c_k = \dfrac{1}{2}\left(a_k-ib_k\right) \\ [6pt]
					& b_k = \dfrac{1}{i}\left(c_{-k}-c_{k}\right) & c_{-k} = \dfrac{1}{2}\left(a_k+ib_k\right) \\
				\end{array}
			\end{equation}

			\noindent
			Continuous Fourier transformation\index{Fourier!Transformation} and inverse transformation:
			\begin{equation}
				\begin{aligned}
					(\mathcal{F}f)(\vec{k}) :=& \frac{1}{\sqrt{2\pi}^n}\int_{\mathbb{R}^n} \phantom{(\mathcal{F})}f(\vec{x})\, e^{-\i\vec{k}\cdot\vec{x}}\;\dd\vec{x} \\
					\phantom{(\mathcal{F})}f(\vec{x}) =& \frac{1}{\sqrt{2\pi}^n}\int_{\mathbb{R}^n} (\mathcal{F}f)(\vec{k}) \,e^{\i\vec{k}\cdot\vec{x}}\;\dd \vec{k} \\
				\end{aligned}
			\end{equation}

			\noindent
			Identities of the continuous Fourier transformation\index{Fourier Transformation} (where $\tilde{f}(k) := \left(\mathcal{F}f\right)(k)$)
			\begin{equation}
				\begin{aligned}
					\mathcal{F}\left(xf\right)(k) &= \phantom{-}i \pder{}{k}\tilde{f}(k) \\
					\mathcal{F}^{-1}(k\tilde{f})(x) &= -i\pder{}{x}f(x) \\
					\int_{\mathbb{R}} f^{*}(x) g(x)\;\dd x &=
					\int_{\mathbb{R}}\tilde{f}^{*}(k)\tilde{g}(k)\;\dd k \\
					\delta(x) &= \frac{1}{2\pi}\int_{\mathbb{R}}e^{\i kx}\;\dd k \\
					\label{Eq:FourierIdentities}
				\end{aligned}
			\end{equation}

			\noindent
			Further properties:
			\begin{equation}
				\begin{aligned}
					f(x)\text{ is purely real} &\Leftrightarrow \tilde{f}(-k) = \phantom{-}\tilde{f}^*(k) \\
					f(x)\text{ is purely imaginary} &\Leftrightarrow \tilde{f}(-k) = -\tilde{f}^*(k) \\
					f(x)\text{ is purely real and even} &\Leftrightarrow \tilde{f}(k)\text{ is purely real and even} \\
					f(x)\text{ is purely real and odd} &\Leftrightarrow \tilde{f}(k)\text{ is purely imaginary and odd} \\
				\end{aligned}
			\end{equation}

			\noindent
			\href{https://en.wikipedia.org/wiki/Fourier_transform#Functional_relationships,_one-dimensional}{Properties listed by wikipedia.org}

		\subsubsection{Legendre Transformation\index{Legendre!Transformation}}
			\noindent
			Legendre transformation (self-adjoint isometry between convex functions)
			\begin{equation}
				(\mathcal{L}f)(p)=\max_x\left\lbrace xp-f(x) \right\rbrace
			\end{equation}


		\subsubsection{Laurent Series\index{Laurent!Reihe}}
			\noindent
			Laurent Series\index{Laurent Reihe}:
			\begin{equation}
				f(z)=\sum_{n\in\mathbb{Z}} a_n(z-z_0)^n
			\end{equation}

			\noindent
			Coefficients:
			\begin{equation}
				a_n = \frac{1}{2 \pi \i}\int_{\left| z-z_0 \right| = r} \frac{f(z)}{(z-z_0)^{n+1}}\;\dd z
			\end{equation}

		\subsubsection{Spherical Harmonics\index{Kugelflächenfunktionen}}
			\noindent
			Expansion in spherical harmonics\index{Kugelflächenfunktionen}:
			\begin{equation}
				f(\theta, \phi) = \sum_{l=0}^{\infty} \sum_{m=-l}^{l} f_{lm} Y_{lm}(\theta,\phi)
			\end{equation}

			\noindent
			Coefficients:
			\begin{equation}
				f_{lm} = \int_0^{2\pi} \int_0^\pi Y_{lm}^{*}(\theta,\phi) f(\theta,\phi) \sin\theta\;\dd\theta\dd\phi
			\end{equation}

	\subsection{Differential Equations}
		\subsubsection{General}
			\noindent
			Cauchy-Riemann differential Equations\index{Cauchy!-Riemann Differentialgleichungen} (Always hold for complex functions $f(z)=u(z)+iv(z)$ with $z=x+iy$, where $x,y,u,v\in\mathbb{R}$)
			\begin{equation}
				\begin{aligned}
					\pder{u}{x} &= \phantom{-}\pder{v}{y} \\
					\pder{u}{y} &= -\pder{v}{x} \\
				\end{aligned}
			\end{equation}


		\subsubsection{Green's Functions\index{Green!Funktion}}
			\noindent
			Green's Function\index{Green!Funktion} ($\mathrm{D}$ is an arbitrary linear differential operator):
			\begin{equation}
				\mathrm{D}_{\pvec{r}}\, G\left(\vec{r},\pvec{r}'\right) = \delta\left(\vec{r} - \pvec{r}'\right)
			\end{equation}
			General solution of the differential equation $\mathrm{D}_{\pvec{r}}\, \phi = f\left(\vec{r}\right)$:
			\begin{equation}
				\phi\left(\vec{r}\right) = \int G\left(\vec{r},\pvec{r}'\right)f\left(\pvec{r}'\right)\;\dd^3\pvec{r}'
			\end{equation}

		\subsubsection{Harmonic Oscillator\index{Harmonischer Oszillator}}
			\noindent
			Differential equation and solution:
			\begin{equation}
				\begin{aligned}
					\ddot{x}+\omega_0^2 x &= 0 \\
					x(t) &= x_0 e^{\pm \i\omega_0 t}
				\end{aligned}
			\end{equation}

		\subsubsection{Forced, Damped Harmonic Oscillator}
			\noindent
			Differential equation and solution:
			\begin{equation}
				\begin{aligned}
					m\ddot{x}+m\gamma\dot{x}+m\omega_0^2 x &= F_0 e^{-\i\omega t} \\
					x(t) &= \frac{F_0}{m} \frac{1}{\left(\omega_0^2-\omega^2\right)-i\gamma\omega_0} e^{-\i\omega t}
				\end{aligned}
			\end{equation}

		\subsubsection{Bessel Differential Equation\index{Bessel!Differentialgleichung}}
			\noindent
			Differential equation:
			\begin{equation}
				B_\nu f = x^2\frac{\dd^2 f}{\dd x^2} + x\tder{f}{x}+\left(x^2-\nu^2\right)f = 0
			\end{equation}

			\noindent
			Bessel function\index{Bessel!Funktion} of the first kind:
			\begin{equation}
				J_\nu(x) = \sum_{r=0}^\infty \frac{(-1)^r\left(\frac{x}{2}\right)^{2r+\nu}}{\Gamma(\nu+r+1)\,r!} = \frac{1}{2\pi}\int_{-\pi}^{\pi} 	e^{\i(x\sin\varphi-\nu\varphi)}\;\dd\varphi
			\end{equation}

		\subsubsection{Poisson Equation\index{Poisson!Gleichung}}
			\noindent
			Poisson equation\index{Poisson!Gleichung} of electrostatics:
			\begin{equation}
				\Nabla^2\phi = -\frac{\rho}{\epsilon_0}
			\end{equation}

			\noindent
			Sufficient conditions for the existence and uniqueness of solutions of the Poisson equation:
			%Ausreichende Bedingungen für Existenz und Eindeutigkeit von Lösungen der Poisson-Gleichung:
			\begin{description}
				\item[Dirichlet boundary conditions\index{Dirichlet!Randbedingungen}]\hfill \\
					The charge density $\rho(\vec{r})$ in $\vec{r}\in\mathcal{V}$ and the potential $\phi(\vec{r})$ on $\vec{r}\in\partial\mathcal{V}$ are known.
				\item[Neumann boundary conditions\index{Neumann!Randbedingungen}]\hfill \\
					The charge density  $\rho(\vec{r})$ in $\vec{r}\in\mathcal{V}$ and the normal gradient of the potential  $\vec{n}(\vec{r})\cdot\Nabla\phi(\vec{r})$ on $\vec{r}\in\partial\mathcal{V}$ are known.
				\item[Boundary conditions with conductors of known total charge]\hfill \\
					The charge density $\rho(\vec{r})$ in $\vec{r}\in\mathcal{V}$ and the total charge $Q_j$ of conductors restricted by $\partial\mathcal{V}$ are known.
			\end{description}
				% \item[Dirichlet-Randbedingungen]\hfill \\
				% 	Die Ladungsdichte $\rho(\vec{r})$ in $\vec{r}\in\mathcal{V}$ und das Potential $\phi(\vec{r})$ auf $\vec{r}\in\partial\mathcal{V}$ sind bekannt.
				% \item[Neumann-Randbedingungen]\hfill \\
				% 	Die Ladungsdichte $\rho(\vec{r})$ in $\vec{r}\in\mathcal{V}$ und der Normalengradient des Potentials $\vec{n}(\vec{r})\cdot\Nabla\phi(\vec{r})$ auf $\vec{r}\in\partial\mathcal{V}$ sind bekannt.
				% \item[Randbedingungen mit Leitern bekannter Gesamtladungen]\hfill \\
				% 	Die Ladungsdichte $\rho(\vec{r})$ in $\vec{r}\in\mathcal{V}$ und die Gesamtladungen $Q_j$ von Leitern die durch $\partial\mathcal{V}$ begrenzt werden sind bekannt.

		\subsubsection{Laplace Equation}
			\noindent
			Laplace equation\index{Laplace!Gleichung}
			\begin{equation}
				\Delta\phi=\Nabla^2\phi = 0
			\end{equation}

		\subsubsection{Legendre Differential Equation}
			\noindent
			Legendre differential equation\index{Legendre!Differentialgleichung} ($l>0$):
			\begin{equation}
				\pder{}{x}\left(\left(1-x^2\right)\pder{f}{x}\right)+l\left(l+1\right)f
				= \frac{\partial^2 f}{\partial x^2} - \frac{2x}{1-x^2}\pder{f}{x} + \frac{l(l+1)}{1-x^2}f = 0
			\end{equation}

			\noindent
			Solutions (Legendre Polynomials\index{Legendre!Polynome}) / Rodrigues Formula\index{Rodrigues!Formel}:
			\begin{equation} \label{Eq:LegendrePolynomials}
				\begin{aligned}
					P_0(x) &= 1 \\
					P_1(x) &= x \\
					P_2(x) &= \frac{1}{2}\left(3x^2-1\right) \\
					P_l(x) &= \frac{1}{2^l l!}\frac{\dd^l}{\dd x^l}\left(x^2-1\right)^l \\
				\end{aligned}
			\end{equation}

			\noindent
			Explicit construction using the binomial coefficient:
			\begin{equation}
				P_l(x)=\frac{1}{2^l} \sum_{k=0}^{l}\binomkoeff{l}{k}^2(x+1)^k(x-1)^{l-k}
			\end{equation}

			\noindent
			Orthogonality:
			\begin{equation}
				\int_{-1}^1 P_l(x) P_k(x)\;\dd x = \delta_{lk} \frac{2}{2l+1}
			\end{equation}

		\subsubsection{Generalized Legendre Differential Equation}
			\noindent
			Differential equation ($l\in\mathbb{N}$ and $m\in\mathbb{Z}, -l\le m\le l$)
			\begin{equation}
				\pder{}{x}\left(\left(1-x^2\right)\pder{f}{x}\right)+\left(l\left(l+1\right)-\frac{m^2}{1-x^2}\right)f= 0
			\end{equation}

			\noindent
			Solutions (Legendre Functions\index{Legendre!Funktionen})
			\begin{equation}
				\begin{aligned}
					P_l^m(x) &= \frac{(-1)^m}{2^l l!}\sqrt{1-x^2}^m
					\frac{\dd^{l+m}}{\dd x^{l+m}}\left(x^2-1\right)^l \\
					&= (-1)^m \frac{(m+l)!}{2^l l! (l-m)!}\frac{1}{\sqrt{1-x^2}^m}
					\frac{\dd^{l-m}}{\dd x^{l-m}}\left(x^2-1\right)^l \\
				\end{aligned}
			\end{equation}

		\subsubsection{Spherical Harmonics}
			\noindent
			Differential equation:
			\begin{equation}
				-\left(
				\frac{1}{\sin\theta}\pder{}{\theta}\sin\theta\pder{}{\theta} + \frac{1}{\sin^2\theta}\frac{\partial^2}{\partial \phi^2}
				\right)
				Y_{lm}(\theta,\phi) = l(l+1)Y_{lm}(\theta,\phi)
			\end{equation}

			\noindent
			Solutions (Spherical Harmonics):
			\begin{equation} \label{Eq:SphericalHarmonics}
				Y_{lm}(\theta,\varphi) = \sqrt{\frac{2l+1}{4\pi}\frac{(l-m)!}{(l+m)!}} e^{\i m\varphi} P_l^m(\cos\theta)
			\end{equation}

			\noindent
			Orthogonality:
			\begin{equation}
				\int_0^{2\pi}\dd\varphi \int_0^{\pi}\dd\theta \sin\theta\, Y_{lm}(\theta,\varphi) Y^{*}_{l'm'}(\theta,\varphi) = 	\delta_{ll'}\delta_{mm'}
			\end{equation}

		\subsubsection{Hermite's Differential Equation}
			\noindent
			Differential equation\index{Hermite!Differentialgleichung} ($n\in\mathbb{N}_0$):
			\begin{equation}
				\frac{\dd^2}{\dd x^2} H_n(x) -2x\tder{}{x} H_n+2 n H_n(x)=0
			\end{equation}

			\noindent
			Solutions (Hermite polynomials\index{Hermite!Polynome}):
			\begin{equation} \label{Eq:HermitePolynomials}
				\begin{aligned}
					H_n(x) &= (-1)^n e^{x^2} \frac{\dd^n}{\dd x^n}e^{-x^2} \\
					&= e^{x^2/2}\left(x-\tder{}{x}\right)^n e^{-x^2/2} \\
				\end{aligned}
			\end{equation}

			\noindent
			Orthogonality:
			\begin{equation}
				\int_{\mathbb{R}} H_m(x) H_n(x) e^{-x^2}\;\dd x = \sqrt{\pi}\,2^n n!\delta_{nm}
			\end{equation}

		\subsubsection{Laguerre Differential Equation}
			\noindent
			Differential Equation\index{Laguerre!Differentialgleichung} ($x>0$, $n\in\mathbb{N}_0$):
			\begin{equation}
				x\frac{\dd^2}{\dd x^2}L_n(x) +(1-x)\tder{}{x}L_n(x) + nL_n(x) = 0
			\end{equation}

			\noindent
			Solutions (Laguerre Polynomials\index{Laguerre!Polynome}):
			\begin{equation}
				L_n(x) = \frac{e^x}{n!}\frac{\dd^n}{\dd x^n}(e^{-x} x^n) = \frac{1}{n!} \left(\tder{}{x}-1\right)^n x^n
			\end{equation}

			\noindent
			Orthogonality:
			\begin{equation}
				\int_0^\infty e^{-x} L_n(x) L_m(x)\;\dd x = \delta_{nm}
			\end{equation}

			\noindent
			Associated Laguerre Polynomials ($k\in\mathbb{N}_0$):
			\begin{equation}
				L_n^k(x) = (-1)^k\frac{\dd^k}{\dd x^k} L_{n+k}(x)
			\end{equation}

	\subsection{Geometry}
		\subsubsection{General}
			\noindent
			Volume of a $n$-dimensional Sphere:
			\begin{equation}
				\Omega_n = \frac{\sqrt{\pi}^n}{\left(\frac{n}{2}\right)!}R^n
				= \frac{\sqrt{\pi}^n}{\Gamma\left(\frac{n}{2}+1\right)}R^n
			\end{equation}


		\subsubsection{Solid Angle\index{Raumwinkel}}
			\noindent
			Solid angle definition (partial Surface of the unit sphere):
			\begin{equation}
				\Omega = \frac{A}{R^2}
			\end{equation}

			\noindent
			Differential solid angle in spherical coordinates:
			\begin{equation}
				\dd \Omega = \sin\theta\;\dd \theta \,\dd \phi
			\end{equation}

			\noindent
			Solid angle of a cone with opening angle $2\theta$:
			\begin{equation}
				\Omega = 4\pi\sin^2\left(\frac{\theta}{2}\right)
			\end{equation}

		\subsubsection{Triangles}
			\noindent
			Law of sines\index{Sinussatz}:
			\begin{equation}
				\frac{a}{\sin\alpha} = \frac{b}{\sin\beta} = \frac{c}{\sin\gamma}
			\end{equation}

			\noindent
			Law of cosines\index{Kosinussatz}:
			\begin{equation}
				c^2 = a^2 + b^2 -2ab \cos\gamma
			\end{equation}


	\subsection{Vector analysis}
		\subsubsection{Integration on Manifolds}
			\noindent
			General Stokes theorem\index{Stokes!Satz} ($\mathcal{M}$ orientable $n$-dimensional Manifold, $\omega$ continuously differentiable alternating differential form of order $n-1$):
			\begin{equation}
				\int_\mathcal{M} \dd \omega = \int_{\partial\mathcal{M}} \omega
			\end{equation}

			\noindent
			Divergence theorem / Gauß theorem\index{Gauß!Integralsatz}:
			\begin{equation}
				\oint_{\partial\mathcal{V}}\vec{f}\cdot\vec{n}\,\mathrm{d}A=\int_{\mathcal{V}}\vec{\nabla}\cdot\vec{f}\;\mathrm{d}V
			\end{equation}

			\noindent
			Green's theorem\index{Green!Integralsatz}:
			\begin{equation}
				\oint_{\partial\mathcal{A}}\vec{f}\cdot\mathrm{d}\vec{s}=\int_{\mathcal{A}}\left(\vec{\nabla}\times\pvec{f}\right)\cdot\vec{n}\;\mathrm{d}A
			\end{equation}

		\subsubsection{Vector Identities}
			\noindent
			Graßmann identity\index{Graßmann!Identität}\index{bac-cab Regel}:
			\begin{equation}
				\vec{a}\times\big(\vec{b}\times\vec{c}\big) = \vec{b}\big(\vec{a}\cdot\vec{c}\big) - \vec{c}\big(\vec{a}\cdot\vec{b}\big)
			\end{equation}

			\noindent
			Graßmann identity\index{Graßmann!Identität} in index notation:
			\begin{equation}
				\varepsilon_{ijk}\,\varepsilon_{imn}=\delta_{jm}\delta_{kn}-\delta_{jn}\delta_{km}
			\end{equation}

			\noindent
			Triple product\index{Spatprodukt}:
			\begin{equation}
				\vec{a}\cdot\left(\vec{b}\times\vec{c}\right) = \det\left(\vec{a},\vec{b},\vec{c}\right)
			\end{equation}

			\noindent
			Jacobi matrix\index{Jacobi!Matrix} and Jacobi determinant\index{Jacobi!Determinante}:
			\begin{equation}
				D\left(\vec{f}\,\right) = \left(\frac{\partial f_j}{\partial x_k}\right)_{jk}
				= \left(\begin{matrix}
				\frac{\partial f_1}{\partial x_1} & \dotsb & \frac{\partial f_1}{\partial x_n} \\
				\vdots & \ddots & \vdots \\
				\frac{\partial f_m}{\partial x_1} & \dotsb & \frac{\partial f_m}{\partial x_n} \\
				\end{matrix}\right);\;\;J_f(\pvec{x})=\left|\det\left(D\vec{f}(\pvec{x})\right)\right|
			\end{equation}

			\noindent
			Identities in connection with the gradient, divergence and rotation of vector fields:
			\begin{equation}
				\begin{aligned}
					\Nabla\cdot\left(\Nabla\times\vec{A}\right) &= 0 \\
					\Nabla\times\left(\Nabla\psi\right) &= 0 \\
					\Nabla\cdot\left(\Nabla\psi\right) &= \Nabla^2\psi \\
					\Nabla\times\left(\Nabla\times\vec{A}\right) &= \Nabla\left(\Nabla\cdot\vec{A}\right) -\Nabla^2\vec{A} \\
					\Nabla\cdot\left(\psi\vec{A}\right) &= \Nabla\psi\cdot\vec{A} + \psi\Nabla\cdot\vec{A}\\
					\Nabla\times\left(\psi\vec{A}\right) &= \Nabla\psi\times\vec{A} + \psi\Nabla\times\vec{A} \\
					\Nabla\cdot\left(\vec{A}\times\vec{B}\right) &= \vec{B}\cdot\left(\Nabla\times\vec{A}\right) - 	\vec{A}\cdot\left(\Nabla\times\vec{B}\right) \\
					\Nabla\times\left(\vec{A}\times\vec{B}\right) &= \vec{A}\left(\Nabla\cdot\vec{B}\right) - \vec{B}\left(\Nabla\cdot{A}\right) + \left(\vec{B}\cdot\Nabla\right)\vec{A} - \left(\vec{A}\cdot\Nabla\right)\vec{B} \\
				\end{aligned}
			\end{equation}

			\noindent
			Helmholtz decomposition\index{Helmholtz!Zerlegung}:
			\begin{equation}
				\forall \,\vec{v}(\pvec{r}):\; \exists\left(\phi\left(\pvec{r}\right),\vec{A}(\pvec{r})\right):\;\vec{v} = \Nabla\phi(\pvec{r}) + 	\Nabla\times\vec{A}(\pvec{r})
			\end{equation}

			\noindent
			Poincaré Lemma\index{Poincaré!Lemma} $\forall\, \vec{r}\in\mathcal{G}$ ($\mathcal{G}$ connected domain):
			\begin{equation}
				\exists U(\vec{r}): \vec{v}(\pvec{r}) = -\Nabla U(\pvec{r})
				\Leftrightarrow \vec{v}(\pvec{r}) \text{ is conservative }
				\Leftrightarrow \Nabla\times\vec{v}(\pvec{r}) = 0
			\end{equation}

		\subsubsection{Coordinate Transformations}
			\noindent
			Transformation between cartesian, cylindrical and spherical coordinates:
			\begin{center}
				\begin{tabular}{| r || l | l | l |}
					\hline\xrowht{10pt}
					Coordinates & Cartesian & Cylindrical & Spherical \\
					\hline\hline\xrowht{45pt}
					Cartesian & $\begin{aligned}  x &= x \\  y &= y \\  z &= z\end{aligned}$ & $\begin{aligned}  x &= \rho \cos\varphi \\  y &= \rho \sin\varphi \\  z &= z\end{aligned}$ & $\begin{aligned}  x &= r \sin\theta \cos\varphi \\  y &= r \sin\theta \sin\varphi \\  z &= r \cos\theta\end{aligned}$ \\
					\hline\xrowht{45pt}
					Cylindrical & ${\displaystyle {\begin{aligned}\rho &={\sqrt {x^{2}+y^{2}}}\\\varphi &=\arctan \left({\frac {y}{x}}\right)\\z&=z\end{aligned}}}$ & ${\displaystyle {\begin{aligned}\rho &=\rho \\\varphi &=\varphi \\z&=z\end{aligned}}}$ & ${\displaystyle {\begin{aligned}\rho &=r\sin \theta \\\varphi &=\varphi \\z&=r\cos \theta \end{aligned}}}$ \\
					\hline\xrowht{70pt}
					Spherical & ${\displaystyle {\begin{aligned}r&={\sqrt {x^{2}+y^{2}+z^{2}}}\\\theta &=\arctan \left({\frac {\sqrt {x^{2}+y^{2}}}{z}}\right)\\\varphi &=\arctan \left({\frac {y}{x}}\right)\end{aligned}}}$ & ${\displaystyle {\begin{aligned}r&={\sqrt {\rho ^{2}+z^{2}}}\\\theta &=\arctan {\left({\frac {\rho }{z}}\right)}\\\varphi &=\varphi \end{aligned}}}$ & ${\displaystyle {\begin{aligned}r&=r\\\theta &=\theta \\\varphi &=\varphi \\\end{aligned}}}$ \\
					\hline
				\end{tabular}
			\end{center}
			\href{https://en.wikipedia.org/wiki/Vector_calculus_identities}{Vector calculus identities (en.wikipedia.org/wiki/Vector\_calculus\_identities)}


			\begin{center}
				\makebox[1\textwidth][c]{
				\begin{tabular}{| r || l | l | l |}
					\hline\xrowht{10pt}
					Unit vectors & Cartesian & Cylindrical & Spherical \\
					\hline\hline\xrowht{45pt}
					Cartesian & N/A & $\begin{aligned}  \hat{\mathbf x} &= \cos\varphi \hat{\boldsymbol \rho} - \sin\varphi \hat{\boldsymbol \varphi} \\  \hat{\mathbf y} &= \sin\varphi \hat{\boldsymbol \rho} + \cos\varphi \hat{\boldsymbol \varphi} \\  \hat{\mathbf z} &= \hat{\mathbf z}\end{aligned}$ & $\begin{aligned}  \hat{\mathbf x} &= \sin\theta \cos\varphi \hat{\mathbf r} + \cos\theta \cos\varphi \hat{\boldsymbol \theta} - \sin\varphi \hat{\boldsymbol \varphi} \\  \hat{\mathbf y} &= \sin\theta \sin\varphi \hat{\mathbf r} + \cos\theta \sin\varphi \hat{\boldsymbol \theta} + \cos\varphi \hat{\boldsymbol \varphi} \\  \hat{\mathbf z} &= \cos\theta \hat{\mathbf r} - \sin\theta \hat{\boldsymbol \theta}\end{aligned}$ \\
					\hline\xrowht{45pt}
					Cylindrical & ${\displaystyle {\begin{aligned}{\hat {\boldsymbol {\rho }}}&={\frac {x{\hat {\mathbf {x} }}+y{\hat {\mathbf {y} }}}{\sqrt {x^{2}+y^{2}}}}\\{\hat {\boldsymbol {\varphi }}}&={\frac {-y{\hat {\mathbf {x} }}+x{\hat {\mathbf {y} }}}{\sqrt {x^{2}+y^{2}}}}\\{\hat {\mathbf {z} }}&={\hat {\mathbf {z} }}\end{aligned}}}$ & N/A & ${\displaystyle {\begin{aligned}{\hat {\boldsymbol {\rho }}}&=\sin \theta {\hat {\mathbf {r} }}+\cos \theta {\hat {\boldsymbol {\theta }}}\\{\hat {\boldsymbol {\varphi }}}&={\hat {\boldsymbol {\varphi }}}\\{\hat {\mathbf {z} }}&=\cos \theta {\hat {\mathbf {r} }}-\sin \theta {\hat {\boldsymbol {\theta }}}\end{aligned}}}$ \\
					\hline\xrowht{70pt}
					Spherical & ${\displaystyle {\begin{aligned}{\hat {\mathbf {r} }}&={\frac {x{\hat {\mathbf {x} }}+y{\hat {\mathbf {y} }}+z{\hat {\mathbf {z} }}}{\sqrt {x^{2}+y^{2}+z^{2}}}}\\{\hat {\boldsymbol {\theta }}}&={\frac {\left(x{\hat {\mathbf {x} }}+y{\hat {\mathbf {y} }}\right)z-\left(x^{2}+y^{2}\right){\hat {\mathbf {z} }}}{{\sqrt {x^{2}+y^{2}+z^{2}}}{\sqrt {x^{2}+y^{2}}}}}\\{\hat {\boldsymbol {\varphi }}}&={\frac {-y{\hat {\mathbf {x} }}+x{\hat {\mathbf {y} }}}{\sqrt {x^{2}+y^{2}}}}\end{aligned}}}$ & ${\displaystyle {\begin{aligned}{\hat {\mathbf {r} }}&={\frac {\rho {\hat {\boldsymbol {\rho }}}+z{\hat {\mathbf {z} }}}{\sqrt {\rho ^{2}+z^{2}}}}\\{\hat {\boldsymbol {\theta }}}&={\frac {z{\hat {\boldsymbol {\rho }}}-\rho {\hat {\mathbf {z} }}}{\sqrt {\rho ^{2}+z^{2}}}}\\{\hat {\boldsymbol {\varphi }}}&={\hat {\boldsymbol {\varphi }}}\end{aligned}}}$ & N/A \\
					\hline
				\end{tabular}}
			\end{center}


			\noindent
			Gradient:
			\begin{equation}
				\begin{aligned}
					\Nabla f = \pder{f}{\vec{x}} &= \frac{\partial f}{\partial x}\hat{\mathbf x} + \frac{\partial f}{\partial y}\hat{\mathbf y}+ \frac{\partial f}{\partial z}\hat{\mathbf z} \\
					&= \frac{\partial f}{\partial \rho}\hat{\boldsymbol \rho}+ \frac{1}{\rho}\frac{\partial f}{\partial \varphi}\hat{\boldsymbol \varphi}+ \frac{\partial f}{\partial z}\hat{\mathbf z} \\
					&= \frac{\partial f}{\partial r}\hat{\mathbf r}+ \frac{1}{r}\frac{\partial f}{\partial \theta}\hat{\boldsymbol \theta}+ \frac{1}{r\sin\theta}\frac{\partial f}{\partial \varphi}\hat{\boldsymbol \varphi} \\
				\end{aligned}
			\end{equation}

			\noindent
			Divergence:
			\begin{equation}
				\begin{aligned}
					\Nabla\cdot\vec{A} &= \frac{\partial A_x}{\partial x} + \frac{\partial A_y}{\partial y} + \frac{\partial A_z}{\partial z} \\
					&= \frac{1}{\rho}\frac{\partial \left( \rho A_\rho  \right)}{\partial \rho}+ \frac{1}{\rho}\frac{\partial A_\varphi}{\partial \varphi}+ \frac{\partial A_z}{\partial z} \\
					&= \frac{1}{r^2}\frac{\partial \left( r^2 A_r \right)}{\partial r}+ \frac{1}{r\sin\theta}\frac{\partial}{\partial \theta} \left(  A_\theta\sin\theta \right)+ \frac{1}{r\sin\theta}\frac{\partial A_\varphi}{\partial \varphi} \\
				\end{aligned}
			\end{equation}

			\noindent
			Rotation:
			\begin{equation}
				\begin{aligned}
					\Nabla\times\vec{A} &= \left(\frac{\partial A_z}{\partial y} - \frac{\partial A_y}{\partial z}\right) \hat{\mathbf x} + \left(\frac{\partial A_x}{\partial z} - \frac{\partial A_z}{\partial x}\right) \hat{\mathbf y} + \left(\frac{\partial A_y}{\partial x} - \frac{\partial A_x}{\partial y}\right) \hat{\mathbf z} \\
					&= {\displaystyle {\left({\frac {1}{\rho }}{\frac {\partial A_{z}}{\partial \varphi }}-{\frac {\partial A_{\varphi }}{\partial z}}\right){\hat {\boldsymbol {\rho }}}+\left({\frac {\partial A_{\rho }}{\partial z}}-{\frac {\partial A_{z}}{\partial \rho }}\right){\hat {\boldsymbol {\varphi }}}+{\frac {1}{\rho }}\left({\frac {\partial \left(\rho A_{\varphi }\right)}{\partial \rho }}-{\frac {\partial A_{\rho }}{\partial \varphi }}\right){\hat {\mathbf {z} }}}} \\
					&= {\displaystyle {{\frac {1}{r\sin \theta }}\left({\frac {\partial }{\partial \theta }}\left(A_{\varphi }\sin \theta \right)-{\frac {\partial A_{\theta }}{\partial \varphi }}\right){\hat {\mathbf {r} }}+\frac{1}{r}\left({\frac {1}{\sin \theta }}{\frac {\partial A_{r}}{\partial \varphi }}-{\frac {\partial }{\partial r}}\left(rA_{\varphi }\right)\right){\hat {\boldsymbol {\theta }}}+{\frac {1}{r}}\left({\frac {\partial }{\partial r}}\left(rA_{\theta }\right)-{\frac {\partial A_{r}}{\partial \theta }}\right){\hat {\boldsymbol {\varphi }}}}} \\
				\end{aligned}
			\end{equation}

			\noindent
			Scalar Laplace operator\index{Laplace!Operator}:
			\begin{equation}
				\begin{aligned}
					\Delta f = \Nabla^2 f &= \frac{\partial^2 f}{\partial x^2} + \frac{\partial^2 f}{\partial y^2} + \frac{\partial^2 f}{\partial z^2} \\
					&= \frac{1}{\rho} \frac{\partial}{\partial \rho}\left(\rho \frac{\partial f}{\partial \rho}\right)+ \frac{1}{\rho^2} \frac{\partial^2 f}{\partial \varphi^2}+ \frac{\partial^2 f}{\partial z^2} \\
					&= {\displaystyle \frac{1}{r^{2}} \frac{\partial }{\partial r}\!\left(r^{2}\frac{\partial f}{\partial r}\right)\!+\!\frac{1}{r^{2}\!\sin \theta } \frac{\partial }{\partial \theta }\!\left(\sin \theta \frac{\partial f}{\partial \theta }\right)\!+\!\frac{1}{r^{2}\!\sin ^{2}\theta }\frac{\partial ^{2}f}{\partial \varphi ^{2}}}
				\end{aligned}
			\end{equation}


	\subsection{Stochastic}
		\subsubsection{General}
			\noindent
			Factorial:
			\begin{equation}
				n!=\Gamma(n+1)=\prod_{k=1}^{n}k=(n)(n-1)(n-2)...
			\end{equation}

			\noindent
			Binomial coefficient ($n\ge k$):
			\begin{equation}
				\binomkoeff{n}{k} = \frac{k!}{\left(n-k\right)!\,k!}
			\end{equation}

			\noindent
			Properties of the binomial coefficient:
			\begin{equation}
				\begin{array}{cl}
					\binomkoeff{n}{0}=\binomkoeff{n}{n} = 1 & \binomkoeff{n}{k} = \binomkoeff{n}{n-k}\\ [8pt]
					\binomkoeff{n+1}{k+1}=\binomkoeff{n}{k}+\binomkoeff{n}{k+1}\\
				\end{array}
			\end{equation}

		\subsubsection{Expectation Value and Variance}
			\noindent
			Definition of the expectation value:
			\begin{equation}
				\mu = E\left[ X \right] := \int_{-\infty}^{\infty} xf(x)\;\dd x
			\end{equation}

			\noindent
			Definition of variance and standard deviation:
			\begin{equation}
				\sigma^2 = V\left[ X \right] := \int_{-\infty}^{\infty} (x-E\left[X\right])^2f(x)\;\dd x = E\left[X^2\right]-E^2\left[X\right]
			\end{equation}

			\noindent
			Arithmetic mean:
			\begin{equation}
				\bar{x}=\frac{1}{n}\sum_{j=1}^n x_i
			\end{equation}

			\noindent
			Weighed mean:
			\begin{equation}
				\begin{aligned}
					\hat{x} &= \frac{\sum_{j=1}^n \frac{x_j}{\sigma_j^2}}{\sum_{j=1}^n \frac{1}{\sigma_j^2}} \\
					\sigma_{\hat{x}}^2 &= \frac{1}{\sum_{j=1}^n \frac{1}{\sigma_j^2}}
				\end{aligned}
			\end{equation}

			\noindent
			Empirical variance:
			\begin{equation}
				s^2 = \frac{1}{n-1}\sum_{j=1}^n (x_j-\bar{x})^2
			\end{equation}

			\noindent
			Definition of variance:
			\begin{equation}
				V_{xy} = E\left[(x-\mu_x)(y-\mu_y)\right] = E\left[xy\right]-\mu_x\mu_y
			\end{equation}

			\noindent
			Correlation coefficient ($-1\le\rho_{xy}\le 1$):
			\begin{equation}
				\rho_{xy} = \frac{V_{xy}}{\sigma_x\sigma_y}
			\end{equation}

			\noindent
			Properties of expectation value and variance (The last equation is only valid for independent $X, Y$):
			\begin{equation}
				\begin{array}{rl}
					E\left[aX\right] = \phantom{^2}a E\left[X\right]
					&\hspace{20pt}
					E\left[X+Y\right] = E\left[X\right] + E\left[Y\right]
					\\
					V\left[aX\right] = a^2 V\left[X\right]
					&\hspace{20pt}
					V\left[X+Y\right] = E\left[X\right] + E\left[Y\right]
					\\
				\end{array}
			\end{equation}

		\subsubsection{Probability-Generating Function}
			\noindent
			For a random quantity $X$, the probability generating function is defined as:
			\begin{equation}
				Z(\lambda) = E\left[ \exp(\lambda X) \right]
			\end{equation}

			\noindent
			The $n$\textsuperscript{th} raw moment is defined as:
			\begin{equation}
				E\left[X^n\right] = \left.\frac{\dd^n Z}{\dd \lambda^n}\right|_{\lambda=0}
			\end{equation}

			\noindent
			From the Moment-generating function $F(\lambda)=\ln Z(\lambda)$ one can extract
			\begin{equation}
				\begin{aligned}
					F(0) &= 0 \\
					\tder{F}{\lambda}(0) &= E\left[X\right] \\
					\frac{\dd^2 F}{\dd \lambda^2}(0) &= V\left[X\right] \\
				\end{aligned}
			\end{equation}


		\subsubsection{Propagation of Uncertainty}
			\noindent
			Gaußian error propagation\index{Gauß!Fehlerfortpflanzungsgesetz}:
			\begin{equation}
				\sigma_y^2 = \sum_{j,k=1}^n \left[\pder{y}{x_j}\pder{y}{x_k}\right]_{\vec{x}=\vec{\mu}} V_{jk}
			\end{equation}

			\noindent
			Gaußian error propagation\index{Gauß!Fehlerfortpflanzungsgesetz} for uncorrelated  $x_i$ (i.e. $V_{jj} = \sigma_{j}^2$ and $V_{jk} = 0 \;\forall j\ne k$):
			\begin{equation}
				\sigma_y^2 = \sum_{j=1}^n \left[\pder{y}{x_j}\right]^2_{\vec{x}=\vec{\mu}} \sigma_j^2
			\end{equation}


		\subsubsection{Conditional Probabilities}
			\noindent
			Definition of conditional probability:
			\begin{equation}
				P(A|B) := \frac{P(A \cap B)}{P(B)}
			\end{equation}

			\noindent
			Bayes theorem\index{Bayes!Satz}:
			\begin{equation}
				P(A|B) = \frac{P(B|A)P(A)}{P(B)}
			\end{equation}

		\subsubsection{Discrete Distributions}
			\noindent
			Binomial distribution\index{Binomialverteilung} ($k$ of $n$ of a Bernoulli experiment\index{Bernoulli!Experiment}):
			\begin{equation}
				\begin{aligned}
					P(k)&=\binomkoeff{n}{k}p^k(1-p)^{n-k} \\
					\mu &= np \\
					\sigma^2 &= np(1-p)
				\end{aligned}
			\end{equation}

			\noindent
			Poisson distribution\index{Poisson!Verteilung}:
			\begin{equation}
				\begin{aligned}
					\rho(k) &= \frac{\nu^k}{k!} e^{-\nu} \\
					\mu &= \sigma^2 = \nu
				\end{aligned}
			\end{equation}

			\noindent
			Hypergeometric distribution ($k$ of $M\le N$ of $n$ draws):
			\begin{equation}
				P(x)=\frac{\binomkoeff{M}{k}\binomkoeff{N-M}{n-k}}{\binomkoeff{N}{n}}
			\end{equation}

		\subsubsection{Continuous Distributions}
			\noindent
			Central limit theorem\index{Zentraler Grenzwertsatz}: \par
				\emph{The sum of $n$ independent continuous random variables with expectation value $\mu_j$ and finite variances $\sigma_j^2$ converge in the limit $n\rightarrow \infty$ to a normal distribution with $\mu = \sum_j \mu_j$ and $\sigma^2 = \sum_j \sigma_j^2$.} \vsp
				% \emph{Die Summe von $n$ unabhängigen kontinuierlichen Zufallsgrößen mit Mittelwert $\mu_j$ und endlichen Varianzen $\sigma_j^2$ konvergiert im Grenzfall $n\rightarrow \infty$ zu einer Gaußverteilung mit $\mu = \sum_j \mu_j$ und $\sigma^2 = \sum_j \sigma_j^2$.} \vsp

			\noindent
			Normal distribution\index{Gauß!Verteilung}:
			\begin{equation}
				\rho(x)=\frac{1}{\sigma\sqrt{2\pi}}e^{-\frac{1}{2}\left(\frac{x-\mu}{\sigma}\right)^2}
			\end{equation}

			\noindent
			Uniform distribution:
			\begin{equation}
				\begin{aligned}
					\rho(x) =& \left\{\begin{array}{ll}
					\frac{1}{\beta-\alpha} & \alpha\le x\le \beta \\
					0 & \text{else} \\
					\end{array}\right. \\
					\mu =& \frac{1}{2}(\alpha+\beta) \\
					\sigma^2 =& \frac{1}{12}(\beta-\alpha)^2 \\
				\end{aligned}
			\end{equation}

	\newpage
	% !TEX root = ../physics.tex
\section{Constants}
	\href{https://en.wikipedia.org/wiki/List_of_physical_constants}{Wikipedia list of physical constants (en.wikipedia.org/wiki/List\_of\_physical\_constants)}\\
	\href{https://physics.nist.gov/cuu/Constants/}{CODATA List of physical constants (physics.nist.gov/cuu/Constants/)}
	\subsection{Fundamental Constants}
		\begin{center}
			\begin{tabular}{| L{.35\textwidth} L{.15\textwidth} L{.4\textwidth} |}
				\hline Name & Symbol & Value \\ \hline \hline
				\href{https://en.wikipedia.org/wiki/Speed_of_light}{Speed of light in a vacuum} & $c$ & $299\,792\,458\;\frac{\mathrm{m}}{\mathrm{s}}$ \exact \\ \hline
				\href{https://en.wikipedia.org/wiki/Planck_constant}{Planck's Constant}\index{Planck!Wirkungsquantum} & $h$ & $6.626\,070\,15\e{-34}\unit{J\,s}$ \exact \\ \hline
				\href{https://en.wikipedia.org/wiki/Gravitational_constant}{Gravitational Constant}\index{Newton!Gravitationskonstante} & $G$ & $6.674\,30(15)\e{-11}\;\frac{\mathrm{m^3}}{\mathrm{kg\,s^2}} $ \\ \hline
				\href{https://en.wikipedia.org/wiki/Elementary_charge}{Elementary charge}\index{Elementarladung} & $e$ & $1.602\,176\,634\e{-19}\unit{C}$ \exact \\ \hline
				\href{https://en.wikipedia.org/wiki/Vacuum_permittivity}{Vacuum permittivity} \index{Vakuum!Permittivität} & $\varepsilon_0$ & $8.854\,187\,812\,8(13)\e{-12}\;\frac{\mathrm{A\,s}}{\mathrm{V\,m}}$ \\ \hline
				\href{https://en.wikipedia.org/wiki/Vacuum_permeability}{Vacuum permeability} \index{Vakuum!Permeabilität} & $\mu_0$ & $1.256\,637\,062\,12(19)\e{-6}\;\frac{\mathrm{N}}{\mathrm{A^2}}$ \\ \hline
				\href{https://en.wikipedia.org/wiki/Boltzmann_constant}{Boltzmann constant}\index{Boltzmann!Konstante} & $\kB$ & $1.380\,649\e{-23}\;\frac{\mathrm{J}}{\mathrm{K}}$ \exact \\ \hline
				\href{https://en.wikipedia.org/wiki/Weinberg_angle}{Weinberg angle}\index{Weinberg!Winkel} & $\theta_W$ & $\sin^2(\theta_W) = 0.223\,05(23)$ \\ \hline
			\end{tabular}
		\end{center}

	\subsection{Particle Constants}
		\label{Sec:ParticleConstants}
		\begin{center}
			\begin{tabular}{| L{.35\textwidth} L{.15\textwidth} L{.4\textwidth} |}
				\hline Name & Symbol & $\phantom{-}$Value \\ \hline \hline
				\href{https://en.wikipedia.org/wiki/Electron_mass}{Electron mass} & $m_e$ & $\phantom{-}9.109\,383\,701\,5(28)\e{-31}\unit{kg} = 0.510\,998\,950\,00(15)\unit{MeV/c^2}$ \\ \hline
				\href{https://en.wikipedia.org/wiki/Proton}{Proton mass} & $m_p$ & $\phantom{-}1.672\,621\,923\,69(51)\e{-27}\unit{kg} = 938.272\,088\,16(29)\unit{MeV/c^2}$ \\ \hline
				\href{https://en.wikipedia.org/wiki/Neutron}{Neutron mass} & $m_n$ & $\phantom{-}1.674\,927\,498\, 04(95)\e{-27}\unit{kg} = 939.565\,420\,52(54)\unit{MeV/c^2}$ \\ \hline
				Landé-Factor of the Electron & $g_e$ & $\phantom{-}2.002\,319\,304\,362\,56(35)$ \\ \hline
				Landé-Factor of the Proton & $g_p$ & $\phantom{-}5.585\,694\,689\,3(1\,6)$ \\ \hline
				Landé-Factor of the Neutron & $g_n$ & $- 3.826\,085\,45(90)$ \\ \hline
				\href{https://en.wikipedia.org/wiki/W_and_Z_bosons#W_bosons}{$W$-Boson mass} & $m_W$ & $\phantom{-}80.377(12)\unit{GeV/c^2}$ \\ \hline
				\href{https://en.wikipedia.org/wiki/W_and_Z_bosons#W_bosons}{$Z$-Boson mass} & $m_Z$ & $\phantom{-}91.1876(21)\unit{GeV/c^2}$ \\ \hline
				\href{https://en.wikipedia.org/wiki/Higgs_boson}{Higgs-Boson mass}\index{Higgs!Boson} & $m_H$ & $\phantom{-}125.11(11)\unit{GeV/c^2}$ \\ \hline
			\end{tabular}
		\end{center}

		\href{https://pdg.lbl.gov/}{Particle Data group (pdg.lbl.gov)}



	\subsection{Fermion Quantum Numbers}
		\label{Sec:FermionConstants}
		$Y$ is the weak hypercharge, $T_3$ is the weak isospin, $Q$ is the electric charge:
		\begin{center}
			\begin{tabular}{| L{.45\textwidth} | L{.45\textwidth} |}
				\hline
				Left-handed quarks: $\doublet{u}{d}_L, \doublet{c}{s}_L, \doublet{t}{b}_L$

				$Y=\doublet{1/6}{1/6}$, $T_3=\doublet{+1/2}{-1/2}$, $Q=\doublet{+2/3}{-1/3}$
				&
				Right-handed quarks: $\doublet{u}{d}_R, \doublet{c}{s}_R, \doublet{t}{b}_R$

				$Y=\doublet{+2/3}{-1/3}$, $T_3=\doublet{0}{0}$, $Q=\doublet{+2/3}{-1/3}$
				\\
				\hline
				Left-handed leptons: $\doublet{\nu_e}{e}_L, \doublet{\nu_\mu}{\mu}_L, \doublet{\nu_\tau}{\tau}_L$

				$Y=\doublet{-1/2}{+1/2}$, $T_3=\doublet{+1/2}{-1/2}$, $Q=\doublet{0}{-1}$
				&
				Right-handed leptons: $\doublet{u}{d}_R, \doublet{c}{s}_R, \doublet{t}{b}_R$

				$Y=\doublet{0}{0}$, $T_3=\doublet{-1}{0}$, $Q=\doublet{0}{-1}$
				\\
				\hline
			\end{tabular}
		\end{center}

		\renewcommand{\arraystretch}{2.0}
	\subsection{Composite Constants}
		\begin{center}
			\begin{tabular}{| L{.35\textwidth} L{.19 \textwidth} L{.36\textwidth} |}
				\hline Name & Definition & Value \\ \hline \hline
				\href{https://en.wikipedia.org/wiki/Planck_constant#Reduced_Planck_constant}{Reduced Planck's Constant} & $\hbar:=\dfrac{h}{2\pi}$ & $1.054\,571\,817...\e{-34}\unit{J\,s}$ \\[3pt] \hline
				\href{https://en.wikipedia.org/wiki/Stefan%E2%80%93Boltzmann_constant}{Stefan-Boltzmann Constant}\index{Stefan!-Boltzmann Konstante} & $\sigma:=\dfrac{2\pi^5 \kB^4}{15h^3c^2}$ & $5.670\,374\,419...\e{-8}\;\frac{\mathrm{W}}{\mathrm{m^2\,K^4}}$ \\[3pt] \hline
				\href{https://en.wikipedia.org/wiki/Rydberg_constant}{Rydberg constant}\index{Rydberg!Konstante} & $R_\infty := \dfrac{m_e e^4}{8 c h^3 \varepsilon_0^2}$ & $1.097\,373\,156\,816\,0(2\,1)\e{7}\;\frac{1}{\mathrm{m}}$ \\[3pt] \hline
				\href{https://en.wikipedia.org/wiki/Rydberg_constant#Rydberg_unit_of_energy}{Rydberg-Energy}\index{Rydberg!Energie} & $R_y := \dfrac{m_e e^4}{8 h^2 \varepsilon_0^2}$ & $2.179\,872\,361\,103\,5(4\,2)\e{-18}\unit{J}$ \\[3pt] \hline
				\href{https://en.wikipedia.org/wiki/Bohr_radius}{Bohr radius}\index{Bohr!Radius} & $\rho := \dfrac{4\pi\hbar^2\varepsilon_0}{m_e e^2}$ & $5.291\,772\,109\,03(80) \e{-11}\unit{m}$ \\[3pt] \hline
				\href{https://en.wikipedia.org/wiki/Fine-structure_constant}{Fine structure constant}\index{Feinstrukturkonstante} & $\alpha := \dfrac{e^2}{4\pi\varepsilon_0\hbar c}$ & $7.297\,352\,537\,6(5\,0) \e{-3} \approx\dfrac{1}{137}$ \\[3pt] \hline
				Fermi coupling constant\index{Fermi!Kopplungskonstante} & $G_F := \dfrac{\sqrt{2} g^2}{8 m_W^2 c^4}$ & $1.166\,378\,7(6)\unit{GeV^{-2}}$ \\[3pt] \hline
				\href{https://en.wikipedia.org/wiki/Bohr_magneton}{Bohr magneton}\index{Bohr!Magneton} & $\mu_\text{B} := \dfrac{e \hbar}{2 m_e}$ & $9.274\,009\,994\,(57)\e{-24}\;\frac{\mathrm{J}}{\mathrm{T}}$ \\[3pt] \hline
				\href{https://en.wikipedia.org/wiki/Nuclear_magneton}{Nuclear magneton}\index{Kern Magneton} & $\mu_\text{N} := \dfrac{e \hbar}{2 m_p}$ & $5.050\,783\,746\,1(1\,5)\e{-27}\;\frac{\mathrm{J}}{\mathrm{T}}$ \\[3pt] \hline
				\href{https://en.wikipedia.org/wiki/Classical_electron_radius}{Classical electron radius}\index{Klassischer Elektronenradius} & $r_e := \dfrac{e^2}{4\pi\varepsilon_0 m_e c^2}$ & $2.817\,940\,322\,7(1\,9)\e{-15}\;\mathrm{m}$ \\[3pt] \hline
			\end{tabular}
		\end{center}
		\renewcommand{\arraystretch}{1.4}

	\subsection{Astronomical Constants}
		\label{Sec:AstronomicalConstants}
		\begin{center}
			\begin{tabular}{| L{.4\textwidth} L{.15\textwidth} L{.35\textwidth} |}
				\hline Name & Symbol & Value \\ \hline \hline
				\href{https://en.wikipedia.org/wiki/Solar_mass}{Solar mass} & $M_\odot$ & $1.988\,92(25)\e{30}\unit{kg}$ \\ \hline
				\href{https://en.wikipedia.org/wiki/Earth_mass}{Earth mass} & $M_\oplus$ & $5.972\,2(6) \e{24}\unit{kg}$ \\ \hline
				\href{https://en.wikipedia.org/wiki/Earth_radius}{Mean earth radius} & $R_\oplus$ & $6.3781 \e{6}\unit{m}$ \\ \hline
				\href{https://en.wikipedia.org/wiki/Solar_constant}{Solar constant}\index{Solarkonstante} & $E_0$ & $1361 \unit{\frac{W}{m^2}}$ \\ \hline
				\href{https://en.wikipedia.org/wiki/Hubble%27s_law}{Current Hubble's constant}\index{Hubble!Konstante} & $H_0$ & $2.33 \e{-18} \unit{\frac{1}{s}}$ \\ \hline
				\href{https://en.wikipedia.org/wiki/Hubble%27s_law}{Cosmological constant} & $\Lambda$ & $1.088(30)\e{-52} \unit{\frac{1}{m^2}}$ \\ \hline
				\href{https://en.wikipedia.org/wiki/Solar_luminosity}{Nominal solar luminosity} & $L_\odot$ & $3.828\e{26} \unit{W}$ \\ \hline
				\href{https://en.wikipedia.org/wiki/Cosmic_microwave_background#cite_note-apj707_2_916-6}{Current temperature of the CMB} & $T_\text{CMB}$ & $2.725\,48 (57) \unit{K}$ \\ \hline
			\end{tabular}
		\end{center}

		\noindent
		\href{https://en.wikipedia.org/wiki/Astronomical_constant}{List of astronomical constants (en.wikipedia.org/wiki/Astronomical\_constant)}

		\noindent
		\href{https://en.wikipedia.org/wiki/Standard_gravitational_parameter}{List of standard gravitational parameters (en.wikipedia.org/wiki/Standard\_gravitational\_parameter)}\index{Gravitationsparameter}

	\subsection{Units}
		\begin{center}
			\begin{tabular}{| L{.35\textwidth} L{.15\textwidth} L{.4\textwidth} |}
				\hline
				Name & Symbol & Definition \\ \hline \hline
				Year & $\mathrm{yr}$ & $31\,557\,600\,\unit{s}$ \\ \hline
				\href{https://en.wikipedia.org/wiki/Light-year}{Light-year}\index{Lichtjahr} & $\mathrm{ly}$ & $9\,460\,730\,472\,580\,800\unit{m}$ \\ \hline
				\href{https://en.wikipedia.org/wiki/Astronomical_unit}{Astronomical Unit} & $\mathrm{AU}$ & $149\,597\,870\,700\unit{m}$ \\ \hline
				\href{https://en.wikipedia.org/wiki/Parsec}{Parsec} & $\mathrm{pc}$ & $\frac{648000}{\pi}\,\mathrm{AU} = 3.085\,677\,581...\e{16}\unit{m}$ \\ \hline
				\href{https://en.wikipedia.org/wiki/Angstrom}{\r{A}ngström} & ${\mbox{\normalfont\AA}}$ & $10^{-10}\unit{m}$ \\ \hline
				\href{https://en.wikipedia.org/wiki/Barn_(unit)}{Barn} & $\mathrm{b}$ & $10^{-18}\unit{m^2}$ \\ \hline
				\href{https://en.wikipedia.org/wiki/Wavenumber#In_spectroscopy}{Kayser} & $\mathrm{K}$ & $10^2\unit{m^{-1}}$ \\ \hline
				\href{https://en.wikipedia.org/wiki/Dalton_(unit)}{Dalton / Atomic mass unit}\index{Atomare Masseneinheit} & $\mathrm{u}$, $\mathrm{Da}$ & $\frac{1}{12}m(\prescript{12}{6}{\mathbf{C}}) = 1.660\,539\,066\,60(50)\e{-27}\unit{kg}$ \\ \hline
				\href{https://en.wikipedia.org/wiki/Erg}{Erg} & $\mathrm{erg}$ & $10^{-7}\,\mathrm{J}$ \\ \hline
				\href{https://en.wikipedia.org/wiki/Calorie}{Calorie\index{Calorie!Einheit}} & $\mathrm{cal}$ & $4\,184\,\mathrm{J}$ \\ \hline
				\href{https://en.wikipedia.org/wiki/Bar_(unit)}{Bar} & $\mathrm{bar}$ & $10^5\unit{Pa}$ \\ \hline
				\href{https://en.wikipedia.org/wiki/Standard_atmosphere_(unit)}{Standard atmosphere} & $\mathrm{atm}$ & $101\,325\unit{Pa}$ \\ \hline
				\href{https://en.wikipedia.org/wiki/Celsius}{Degree Celsius} & $\mathrm{^\circ C}$ & $(x)\mathrm{^\circ C}=(x+273.15)\mathrm{K}$ \\ \hline
				\href{https://en.wikipedia.org/wiki/Gauss_(unit)}{Gauß}\index{Gauß!Einheit} & $\mathrm{Gs}$ & $10^{-4}\unit{T}$ \\ \hline
				\href{https://en.wikipedia.org/wiki/Mole_(unit)}{Mole} & $\mathrm{mol}$ & $6.022\,140\,76\e{23}$ \\ \hline
			\end{tabular}
		\end{center}

		\subsubsection{Logarithmic Units}
			\noindent
			Bel\index{Bel} and Decibel\index{Dezibel} ($L_P$ is the power ratio and $P_0$ is the reference power):
			\begin{equation}
				L_P = \log_{10}\qty(\frac{P}{P_0})\unit{B}
				=10\log_{10}\qty(\frac{P}{P_0})\unit{dB}
			\end{equation}

			\noindent
			Apparent Magnitude ($I_0$ is the reference intensity, originally Vega):
			\begin{equation}
				m = -2.5 \log_{10} \qty( \frac{I}{I_0} )
			\end{equation}

			\noindent
			Absolute Magnitude (Apparent magnitude at a distance of $10\unit{pc}$):
			\begin{equation}
				M = m - 5 \log_{10} \qty( \frac{I(10\unit{pc})}{I_0} )
			\end{equation}

			\noindent
			Distance modulus:
			\begin{equation}
				m - M = 5 \log_{10} \qty( \frac{d}{10\unit{pc}} )
			\end{equation}

		\subsubsection{Other Units}
			\href{https://en.wikipedia.org/wiki/Sievert}{Sievert\index{Sievert!Einheit} $\mathrm{Sv}$} \href{https://en.wikipedia.org/wiki/Equivalent_dose}{(Equivalent dose $H_T$)}:
			\begin{equation}
				\frac{H_T}{\mathrm{Sv}} = \frac{1}{\mathrm{J}}\sum_R W_R D_{T,R}
			\end{equation}
			where $D_{T,R}$ is the absorbed energy dose and $W_R$ is the weighting factor:
			\begin{equation}
				\begin{aligned}
					W_R =
					\begin{cases}
						\gamma,\beta,\mu: & 1 \\
						p, \pi^\pm: & 2 \\
						\alpha, \prescript{A}{Z}{\mathbf{X}}\quad\forall Z>1: & 20 \\
						n: &\begin{cases}
							E<1\unit{MeV}: & \qty(2.5+18.2 \,\ex^{\ln^2(E)/6}) \\
							1\unit{MeV} <E<50\unit{MeV}: & \qty(5.0+17.0 \,\ex^{\ln^2(2E)/6}) \\
							50\unit{MeV}<E: & \qty(2.5+3.25 \,\ex^{\ln^2(0.04E)/6}) \\
						\end{cases}
					\end{cases}\\
				\end{aligned}
			\end{equation}

	\subsection{Natural Units}
		Using $c=\hbar=\kB=1$, where $\qty{x}=x/\qty[x]$ is the numerical value of a quantity in SI units.
		\begin{center}
			\begin{tabular}{| L{.15\textwidth} L{.1\textwidth} L{.1\textwidth} L{.5\textwidth} |}
				\hline
				Dimension & Natural & SI & Conversion \\ \hline \hline
				Mass & $1\unit{eV}$ & $1\unit{eV} / c^2$ & $1\unit{eV}\doteq \qty{e}\qty{c}^{-2}\unit{kg}=1.782\,661\,921\e{-36}\unit{kg}$ \\ \hline
				Length & $1\unit{eV^{-1}}$ & $c \hbar / 1 \unit{eV}$ & $1\unit{eV^{-1}}\doteq \qty{c}\qty{\hbar}\qty{e}^{-1}\unit{m}=1.602\,176\,634\e{-7}\unit{m}$ \\ \hline
				Time & $1\unit{eV^{-1}}$ & $\hbar / 1 \unit{eV}$ & $1\unit{eV^{-1}}\doteq \qty{\hbar}\qty{e}^{-1}\unit{s}=6.582\,119\,569\e{-16}\unit{s}$ \\ \hline
				Temperature & $1\unit{eV}$ & $1\unit{eV} / \kB$ & $1\unit{eV}\doteq \qty{e}\qty{\kB}^{-1}\unit{K}=1.160\,451\,812\e{4}\unit{K}$ \\ \hline
			\end{tabular}
		\end{center}
	\newpage

	\printindex
\end{document}
