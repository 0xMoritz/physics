% !TEX root = ../physics.tex
\section{Quantum Field Theory\index{QFT}\index{Quantenfeldtheorie}} % QFT

	Conventions in this section:
	\begin{itemize}
		\item $\hbar = c = 1$
		\item Metric signature: $\eta_{\mu\nu} = \mathrm{diag}(+1,-1,-1,-1)_{\mu\nu}$
	\end{itemize}


	\subsection{Basics}
		\noindent
		Inner product of 4-vectors:
		\begin{equation}
			q \cdot p = q^\mu p_\mu
		\end{equation}

		\noindent
		Momentum state for a scalar field:
		\begin{equation}
			\Ket{\vec{p}} = \sqrt{2 \omega_{\vec{p}}} \hat{b}^\dagger(\vec{p}) \Ket{0}
		\end{equation}
		Note: the momentum state is defined such that $\Braket{\vec{q}|\vec{p}} = 2\omega_{\vec{p}} (2\pi)^3 \delta(\vec{p}-\vec{q})$ is Lorentz invariant.

		\noindent
		Fock space\index{Fock!Raum} (direct sum of Hilbert spaces\index{Hilbert!Raum} for all possible numbers of particles $N$; $\mathcal{H}^{(0)} = \qty{\Ket{0}}$ consists only of the vacuum state, for bosons, the Fock space consists only of symmetrized states ``$+$'', for fermions, the Fock space consists only of antisymmetrized states ``$-$''):
		\begin{equation}
			\mathcal{F_\pm}
			= \bigoplus_{N=0}^{\infty} \mathcal{H_\pm}^{(N)}
			= \mathcal{H_\pm}^{(0)} \oplus \mathcal{H_\pm}^{(1)} \oplus \mathcal{H_\pm}^{(2)} \oplus ...
		\end{equation}

		\noindent
		Wick rotation\index{Wick!Rotation} (rotating Minkowski spacetime $(c t,x,y,z)$ into Euclidean $(\tau,x,y,z)$ space):
		\begin{equation}
			\tau \longleftrightarrow \i c t
		\end{equation}
		Connection between QFT and Statistical Mechanics:
		\begin{equation}
			\frac{\i t}{\hbar} \longleftrightarrow \frac{1}{\kB T}
		\end{equation}

		\subsubsection{Scattering}
			\noindent
			Mandelstam Variables\index{Mandelstam!Variablen}, \ie Lorentz scalars\index{Lorentz!Skalare} in 4 particle scattering processes: $P_1 + P_2 \to P_3 + P_4$:
			\begin{itemize}\itemsep -0pt
				\item $s\defeq (p_1+p_2)^2=(p_3+p_4)^2$ \hfill{(Square of the invariant mass\index{Schwerpunktsenergie})}
				\item $t\defeq (p_1-p_3)^2=(p_2-p_4)^2$ \hfill{(Square of the four-momentum transfer)}
				\item $u\defeq (p_1-p_4)^2=(p_2-p_3)^2$ \hfill{(Four-momentum transfer of the rebound particle)}
			\end{itemize}

			\noindent
			Connection:
			\begin{equation}
				s+t+u = \sum_i m_i^2
			\end{equation}

			\noindent
			Non-Relativistic Limit ($c \to\infty$):
			\begin{equation}
				\phi(x) \to \phi(x) \ex^{-\i m c^2 t/\hbar}
			\end{equation}

		\subsubsection{Pauli Matrices}
			The Pauli matrices are self hermitian, traceless, $2\times 2$-matrices with determinant $-1$, they are the generators of the Lie algebra of $SU(2)$:

			Clifford Algebra\index{Clifford!Algebra} (defining properties of the Pauli matrices):
			\begin{equation}
				\acomm{\sigma_i}{\sigma_j} = 2\delta_{ij}\hat{1}
			\end{equation}


			\noindent
			Matrix representation of the Pauli matrices\index{Pauli!Matrizen}:
			\begin{equation}
				\label{Eq:PauliMatricesQFT}
				\begin{aligned}
					\sigma^0 \doteq \mqty(
					1 & 0 \\
					0 & 1 \\
					) &&\hsp
					\sigma^1 \doteq \mqty(
					0 & 1 \\
					1 & 0 \\
					) &&\hsp
					\sigma^2 \doteq \mqty(
					0 & -\i \\
					\i & 0 \\
					) &&\hsp
					\sigma^3 \doteq \mqty(
					1 & 0 \\
					0 & -1 \\
					)
				\end{aligned}
			\end{equation}

			\noindent
			Properties of the Pauli matrices\index{Pauli!Spinmatrizen} (see also Eq.~\ref{Eq:PauliMatricesQFT}):
			\begin{align}
				\label{Eq:PauliMatrices}
				\hat{\sigma}_j^\dagger &= \hat{\sigma}_j \\
				\det\qty(\hat{\sigma}_j) &= -1 \\
				\tr\qty(\hat{\sigma}_j) &= 0 \\
				\qty(\hat{\sigma}_1)^2 = \qty(\hat{\sigma}_2)^2 = \qty(\hat{\sigma}_3)^2 &= -\i\hat{\sigma}_1\hat{\sigma}_2\hat{\sigma}_3 = \hat{1} \\
				\hat{\sigma}_j \hat{\sigma}_k &= \delta_{jk}\hat{1} + \i\levicivita_{jkl}\hat{\sigma}_l \\
				\comm{\sigma_\mu}{\sigma_\nu} &= 2\i\levicivita_{\mu\nu\lambda}\sigma_\lambda \\
				\acomm{\sigma_i}{\sigma_j} &= 2\delta_{ij}\hat{1} \\
				\sigma_\mu \sigma_\nu \sigma_\lambda &= \i\levicivita_{\xi\mu\nu\lambda}\sigma_\xi \\
			\end{align}

			\noindent
			Fierz completeness relation:
			\begin{align}
				\sigma^k_{ij} \sigma^k_{kl} &= 2 \delta_{il} \delta_{jk} - \delta_{ij} \delta_{kl}
			\end{align}

		\subsubsection{Generalized Pauli Matrices}
		Definition of the generalized Pauli matrices\index{Pauli!Matrizen}:
		\begin{equation}
			\bar{\sigma}^\mu \defeq \mqty(\mathds{1}_2 & -\vec{\sigma})^\mu
		\end{equation}

		\noindent
		Clifford Algebra
		\begin{equation}
			\acomm{\bar{\sigma}^\mu}{\bar{\sigma}^\nu} = 2\eta^{\mu\nu}
		\end{equation}
		

		\subsubsection{Gamma Matrices}
			\noindent
			Clifford Algebra\index{Clifford!Algebra} (defining properties of the gamma matrices):
			\begin{equation}
				\acomm{\gamma^\mu}{\gamma^\nu} = \gamma^\mu \gamma^\nu + \gamma^\nu \gamma^\mu = 2 \eta^{\mu\nu}
			\end{equation}

			\noindent
			Identities:
			\begin{equation}
				\qty(\gamma^\mu)^\dagger = \eta^{\mu\mu} \gamma^\mu = \gamma^0 \gamma^\mu \gamma^0
				\hsp \gamma^\mu \gamma^\mu = \eta^{\mu\mu} \mathds{1}_4
				\hsp \gamma_\mu \gamma^\mu = 4 \,\mathds{1}_4
			\end{equation}

			\noindent
			Feynman slash notation\index{Feynman!Slash Notation}:
			\begin{equation}
				\slashed{A} \defeq \gamma_\mu A^\mu
			\end{equation}

			\noindent
			Identities
			\begin{equation}
				\slashed{a}^2 = a^2
				\hsp \inv{\slashed{a}} = \frac{\slashed{a}}{a^2}
				\hsp \qty(\eta^{\mu\nu}\eta^{\rho\sigma} + \eta^{\mu\sigma}\eta^{\nu\rho} - \eta^{\mu\rho}\eta^{\nu\sigma})
				\gamma_\mu \slashed{a} \gamma^\mu = -2 \slashed{a}
			\end{equation}

			\noindent
			Trace Identities:
			\begin{align}
				\tr(\gamma^\mu) &= 0 \\
				\tr\overbrace{({\gamma^\mu\dots\gamma^\omega})}^{\text{odd \#}} &= 0 \\
				\tr(\gamma^\mu \gamma^\nu) &= 4 \eta^{\mu\nu} \\
				\tr(\gamma^\mu \gamma^\nu \gamma^\rho \gamma^\sigma) &= 4\qty(\eta^{\mu\nu}\eta^{\rho\sigma} - \eta^{\mu\rho}\eta^{\nu\sigma} + \eta^{\mu\sigma}\eta^{\nu\rho})
			\end{align}

			\noindent
			Weyl representation / Chiral representation\index{Weyl!Darstellung}:
			\begin{equation}
				\gamma^\mu = \mqty(
				0 & \sigma^\mu \\
				\bar{\sigma}^\mu & 0
				)
				\hsp
				\gamma^0 = \mqty(
				0 & \mathds{1}_2 \\
				\mathds{1}_2 & 0
				)
				\hsp
				\gamma^i = \mqty(
				0 & \sigma^i \\
				-\sigma^i & 0
				)
				\hsp
				\gamma^5 = \mqty(
				-\mathds{1}_2 & 0 \\
				0 & \mathds{1}_2
				)
			\end{equation}

			\noindent
			Dirac representation\index{Dirac!Darstellung}:
			\begin{equation}
				\begin{aligned}
					\gamma^0
					= \left( \begin{matrix}
							\sigma^0 & \M0       \\
							0        & -\sigma^0
						\end{matrix} \right)
					&\hsp
					\gamma^i = \mqty(
					0 & \sigma^i \\
					-\sigma^i & 0
					)
					&\hsp
					\gamma^5 = \mqty(
					0 & \mathds{1}_2 \\
					\mathds{1}_2 & 0
					)
				\end{aligned}
			\end{equation}

			\noindent
			Transformation to any other viable representation ($U$ unitary: $U U^\dagger=\mathds{1}_4$):
			\begin{equation}
				\tilde{\gamma} = U \gamma^\mu U^\dagger
			\end{equation}

			\noindent
			$\gamma^5$-Matrix (using the convention $\levicivita_{0123} = +1$):
			\begin{equation}
				\gamma^5 \defeq \i\gamma^0 \gamma^1 \gamma^2 \gamma^3
				= \frac{\i}{4!}\levicivita_{\mu\nu\rho\sigma} \gamma^\mu \gamma^\nu \gamma^\rho \gamma^\sigma
			\end{equation}

			\noindent
			Properties of the $\gamma^5$-matrix:
			\begin{align}
				\tr \gamma_5 &= 0 \\
				\gamma^5 &= \gamma_5 \\
				\gamma_5^\dagger &= \gamma_5 \\
				\gamma_5 \gamma_5 &= \mathds{1}_4 \\
				\acomm{\gamma^\mu}{\gamma^5} &= 0
			\end{align}

			\noindent
			$\sigma^{\mu\nu}$ matrices:
			\begin{equation}
				\sigma^{\mu\nu} = \frac{\i}{2}\comm{\gamma^\mu}{\gamma^\nu} = -\sigma^{\nu\mu}
			\end{equation}

			\noindent
			Generalized $\gamma$-matrices of $n$ Lorentz indices:
			\begin{equation}
				\gamma^{\mu\dots\nu} = \frac{1}{n!} \gamma^{[\mu} \dots \gamma^{\nu]}
				= \frac{1}{n!}\levicivita_{\mu\dots\nu\alpha\dots\beta} \gamma^\alpha \dots \gamma^\beta
			\end{equation}

		\subsubsection{Gell-Mann Matrices}
			Matrix representation of the Gell-Mann matrices\index{Gell-Mann!Matrizen} (conventional choice of linearly independent traceless Hermitian matrices spanning the Lie algebra of $SU(3)$):
			\begin{alignat}{2}
				\lambda_1 = \mqty(
				0 & 1 & 0 \\
				1 & 0 & 0 \\
				0 & 0 & 0
				) &\hsp
				\lambda_2 = \mqty(
				0 & -\i & 0 \\
				\i & 0 & 0 \\
				0 & 0 & 0
				) &&\hsp
				\lambda_3 = \mqty(
				1 & 0 & 0 \\
				0 & -1 & 0 \\
				0 & 0 & 0
				) \\
				\lambda_4 = \mqty(
				0 & 0 & 1 \\
				0 & 0 & 0 \\
				1 & 0 & 0
				) &\hsp
				\lambda_5 = \mqty(
				0 & 0 & -\i \\
				0 & 0 & 0 \\
				\i & 0 & 0
				) &&\\
				\lambda_6 = \mqty(
				0 & 0 & 0 \\
				0 & 0 & 1 \\
				0 & 1 & 0
				) &\hsp
				\lambda_7 = \mqty(
				0 & 0 & 0 \\
				0 & 0 & -\i \\
				0 & \i & 0
				) &&\hsp
				\lambda_8 = \frac{1}{\sqrt{3}}\mqty(
				1 & 0 & 0 \\
				0 & 1 & 0 \\
				0 & 0 & -2
				)
			\end{alignat}

			\noindent
			Properties:
			\begin{align}
				\tr(\lambda_a) &= 0 \\
				\tr(\lambda_a \lambda_b) &= 2\delta_{ab} \\
				\comm{\lambda_a}{\lambda_b} &= 2\i f_{abc} \lambda_c \\
				\lambda_a^\dagger &= \lambda_a \\
			\end{align}

	\subsection{Gauge and Symmetry}
		\subsubsection{Symmetry Operations}
			Relation of generators $\hat{G}$ to transformations $\hat{T}(\omega)$ acting on spacetime and resulting unitary Operators $\hat{U}(\omega)$ acting on states or operators:
			\begin{equation}
				\begin{aligned}
					\psi(\hat{T}(\delta \omega)x) &= \qty(1+ \i \delta\omega \hat{G}) \psi(x)\\
					\hat{G} &= \i \pdv{\hat{T}}{\omega} \\
					\hat{U} &= \lim_{N \to \infty} \qty(1+\i\hat{G}\delta\omega)^N = \exp(\i \hat{G} \cdot \omega) \\
					\Ket{\psi'} &= \hat{U}(\omega)\Ket{\psi} = \Ket{\hat{T}(\omega)\psi} \\
					\hat{V}'(x) &= \hat{T}(\omega) \hat{V}(\inv{\hat{T}}(\omega)x) = \hat{U}^\dagger(\omega) \hat{V}(x) \hat{U}(\omega) \\
				\end{aligned}
			\end{equation}

			\noindent
			Infinitesimal Lorentz Transformation (With parameters $\omega$ and generators $M$):
			\begin{equation}
				\begin{aligned}
					\Lambda^\mu_\nu &= \delta^\mu_\nu - \frac{\i}{2}\omega^\mu_\nu = \delta^\mu_\nu - \frac{\i}{2}\delta\omega_{\alpha\beta}M^{\alpha\beta\mu}{}_\nu \\
				\end{aligned}
			\end{equation}

			\noindent
			Generators of the Lorentz group (Angular momentum $J^i = \frac{1}{2}\levicivita^{ijk} M^{jk}$ and Boosts $K^i = M^{0i}$):
			\begin{equation}
				\begin{aligned}
					(M^{\alpha\beta})_{\mu\nu} &= \i\qty(\delta^{\alpha}_{\mu}\delta^{\beta}_{\nu} - \delta^{\beta}_{\mu}\delta^{\alpha}_\nu ) \\
					%&= \i \qty(x^\mu \partial^\nu - x^\nu \partial^\mu)				
				\end{aligned}
			\end{equation}

			\noindent
			Lie Algebra of the Lorentz group $SO(1,3)$:
			\begin{equation}
				\begin{aligned}
					\comm{M^{\alpha\beta}}{M^{\gamma\delta}} = \i \left(\eta^{\beta\gamma}M^{\alpha\delta}+\eta^{\alpha\delta}M^{\beta\gamma}-\eta^{\alpha\gamma}M^{\beta\delta}-\eta^{\beta\delta}M^{\alpha\gamma}\right)
				\end{aligned}
			\end{equation}

			\noindent
			Implications for boosts and rotations:
			\begin{equation}
				\begin{aligned}
					\comm{\hat{K}^i}{\hat{K}^j} &= -\i \levicivita_{ijk}\hat{J}^k \\
					\comm{\hat{J}^i}{\hat{K}^j} &= \i \levicivita_{ijk}\hat{K}^k \\
					\comm{\hat{J}^i}{\hat{J}^j} &= \i \levicivita_{ijk}\hat{J}^k \\
				\end{aligned}
			\end{equation}

			\noindent
			General Lorentz Transformation $\Lambda\in SO(3,1)$ (angles $\vec{\theta}$ and rapidities $\vec{\psi}$)
			\begin{equation}
				\label{Eq:GeneralLorentzTransformation}
				\Lambda(\vec{\theta}, \vec{\psi})
				= \exp(-\i(\hat{\vec{J}}\cdot\vec{\theta} - \hat{\vec{K}}\cdot\vec{\psi}))
				= \exp(\frac{1}{2}\i \theta^{\alpha\beta} M_{\alpha\beta})
			\end{equation}

		\subsubsection{Gauge Invariance}
			Gauge theory ($A_\mu$ is the gauge field, $q$ is the coupling strength):
			\begin{equation}
				\begin{aligned}
					\psi(x) &\to \ex^{\i q \alpha(x)}\psi(x) \\
					A_\mu(x) &\to A_\mu(x) - \frac{1}{q}\partial_\mu\alpha(x) \\
				\end{aligned}
			\end{equation}

			\noindent
			Covariant derivative:
			\begin{equation}
				\partial_\mu \to D_\mu = \qty(\partial_\mu - \i q A_\mu) \\
			\end{equation}

			\noindent
			The \emph{gauge principle} states, that the gauge field -- introduced to guarantee local gauge symmetry -- dictates the form of coupling.

			\noindent
			Application in Electromagnetism (Note that $F_{\mu\nu}$ is invariant under gauge transformation):
			\begin{equation}
				F_{\mu\nu} =  \frac{\i}{q} \comm{D_\mu}{D_\nu} = \partial_\mu A_\nu - \partial_\nu A_\mu
			\end{equation}


			\noindent
			Wilson Loop\index{Wilson!Loop} (Path ordered $\mathcal{P}$ exponential of the gauge field along a closed path $\Gamma$):
			\begin{equation}
				W(\Gamma) =\tr{\mathcal{P}\exp(\i\oint_\Gamma A_\mu \dd{x^\mu})}
			\end{equation}

		\subsubsection{Discrete Symmetry}
			Charge conjugation $\hat{\mathsf{C}}$ (For any quantum charge operator $\hat{Q}$, i.e. electric charge, lepton / baryon number, hypercharge,\dots; $\bar{p}$ is the antiparticle of $p$):
			\begin{equation}
				\begin{aligned}
					\hat{\mathsf{C}} \Ket{p} &= \Ket{\bar{p}}
					\phantom{Q}\hsp \hat{\mathsf{C}} \Ket{\bar{p}} = \Ket{p} \\
					\hat{\mathsf{C}}^2 &= \hat{1} \\
					\hat{Q} \Ket{p} &= Q \Ket{p}
					\hsp \hat{Q} \Ket{\bar{p}} = -Q\Ket{\bar{p}} \\
					\inv{\hat{\mathsf{C}}} \hat{Q} \hat{\mathsf{C}} &= -\hat{Q} \\
					\inv{\hat{\mathsf{C}}}\hat{a}_{\vec{p}} \hat{\mathsf{C}} &= \hat{b}_{\vec{p}}
					\hsp \inv{\hat{\mathsf{C}}}\hat{b}^\dagger_{\vec{p}} \hat{\mathsf{C}} = \hat{a}^\dagger_{\vec{p}} \\
				\end{aligned}
			\end{equation}
			in Chiral representation
			\begin{equation}
				\hat{C} = \i \gamma^2 \gamma^0
			\end{equation}

			\noindent
			Parity inversion $\hat{\mathsf{P}}$
			\begin{equation}
				\begin{aligned}
					\inv{\hat{\mathsf{P}}} \hat{x} \hat{\mathsf{P}} &= - \hat{x}
					\hsp \inv{\hat{\mathsf{P}}} \hat{p} \hat{\mathsf{P}} = - \hat{p} \\
					\hat{\mathsf{P}}^2 &= \hat{1} \\
					\inv{\hat{\mathsf{P}}}\hat{a}_{\vec{p}} \hat{\mathsf{P}} &= \hat{b}_{\vec{p}}
					\hsp \inv{\hat{\mathsf{P}}}\hat{b}^\dagger_{\vec{p}} \hat{\mathsf{P}} = \hat{a} ^\dagger_{\vec{p}} \\
				\end{aligned}
			\end{equation}
			in Chiral representation
			\begin{equation}
				\hat{C} = \gamma^0
			\end{equation}

			\noindent
			Time reversal $\hat{\mathsf{T}}$:
			\begin{equation}
				\begin{aligned}
					\hat{\mathsf{T}} \Ket{\psi(t)} &= \Ket{\psi(-t)} \\
					\hat{\mathsf{T}}^2 &= - \hat{1} \\
					\inv{\hat{\mathsf{T}}} \i \hat{\mathsf{T}} &= -\i \\
				\end{aligned}
			\end{equation}

			\noindent
			CPT-Theorem, valid for $\mathcal{L}$ Lorentz invariant, local, hermitian and normal ordered:
			\begin{equation}
				\comm{\hat{\mathsf{C}}\hat{\mathsf{P}}\hat{\mathsf{T}}}{\mathcal{H}} = 0
			\end{equation}


		\subsubsection{Conformal Symmetry}
			A map $x\to f(x)$ is a conformal symmetry if the Jacobian $J$ fulfills:
			\begin{equation}
				J^\mu{}_\nu(x) = \pdv{x^\nu} f^\mu(x) = \Omega(x) \Lambda^\mu{}_\nu(x)
			\end{equation}
			where $\Lambda$ is a general Lorentz transformation (\ie combination of boost and rotation).
			A special conformal transformation $S_\mu$ can be created from any translation $T_a(x) = x+a$ as
			\begin{equation}
				(S_a)_{\mu} = I\circ T_a \circ I \hsp\text{with}\quad I(x^\mu) = \frac{x^\mu}{x^2}.
			\end{equation}
			The conformal group is locally isomorphic to $SO(2,d)$ in a Minkowskian spacetime which includes translations, Lorentz transformations and dilations $x^\mu \to \omega x^\mu$. The Lie algebra can be deconstructed into translations $P$, dilations $D$, special conformal transformations $S_\mu$ and general Lorentz transformations $M_{\mu\nu}$ which satisfy the commutation relations
			\begin{alignat}{3}
				\comm{P_\mu}{P_\nu} &= 0,\hsp&&\comm{S_\mu}{S_\nu} &&= 0, \\
				\comm{D}{S_\mu} &= - S_\mu,\hsp&& \comm{D}{P_\mu} &&= P_\mu, \\
				\comm{S_\mu}{P_\nu} &= \eta_{\mu\nu} D - \i M_{\mu\nu}
			\end{alignat}


	\subsection{Operators}
		\subsubsection{Operator Ordering}
			Time ordering symbol (``$+$'' for Bosons, ``$-$'' for Fermions). Linear operator, that orders all time dependent operators from right (earliest) to left (latest):
			\begin{equation}
				\TimeOrder{\hat{A}(t)\hat{B}(t')}
				= \hat{A}(t) \hat{B}(t') \Theta(t-t') \pm \hat{B}(t') \hat{A}(t) \Theta(t'-t)
				= \begin{cases}
					\phantom{\pm} \hat{A}(t) \hat{B}(t') & t>t' \\
					\pm \hat{B}(t') \hat{A}(t) & t<t'
				\end{cases}
			\end{equation}

			\noindent
			Normal order symbol (Alternative notation: $\NormalOrder{\hat{A},\hat{B}} = \;\normord{\hat{A} \hat{B}}$). Linear operator, that orders all creation operators to the left and all annihilation operators to the right -- considered normal order (``$+$'' for Bosons, ``$-$'' for Fermions):
			\begin{equation}
				\begin{aligned}
					\NormalOrder{\hat{a}(\vec{p})\hat{a}^\dagger(\vec{q})}
					&= \pm \NormalOrder{\hat{a}^\dagger(\vec{q})\hat{a}(\vec{p})} = \pm \hat{a}^\dagger(\vec{p}) \hat{a}(\vec{q}) \\
				\end{aligned}
			\end{equation}

			\noindent
			Wick\index{Wick!Kontraktion} contraction definition:
			\begin{equation}
				\wick{\c{\hat{A}} \c{\hat{B}}} = \hat{A}\hat{B} - \NormalOrder{\hat{A}\hat{B}}
			\end{equation}

			\noindent
			Wick's Theorem\index{Wick!Theorem} (where all contractions means all possible simple, double and higher order contractions, until all operators are contracted):
			\begin{equation}
				\TimeOrder{\phi(x_1)\dotsb\phi(x_n)} = \NormalOrder{\phi(x_1)\dotsb\phi(x_n) + \sum(\text{all contractions})}
			\end{equation}

		\subsubsection{Multiparticle Operators}
			Many Body generalization of a single particle Operator $\hat{A}$:
			\begin{equation}
				\hat{A}' = \sum_{\alpha,\beta} \mel{\alpha}{\hat{A}}{\beta} \hat{a}_\alpha^\dagger \hat{a}_\beta
			\end{equation}

		\subsubsection{Noether Operator}
			\noindent
			Conserved operator by Noether's Theorem (Where $D$ is the symmetry operation and $\hat{Q}$ is the conserved operator, \ie eigenvalues are conserved charges)\index{Noether!Theorem}
			\begin{equation}
				\comm{\hat{Q}}{\hat{\phi}} = -\i D\hat{\phi}
			\end{equation}

	\subsection{Pertubation Theory}
		Dyson equation / Dyson expansion\index{Dyson!Gleichung} ($t > t_0$):
		\begin{equation}
			U(t,t_0) = T \qty[ \exp(-\i \int_{t_0}^\trp \dd t' \hat{H}_{\mathrm{int}}(t')) ]
		\end{equation}

		\noindent
		Propagator for a $n$-point correlation function (Where $\Ket{\Omega}$ is the vacuum state of the full theory\index{Vakuumzustand}, $\Ket{0}$ is the vacuum state of the free theory, $\hat{\phi}_H$ is the field operator in the Heisenberg picture and $\hat{\phi}_I$ is the field operator in the interaction picture):
		\begin{equation}
			\Bra{\Omega} T \hat{\phi}_H(x_1) \dotsb \hat{\phi}_H(x_n) \Ket{\Omega}
			= \frac{ \mel**{0}{T \hat{\phi}_I(x_1) \dotsb \hat{\phi}_I(x_n) \exp{-\i \displaystyle\int \dd t \, H_{\mathrm{int}}(t)}}{0}}
			{\mel**{0}{T\exp{-\i \displaystyle\int \dd t \, H_{\mathrm{int}}(t)}}{0}}
		\end{equation}

		\noindent
		$S$-Matrix\index{S-Matrix} (Where $\hat{U}$ is the time evolution operator, and $\hat{T}$ is the scattering term / Transition matrix and $\mathcal{M}$ is the invariant amplitude):
		\begin{equation}
			\begin{aligned}
				\hat{S} &= \lim_{\substack{t \to\infty\\t_0 \to-\infty}} \hat{U}(t,t_0) = \hat{1} + \i \hat{T} \\
				\mel**{p_{1}p_{2}}{\i \hat{T}}{k_{1}k_{2}} &= \qty(2\pi)^4\delta(p_1 + p_2 - k_1 - k_2) \i \mathcal{M} \\
			\end{aligned}
		\end{equation}

		\noindent
		Cross section\index{Wirkungsquerschnitt} of a collider experiment with beams $A$ and $B$ and $n$ particles in the product state, where $\dd \Pi_\mathrm{LIPS}$ is the Lorentz invariant phase space element:
		\begin{equation}
			\begin{aligned}
				\dd \sigma
				&= \frac{\abs{\mathcal{M}(p_A,p_B \to\set{p_f})}^2}{4p_A^0 p_B^0 \abs{\vec{v}_A - \vec{v}_B}} \dd \Pi_\mathrm{LIPS} \\
				&= \frac{\abs{\mathcal{M}(p_A,p_B \to\set{p_f})}^2}{4p_A^0 p_B^0 \abs{\vec{v}_A - \vec{v}_B}}  \qty(2\pi)^4 \delta^4 \qty(\sum p_{f} - p_A - p_B) \qty(\prod_{i=1}^{n} \frac{\dd[3] \vec{p}_i}{(2 \pi)^3} \frac{1}{2 p_i^0})
			\end{aligned}
		\end{equation}

		\noindent
		One-particle decay rate of a particle $A$:
		\begin{equation}
			\dd \Gamma = \frac{\abs{\mathcal{M}(p_A \to\set{p_f})}^2}{2 p_A^0} \dd \Pi_\mathrm{LIPS}
		\end{equation}

		\noindent
		Lehmann--Symanzik--Zimmermann reduction formula\index{Lehmann!LSZ reduction formula}\index{Symanzik!LSZ reduction formula}\index{Zimmermann!LSZ reduction formula}\index{LSZ Reduktionsformel} (LSZ reduction formula):
		\begin{equation}
			\begin{aligned}
				\Braket{p_k \hdots p_n|S|p_1 \hdots p_{k-1}} = &\prod_{i=1}^{k-1} \qty[\i \int \dd[4]{x_i} \ex^{-\i p_i\cdot x_i} \qty(\partial^2_i+m^2)] \prod_{i=k}^{n} \qty[\i \int \dd[4]{x_i} \ex^{+\i p_i\cdot x_i} \qty(\partial^2_i+m^2)] \\
				&\times\Braket{\Omega|T \phi(x_1) \hdots \phi(x_n)|\Omega}
			\end{aligned}
		\end{equation}

		\noindent
		Gell-Mann and Low Theorem\index{Gell-Mann! und Low Theorem}:
		\begin{equation}
			\begin{aligned}
				&\hat{H} \defeq \hat{H}_0 + \ex^{-\epsilon \abs{t}} \hat{H}' \\
				&\hat{H}_0 \Ket{\phi_0} = E_0 + \Ket{\phi_0} \\
				&\text{if}\quad \lim_{\epsilon\to 0} \frac{\hat{U}_\epsilon(0,-\infty)\Ket{\phi_0}}{\Bra{\phi_0} \hat{U}_\epsilon(0,-\infty)\Ket{\phi_0}}
				= \frac{\Ket{\psi_0}}{\Braket{\phi_0|\psi_0}} \quad\text{exists}\quad
				\to&\quad \hat{H} \frac{\Ket{\psi_0}}{\Braket{\phi_0|\psi_0}} = E \frac{\Ket{\psi_0}}{\Braket{\phi_0|\psi_0}} \\
			\end{aligned}
		\end{equation}

		\subsubsection{Green's Functions}
			Single particle's full Green's function / Propagator \index{Green!Funktion} ($x=(t,\vec{x})$; $\Ket{\Psi_0}$ is the Heisenberg ground state)
			\begin{equation}
				\begin{aligned}
					\i G_{\alpha\beta}(x, x') &= \frac{\Bra{\Psi_0} T\qty[\hat{\Psi}_{H\alpha}(x)\hat{\Psi}^\dagger_{H\beta}(x')] \Ket{\Psi_0}}{\Braket{\Psi_0|\Psi_0}} \\
				\end{aligned}
			\end{equation}


	\subsection{Path Integral Formulation\index{Pfadintegral}}
		Path Integral in phase space form(between initial state $q_i$ and final state $q_f$):
		\begin{equation}
			\mel**{q_f}{U(t_f, t_i)}{q_i} = \int_{q_i}^{q_f} \mathcal{D}q \mathcal{D}p \, \exp(\i \int_{t_i}^{t_f} \dd t \, \qty[ \dot{q}(t)p(t) - H\qty(p(t),q(t))])
		\end{equation}

		\noindent
		Path Integral for separable Hamiltonian (\ie $\hat{H} = \frac{\hat{p}^2}{2m} + V(\hat{q})$)
		\begin{equation}
			\mel**{q_f}{U(t_f, t_i)}{q_i} = \frac{1}{N} \int \mathcal{D}q \, \ex^{\i S[q]}
		\end{equation}

		\noindent
		Path Integral measure:
		\begin{equation}
			\mathcal{D}q = \eval{\prod_{j=0}^{n-1} \dd q(t_j)}_{\substack{q(t_0)=q_i\\q(t_n)=q_f}}
			\hsp
			\mathcal{D}p = \prod_{j=0}^{n-1} \dd p(t_j)
		\end{equation}

		\noindent
		$n$-point correlation function:
		\begin{equation}
			\mel**{\Omega}{T\hat{q}(t_1)\dotsb\hat{q}(t_n)}{\Omega} = \frac{\int\mathcal{D}q\,q(t_1)\dotsb q(t_n)\ex^{\i S \qty[q]}}{\int\mathcal{D}q\,\ex^{\i S\qty[q]}}
		\end{equation}

		\subsubsection{Generating Functionals}
			Generating Functional / partition function:
			\begin{equation}
				Z[J] = \frac{1}{N} \int \mathcal{D}\varphi\, \exp(\i\qty(S[\varphi] + \int \dd^d x \, J(x) \varphi(x)))
			\end{equation}

			\noindent
			Correlation Functions:
			\begin{equation}
				\mel**{\Omega}{T \hat{\varphi}(x_1) \dotsb \hat{\varphi}(x_n)}{\Omega} = \eval{(-\i)^{n} \frac{\delta^n Z[J]}{\delta J(x_1) \dotsb \delta J(x_n)}}_{J=0}
			\end{equation}

			\noindent
			Free Generating Functional:
			\begin{equation}
				Z_0\qty[J] = \exp(-\frac{1}{2}\int\dd^d x \dd^d x\, J(x) G(x,y)J(y))
			\end{equation}

			\noindent
			Schwinger\index{Schwinger!Funktional} functional (Generates connected correlation functions):
			\begin{equation}
				W\qty[J] = \ln Z\qty[J]
			\end{equation}

			\noindent
			Propagator:
			\begin{equation}
				\frac{\delta^2 W[J]}{\delta J(x) \delta J(y)} = G(x,y) = \qty(\frac{\delta^2\Gamma[\phi]}{\delta \phi(x) \delta \phi(y)})^{-1}
			\end{equation}

			\noindent
			Effective Action\index{Effektive!Wirkung} (\ie the Legendre Transformation\index{Legendre!Transformation} of the Schwinger Functional\index{Schwinger!Funktional}; generates 1 particle irreducible (1PI) correlation functions; $\phi(x) = \Avg{\varphi(x)}_J = \fdv{W\qty[J]}{J(x)}$):
			\begin{equation}
				\Gamma(\phi) = \sup_{J} \qty( \int \dd^d x \, J(x) \phi(x) - W[J])
			\end{equation}

			\noindent
			Quantum equations of motion:
			\begin{equation}
				J(x)=\fdv{\Gamma\qty[\phi]}{\phi(x)}
			\end{equation}

			\noindent
			Expansion of the $n$-point correlation function:
			\begin{equation}
				\Avg{\varphi(x_1) \hdots\varphi(x_n)}_J = \prod_{i=1}^{n} \qty(\int \dd^d x'_i \;G(x_i,x'_i) \fdv{\phi(x'_i)} + \phi(x_i))
			\end{equation}

			\noindent
			Functional Dyson--Schwinger Equation\index{Dyson--Schwinger Gleichung!funktional}:
			\begin{equation}
				\fdv{\Gamma[\phi]}{\phi(x)} = \fdv{S}{\varphi(x)}\qty[\varphi = \phi + \int \dd^d y\; G(x,y) \fdv{}{\phi(y)}]
			\end{equation}

			\noindent
			General Dyson--Schwinger Equation:
			\begin{equation}
				\Avg{\Psi[\varphi]} \fdv{\Gamma[\phi]}{\phi(x)} = \Avg{\Psi[\varphi] \fdv{S[\varphi]}{\varphi(x)}} - \Avg{\fdv{\Psi[\varphi]}{\varphi(x)}}
			\end{equation}

			\noindent
			Ward--Takahashi identity\index{Ward!--Takahashi Identität}\index{Takahashi!Ward--Takahashi Identität} for a gauge invariance of the effective action $\Gamma[\phi]$ with generator $G$ and the corresponding gauge fixing and ghost actions:
			\begin{equation}
				0 = G \Gamma[\phi] - \Avg{G(S_\text{g.f.} + S_\text{gh.})}_{J[\phi]}
			\end{equation}

	\subsection{Functional Renormalization}
		\subsubsection{General Concepts}
			Infrared regularized generating functional with a hard infrared cutoff scale $k$:
			\begin{equation}
				Z_k[J] = \int \mathcal{D}\varphi_{p^2 \ge k^2} \exp(-S[\varphi] - \int \dd^d x \, J(x) \varphi(x))
				\hsp \mathcal{D}\varphi_{p^2 \ge k^2} = \prod_{p^2 \ge k^2} \frac{\dd \varphi_p}{(2\pi)^d}
			\end{equation}

			\noindent
			Soft cutoff integration measure $\mathcal{D}\varphi_{p^2 \ge k^2} \defeq \mathcal{D}\varphi \exp(-\Delta S_k[\varphi])$:
			\begin{equation}
				\Delta S_k[\varphi]
				\defeq \frac{1}{2} \int \frac{\dd^d p}{(2\pi)^d}\, \varphi(-p) R_k(p^2) \varphi(p)
				= \frac{1}{2} \int \dd[d]{x} \varphi(x) R_k(-\partial^2) \varphi(x)
			\end{equation}
			with a Regulator function\index{Regulator Funktion} $R_k(p^2)$ that satisfies
			\begin{itemize}\itemsep -0pt
				\item $\displaystyle\lim_{p^2 \to 0} R_k(p^2) > 0$ \hfill{(Suppression of IR modes)}
				\item $\displaystyle\lim_{k \to 0} R_k(p^2) = 0$ \hfill{(Physical limit)}
				\item $\displaystyle\lim_{k \to \Lambda \to \infty} R_k(p^2) = \infty$ \hfill{(UV limit)}
			\end{itemize}
			hence, the generating functional with a soft cutoff is
			\begin{equation}
				Z_k[J] \defeq \ex^{-\Delta S_k [\fdv{J}]}Z[J]
			\end{equation}
			Note that there are complications when dealing with fermionic regulators

			\noindent
			Example regulators ($R_k(p^2) = p^2 r(y)$ with $y=p^2/k^2$):
			\begin{itemize} \itemsep -0pt
				\item Litim regulator\index{Litim!Regulator}: $r(y) = \qty(\dfrac{1}{y}-1)\theta(1-y)$, \ie $R_k(p^2) = (k^2 - p^2) \Theta(k^2 - p^2)$
				\item Exponential regulator: $r(y) = \dfrac{cy^{b-1}}{\exp(y^b) - 1}$
				\item Sharp regulator $r(y) = \dfrac{1}{\Theta(y-1)}-1$
			\end{itemize}

			\noindent
			Renormalization group time / RG time and beta functions (using a UV cutoff $\Lambda$):
			\begin{equation}
				t = \ln \frac{k}{\Lambda}
				\hsp \beta_g \defeq \dot{g} = \partial_t g = k \partial_k g
			\end{equation}

			\noindent
			Anomalous dimension (where $Z_\psi(k)$ is the wave function renormalization):
			\begin{equation}
				\eta_\psi = -\partial_t \ln Z_\psi
			\end{equation}

			\noindent
			Flow equation for the Schwinger Functional\index{Schwinger!Funktional}:
			\begin{equation}
				\partial_t W_k[J] = -\frac{1}{2}\int \frac{\dd^d p}{(2\pi)^d} \, \qty[W^{(2)}_k[J]+(W_k^{(1)}[J])^2](p,-p) \partial_t R_k(p^2)
			\end{equation}

			\noindent
			Flowing / Scale dependent effective action / Effective average action and modified quantum equations of motion / effective field equations ($\phi(x)= \fdv{W_k[J_k]}{J_k(x)}$):
			\begin{equation}
				\Gamma_k[\phi] = \sup_{J_k} \qty{\int \dd^d x\, J_k(x) \phi(x) - W_k[J_k] - \Delta S_k[\phi]}
				\hsp \fdv{(\Gamma_k[\phi] + \Delta S_k[\phi])}{\phi(x)} = J_k(x)
			\end{equation}

			\noindent
			The flowing effective action interpolates between the classical action $S = \Gamma_\Lambda$ where quantum fluctuations are fully suppressed and the full quantum effective action $\Gamma$ where all fluctuations are included:
			\begin{equation}
				\Gamma \;\stackrel{k \to 0}{\longleftarrow}\; \Gamma_{k} \stackrel{k \to \infty}{\longrightarrow} S
			\end{equation}

			\noindent
			Wetterich Equation\index{Wetterich!Gleichung} / flow equation for the effective action / functional renormalization group equation (Note the inclusion of a minus sign in tracing out fermionic fields, and the change of sign for the momentum of transposed field derivatives, the multi-Index $I$ denotes the combination of all occurring internal indices of the superfield $\vec{\Phi}$ and $\Str$ is the supertrace):
			\begin{equation}
				\begin{aligned}
					\partial_t \Gamma_k[\vec{\Phi}]
					&= \frac{1}{2} \Str G_k \;\partial_t R_k \\
					&= \frac{1}{2} \Str \frac{1}{\Gamma^{(2)}_k + R_k} \partial_t R_k \\
					&= \frac{1}{2} \Str \int \frac{\dd^d p}{(2\pi)^d} \, \frac{1}{\Gamma^{(2)}_k[\vec{\Phi}] + R_k}(p,-p) \partial_t R_k(p^2) \\
					&= \frac{1}{2} \sum_I \pm \int \frac{\dd^d p}{(2\pi)^d} \, \inv{\qty(\frac{\delta^2 \Gamma_k[\vec{\Phi}]}{\delta \Phi^I(-p) \delta \Phi_I(p)} + R_k(p^2))} \cdot \partial_t R_k(p^2) \\
					&= \frac{1}{2} \Str \eval{\partial_{\tilde{t}} \ln(\Gamma^{(2)}_k[\vec{\Phi}] + R_{\tilde{k}})}_{\tilde{k} = k}
				\end{aligned}
			\end{equation}

		\subsubsection{Asymptotic Safety}
			\noindent
			Definition of a fixed point $\vec{g}^{*}$:
			\begin{equation}
				\beta_{g_i}(\vec{g}^{*}) = 0 \quad \forall i
			\end{equation}

			\noindent
			Stability Matrix:
			\begin{equation}
				\mathcal{M}_{ij} = \eval{\pdv{\beta_{g_i}}{g_j}}_{\vec{g} = \vec{g_*}}
			\end{equation}
			Critical Exponent (they correspond to eigenvectors $\vec{V}_I$ of the stability matrix)
			\begin{equation}
				\theta_I = - \mathrm{eig}(\mathcal{M})
				\hsp \theta_I \vec{V}_I = \mathcal{M} \vec{V}_I
			\end{equation}
			Fixed point expansion of the beta functions:
			\begin{equation}
				\beta_{g_i} = \sum_{j} \mathcal{M}_{ij} \qty(g_j - g^{*}_{j}) + \mathcal{O}\qty(g_j - g^{*}_{j})^2
			\end{equation}
			Solution (where $c_I$ are integration constants and $k_0$ is a reference scale):
			\begin{equation}
				\vec{g}(k) = \vec{g}^{*} + \sum_I c_I \vec{V}_I \qty(\frac{k}{k_0})^{-\theta_I}
			\end{equation}
			\begin{itemize}
				\item $\Re(\theta_I) > 0$: relevant direction $\vec{V}_I$ (IR repulsive, UV attractive), contributing a free parameter $c_I$
				\item $\Re(\theta_I) < 0$: irrelevant direction $\vec{V}_I$ (IR attractive, UV repulsive), no free parameter $c_I$ insignificant
				\item $\Re(\theta_I) = 0$: marginal, look at higher order contributions
			\end{itemize}

	\subsection{Bosons}
		\subsubsection{Bosonic Ladder Operators}
			\label{Sec:BosonicCreationAndAnnihilationOperators}
			Definition
			\begin{equation}
				\comm{\hat{b}_k}{\hat{b}_{k'}^\dagger} = \delta_{kk'}
				\hsp
				\comm{\hat{b}_k}{\hat{b}_{k'}} = \comm{\hat{b}_k^\dagger}{\hat{b}_{k'}^\dagger} = 0
			\end{equation}

			\noindent
			For the basis of occupation numbers $\Ket{n_1, n_2, ..., n_\infty}$ with $\forall i:\;n_i\in \\naturals _0$ the operators have the following qualities:
			\begin{equation}
				\begin{aligned}
					\hat{b}_k \Ket{...,n_k,...} &= \sqrt{n_k} \Ket{...,n_k-1,...} \\
					\hat{b}_k^\dagger \Ket{...,n_k,...} &= \sqrt{n_k+1} \Ket{...,n_k+1,...} \\
				\end{aligned}
			\end{equation}
			\begin{equation}
				\hat{b}_k^\dagger \hat{b}_k \Ket{...,n_k,...} = n_k \Ket{...,n_k,...}
			\end{equation}

			\noindent
			Scalar bosonic field and conjugate momentum operators in terms of creation and annihilation operators (with the on-shell frequency $\omega_{\vec{p}} = p^0 = \sqrt{\vec{p}^2 + m^2}$):
			\begin{equation}
				\begin{aligned}
					\hat{\phi}(x) &= \int \frac{\dd[3]{p}}{(2\pi)^3} \frac{1}{\sqrt{2 \omega_{\vec{p}}}} \qty( \hat{b}(\vec{p}) \ex^{-\i p_\mu x^\mu} + \hat{b}^\dagger(\vec{p}) \ex^{\i p_\mu x^\mu} ) = \hat{\phi}_{-}(\hat{b}(\vec{p}),x) + \hat{\phi}_{+}(\hat{b}^\dagger(\vec{p}),x) \\
					\hat{\pi}(x) &= -\i \int \frac{\dd[3]{p}}{(2\pi)^3} \sqrt{\frac{\omega_{\vec{p}}}{2}} \qty( \hat{b}(\vec{p}) \ex^{-\i p_\mu x^\mu} - \hat{b}^\dagger(\vec{p}) \ex^{\i p_\mu x^\mu} ) \\
				\end{aligned}
			\end{equation}

			\noindent
			Creation and Annihilation operators\index{Zerstörungsoperator}\index{Erzeugungsoperator} and their commutation relations for scalar bosonic fields:
			\begin{equation}
				\begin{aligned}
					\hat{b}(\vec{p}) &= \sqrt{\frac{\omega_{\vec{p}}}{2}} \hat{\tilde{\phi}}(\vec{p}) + \i \frac{1}{\sqrt{2 \omega_{\vec{p}}}} \hat{\tilde{\pi}}(\vec{p}) \\
					\hat{b}^\dagger(\vec{p}) &= \sqrt{\frac{\omega_{\vec{p}}}{2}} \hat{\tilde{\phi}}(-\vec{p}) - \i \frac{1}{\sqrt{2 \omega_{\vec{p}}}} \hat{\tilde{\pi}}(-\vec{p})
				\end{aligned}
				\hsp
				\begin{aligned}
					\comm{\hat{b}(\vec{p})}{\hat{b}^\dagger(\vec{q})} &= (2\pi)^3 \delta^{(3)}(\vec{p}-\vec{q}) \\
					\Big[ \hat{b}(\vec{p}) , \hat{b}(\vec{q}) \Big] &= \comm{\hat{b}^\dagger(\vec{p})}{\hat{b}^\dagger(\vec{q})} = 0 \\
				\end{aligned}
			\end{equation}

		\subsubsection{Free Klein--Gordon Equation\index{Klein!--Gordon Gleichung}\index{Gordon!Klein--Gordon Gleichung}}
			\noindent
			Klein--Gordon Equation (spin $s=0$)
			\begin{equation}
				\left(\partial^\mu\partial_\mu+\qty(\frac{mc}{\hbar})^2\right) \phi(x) = 0
			\end{equation}

			\noindent
			Solution (plane waves):
			\begin{equation}
				{\phi}^{ ( \pm ) }_{\vec{p}} (x) = \ex^{\mp \i p_\mu x^\mu}
			\end{equation}

			\noindent
			Free Feynman Propagator\index{Feynman!Propagator} for a scalar field (\ie Green's function of the Klein--Gordon equation):
			\begin{equation}
				\begin{aligned}
					D_F (x-y)
					&= \Bra{0} T \hat{\phi}(x)\hat{\phi}(y) \Ket{0} \\
					&= \comm{\hat{\phi}_{-}(x)}{\hat{\phi}_{+}(y)} \Theta(x^0-y^0) + \comm{\hat{\phi}_{-}(y)}{\hat{\phi}_{+}(x)} \Theta(y^0-x^0) \\
					&= \lim_{\epsilon \searrow 0} \int \frac{\dd^4 p}{(2\pi)^4} \frac{\i}{p^\mu p_\mu - m^2 + \i \epsilon} \ex^{-\i p_\mu (x-y)^\mu} \\
					&= \wick{\c{\hat{\phi}}(x) \c{\hat{\phi}}(y)} \\
				\end{aligned}
			\end{equation}

			\noindent
			$n$-point correlation function for a scalar field:
			\begin{equation}
				G^{(n)}(x_1, ..., x_n) = \Bra{\Omega} T \hat{\phi}(x_1) \dotsb \hat{\phi}(x_n) \Ket{\Omega}
			\end{equation}

	\subsection{Fermions}
		\subsubsection{Fermionic Creation an Annihilation Operators}
			\label{Sec:FermionicCreationAndAnnihilationOperators}
			Definition of fermionic creation an annihilation operators
			\begin{equation}
				\acomm{\hat{a}_k}{\hat{a}_{k'}^\dagger} = \delta_{kk'}
				\hsp
				%\acomm{\hat{a}_k}{\hat{a}_{k'}} = \acomm{\hat{a}_k^\dagger}{\hat{a}_{k'}^\dagger} = 0
				\Big\lbrace\hat{a}_k,\hat{a}_{k'}\Big\rbrace = \acomm{\hat{a}_k^\dagger}{\hat{a}_{k'}^\dagger} = 0
			\end{equation}
			In particular $\hat{a}_k^\dagger \hat{a}_k^\dagger = 0 = \hat{a}_k \hat{a}_k$, which ensures $n_k\in\qty{0,1}$, \ie the Pauli exclusion principle\index{Pauli!Ausschlussprinzip}.

			\noindent
			For the basis of occupation numbers $\Ket{n_1, n_2, ..., n_\infty}$ with $\forall i:\;n_i\in \qty{0,1}$ the operators have the following qualities (where the phase factor $S = n_1 + n_2 + ... + n_{k-1}$):
			\begin{equation}
				\begin{aligned}
					\hat{a_k}^\dagger \Ket{...,0,...} &= (-1)^S \Ket{...,1,...}
					&\hsp
					\hat{a_k} \Ket{...,1,...} &= (-1)^S \Ket{...,0,...}
					\\
					\hat{a_k}^\dagger \Ket{...,1,...} &= 0
					&\hsp
					\hat{a_k} \Ket{...,0,...} &= 0
				\end{aligned}
			\end{equation}
			\begin{equation}
				\hat{a_k}^\dagger \hat{a_k} \Ket{...,n_k,...} = n_k \Ket{...,n_k,...}
			\end{equation}

			\noindent
			Slater determinant\index{Slater!Determinante}:
			\begin{equation}
				\Ket{\psi(x_1,x_2,...,x_N)} = \frac{1}{\sqrt{N!}} \det\left( \begin{matrix}
						\Ket{\psi}_1(x_1) & \dotsb & \Ket{\psi}_1(x_N) \\
						\vdots            & \ddots & \vdots            \\
						\Ket{\psi}_N(x_1) & \dotsb & \Ket{\psi}_N(x_N) \\
					\end{matrix} \right)
			\end{equation}

		\subsubsection{Weyl Spinors\index{Weyl!Spinor}}
			\noindent
			Decomposition of the Dirac-Spinor $(\frac{1}{2},0)\oplus(0,\frac{1}{2})$ Lorentz transformation $\Lambda\in SO(3,1)$ into the left-handed $(\frac{1}{2},0)$ representation $\Lambda_L$ and right-handed $(0,\frac{1}{2})$ representation $\Lambda_R$ (using \ref{Eq:GeneralLorentzTransformation} with the generators $\hat{\vec{J}} = \vec{\sigma}/2$ and $\hat{\vec{K}} = \pm \i\vec{\sigma}/2$):
			\begin{equation}
				\Lambda
				= \mqty(\Lambda_\text{L} & 0 \\ 0 & \Lambda_\text{R})
				= \mqty(
				\exp(\frac{\vec{\sigma}}{2}\cdot(-\i\vec{\theta} - \vec{\psi})) & 0 \\
				0 & \exp(\frac{\vec{\sigma}}{2}\cdot(-\i\vec{\theta} + \vec{\psi}))
				)
			\end{equation}
			\begin{equation}
				\Lambda_\text{L}^{*} = \inv{\Lambda_\text{R}} \hsp \Lambda_\text{R}^{*} = \inv{\Lambda_\text{L}}
			\end{equation}

			\noindent
			Pauli 4-Vector\index{Pauli!4-Vektor}
			\begin{equation}
				\begin{aligned}
					\tilde{x} &= \sigma_\mu x^\mu \doteq \mqty(
					-x^0 + x^3 & \phantom{-}x^1 - \i x^2 \\
					\phantom{-}x^1 + \i x^2 & -x^0 - x^3 \\
					) \hsp \\
					x^\mu x_\mu &= \det(\tilde{x}) = x^2 \\
				\end{aligned}
			\end{equation}

			\noindent
			Weyl Spinor\index{Weyl!Spinor} transformation rules
			\begin{equation}
				\psi_\text{L}{}_{\alpha} \to \Lambda_\text{L}{}_{\alpha}{}^\beta \psi_\text{L}{}_{\beta} \Lambda_\text{L}^\dagger \hsp
				\psi_\text{R}{}_{\dot{\alpha}} \to \Lambda_\text{R}{}_{\dot{\alpha}}{}^{\dot{\beta}} \psi_\text{R}{}_{\dot{\beta}} \Lambda_\text{R}^\dagger
			\end{equation}
			\begin{equation}
				\psi_{\alpha} \to M_{\alpha}{}^\beta \psi_\beta
				\psi_\text{R}{}_{\dot{\alpha}} \to \Lambda_\text{R}{}_{\dot{\alpha}}{}^{\dot{\beta}} \psi_\text{R}{}_{\dot{\beta}} \Lambda_\text{R}^\dagger
			\end{equation}

			\noindent
			Weyl Equations\index{Weyl!Gleichungen} (Consequence of the massless Dirac equation, see Eq.~\ref{Eq:DiracInTermsOfWeylSpinors}):
			\begin{equation}
				\begin{aligned}
					\i \bar{\sigma}^\mu \partial_\mu \psi_\text{L} &= 0 \\
					\i \sigma^\mu \partial_\mu \psi_\text{R} &= 0 \\
				\end{aligned}
			\end{equation}

		\subsubsection{Two-component Spinor Formalism}
			Spinor representation of the Lorentz transformation:
			\begin{equation}
				M=M_{\frac{1}{2},0} = \exp(-\frac{\i}{2} \vec{\theta}\cdot\vec{\sigma} - \frac{1}{2} \vec{\psi}\cdot\vec{\sigma})
			\end{equation}

			\noindent
			Weyl Spinors:
			\begin{itemize}
				\item A left-handed two-component $(\frac{1}{2},0)$ spinor is denoted $\psi_{\alpha}$ and transforms as $\psi_{\alpha} \to (M)_{\alpha}{}^{\beta} \psi_{\beta}$ under a Lorentz transformation $M=M_{(\frac{1}{2},0)}$.
				\item A right-handed two-component $(0,\frac{1}{2})$ spinor is denoted $\psi_{\dot{\alpha}}^\dagger$ and transforms as $\psi_{\dot{\alpha}}^{*} \to \psi_{\dot{\beta}}^{*} (M^{*}){}^{\dot{\beta}}{}_{\dot{\alpha}}$ under a Lorentz transformation $M^{*}=M^\dagger =M_{(0,\frac{1}{2})}$.
			\end{itemize}
			where $\psi_\alpha$ and $\psi^{{*}\dot{\alpha}}$ are column vectors, $\psi^\alpha$ and $\psi^{*}_{\dot{\alpha}}$ are row vectors, and $\alpha,\dot{\alpha}\in\set{1,2}$.
			\begin{equation}
				\psi^{*}_{\dot{\alpha}} = (\psi^{*})_{\dot{\alpha}} = (\psi_\alpha)^{*}
				\hsp
				(\psi^{*})^{\dot{\alpha}} = (\psi^\alpha)^{*}
			\end{equation}
			the spinor metric tensor is the totally antisymmetric two-component Levi-Civita tensor (equally for dotted as for undotted indices):
			\begin{equation}
				\levicivita^{12} = - \levicivita^{21} = \levicivita_{21} = - \levicivita_{12} = 1
			\end{equation}

			\noindent
			Dirac fermion\index{Dirac!Fermion} $\Psi$ is described by a pair of Weyl fermions $\psi_{\alpha}, \psi^{*c}_{\dot{\alpha}}$.
			\begin{alignat}{3}
				\Psi &\doteq \mqty(\psi_{\alpha} \\ \psi^{*c}_{\dot{\alpha}})
				\hsp &&
				\bar{\Psi} &&= \Psi^\dagger\gamma^0 \doteq \mqty(\psi^c_{\alpha} & \psi^{*}_{\dot{\alpha}})
			\end{alignat}
			these can be singled out using the projection operators
			\begin{alignat}{3}
				\psi_L &= \hat{P}_L \Psi \doteq \mqty(\psi \\ 0)
				\hsp &&
				\psi_R^{*} &&= \hat{P}_R \Psi \doteq \mqty(0 \\ \psi^{*c})
			\end{alignat}

			\noindent
			Contractions can be formed with $\sigma^\mu=\sigma^\mu_{\alpha\dot{\alpha}}$ and $\bar{\sigma}_\mu=\bar{\sigma}_\mu^{\dot{\alpha}\alpha} = \levicivita^{\alpha\beta}\levicivita^{\dot{\alpha}\dot{\beta}}\sigma_{\mu\,\beta\dot{\beta}}$:
			\begin{equation}
				\psi^{*} \bar{\sigma}^\mu \partial_\mu \psi = \psi^{*}_{\dot{\alpha}} (\bar{\sigma}^\mu)^{\dot{\alpha}\alpha}\partial_\mu \psi_{\alpha}
				\hsp \psi^c \sigma^\mu \partial_\mu \psi^{*c} = \psi^{c\,\alpha} (\sigma^\mu)_{\alpha\dot{\alpha}}\partial_\mu \psi^{*c\,\dot{\alpha}}
			\end{equation}

			\noindent
			Weak isospinor\index{Weak!Isospinor} doublet:
			\begin{equation}
				q_{\alpha}^i \doteq \mqty(u_{\alpha} \\ d_{\alpha}) = \mqty(u^L_{\alpha} \\ d^L_{\alpha})
			\end{equation}




		\subsubsection{Dirac Spinors\index{Dirac!Spinor}}
			Adjoint spinor $\bar{\psi}$ of a spinor $\psi$:
			\begin{equation}
				\bar{\psi} \defeq \psi^{\dagger} \gamma^0
			\end{equation}

			\noindent
			Dirac spinor\index{Dirac!Spinor} composition of Weyl spinors\index{Weyl!Spinor} (Chiral representation):
			\begin{equation}
				\psi = \mqty(\psi_\text{L} \\ \psi_\text{R})
				\hsp
				\psi^\dagger = \mqty(\bar{\psi}_\text{L} & \bar{\psi}_\text{R})
				\hsp
				\bar{\psi} = \mqty(\bar{\psi}_\text{R} & \bar{\psi}_\text{L})
			\end{equation}

			\noindent
			Weyl spinors in terms of Dirac spinors (Eq.~\ref{Eq:SpinorProjectionOperators}):
			\begin{equation}
				\psi_L = \hat{P}_L \psi \hsp \psi_R = \hat{P}_R \psi
				\hsp
				\bar{\psi}_L = \bar{\psi} \hat{P}_R \hsp \bar{\psi}_R = \bar{\psi} \hat{P}_L
			\end{equation}

			\noindent
			Dirac Equation\index{Dirac!Equation} (Spin $s=1/2$ particles. The Dirac Equation implies the Klein--Gordon Equation\index{Klein!--Gordon Gleichung}\index{Gordon!Klein--Gordon Gleichung}):
			\begin{equation}
				\qty(\i\slashed{\partial} - m ) \psi\qty(x) = 0
			\end{equation}
			using $\slashed{\hat{p}} = \gamma^\mu(\i\partial_\mu)$.
			\begin{equation}
				\qty(\slashed{\hat{p}} - m) \psi\qty(x) = 0
			\end{equation}

			\noindent
			Adjoint Dirac Equation:
			\begin{equation}
				-\i \partial_\mu \bar{\psi}(x)\gamma^\mu-\bar\psi(x)m = 0
			\end{equation}

			\noindent
			Dirac Equation in Terms of Weyl Spinors:
			\begin{equation}
				\label{Eq:DiracInTermsOfWeylSpinors}
				\mqty( -m & \i (\partial_0 + \vec{\sigma}\cdot\partial_{\vec{x}}) \\ \i (\partial_0 - \vec{\sigma}\cdot\partial_{\vec{x}}) & -m ) \mqty( \psi_\text{L} \\ \psi_\text{R} ) = 0
			\end{equation}

			\noindent
			Solutions to the Dirac equation\index{Dirac!Gleichung}:
			\begin{align}
				\text{particles}
				& \hsp
				\psi(x) = u(p) \ex^{-\i p \cdot x}
				& \hsp
				u(p) &= \mqty(
				u_\text{L}(p) \\
				u_\text{R}(p)
				) = \mqty(
				\sqrt{p\cdot\sigma}\,\xi \\
				\sqrt{p\cdot\bar{\sigma}}\,\xi
				) \\
				\text{antiparticles}
				& \hsp
				\bar{\psi}(x) = v(p) \ex^{\i p \cdot x}
				& \hsp
				v(p) &= \mqty(
				v_\text{L}(p) \\
				v_\text{R}(p)
				) = \mqty(
				\M\sqrt{p\cdot\sigma}\,\chi \\
				-\sqrt{p\cdot\bar{\sigma}}\,\chi
				)
			\end{align}
			\begin{itemize} \itemsep -0pt
				\item particle with spin up: $\xi=\xi^1 = (1, 0)^\trp$
				\item particle with spin down: $\xi=\xi^2 = (0, 1)^\trp$
				\item antiparticle with spin up: $\chi=\chi^1 = (0, 1)^\trp$
				\item antiparticle with spin down: $\chi=\chi^2 = (1, 0)^\trp$
			\end{itemize}

			\noindent
			Spinor normalization
			\begin{align}
				\bar{u}^r(p) u^s(p) &= 2m \xi^{s\dagger} \xi^s = 2m \delta^{rs}
				&
				\bar{v}^r(p) v^s(p) &= - 2m\chi^{s\dagger}\chi^s = - 2m\delta^{rs} \\
				u^{r\dagger}(p) u^s(p) &= 2E(\vec{p}) \delta^{rs}
				&
				v^{r\dagger}(p) v^s(p) &= 2E(\vec{p}) \delta^{rs}
			\end{align}

			\noindent
			Spinor orthogonality
			\begin{align}
				\bar{u}^r(p) v^s(p) &= \bar{v}^r(p) u^s(p) = 0 \\
				u^{r \dagger}(\vec{p}) v^s(-\vec{p}) &= v^{r \dagger}(-\vec{p}) u^s(\vec{p}) = 0
			\end{align}

			\noindent
			Completeness relations:
			\begin{equation}
				\begin{aligned}
					\sum_s u^s(p) \bar{u}^s(p) &= \slashed{p} + m \\
					\sum_s v^s(p) \bar{v}^s(p) &= \slashed{p} - m
				\end{aligned}
			\end{equation}

			\noindent
			Closure relation of polarization vectors:
			\begin{equation}
				\levicivita^{\lambda}{}_{\mu}(k)\levicivita^{\lambda^{\prime}{}^{\mu}\ast}(k)=\eta^{\lambda\lambda^{\prime}},
				\hsp
				\levicivita^{\lambda}{}_{\mu}(k)\levicivita_{\lambda\nu}{}^{\ast}(k)=\eta_{\mu\nu}
			\end{equation}

			\noindent
			Full particle $\psi^{-}(x)$ and antiparticle $\psi^{+}(x)$ solutions (with amplitudes $a_{s,\vec{p}}$ and $b_{s,\vec{p}}$)
			\begin{equation}
				\begin{aligned}
					\psi^{-}(x) &= \int\frac{\dd^{3}p}{(2\pi)^{\frac{3}{2}}}\frac{1}{\sqrt{2E_{\vec{p}}}}\sum_{s=1}^{2}a_{s, \vec{p}}u^{s}(p)\ex^{-\i p\cdot x} \\
					\psi^{+}(x) &= \int\frac{\dd^{3}p}{(2\pi)^{\frac{3}{2}}}\frac{1}{\sqrt{2E_{\vec{p}}}}\sum_{s=1}^{2}{b_{s, \vec{p}}^{*}}v^{s}(p)\ex^{\i p\cdot x}
				\end{aligned}
			\end{equation}

			\noindent
			Pauli Equation\index{Pauli!Gleichung} (non-relativistic limit of the Dirac equation):
			\begin{equation}
				\hat{H}\psi = \frac{\qty(\vec{\sigma}\cdot\hat{\vec{p}})^2}{2m} \psi
			\end{equation}

			\noindent
			Spinor representation of the Lorentz group (Generator $\hat{S}^{\mu\nu} = \frac{\i}{4}[\gamma^\mu,\gamma^\nu]$):
			\begin{align}
				\Lambda_{1/2} &= \exp(-\frac{\i}{2}\omega_{\mu\nu}S^{\mu\nu}) = \mqty(\Lambda_\text{L} & 0 \\ 0 & \Lambda_\text{R})
			\end{align}
			Dirac Spinor\index{Dirac!Spinor} transformation rules:
			\begin{align}
				\gamma^0 \Lambda_{1/2}^\dagger \gamma^0	&= \Lambda_{1/2}^{-1}
				&
				\Lambda_{1/2}^{-1} \gamma^\mu \Lambda_{1/2} &= \Lambda^\mu{}_\nu \gamma^\nu \\
				\psi' &= \Lambda_{1/2} \psi
				&
				\bar{\psi}' &= \bar{\psi} \Lambda_{1/2}^{-1}
			\end{align}

			\noindent
			Lagrangian for a Dirac-Field:
			\begin{equation}
				\mathcal{L} = \bar{\psi} \qty(\i \slashed{\partial} - m) \psi
			\end{equation}

			\noindent
			Noether current operator / probability current operator (resulting from the symmetry $\psi \to \ex^{\i \alpha} \psi$):
			\begin{equation}
				\hat{J}^\mu = \hat{\bar{\psi}} \gamma^\mu \hat{\psi}
			\end{equation}

			\noindent
			Axial current operator (conserved if the chiral symmetry $\psi \to \ex^{\i \alpha \gamma^5} \psi$ is present, typically when $m=0$):
			\begin{equation}
				\hat{J}^\mu_5 = \hat{\bar{\psi}} \gamma^\mu \gamma^5 \hat{\psi}
			\end{equation}

			\noindent
			Free Feynman Propagator\index{Feynman!Propagator} for a spinor field:
			\begin{equation}
				\begin{aligned}
					D_\text{F}(x-y) &= \mel**{0}{T\hat{\psi}(x)\hat{\bar\psi}(y)}{0} \\
					&= \int \frac{\dd^4 p}{(2\pi)^4} \frac{\i(\slashed{p}+m)}{p^2-m^2+\i\epsilon} \ex^{-\i p\cdot(x-y)} \\
					&= \wick{\c{\hat{\psi}}(x) \c{\hat{\bar{\psi}}}(y)} = - \wick{\c{\hat{\bar{\psi}}}(y) \c{\hat\psi}(x)}
				\end{aligned}
			\end{equation}

			\noindent
			Orthogonal basis for spinor bilinears (scalar, vector, tensor, pseudovector, pseudoscalar):
			\begin{equation}
				O_S = \mathds{1}_4
				\hsp
				O_V = \gamma_\mu
				\hsp
				O_T = \frac{1}{\sqrt{2}} \sigma_{\mu\nu}
				\hsp
				O_A = \i \gamma_\mu \gamma_5
				\hsp
				O_P = \gamma_5
			\end{equation}

			\noindent
			Bispinor decomposition:
			\begin{align}
				\psi\bar{\chi} &= \frac{1}{4} \qty( c_S \mathds{1}_4 + c_V^\mu \gamma_\mu + c_T^{\mu\nu} \sigma_{\mu\nu} + c_A^\mu \gamma_\mu \gamma_5 + c_P \gamma_5 ) \\
				c_X &= \frac{1}{4} \tr(O_X O_X) \tr(\psi \bar{\chi} O_X)
			\end{align}

			\noindent
			Fierz identities\index{Fierz!Identitäten}:
			\begin{equation}
				\qty(\bar{\psi}_a O_X \psi_b) \qty(\bar{\psi}_c O_X \psi_d)
				= \sum_Y C_{XY} \qty(\bar{\psi}_a O_Y \psi_d) \qty(\bar{\psi}_c O_Y \psi_b)
			\end{equation}

			\noindent
			Fierz Table\index{Fierz!Tabelle} ($X\in\qty{S,V,T,A,P}$; note: signs depend on metric signature, order of $\gamma^5 \gamma^\mu$ for the pseudovector and euclidean vs. minkowski space but not representation):
			\begin{equation}
				C_{XY} = \mqty(
				+\frac{1}{4} & +\frac{1}{4} & -\frac{1}{4} & -\frac{1}{4} & +\frac{1}{4} \\
				+1 & -\frac{1}{2} & 0 & -\frac{1}{2} & -1 \\
				-\frac{3}{2} & 0 & -\frac{1}{2} & 0 & -\frac{3}{2} \\
				-1 & -\frac{1}{2} & 0 & -\frac{1}{2} & +1 \\
				+\frac{1}{4} & -\frac{1}{4} & -\frac{1}{4} & +\frac{1}{4} & +\frac{1}{4} \\
				)
			\end{equation}

			\noindent
			Selection of important Fierz identities:
			\begin{align}
				(\bar{\psi}_1 \bar{\sigma}^\mu \psi_2) (\bar{\psi}_3 \bar{\sigma}^\mu \psi_4) = (\bar{\psi}_1 \bar{\sigma}^\mu \psi_4) (\bar{\psi}_3 \bar{\sigma}^\mu \psi_2)
			\end{align}

		\subsubsection{Chirality}
			\noindent
			Chiral spinors (eigenstates of $\gamma^5$ with eigenvalues / chirality\index{Chiralität} $\pm 1$; note that while being Lorentz invariant, chirality is not a constant of motion):
			\begin{equation}
				\gamma_5 \psi_L = - \psi_L
				\hsp \gamma_5 \psi_R = \psi_R
				\hsp \bar{\psi}_L \gamma_5 = \bar{\psi}_L
				\hsp \bar{\psi}_R \gamma_5 = -\bar{\psi}_R
			\end{equation}

			\noindent
			Left- and right-handed projection operators:
			\begin{equation}
				\label{Eq:SpinorProjectionOperators}
				\hat{P}_{\mathrm{L}}=\frac{1-\gamma^{5}}{2} \hsp
				\hat{P}_{\mathrm{R}}=\frac{1+\gamma^{5}}{2}
			\end{equation}

			\noindent
			Helicity operator (\ie right-handed particles have positive helicity and left-handed particles have negative helicity, while right-handed antiparticles have negative helicity and left-handed antiparticles have positive helicity; Note that while helicity is a constant of motion, it is not Lorentz invariant):
			\begin{equation}
				\hat{h} = \frac{\vec{\sigma}\cdot\hat{\vec{p}}}{\abs{\vec{p}}}
			\end{equation}

			\noindent
			Chiral $U(1)_A$ transformation / Axial phase transformation:
			\begin{equation}
				\psi(x) \to \ex^{\i \gamma_5 \alpha} \psi(x)
				\hsp \bar{\psi}(x) \to \bar{\psi}(x) \ex^{\i \gamma_5 \alpha}
			\end{equation}
			left- and right-handed chiral transformations:
			\begin{align}
				U(1)_L:&\quad U(\alpha) = U(\alpha/2)_V U(-\alpha/2)_A,
				\hsp \psi \to \ex^{\i P_L \alpha} \psi
				\hsp \bar{\psi} \to \bar{\psi} \ex^{-\i P_R \alpha} \\
				U(1)_R:&\quad U(\alpha) = U(\alpha/2)_V U(\alpha/2)_A, \phantom{-}
				\hsp \psi \to \ex^{\i P_R \alpha} \psi
				\hsp \bar{\psi} \to \bar{\psi} \ex^{-\i P_L \alpha}
			\end{align}


		\subsubsection{Yukawa Theory\index{Yukawa!Theorie}}
			Lagrangian density:
			\begin{equation}
				\mathcal{L}=\partial^{\mu}\psi^{\dagger}\partial_{\mu}\psi-m^{2}\psi^{\dagger}\psi+\frac{1}{2}(\partial_{\mu}\phi)^{2}-\frac{1}{2}\mu^{2}\phi^{2}-g\psi^{\dagger}\psi\phi
			\end{equation}

			\noindent
			Free field representations:
			\begin{equation}
				\begin{aligned}
					\hat{\psi}(x) &= \int\frac{\dd^{3}p}{\left(2\pi\right)^{\frac{3}{2}}}\frac{1}{(2E_{p})^{\frac{1}{2}}}\left(\hat{a}_{p}\mathrm{e}^{-\mathrm{i}p\cdot x}+\hat{b}_{p}^{\dagger}\mathrm{e}^{\mathrm{i}p\cdot x}\right), \\
					\hat{\psi}^{\dagger}(x) &= \int\frac{\dd^{3}p}{(2\pi)^{\frac{3}{2}}}\frac{1}{(2E_{p})^{\frac{1}{2}}}\left(\hat{a}_{p}^{\dagger}\mathrm{e}^{\mathrm{i}p\cdot x}+\hat{b}_{p}\mathrm{e}^{-\mathrm{i}p\cdot x}\right) \\
					\hat{\phi}(x) &=\int\frac{\dd^{3}q}{\left(2\pi\right)^{\frac{3}{2}}}\frac{1}{\left(2\varepsilon_{q}\right)^{\frac{1}{2}}}\left(\hat{c}_{q}\mathrm{e}^{-\mathrm{i}q\cdot x}+\hat{c}_{q}^{\dagger}\mathrm{e}^{\mathrm{i}q\cdot x}\right) \\
				\end{aligned}
			\end{equation}

			\noindent
			Feynman rules in momentum space:
			\begin{itemize}\itemsep -0pt
				\item Each vertex contributes a factor $-ig$
				\item For each internal phion line carrying momentum $q$ include a propagator $\frac{\i}{q^2 -\mu^2+\i\epsilon}$
				\item For each internal psion line carrying momentum $q$ include a
					propagator $\frac{\i}{q^2 -m^2+\i\epsilon}$
				\item Integrate over all undetermined momenta
				\item All symmetry factors are 1
				\item Incoming and outgoing lines contribute a factor 1
				\item Include an overall energy-momentum conserving delta function
					for each diagram
			\end{itemize}



		\subsubsection{NJL Model}
			The Nambu--Jona-Lasino\index{Nambu--Jona-Lasino Model} is the chiral limit of QCD (with $N$ flavors):
			\begin{equation}
				\begin{aligned}
					\mathcal{L}
					&= \i \bar{\psi}_a \slashed{\partial} \psi^a + \frac{\lambda}{4 N} \qty[(\bar{\psi}_a\psi^b)(\bar{\psi}_b\psi^a) - (\bar{\psi}_a\gamma_5 \psi^b) (\bar{\psi}_b\gamma_5 \psi^a)] \\
					&= \i \bar{\psi}_{La} \slashed{\partial} \psi_L^a + \i \bar{\psi}_{Ra} \slashed{\partial} \psi_R^a + \frac{\lambda}{N} (\bar{\psi}_{La}\psi_R^b)(\bar{\psi}_{Rb}\psi_L^a)
				\end{aligned}
			\end{equation}

	\subsection{Yang--Mills Theory\index{Yang!--Mills Theorie}\index{Mills!Yang--Mills Theorie}}
		Lagrangian of Yang--Mills Theory ($SU(N)$ gauge)
		\begin{equation}
			\mathcal{L}_\text{YM} = -\frac{1}{4} F^a_{\mu\nu} F^{a\mu\nu} + \bar{\psi}(\i\slashed{D}-m)\psi
		\end{equation}

		\noindent
		Gauge Fixing and ghosts:
		\begin{equation}
			\mathcal{L}_\text{GF} = -\frac{1}{2\xi} \qty(\partial_\mu A^\mu)^2 - \bar{c}^a \partial_\mu D^\mu_{ab} c^b
		\end{equation}

		\noindent
		Lie Algebra\index{Lie!Algebra} of $SU(N)$ (with $T^a$ the generators of the Lie algebra):
		\begin{equation}
			\comm{T^a}{T^b} = \i f^{abc} T^c
		\end{equation}

		\noindent
		Casimir operators\index{Casimir!Operator} $\hat{C}_R$ (invariants of the algebra, for a representation $R$):
		\begin{equation}
			\comm{T_R^a}{\hat{C}(R)} = 0
		\end{equation}

		\noindent
		Quadratic Casimir operator:
		\begin{equation}
			T^a_R T^a_R = \hat{C}_2(R) = C_2(R) \hat{1}
		\end{equation}

		\noindent
		Field strength tensor\index{Feldstärketensor} for Yang--Mills Theory (with $f^{abc}$ the structure constants of the Lie algebra, which are totally antisymmetric for normalized $T^a$ such that $\tr(T^a T^b) = \frac{1}{2} \delta^{ab}$):
		\begin{equation}
			\begin{aligned}
				F^a_{\mu\nu} &= \partial_\mu A^a_\nu - \partial_\nu A^a_\mu + g f^{abc} A^b_\mu A^c_\nu \\
				\comm{D_\mu}{D_\nu} &= -\i g F_{\mu\nu}^a T^a
			\end{aligned}
		\end{equation}

		\noindent
		Structure constant identities:
		\begin{align}
			%WTF ? % f^{abc} f^{ade} &= f^{abc} f^{dae} \\
			% f^{abc} f^{cde} &= \frac{1}{2} f^{abc} f^{cde} + \frac{1}{2} f^{abc} f^{ced} = \frac{1}{2} C_A \delta^{ad} \\
			f^{acd} f^{bcd} &= C_A \delta^{ab} \\
			0 &= f^{ade} f^{bcd} + f^{bde} f^{cad} + f^{cde} f^{abd}
		\end{align}

		\noindent
		Covariant derivative\index{Kovariante Ableitung} for Yang--Mills Theory:
		\begin{equation}
			D_\mu = \partial_\mu - \i g A_\mu^a T^a
		\end{equation}

		\noindent
		Gauge Transformation ($\alpha(x) = \alpha^a(x) T^a$):
		\begin{equation}
			\begin{aligned}
				\psi(x) & \to \psi'(x) = \ex^{\i g \alpha(x)} \psi(x) \\
				A_\mu^a(x) & \to {A'_\mu}^a(x) = A_\mu^a(x) - \frac{1}{g} \partial_\mu \alpha^a(x) \\
				F_{\mu\nu}^a(x) & \to {F'_{\mu\nu}}^a(x) = F_{\mu\nu}^a(x) - f^{abc} \alpha^b F_{\mu\nu}^c(x)
			\end{aligned}
		\end{equation}

		\noindent
		Feynman Rules for Yang--Mills Theory:
		\begin{itemize}\itemsep -0pt
			\item Gluon propagator: $D_\text{F}^{ab,\mu\nu}(x-y) = -\frac{\i \delta^{ab}}{k^2+\i\epsilon} \qty(\eta^{\mu\nu}-(1-\xi)\frac{k^\mu k^\nu}{k^2+\i\epsilon} )$
			\item Quark propagator: $D_\text{F}(x-y) = \i\qty(\frac{\slashed{p}+m_q}{p^2-m_q^2+\i\epsilon})$
			\item Ghost propagator: $C^{ab}(p) = \frac{\i\delta^{ab}}{p^2+\i\epsilon}$
			\item 3 gluon vertex factor: $\Gamma^{abc}_{\mu\nu\rho}(p,q,k) = -gf^{abc} \qty[(p-q)_\rho \eta_{\mu\nu} + (q-k)_\mu \eta_{\nu\rho} + (k-p)_\nu \eta_{\mu\rho}]$
			\item 4 gluon vertex factor: $\Gamma^{abcd}_{\mu\nu\rho\sigma} = -\i g^2 f^{abe}f^{cde}(\eta_{\mu\rho}\eta_{\nu\sigma} - \eta_{\mu\sigma}\eta_{\nu\rho}) - \i g^2 f^{ace}f^{bde} (\eta_{\mu\nu}\eta_{\rho\sigma} - \eta_{\mu\sigma}\eta_{\nu\rho}) - \i g^2 f^{ade} f^{bce} (\eta_{\mu\nu}\eta_{\rho\sigma} - \eta_{\mu\rho}\eta_{\nu\sigma})$
			\item Ghost-ghost gluon vertex: $\Gamma^{abc}_\mu(p) = gf^{abc} p_\mu$
			\item Incoming gluon: $\epsilon^a_\mu(k)$, outgoing gluon: $\epsilon^{*a}_\mu(k)$
			\item Incoming quark: $u(p)$, outgoing quark: $\bar{u}(p)$, incoming anti-quark: $\bar{v}(p)$, outgoing anti-quark: $v(p)$
			\item For each loop integrate over loop momentum $\int \frac{\dd^4 q}{(2\pi)^4}$, include a minus for fermion loops
			\item Ensure overall momentum conservation $(2\pi)^4 \delta(\sum_i p_i)$
		\end{itemize}

		\subsubsection{BRST Invariance}
			BRST Transformation (Becchi, Rouet, Stora and Tyutin)\index{BRST Transformation}:
			\begin{equation}
				\begin{aligned}
					\phi_i & \to \phi_i + i\theta c^a T^a_{ij} \phi_j \\
					A^a_\mu & \to A^a_\mu + \frac{1}{g}\theta D_\mu c^a \\
					\bar{c}^a & \to \bar{c}^a - \frac{1}{g} \theta \frac{1}{\xi} \partial^\mu A^a_\mu \\
				\end{aligned}
			\end{equation}


	\subsection{Standard Model of particle Physics\index{SM}}

		\subsubsection{Quantum Electrodynamics}
			Lagrangian of Quantum Electrodynamics\index{Quanten Elektrodynamik} ($U(1)$ gauge)
			\begin{equation}
				\mathcal{L}_\mathrm{QED} = -F^{\mu\nu}F_{\mu\nu} + \bar{\psi}(\i\slashed{D}-m)\psi
			\end{equation}

			\noindent
			Feynman Rules for Quantum Electrodynamics ($\xi$ is the gauge fixing parameter):
			\begin{itemize}\itemsep -0pt
				\item Photon propagator: $D_\text{F}^{\mu\nu}(x-y) = -\frac{\i}{k^2+\i\epsilon} \qty(\eta_{\mu\nu}-(1-\xi)\frac{k_\mu k_\nu}{k^2+\i\epsilon} )$
				\item Charged particle propagator $\psi \in \qty{e^\pm, \mu^\pm ,\tau^\pm, u,d,s,c,t,b}$: $D_\text{F}(x-y) = \i\qty(\frac{\slashed{p}+m_\psi}{p^2-m_\psi^2+\i\epsilon})$
				\item Vertex factor: $-e\gamma^\mu$
				\item Incoming photon: $\epsilon_\mu(k)$, outgoing photon: $\epsilon^{*}_\mu(k)$
				\item Incoming lepton: $u(p)$, outgoing lepton: $\bar{u}(p)$, incoming anti-lepton: $\bar{v}(p)$, outgoing anti-lepton: $v(p)$
				\item For each loop integrate over loop momentum $\int \frac{\dd^4 q}{(2\pi)^4}$, include a minus for fermion loops
				\item Ensure overall momentum conservation $(2\pi)^4 \delta(\sum_i p_i)$
			\end{itemize}

		\subsubsection{Electroweak Theory\index{EW Theory}}
			$SU(2)_L \times U(1)_Y$ gauge theory.

			Electroweak Lagrangian before symmetry breaking ($W_a^{\mu\nu}$ is the field strength tensor of the weak isospin gauge field, $B^{\mu\nu}$ is the field strength tensor of the hypercharge gauge field; $j$ denotes flavor; $Q_j$ are the left-handed quark doublets, $u_j$ are the right-handed up-like singlets, $d_j$ are the right-handed down-like singlets, $L_j$ are the left-handed lepton doublets, $e_j$ are the right-handed charged lepton singlets; $H$ is the higgs field, with vacuum expectation value $v$, fluctuation $h$, with $\tilde{H}=\i\sigma_2 H^{*}$; $y^a_{ij}$ are the Yukawa couplings for $a \in \set{u,d,e}$. Note the absence of neutrino masses):
			\begin{equation}
				\begin{aligned}
					\lagrangian_{EW} &= \lagrangian_{\text{gauge}} + \lagrangian_{\text{fermion}} + \lagrangian_{\text{Higgs}} + \lagrangian_{\text{Yukawa}} \\
					\lagrangian_{\text{gauge}} &= -\frac{1}{4} W_a^{\mu\nu} W^a_{\mu\nu} -\frac{1}{4} B^{\mu\nu} B_{\mu\nu} \\
					\lagrangian_{\text{fermion}} &= \bar{Q}_j \i \slashed{D} Q_j + \bar{u}_j \i \slashed{D} u_j + \bar{d}_j \i \slashed{D} d_j + \bar{L}_j \i \slashed{D} L_j + \bar{e}_j \i \slashed{D} e_j \\
					\lagrangian_{\text{Higgs}} &= \abs{D_\mu h}^2 - \lambda\qty(\abs{h}^2 -\frac{v^2}{2})^2 \\
					\lagrangian_{\text{Yukawa}} &= - y^u_{ij} \bar{Q}_{i} \tilde{H}  u_j - y^d_{ij} \bar{Q}_i H d_j - y^e_{ij} \bar{L}_i H e_j + \text{h.c.}
				\end{aligned}
			\end{equation}

			\noindent
			Gauge transformation ($\vec{T}=T\vec{\sigma}=\frac{1}{2}\vec{\sigma}$ for doublets and $0$ for singlets):
			\begin{equation}
				\Psi \to \Psi' = \exp(\i g \vec{\omega}(x)\cdot\vec{T}	+ \i g' \omega_0(x) Y) \Psi
			\end{equation}

			\noindent
			Fermion masses can be acquired after symmetry breaking via the Yukawa couplings (with $U,K$ unitary and $M$ diagonal with eigenvalues $m_{u,d}^i$):
			\begin{equation}
				y_{u,d} = U_{u,d} M_{u,d} K_{u,d}^\dagger
			\end{equation}

			\noindent
			Transformation from flavor basis to mass basis:
			\begin{alignat}{3}
				u_L^i &\leftrightarrow U^{ij}_u u_L^j
				\hsp && d_L^i &&\leftrightarrow U^{ij}_d d_L^j
				\\
				u_R^i &\leftrightarrow K^{ij}_u u_R^j
				\hsp && d_R^i &&\leftrightarrow K^{ij}_d d_R^j
			\end{alignat}

			\noindent
			CKM (Cabibbo--Kobayashi--Maskawa) matrix\index{CKM Matrix} (which taken arbitrary phase factors into account can be parametrized by 4 real \dof $\theta_{12}, \theta_{13}, \theta_{23}, \delta$):
			\begin{equation}
				\begin{aligned}
					V &\defeq U_u^\dagger U_d = \mqty(V_{ud} & V_{us} & V_{ub} \\ V_{cd} & V_{cs} & V_{cb} \\ V_{td} & V_{ts} & V_{tb}) \\
					&= \mqty(
					1 & 0 & 0 \\
					\cos\theta_{23} & \sin\theta_{23} & 0 \\
					-\sin\theta_{23} & \cos\theta_{23} & 0)
					\mqty(
					\cos\theta_{13} & 0 & \sin\theta_{13} \ex^{-\i\delta} \\
					0 & 1 & 0 \\
					-\sin\theta_{13} \ex^{\i\delta} & 0 & \cos\theta_{13})
					\mqty(
					\cos\theta_{12} & \sin\theta_{12} & 0 \\
					-\sin\theta_{12} & \cos\theta_{12} & 0 \\
					0 & 0 & 1)
				\end{aligned}
			\end{equation}

			\noindent
			Gell-Mann--Nishijima formula\index{Gell-Mann!--Nishijima Formel}\index{Nishijima!Gell-Mann--Nishijima Formel} (Charge $Q$, third component of the weak isospin $T_3$, hypercharge $Y$):
			\begin{equation}
				\label{Eq:GellMannNishijima}
				Q = T_3 + Y
			\end{equation}

			\noindent
			Covariant derivative:
			\begin{equation}
				D^\mu = \partial^\mu - \i \frac{1}{2} g' B^\mu Y - \i \frac{1}{2} g \vec{W}^\mu \cdot \vec{T}
			\end{equation}

			\noindent
			Relation of weak isospin coupling $g=g_W$ and weak hypercharge coupling $g'$ and effective neutral current coupling $g_Z$ to the Weinberg angle\index{Weinberg!Winkel}:
			\begin{align}
				e &= g \sin\theta_W = g' \cos\theta_W
				\hsp g_Z = \frac{g}{\cos\theta} \\
				\sin \theta_W &= \frac{g'}{\sqrt{g^2 + g'^2}}
				\hsp \cos \theta_W = \frac{g}{\sqrt{g^2 + g'^2}}
			\end{align}

			\noindent
			Weak boson mass relation:
			\begin{equation}
				m_Z \cos(\theta_W) = m_W = \frac{v}{2} g
			\end{equation}

			\noindent
			Boson mixing due to spontaneous symmetry breaking:
			\begin{align}
				A_\mu &= \phantom{-}\cos\theta_W B_\mu + \sin\theta_W W^3_\mu \\
				Z^0_\mu &= -\sin\theta_W B_\mu + \cos\theta_W W^3_\mu
			\end{align}

			\noindent
			Charged weak boson field decomposition:
			\begin{equation}
				W^\pm = \frac{1}{\sqrt{2}}(W^1 \mp \i W^2) \\
			\end{equation}


		\subsubsection{Quantum Chromodynamics\index{QCD}}
			Lagrangian:
			\begin{equation}
				\mathcal{L}_\text{QCD} = -\frac{1}{4} G^a_{\mu\nu} G^{a\mu\nu} + \bar{\psi}(\i\slashed{D}-m)\psi
			\end{equation}

			\noindent
			QCD field strength $G^a_{\mu\nu}$ and gauge field $G^a_\mu$ (gauge coupling $g_s$) and relation to covariant derivative:
			\begin{equation}
				G^a_{\mu\nu} = \partial_\mu G^a_\nu - \partial_\nu G^a_\mu + g_s f^{abc} G^b_\mu G^c_\nu
			\end{equation}

			\noindent
			Covariant derivative
			\begin{equation}
				D_\mu = \partial_\mu + \i g_s G^a_\mu T^a
			\end{equation}



		\subsubsection{Higgs Mechanism}



			% \noindent
			% Feynman Rules for scalar field theory: \\
			% \begin{center}
			% 	\begin{tikzpicture}				
			% 		\begin{feynman}
			% 			\vertex(a);
			% 			\vertex[above left=of a](b){$f^+$};
			% 			\vertex[below left=of a](c){$f^-$};
			% 			\vertex[right=of a](d);
			% 			\vertex[above right=of d](e){$f^+$};
			% 			\vertex[below right=of d](f){$f^-$};
			% 			\diagram{
			% 				(b)--[fermion] (a) -- [fermion](c);
			% 				(a) --[boson, edge label=$\gamma$, momentum'=$k$] (d);
			% 				(e)--[fermion] (d) -- [fermion](f);
			% 				};
			% 		\end{feynman}
			% 	\end{tikzpicture}
			% \end{center}