% !TEX root = ../physics.tex
\section{Quantum Field Theory\index{QFT}\index{Quantenfeldtheorie}} % QFt
	\subsection{Postulates}
		\subsubsection{Symmetries?}
			%Locality
			%CPT
			%...
			%TODO
	\subsection{Definitions}
		Feynman slash notation\index{Feynman!Slash Notation}
		\begin{equation}
			\slashed{A} := \gamma_\mu A^\mu
		\end{equation}

		\noindent
		Pauli matrices\index{Pauli!Matrizen} (see Eq.~\ref{Eq:PauliMatrices}):
		\begin{equation}
			\label{Eq:PauliMatricesQFT}
			\begin{aligned}
				\sigma^0 = \left(\begin{matrix}
					1 & 0 \\
					0 & 1 \\
				\end{matrix}\right) &&\hspace{30pt}
				\sigma^1 = \left(\begin{matrix}
					0 & 1 \\
					1 & 0 \\
				\end{matrix}\right) &&\hspace{30pt}
				\sigma^2 = \left(\begin{matrix}
					0 & -\i \\
						\i & 0 \\
				\end{matrix}\right) &&\hspace{30pt}
				\sigma^3 = \left(\begin{matrix}
					1 & 0 \\
					0 & -1 \\
				\end{matrix}\right)
			\end{aligned}
		\end{equation}

		\noindent
		Properties:
		\begin{equation}
			\begin{aligned}
				\comm{\sigma_\mu}{\sigma_\nu} &= 2\i\varepsilon_{\mu\nu\lambda}\sigma_\lambda \\
				\sigma_\mu \sigma_\nu \sigma_\lambda &= \i\varepsilon_{\xi\mu\nu\lambda}\sigma_\xi \\
			\end{aligned}
		\end{equation}

		\subsubsection{Gamma Matrices}
			\noindent
			Defining properties:
			\begin{equation}
				\acomm{\gamma^\mu}{\gamma^\nu} = \gamma^\mu \gamma^\nu + \gamma^\nu \gamma^\nu = 2g^{\mu\nu}
			\end{equation}

			\noindent
			Further properties:
			\begin{equation}
				\begin{aligned}
					\qty(\gamma^0)^\dagger &= +\gamma^0 \\
					\qty(\gamma^j)^\dagger &= -\gamma^j \\
					\gamma^0 \gamma^0 &= 1_4 \\ % TODO: replace hat with \mathbb{1}
					\gamma^i \gamma^j &= - \delta^{ij} 1_4 \\
					\qty(\gamma^\mu)^\dagger &= \gamma^0 \gamma^\mu \gamma^0 \\
				\end{aligned}
			\end{equation}

			\noindent
			Dirac $\gamma$-matrices\index{Dirac!Gamma Matrizen}:
			\begin{equation}
				\begin{aligned}
					\gamma^0
					= \left( \begin{matrix}
					\sigma^0 & \M0 \\
					0 &  -\sigma^0
					\end{matrix} \right)
					= \left( \begin{matrix}
					\M 1 & \M 0 & \M 0 & \M 0 \\
					\M 0 & \M 1 & \M 0 & \M 0 \\
					\M 0 & \M 0 &   -1 & \M 0 \\
					\M 0 & \M 0 & \M 0 &   -1 \\
					\end{matrix} \right)
					&\hspace{20pt}
					\gamma^1
					= \left( \begin{matrix}
					\M0 & \sigma^1 \\
					-\sigma^1 &  0
					\end{matrix} \right)
					= \left( \begin{matrix}
					\M 0 & \M 0 & \M 0 & \M 1 \\
					\M 0 & \M 0 & \M 1 & \M 0 \\
					\M 0 &   -1 & \M 0 & \M 0 \\
					-1 & \M 0 & \M 0 & \M 0 \\
					\end{matrix} \right) \\[8pt]
					\gamma^2
					= \left( \begin{matrix}
					\M0 & \sigma^2 \\
					-\sigma^2 &  0
					\end{matrix} \right)
					= \left( \begin{matrix}
					\M 0 & \M 0 & \M 0 &  -\i \\
					\M 0 & \M 0 & \M\i & \M 0 \\
					\M 0 & \M\i & \M 0 & \M 0 \\
					-\i & \M 0 & \M 0 & \M 0 \\
					\end{matrix} \right)
					&\hspace{20pt}
					\gamma^3
					= \left( \begin{matrix}
					\M 0 & \sigma^3 \\
					-\sigma^3 &  0
					\end{matrix} \right)
					= \left( \begin{matrix}
					\M 0 & \M 0 & \M 1 & \M 0 \\
					\M 0 & \M 0 & \M 0 &   -1 \\
					-1 & \M 0 & \M 0 & \M 0 \\
					\M 0 & \M 1 & \M 0 & \M 0 \\
					\end{matrix} \right) \\
				\end{aligned}
			\end{equation}

			\noindent
			Transformation ($U$ unitary: $U U^\dagger=1_4$):
			\begin{equation}
				\tilde{\gamma} = U \gamma^\mu U^\dagger
			\end{equation}

			\noindent
			$\gamma^5$-Matrix:
			\begin{equation}
				\gamma^5 = \gamma_5 = \gamma_5 = i\gamma^0 \gamma^1 \gamma^2 \gamma^3 = -\frac{1}{4!}\varepsilon_{\mu\nu\rho\sigma} \gamma^\mu \gamma^\nu \gamma^\rho \gamma^\sigma
			\end{equation}

			\noindent
			Properties of the $\gamma^5$-matrix:
			\begin{equation}
				\begin{aligned}
					\gamma_5^\dagger &= \gamma_5 \\
					\gamma_5 \gamma_5 &= 1_4 \\
					\acomm{\gamma^\mu}{\gamma^5} &= 0 \\
				\end{aligned}
			\end{equation}

			\noindent
			$\sigma^{\mu\nu}$ matrices:
			\begin{equation}
				\sigma^{\mu\nu} = \frac{i}{2}\comm{\gamma^\mu}{\gamma^\nu} = -\sigma^{\nu\mu}
			\end{equation}

	\subsection{Important Operators}
		Time ordering operator ($+$ for Bosons, $-$ for Fermions)
		\begin{equation}
			\hat{T}\qty[\hat{A}(t)\hat{B}(t')]
				= \hat{A}(t) \hat{B}(t') \Theta(t-t') \pm \hat{B}(t') \hat{A}(t) \Theta(t'-t)
				= \begin{cases}
					\phantom{\pm} \hat{A}(t) \hat{B}(t') & t>t' \\
					\pm \hat{B}(t') \hat{A}(t) & t<t'
				\end{cases}
		\end{equation}

		\noindent
		Normal order operator
		\begin{equation}
			\NormalOrder{\hat{A}\hat{B}}
		\end{equation}

		\noindent
		Wick\index{Wick!Kontraktion} contraction definition:
		\begin{equation}
			\wick{\c{\hat{A}} \c{\hat{B}}} = \hat{A}\hat{B} - \NormalOrder{\hat{A}\hat{B}}
		\end{equation}
		%\wick{\c \hat{A} \c \hat{B}}

		%TODO: Wick Theorem

	\subsection{Pertubation Theory}
		Gell-Mann and Low Theorem\index{Gell-Mann! und Low Theorem}:
		\begin{equation}
			\begin{aligned}
				&\hat{H} := \hat{H}_0 + e^{-\varepsilon \abs{t}} \hat{H}' \\
				&\hat{H}_0 \Ket{\phi_0} = E_0 + \Ket{\phi_0} \\
				&\text{if}\quad \lim_{\varepsilon\rightarrow 0} \frac{\hat{U}_\varepsilon(0,-\infty)\Ket{\phi_0}}{\Bra{\phi_0} \hat{U}_\varepsilon(0,-\infty)\Ket{\phi_0}}
				= \frac{\Ket{\psi_0}}{\Braket{\phi_0|\psi_0}} \quad\text{exists}\quad
				\Rightarrow&\quad \hat{H} \frac{\Ket{\psi_0}}{\Braket{\phi_0|\psi_0}} = E \frac{\Ket{\psi_0}}{\Braket{\phi_0|\psi_0}} \\
			\end{aligned}
		\end{equation}

		\subsubsection{Green's Functions}
			Single particle Green's function / Propagator \index{Green!Funktion} ($x=(t,\vec{x})$; $\Ket{\Psi_0}$ is the Heisenberg ground state)
			\begin{equation}
				\begin{aligned}
					\i G_{\alpha\beta}(x, x') &= \frac{\Bra{\Psi_0} \hat{T}\qty[\hat{\Psi}_{H\alpha}(x)\hat{\Psi}^\dagger_{H\beta}(x')] \Ket{\Psi_0}}{\Braket{\Psi_0|\Psi_0}} \\					
				\end{aligned}
			\end{equation}

	%\subsection{Feynman Path Integral\index{Feynman!Pfadintegral}}
		%TODO

	\subsection{Bosons}
		\subsubsection{Bosonic Creation an Annihilation Operators}
		\label{Sec:BosonicCreationAndAnnihilationOperators}
			Definition
			\begin{equation}
				\comm{\hat{b}_k}{\hat{b}_{k'}^\dagger} = \delta_{kk'}
				\hspace{20pt}
				\comm{\hat{b}_k}{\hat{b}_{k'}} = \comm{\hat{b}_k^\dagger}{\hat{b}_{k'}^\dagger} = 0
			\end{equation}

			\noindent
			For the basis of occupation numbers $\Ket{n_1, n_2, ..., n_\infty}$ with $\forall i:\;n_i\in \mathbb{N}_0$ the operators have the following qualities:
			\begin{equation}
				\begin{aligned}
					\hat{b}_k^\dagger \hat{b}_k \Ket{...,n_k,...} &= n_k \Ket{...,n_k,...} \\
					\hat{b}_k \Ket{...,n_k,...} &= \sqrt{n_k} \Ket{...,n_k-1,...} \\
					\hat{b}_k^\dagger \Ket{...,n_k,...} &= \sqrt{n_k+1} \Ket{...,n_k+1,...} \\
				\end{aligned}
			\end{equation}

		\subsubsection{Free Klein-Gordon Equation\index{Klein!-Gordon Gleichung}}
			\noindent
			Klein-Gordon Equation (spin $s=0$)
			\begin{equation}
				\left(\partial^\mu\partial_\mu+m^2\right) \phi(x) = 0
			\end{equation}

			\noindent
			Solution (plane waves):
			\begin{equation}
				{\phi}^{ ( \pm ) }_{\vec{p}} (x) = e^{\mp p_\mu x^\mu}
			\end{equation}

	\subsection{Fermions}
		\subsubsection{Fermionic Creation an Annihilation Operators}
		\label{Sec:FermionicCreationAndAnnihilationOperators}
			Definition of fermionic creation an annihilation operators
			\begin{equation}
				\acomm{\hat{a}_k}{\hat{a}_{k'}^\dagger} = \delta_{kk'}
				\hspace{20pt}
				%\acomm{\hat{a}_k}{\hat{a}_{k'}} = \acomm{\hat{a}_k^\dagger}{\hat{a}_{k'}^\dagger} = 0
				\Big\lbrace\hat{a}_k,\hat{a}_{k'}\Big\rbrace = \acomm{\hat{a}_k^\dagger}{\hat{a}_{k'}^\dagger} = 0
			\end{equation}
			In particular $\hat{a}_k^\dagger \hat{a}_k^\dagger = 0 = \hat{a}_k \hat{a}_k$, which ensures $n_k\in\qty{0,1}$, \ie the Pauli exclusion principle\index{Pauli!Ausschlussprinzip}.

			\noindent
			For the basis of occupation numbers $\Ket{n_1, n_2, ..., n_\infty}$ with $\forall i:\;n_i\in \qty{0,1}$ the operators have the following qualities (where the phase factor $S = n_1 + n_2 + ... + n_{k-1}$):
			\begin{equation}
				\begin{aligned}
					\hat{a_k}^\dagger \Ket{...,0,...} &= (-1)^S \Ket{...,1,...}
					&\hspace{20pt}
					\hat{a_k} \Ket{...,1,...} &= (-1)^S \Ket{...,0,...}
					\\
					\hat{a_k}^\dagger \Ket{...,1,...} &= 0
					&\hspace{20pt}
					\hat{a_k} \Ket{...,0,...} &= 0
				\end{aligned}
			\end{equation}
			\begin{equation}
				\hat{a_k}^\dagger \hat{a_k} \Ket{...,n_k,...} = n_k \Ket{...,n_k,...}
			\end{equation}

			\noindent
			Slater determinant\index{Slater!Determinante}:
			\begin{equation}
				\Ket{\psi(x_1,x_2,...,x_N)} = \frac{1}{\sqrt{N}} \det\left( \begin{matrix}
					\Ket{\psi}_1(x_1) & \dotsb & \Ket{\psi}_1(x_N) \\
					\vdots & \ddots & \vdots \\
					\Ket{\psi}_N(x_1) & \dotsb & \Ket{\psi}_N(x_N) \\
				\end{matrix} \right)
			\end{equation}



		\subsubsection{Free Dirac Equation\index{Dirac!Gleichung}}
			\noindent
			Dirac Equation\index{Dirac!Equation} (Spin $s=1/2$):
			\begin{equation}
				\qty( i\slashed{\partial} - m ) \psi\qty(x) = 0
			\end{equation}

	\subsection{Second Quantization}
		For Creation and Annihilation operators see Sec.~\ref{Sec:BosonicCreationAndAnnihilationOperators} and Sec.~\ref{Sec:FermionicCreationAndAnnihilationOperators}

		\noindent
		Fock space\index{Fock!Raum} (direct sum of Hilbert spaces\index{Hilbert!Raum} for all possible numbers of particles $N$; $\mathcal{H}^{(0)} = \qty{\Ket{0}}$ consists only of the vacuum state):
		\begin{equation}
			\mathcal{F}
			= \bigoplus_{N=0}^{\infty} \mathcal{H}^{(N)}
			= \mathcal{H}^{(0)} \oplus \mathcal{H}^{(1)} \oplus \mathcal{H}^{(2)} \oplus ...
		\end{equation}


		\subsubsection{Field Operators}
			Definition of the field operators (summing over the complete set of single-particle quantum numbers $k$)
			\begin{equation}
				\begin{aligned}
					\hat{\psi} (x) &= \sum_k \psi_k(x) \hat{c}_k \\
					\hat{\psi}^\dagger (x) &= \sum_k \psi_k(x) \hat{c}_k^\dagger \\
				\end{aligned}
			\end{equation}

			\noindent
			Properties (spin $\alpha$, $\beta$; using commutator for bosons and anticommutator for fermions):
			\begin{equation}
				\begin{aligned}
					\comm{\hat{\psi}_\alpha(x)}{\hat{\psi}_\beta^\dagger(x')}_\mp &= \delta_{\alpha,\beta} \delta(x-x')
					&\hspace{20pt}
					\comm{\hat{\psi}_\alpha(x)}{\hat{\psi}_\beta(x')}_\mp
					&= \comm{\hat{\psi}_\alpha^\dagger(x)}{\hat{\psi}_\beta^\dagger(x')}_\mp
					&= 0
				\end{aligned}
			\end{equation}

			\noindent
			General form of a one-particle operator in second quantization
			\begin{equation}
				\begin{aligned}
					\hat{A} &= \sum_{i=1}^N \hat{J}(x_i) \\
					\hat{A} &= \sum_{r,s} \Bra{r}\hat{A}\Ket{s} \hat{c}_r^\dagger \hat{c}_s
					&= \int \hat{\psi}^\dagger(x) \hat{A}(x) \hat{\psi}(x) \; \dd x
				\end{aligned}
			\end{equation}

	\subsection{Interaction}
		\subsubsection{Mandelstam Variables\index{Mandelstam!Variablen}}
			\noindent
			Lorentz scalars\index{Lorentz!Skalare} in scattering processes if the form: $P_1 + P_2 \rightarrow P_3 + P_4$:
			\begin{itemize}\itemsep -0pt	% reduce space between items
				\item $s:=(p_1+p_2)^2=(p_3+p_4)^2$ \hfill{(Square of the invariant mass\index{Schwerpunktsenergie})}
				\item $t:=(p_1-p_3)^2=(p_2-p_4)^2$ \hfill{(Square of the four-momentum transfer)}
				\item $u:=(p_1-p_4)^2=(p_2-p_3)^2$ \hfill{(Four-momentum transfer of the rebound particle)}
			\end{itemize}

			\noindent
			Connection:
			\begin{equation}
				s+t+u = \sum_i m_i^2
			\end{equation}
