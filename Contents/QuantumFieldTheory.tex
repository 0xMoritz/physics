% !TEX root = ../physics.tex
\section{Quantum Field Theory\index{QFT}\index{Quantenfeldtheorie}} % QFt
	\subsection{Postulates}
		\subsubsection{Symmetries?}
			Locality
			CPT
			...
			TODO
	\subsection{Definitions}
		Feynman slash notation\index{Feynman!Slash Notation}
		\begin{equation}
			\slashed{A} := \gamma_\mu A^\mu
		\end{equation}

		\noindent
		Pauli matrices\index{Pauli!Matrizen}:
		\begin{equation}
			\begin{aligned}
				\sigma^0 = \left(\begin{matrix}
					1 & 0 \\
					0 & 1 \\
				\end{matrix}\right) &&\hspace{30pt}
				\sigma^1 = \left(\begin{matrix}
					0 & 1 \\
					1 & 0 \\
				\end{matrix}\right) &&\hspace{30pt}
				\sigma^2 = \left(\begin{matrix}
					0 & -i \\
						i & 0 \\
				\end{matrix}\right) &&\hspace{30pt}
				\sigma^3 = \left(\begin{matrix}
					1 & 0 \\
					0 & -1 \\
				\end{matrix}\right)
			\end{aligned}
		\end{equation}

		\subsubsection{Gamma Matrices}
			\noindent
			Defining properties:
			\begin{equation}
				\anticom{\gamma^\mu}{\gamma^\nu} = \gamma^\mu \gamma^\nu + \gamma^\nu \gamma^\nu = 2g^{\mu\nu}
			\end{equation}

			\noindent
			Further properties:
			\begin{equation}
				\begin{aligned}
					\Br{\gamma^0}^\dagger &= +\gamma^0 \\
					\Br{\gamma^j}^\dagger &= -\gamma^j \\
					\gamma^0 \gamma^0 &= 1_4 \\ % TODO: replace hat with mathbb{1}
					\gamma^i \gamma^j &= - \delta^{ij} 1_4 \\
					\Br{\gamma^\mu}^\dagger &= \gamma^0 \gamma^\mu \gamma^0 \\
				\end{aligned}
			\end{equation}

			\noindent
			Dirac $\gamma$-matrices\index{Dirac!Gamma Matrizen}:
			\begin{equation}
				\begin{aligned}
					\gamma^0
					= \left( \begin{matrix}
					\sigma^0 & \M0 \\
					0 &  -\sigma^0
					\end{matrix} \right)
					= \left( \begin{matrix}
					\M 1 & \M 0 & \M 0 & \M 0 \\
					\M 0 & \M 1 & \M 0 & \M 0 \\
					\M 0 & \M 0 &   -1 & \M 0 \\
					\M 0 & \M 0 & \M 0 &   -1 \\
					\end{matrix} \right)
					&\hspace{20pt}
					\gamma^1
					= \left( \begin{matrix}
					\M0 & \sigma^1 \\
					-\sigma^1 &  0
					\end{matrix} \right)
					= \left( \begin{matrix}
					\M 0 & \M 0 & \M 0 & \M 1 \\
					\M 0 & \M 0 & \M 1 & \M 0 \\
					\M 0 &   -1 & \M 0 & \M 0 \\
					-1 & \M 0 & \M 0 & \M 0 \\
					\end{matrix} \right) \\[8pt]
					\gamma^2
					= \left( \begin{matrix}
					\M0 & \sigma^2 \\
					-\sigma^2 &  0
					\end{matrix} \right)
					= \left( \begin{matrix}
					\M 0 & \M 0 & \M 0 &  -\i \\
					\M 0 & \M 0 & \M\i & \M 0 \\
					\M 0 & \M\i & \M 0 & \M 0 \\
					-\i & \M 0 & \M 0 & \M 0 \\
					\end{matrix} \right)
					&\hspace{20pt}
					\gamma^3
					= \left( \begin{matrix}
					\M 0 & \sigma^3 \\
					-\sigma^3 &  0
					\end{matrix} \right)
					= \left( \begin{matrix}
					\M 0 & \M 0 & \M 1 & \M 0 \\
					\M 0 & \M 0 & \M 0 &   -1 \\
					-1 & \M 0 & \M 0 & \M 0 \\
					\M 0 & \M 1 & \M 0 & \M 0 \\
					\end{matrix} \right) \\
				\end{aligned}
			\end{equation}

			\noindent
			Transformation ($U$ unitary: $U U^\dagger=1_4$):
			\begin{equation}
				\tilde{\gamma} = U \gamma^\mu U^\dagger
			\end{equation}

			\noindent
			$\gamma^5$-Matrix:
			\begin{equation}
				\gamma^5 = \gamma_5 = \gamma_5 = i\gamma^0 \gamma^1 \gamma^2 \gamma^3 = -\frac{1}{4!}\varepsilon_{\mu\nu\rho\sigma} \gamma^\mu \gamma^\nu \gamma^\rho \gamma^\sigma
			\end{equation}

			\noindent
			Properties of the $\gamma^5$-matrix:
			\begin{equation}
				\begin{aligned}
					\gamma_5^\dagger &= \gamma_5 \\
					\gamma_5 \gamma_5 &= 1_4 \\
					\anticom{\gamma^\mu}{\gamma^5} &= 0 \\
				\end{aligned}
			\end{equation}

			\noindent
			$\sigma^{\mu\nu}$ matrices:
			\begin{equation}
				\sigma^{\mu\nu} = \frac{i}{2}\com{\gamma^\mu}{\gamma^\nu} = -\sigma^{\nu\mu}
			\end{equation}


	\subsection{Feynman Path Integral\index{Feynman!Pfadintegral}}
		TODO

	\subsection{Bosons}
		\subsubsection{Free Klein-Gordon Equation\index{Klein!-Gordon Gleichung}}
			\noindent
			Klein-Gordon Equation (spin $s=0$)
			\begin{equation}
				\left(\partial^\mu\partial_\mu+m^2\right) \phi(x) = 0
			\end{equation}

			\noindent
			Solution (plane waves):
			\begin{equation}
				{\phi}^{ ( \pm ) }_{\vec{p}} (x) = e^{\mp p_\mu x^\mu}
			\end{equation}

	\subsection{Fermions}
		\subsubsection{Free Dirac Equation\index{Dirac!Gleichung}}
			\noindent
			Dirac Equation\index{Dirac!Equation} (Spin $s=1/2$):
			\begin{equation}
				\Br{ i\slashed{\partial} - m } \psi\Br{x} = 0
			\end{equation}

	\subsection{Interaction}
		\subsubsection{Mandelstam Variables\index{Mandelstam!Variablen}}
			\noindent
			Lorentz scalars\index{Lorentz!Skalare} in scattering processes if the form: $P_1 + P_2 \rightarrow P_3 + P_4$:
			\begin{itemize}\itemsep -0pt	% reduce space between items
				\item $s:=(p_1+p_2)^2=(p_3+p_4)^2$ \hfill{(Square of the invariant mass\index{Schwerpunktsenergie})}
				\item $t:=(p_1-p_3)^2=(p_2-p_4)^2$ \hfill{(Square of the four-momentum transfer)}
				\item $u:=(p_1-p_4)^2=(p_2-p_3)^2$ \hfill{(Four-momentum transfer of the rebound particle)}
			\end{itemize}

			\noindent
			Connection:
			\begin{equation}
				s+t+u = \sum_i m_i^2
			\end{equation}
