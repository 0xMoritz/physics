% !TEX root = ../physics.tex
\section{Quantum Field Theory\index{QFT}\index{Quantenfeldtheorie}} % QFt
	\subsection{Postulate}
		\subsubsection{Symmetrien?}
			Lokalität
			CPT
			...
			TODO
	\subsection{Definitionen}
		\begin{equation}
			\slashed{A} := \gamma_\mu A^\mu
		\end{equation}

		\noindent
		Paulimatrizen:
		\begin{equation}
			\begin{aligned}
				\sigma^0 = \left(\begin{matrix}
					1 & 0 \\
					0 & 1 \\
				\end{matrix}\right) &&\hspace{30pt}
				\sigma^1 = \left(\begin{matrix}
					0 & 1 \\
					1 & 0 \\
				\end{matrix}\right) &&\hspace{30pt}
				\sigma^2 = \left(\begin{matrix}
					0 & -i \\
						i & 0 \\
				\end{matrix}\right) &&\hspace{30pt}
				\sigma^3 = \left(\begin{matrix}
					1 & 0 \\
					0 & -1 \\
				\end{matrix}\right)
			\end{aligned}
		\end{equation}

		\subsubsection{Gamma Matrizen}
			\noindent
			Definierende Eigenschaft:
			\begin{equation}
				\anticom{\gamma^\mu}{\gamma^\nu} = \gamma^\mu \gamma^\nu + \gamma^\nu \gamma^\nu = 2g^{\mu\nu}
			\end{equation}

			\noindent
			Weitere Eigenschaften:
			\begin{equation}
				\begin{aligned}
					\Br{\gamma^0}^\dagger &= +\gamma^0 \\
					\Br{\gamma^j}^\dagger &= -\gamma^j \\
					\gamma^0 \gamma^0 &= 1_4 \\ % TODO: replace hat with mathbb{1}
					\gamma^i \gamma^j &= - \delta^{ij} 1_4 \\
					\Br{\gamma^\mu}^\dagger &= \gamma^0 \gamma^\mu \gamma^0 \\
				\end{aligned}
			\end{equation}

			\noindent
			Transformation ($U$ unitär $U U^\dagger=1_4$):
			\begin{equation}
				\tilde{\gamma} = U \gamma^\mu U^\dagger
			\end{equation}

			\noindent
			$\gamma^5$-Matrix:
			\begin{equation}
				\gamma^5 = \gamma_5 = \gamma_5 = i\gamma^0 \gamma^1 \gamma^2 \gamma^3 = -\frac{1}{4!}\varepsilon_{\mu\nu\rho\sigma} \gamma^\mu \gamma^\nu \gamma^\rho \gamma^\sigma
			\end{equation}

			\noindent
			Eigenschaften der $\gamma^5$-Matrix:
			\begin{equation}
				\begin{aligned}
					\gamma_5^\dagger &= \gamma_5 \\
					\gamma_5 \gamma_5 &= 1_4 \\
					\anticom{\gamma^\mu}{\gamma^5} &= 0 \\
				\end{aligned}
			\end{equation}

			\noindent
			$\sigma^{\mu\nu}$ Matrizen:
			\begin{equation}
				\sigma^{\mu\nu} = \frac{i}{2}\com{\gamma^\mu}{\gamma^\nu} = -\sigma^{\nu\mu}
			\end{equation}


	\subsection{Pfadintegral}
		TODO

	\subsection{Bosonen}
		\subsubsection{Freie Klein-Gordon-Gleichung}
			\noindent
			Klein Gordon Gleichung (spin $s=0$)
			\begin{equation}
				\left(\partial^\mu\partial_\mu+m^2\right) \phi(x) = 0
			\end{equation}

			\noindent
			Lösung (ebene Wellen):
			\begin{equation}
				{\phi}^{ ( \pm ) }_{\vec{p}} (x) = e^{\mp p_\mu x^\mu}
			\end{equation}

	\subsection{Fermionen}
		\subsubsection{Freie Dirac-Gleichung}
			\noindent
			Dirac-Gleichung (Spin $s=1/2$):
			\begin{equation}
				\Br{ i\slashed{\partial} - m } \psi\Br{x} = 0
			\end{equation}





	\subsection{Wechselwirkungen}
		\subsubsection{Mandelstam Variablen}
			\noindent
			Lorentzskalare bei Streuprozess der Form: $P_1 + P_2 \rightarrow P_3 + P_4$:
			\begin{itemize}\itemsep -0pt	% reduce space between items
				\item $s:=(p_1+p_2)^2=(p_3+p_4)^2$ \hfill{(Quadrat der Schwerpunktsenergie)}
				\item $t:=(p_1-p_3)^2=(p_2-p_4)^2$ \hfill{(Quadrat des Viererimpuls-Übertrags)}
				\item $u:=(p_1-p_4)^2=(p_2-p_3)^2$ \hfill{(Viererimpuls-Übertrag auf Rückstoßteilchen)}
			\end{itemize}

			\noindent
			Zusammenhang:
			\begin{equation}
				s+t+u = \sum_i m_i^2
			\end{equation}
