% !TEX root = ../physics.tex
\section{Quantum Mechanics\index{Quantenmechanik}}
	\subsection{Postulates}
		\begin{description}
			\item[Postulate 1]\hfill \\
				Every closed quantum system has an associated Hilbert space\index{Hilbert!Raum} $\mathcal{H}$. The state of the system at a given time $t$ is represented by an normed element $\Ket{\psi(t)} \in \mathcal{H}$, such that $\braket{\psi(t)|\psi(t)} = 1$.
			\item[Postulate 2]\hfill \\
				Every measurable physical quantity $\mathcal{A}$ is described by a linear, self-adjoint operator $\hat{A}$ on $\mathcal{H}$.
				$\hat{A}$ has a complete system of eigenvectors, \ie there exists a partition of unity and a spectral representation\index{Spektraldarstellung} of the operator composed by eigenvectors:
				\begin{equation}
					\begin{aligned}
						\hat{1} &= \SumInt_n \SumInt_\nu \Ket{a_n,\nu}\Bra{a_n,\nu} \\
						\hat{A} &= \SumInt_n \SumInt_\nu a_n\Ket{a_n,\nu}\Bra{a_n,\nu}. \\
					\end{aligned}
				\end{equation}
				$\hat{A}$ is called observable.
			\item[Postulate 3]\hfill \\
				The possible measurement values of $\mathcal{A}$ are the eigenvalues of $\hat{A}$.
			\item[Postulate 4]\hfill \\
				When measuring the observable $\hat{A}$ of a system in a state $\Ket{\psi}$, the probability of measuring the eigenvalue
				\begin{itemize}
					\item[i)] $a_n$, where $a_n$ is non-degenerate and discrete eigenvalue to the eigenstate $\Ket{a_n}$, is given by
						\begin{equation}
							w_{a_n}(\Ket{\psi}) = \left| \braket{a_n|\psi} \right|^2.
						\end{equation}
					\item[ii)] $a_n$, where $a_n$ is a degenerate and discrete eigenvalue to the eigenstate $\Ket{a_n,\nu}$, is given by
						\begin{equation}
							w_{a_n}(\Ket{\psi}) = \SumInt_\nu \left|\braket{a_n ,\nu|\psi}\right|^2.
						\end{equation}
					\item[iii)] $a$, where $a$ is a non-degenerate and continuous eigenvalue to the eigenstate $\Ket{a}$, is given by
						\begin{equation}
							\dd w_a(\Ket{\psi}) = \left| \braket{a|\psi} \right|^2 \dd a.
						\end{equation}
					\item[iv)] $a$, where $a$ is a degenerate and continuous eigenvalue to the eigenstate $\Ket{a, \nu}$, is given by
						\begin{equation}
							\dd w_a(\Ket{\psi}) = \qty( \SumInt_\nu \abs{\braket{a,\nu |\psi} }^2 ) \dd a.
						\end{equation}
				\end{itemize}
				Either summation or integration is performed depending on whether $\nu$ is discrete or continuous.
			\item[Postulate 5]\hfill \\
				If a measurement of an observable $\hat{A}$ yields the eigenvalue $a_n$, then the state of the system after the measurement will be the normed projection on the subspace to $a_n$
				\begin{equation}
					\begin{aligned}
						\Ket{\psi} & \to \frac{\hat{P}_n\Ket{\psi}} {\sqrt{\Bra{\psi}\hat{P}_n\Ket{\psi}}} \\
						\hat{P}_n &:= \SumInt_\nu \Ket{a_n,\nu}\Bra{a_n,\nu}. \\
					\end{aligned}
				\end{equation}
			\item[Postulate 6]\hfill \\
				The time evolution of a closed quantum system is governed by the Schrödinger equation\index{Schrödinger!Gleichung}
				\begin{equation}
					\i\hbar\dv{t}\Ket{\psi} = \hat{H}\Ket{\psi},
				\end{equation}
				where, the Hamiltonian $\hat{H}$ corresponds to the observable $\mathcal{H}$ associated with the total energy of the system.
			\item[Postulate 7]\hfill \\
				The Hilbert space of a composed quantum system is given by the tensor product of the sub-systems
				\begin{equation}
					\mathcal{H} = \mathcal{H}_1 \otimes \mathcal{H}_2.
				\end{equation}
			\item[Postulate 8]\hfill \\
				Micro-objects with identical properties are described by either totally symmetric (bosonic) or by totally antisymmetric (fermionic) states. Fermions\index{Fermionen} have half-integer, bosons\index{Bosonen} have integer spin values.
		\end{description}

	\subsection{Basics}
		\noindent
		De Broglie wavelength\index{De Broglie!Wellenlänge} (Scale at which quantum effects become important):
		\begin{equation}
			p = \frac{h}{\lambda}
			\hsp
			\vec{p} = \hbar \vec{k}
		\end{equation}

		\noindent
		Compton wavelength\index{Compton!Wellenlänge} (Scale at which quantum relativistic effects become important):
		\begin{equation}
			\lambda_\mathrm{C} = \frac{h}{mc}
		\end{equation}

		\noindent
		Energy of a photon:
		\begin{equation}
			E_\gamma = \hbar\omega = h\nu
		\end{equation}

		\noindent
		Classical Schrödinger equation in position space representation:
		\begin{equation}
			\i\hbar\pdv{t}\psi(\vec{x}) = -\frac{\hbar^2}{2m}\frac{\partial^2}{\partial \vec{x}^2}\psi(\vec{x}) + V(x)\psi(\vec{x})
		\end{equation}

		\noindent
		Heisenberg uncertainty principle\index{Heisenberg!Unschärferelation} ($\Delta_{\Ket{\psi}} (\hat{A}) = \sqrt{\Bra{\psi}\hat{A}^2\Ket{\psi} - \Bra{\psi}\hat{A}\Ket{\psi}^2}$):
		\begin{equation}
			\begin{aligned}
				\Delta_{\Ket{\psi}}(\hat{A}) \Delta_{\Ket{\psi}}(\hat{B}) &\ge
				\frac{1}{2} \abs{\mel**{\psi}{\comm{\hat{A}}{\hat{B}}}{\psi}} \\
				\Delta_{\Ket{\psi}}(\hat{x}) \Delta_{\Ket{\psi}}(\hat{p}) &\ge
				\frac{\hbar}{2}
			\end{aligned}
		\end{equation}

		\noindent
		Energy time uncertainty ($\Delta t$ is the required measurement time for the reduction of the energy uncertainty to $\Delta E$):
		\begin{equation}
			\Delta E \Delta t \ge \frac{\hbar}{2}
		\end{equation}

		\noindent
		Unitarity:
		\begin{equation}
			\begin{aligned}
				\Ket{\psi(t)} &= \hat{U}(t) \Ket{\psi(0)} \\
				\hat{1} &= \hat{U}(t)\hat{U}^\dagger(t) = \hat{U}^\dagger(t)\hat{U}(t) \\
			\end{aligned}
		\end{equation}

		\noindent
		Position space and momentum space representations\index{Ortsdarstellung}\index{Impulsdarstellung}:
		\begin{equation}
			\begin{aligned}
				\Braket{\vec{x} | \psi} &= \psi(\vec{x}) &\hsp
				\Braket{\vec{x} | \vec{x}'} &= \delta(\vec{x} - \vec{x}') \\
				\Braket{\vec{p} | \psi} &= \tilde{\psi}(\vec{p}) &\hsp
				\Braket{\vec{p} | \vec{p}'} &= \delta(\vec{p} - \vec{p}') \\
			\end{aligned}
		\end{equation}
		\begin{equation}
			\begin{aligned}
				\Braket{\vec{x} | \vec{p}} &= \left( \frac{1}{\sqrt{2\pi\hbar}} \right)^\mathrm{dim} \ex^{\i \vec{p}\cdot\vec{x} / \hbar} \\
			\end{aligned}
		\end{equation}

		\noindent
		Classical limit:
		\begin{equation}
			\hbar \to 0
		\end{equation}

		\subsubsection{Continuity Equation}
			\noindent
			Probability density (Born rule\index{Born!Wahrscheinlichkeitsinterpretation}) and probability current density:
			\begin{equation}
				\begin{aligned}
					\rho(t,x) :=& \left|\psi(t,x)\right|^2 \\
					j(t,x) :=& \frac{\hbar}{2mi}\left(
					\psi^*(t,x)\pdv{x}\psi(t,x) - \psi(t,x)\pdv{x}\psi^*(t,x)
					\right) \\
					=& \frac{\hbar}{m} \mathrm{Im}\left(
					\psi^*(t,x) \pdv{x}\psi(t,x)
					\right)\\
				\end{aligned}
			\end{equation}

			\noindent
			Continuity equation\index{Kontinuitätsgleichung!Quantenmechanik}:
			\begin{equation}
				\pdv{t}\rho(t,x) + \pdv{x}j(t,x) = 0
			\end{equation}

	\subsection{Operators}
		\noindent
		Position operator\index{Ortsoperator}:
		\begin{equation}
			\begin{aligned}
				\hat{\vec{x}} :=& \int_{\reals ^n} \Ket{\vec{x}} \vec{x} \Bra{\vec{x}\,}\dd[n]{\vec{x}} &\hsp
				\Bra{\vec{x}\,}\hat{\vec{x}}\Ket{\psi} =& \,\vec{x}\, \psi(\vec{x}) &\hsp
				\Bra{\vec{p}\,}\hat{\vec{x}}\Ket{\psi} =& \,\i\hbar\pdv{\vec{p}}\tilde{\psi}(\vec{p}) \\
			\end{aligned}
		\end{equation}

		\noindent
		Momentum operator\index{Impulsoperator}:
		\begin{equation}
			\begin{aligned}
				\hat{\vec{p}} :=& \int_{\reals ^n} \Ket{\vec{p}} \vec{p} \Bra{\vec{p}\,}\dd^n\vec{p} &\hsp
				\Bra{\vec{p}\,}\hat{\vec{p}}\Ket{\psi} =& \,\vec{p}\, \tilde{\psi}(\vec{p}) &\hsp
				\Bra{\vec{x}}\hat{\vec{p}}\Ket{\psi} =& -\i\hbar\pdv{\vec{x}} \psi(\vec{x}) \\
			\end{aligned}
		\end{equation}

		\noindent
		Classical (non-relativistic) Hamiltonian:
		\begin{equation}
			\hat{H} = \frac{\hat{p}^2}{2m}+\hat{V}(\vec{x}) = -\frac{\hbar^2}{2m}\frac{\partial^2}{\partial \vec{x}^2} + V(\vec{x})
		\end{equation}

		\noindent
		Definition of the time evolution operator\index{Zeitentwicklungsoperator}:
		\begin{equation}
			\Ket{\psi(t)} =: \hat{U}(t)\Ket{\psi(0)}
		\end{equation}
		Time evolution operator for $\left[\hat{H}(t),\hat{H}(t')\right] = 0\quad\forall t,t'$:
		\begin{equation}
			\hat{U}(t) = \exp\qty(-\frac{\i}{\hbar}\int_0^\trp \hat{H}(t')\dd{t'})
		\end{equation}
		Further simplification if $\pdv{\hat{H}}{t}=0$:
		\begin{equation}
			\hat{U}(t) = \exp\qty(-\frac{\i}{\hbar}\hat{H} t)
		\end{equation}

		\noindent
		Canonical commutation relations\index{Kanonische Kommutatorrelationen} (As far as I understand it, depending on the approach, they are either axiomatic or follow from the correspondence principle which then must in turn be axiomatic):
		\begin{equation}
			\begin{aligned}
				\comm{\hat{x}_i}{\hat{x}_j } &= \comm{\hat{p}_i}{\hat{p}_j} = 0 \\
				\comm{\hat{x}_i}{\hat{p}_j } &= \i\hbar\, \delta_{ij}\hat{1} \\
			\end{aligned}
		\end{equation}

		\noindent
		Parity operator\index{Paritätsoperator} and properties (Definition, self-adjointness, unitarity, eigenvalue, eigenstate):
		\begin{equation}
			\begin{aligned}
				\Bra{\vec{x}} \hat{\Pi} \Ket{\Psi} :=& \Braket{-\vec{x} | \psi} = \psi(-\vec{x}) \\
				\hat{\Pi} =& \hat{\Pi}^\dagger\\
				1 =& \hat{\Pi} \,\hat{\Pi}^\dagger = \hat{\Pi}^\dagger \hat{\Pi} \\
				\psi(\vec{x}) = \psi(-\vec{x}) \implies&\; \hat{\Pi}\Ket{\psi} = \Ket{\psi} \\
				\psi(\vec{x}) = -\psi(-\vec{x}) \implies&\; \hat{\Pi}\Ket{\psi} = -\Ket{\psi} \\
			\end{aligned}
		\end{equation}

		\noindent
		Displacement operator\index{Translationsoperator}:
		\begin{equation}
			\begin{aligned}
				\hat{T}_{\vec{y}}\Ket{\vec{x}} :=& \Ket{\vec{x} + \vec{y}} = \ex^{-\i\hat{\vec{p}}\cdot\vec{y}/\hbar}\Ket{\vec{x}} \\
			\end{aligned}
		\end{equation}

		\subsubsection{Angular Momentum Operators\index{Drehimpulsoperatoren}}
			\noindent
			Defining property (Angular momentum commutator relations):
			\begin{equation}
				\left[ \hat{j}_i, \hat{j}_j \right] = \i\hbar\, \levicivita_{ijk} \hat{j}_k
			\end{equation}

			\noindent
			Orbital angular momentum operator \index{Bahndrehimpulsoperator}:
			\begin{equation}
				\begin{aligned}
					\hat{\vec{L}} &= \hat{\vec{x}}\times \hat{\vec{p}} = \vec{x}\times\left(-\i\hbar\pdv{\vec{x}}\right) \\
					\left< \hat{\vec{L}}^2 \right> &= l(l+1)\hbar^2 \\
					\left< \hat{L}_z \right> &= m_l \hbar \\
				\end{aligned}
			\end{equation}

			\noindent
			Relation to the Laplace operator\index{Laplace!Operator}:
			\begin{equation}
				\begin{aligned}
					\hat{\vec{p}}^{\,2} &= -\hbar^2 \Nabla^2 = \hat{\vec{p}}_r^{\,2} + \frac{1}{r^2} \hat{\vec{L}}^2 \\
					&= -\hbar^2 \qty[{\displaystyle \frac{1}{r^{2}} \frac{\partial }{\partial r}\!\left(r^{2}\frac{\partial }{\partial r}\right)\!+\!\frac{1}{r^{2}\!\sin \theta } \frac{\partial }{\partial \theta }\!\left(\sin \theta \frac{\partial }{\partial \theta }\right)\!+\!\frac{1}{r^{2}\!\sin ^{2}\theta }\frac{\partial ^{2}}{\partial \varphi ^{2}}}]\\
					\hat{\vec{p}}_r &= -\i\hbar\frac{1}{r}\pdv{r}r \\
				\end{aligned}
			\end{equation}

			\noindent
			Spin operator \index{Spinoperator} (For a magnetic field $\vec{B}\parallel\vec{e}_z$):
			\begin{equation}
				\begin{aligned}
					\left<\hat{\vec{S}}^2\right> &= s(s+1)\hbar^2 \\
					\left<\hat{S}_z\right> &= m_s \hbar \\
				\end{aligned}
			\end{equation}

			\noindent
			Spin representation in $\vec{e}_3$-basis: $\mathcal{B} = \{ \Ket{\vec{e}_3,+}, \Ket{\vec{e}_3,-} \}$:
			\begin{equation}
				\begin{aligned}
					\hat{\vec{\sigma}} \cdot \vec{n} &\doteq
					\left( \begin{matrix}
							\cos\theta             & \ex^{-\i\phi}\sin\theta \\
							\ex^{\i\phi}\sin\theta & -\cos\theta             \\
						\end{matrix} \right) \\
					\exp\left( \i\alpha \; \hat{\vec{\sigma}} \cdot \vec{n} \right) &= \cos\alpha \;\hat{1} + \i\sin\alpha \; \hat{\vec{\sigma}} \cdot \vec{n} \\
				\end{aligned}
			\end{equation}

			\noindent
			Pauli spin matrices\index{Pauli!Spinmatrizen} (spin operator $\hat{\vec{s}} = \frac{\hbar}{2} \hat{\vec{\sigma}}$):
			\begin{equation}
				\begin{aligned}
					\hat{\sigma}_1 &= \phantom{-\i}\Ket{\vec{e}_3,+} \Bra{\vec{e}_3,-} + \phantom{\i}\Ket{\vec{e}_3,-} \Bra{\vec{e}_3,+} \\
					\hat{\sigma}_2 &= -\i\Ket{\vec{e}_3,+} \Bra{\vec{e}_3,-} + \i\Ket{\vec{e}_3,-} \Bra{\vec{e}_3,+} \\
					\hat{\sigma}_3 &= \phantom{-\i}\Ket{\vec{e}_3,+} \Bra{\vec{e}_3,+} - \phantom{\i}\Ket{\vec{e}_3,-} \Bra{\vec{e}_3,-} \\
				\end{aligned}
			\end{equation}

			\noindent
			Matrix representation of the Pauli matrices\index{Pauli!Matrizen}:
			\begin{equation}
				\label{Eq:PauliMatricesQFT}
				\begin{aligned}
					\sigma^0 \doteq \mqty(
					1 & 0 \\
					0 & 1 \\
					) &&\hsp
					\sigma^1 \doteq \mqty(
					0 & 1 \\
					1 & 0 \\
					) &&\hsp
					\sigma^2 \doteq \mqty(
					0 & -\i \\
					\i & 0 \\
					) &&\hsp
					\sigma^3 \doteq \mqty(
					1 & 0 \\
					0 & -1 \\
					)
				\end{aligned}
			\end{equation}

			\noindent
			Properties of the Pauli matrices\index{Pauli!Spinmatrizen} (see also Eq.~\ref{Eq:PauliMatricesQFT}):
			\begin{align}
				\label{Eq:PauliMatrices}
				\hat{\sigma}_j^\dagger &= \hat{\sigma}_j \\
				\det\qty(\hat{\sigma}_j) &= -1 \\
				\tr\qty(\hat{\sigma}_j) &= 0 \\
				\qty(\hat{\sigma}_1)^2 = \qty(\hat{\sigma}_2)^2 = \qty(\hat{\sigma}_3)^2 &= -\i\hat{\sigma}_1\hat{\sigma}_2\hat{\sigma}_3 = \hat{1} \\
				\hat{\sigma}_j \hat{\sigma}_k &= \delta_{jk}\hat{1} + \i\levicivita_{jkl}\hat{\sigma}_l \\
				\comm{\sigma_\mu}{\sigma_\nu} &= 2\i\levicivita_{\mu\nu\lambda}\sigma_\lambda \\
				\acomm{\sigma_i}{\sigma_j} &= 2\delta_{ij}\hat{1} \\
				\sigma_\mu \sigma_\nu \sigma_\lambda &= \i\levicivita_{\xi\mu\nu\lambda}\sigma_\xi \\
			\end{align}

			\noindent
			Pauli Vector\index{Pauli!Vektor}
			\begin{equation}
				V
				= \sum_{i=1}^3 x_i \hat{\sigma}_i
				\doteq \left( \begin{matrix}
						x_3          & x_1 - \i x_2 \\
						x_1 + \i x_2 & - x_3        \\
					\end{matrix} \right)
			\end{equation}

			\noindent
			Spinor\index{Spinor} representation for $\det V = 0$ (not unique)
			\begin{equation}
				\begin{aligned}
					V &= \mqty(\xi^1\\\xi^2) \mqty(-\xi^2&\xi^1) \\
					\xi^1 &= \sqrt{x_1 - \i x_2} \quad \xi^2 = -\i \sqrt{x_1 + \i x_2} \\
				\end{aligned}
			\end{equation}


			\noindent
			Total angular momentum operator:
			\begin{equation}
				\begin{aligned}
					\hat{\vec{J}} &= \hat{\vec{L}} + \hat{\vec{S}} \\
					\left< \hat{\vec{J}}^2 \right> &= j(j+1)\hbar^2 \\
					\left< \hat{J}_z \right> &= m_j\hbar \\
				\end{aligned}
			\end{equation}

		\subsubsection{Commutator}
			\noindent
			Definition of the commutator:
			\begin{equation}
				\comm{\hat{A}}{\hat{B}} = \hat{A}\hat{B} - \hat{B}\hat{A}
			\end{equation}

			\noindent
			Definition of the anticommutators:
			\begin{equation}
				\acomm{\hat{A}}{\hat{B}} = \hat{A}\hat{B} + \hat{B}\hat{A}
			\end{equation}

			\noindent
			Identities:
			\begin{equation}
				\begin{aligned}
					\comm{\hat{A}}{\hat{B}\hat{C}}
					&= \hat{B}\comm{\hat{A}}{\hat{C}} + \comm{\hat{A}}{\hat{B}}\hat{C} \\
					\acomm{\hat{A}}{\hat{B}\hat{C}}
					&= \hat{B}\comm{\hat{A}}{\hat{C}} + \acomm{\hat{A}}{\hat{B}}\hat{C} \\
				\end{aligned}
				\hsp
				\begin{aligned}
					\comm{\hat{A}}{\hat{B}}^\dagger &= -\comm{\hat{A}^\dagger}{\hat{B}^\dagger} \\
					\acomm{\hat{A}}{\hat{B}}^\dagger &= \acomm{\hat{A}^\dagger}{\hat{B}^\dagger} \\
				\end{aligned}
			\end{equation}

			\noindent
			Commutators of functions (assuming $\comm{\comm{\hat{A}}{\hat{B}}}{\hat{B}}=0$)
			\begin{equation}
				\comm{\hat{A}}{f(\hat{B})}
				= \comm{\hat{A}}{\hat{B}}\dv{f}{\hat{B}}(\hat{B}) \\
			\end{equation}

			\noindent
			Jacobi identity\index{Jacobi!Identität}:
			\begin{equation}
				0 = \comm{\hat{A}}{\comm{\hat{B}}{\hat{C}}} + \comm{\hat{B}}{\comm{\hat{C}}{\hat{A}}} + \comm{\hat{C}}{\comm{\hat{A}}{\hat{B}}}
			\end{equation}


			\noindent
			Baker--Campbell--Hausdorff\index{Baker!--Campbell--Hausdorff formel}\index{Campbell!Baker--Campbell--Hausdorff formel}\index{Hausdorff!Baker--Campbell--Hausdorff formel} formula (assuming $\comm{\hat{A}}{\comm{\hat{A}}{\hat{B}}} = \comm{\hat{B}}{\comm{\hat{B}}{\hat{A}}} = 0$):
			\begin{equation}
				\exp(\hat{A})\exp(\hat{B}) = \exp(\hat{B}) \exp(\hat{A}) \exp(\comm{\hat{A}}{\hat{B}})
			\end{equation}

		\subsubsection{Operators in the Heisenberg Picture\index{Heisenberg!Bild}}
			\noindent
			Time dependent operators:
			\begin{equation}
				\begin{aligned}
					\hat{A}_H(t) &= \hat{U}^\dagger \hat{A} \hat{U} \\
					\langle \hat{A} \rangle_{\Ket{\psi(t)}} &= \Bra{\psi(0)}\hat{A}_H(t)\Ket{\psi(0)}
				\end{aligned}
			\end{equation}

			\noindent
			Time evolution of observables (Heisenberg equation\index{Heisenberg!Gleichung}):
			\begin{equation}
				\i\hbar \dv{t} \hat{A}_H(t) = \left(\left[\hat{A}, \hat{H}\right] + \i\hbar \pdv{t} \hat{A}\right)_H (t)
			\end{equation}

			\noindent
			Ehrenfest theorem\index{Ehrenfest!Theorem}:
			\begin{equation}
				\begin{aligned}
					\dv{t} \hat{\vec{x}}_H(t) &= \frac{1}{m}\hat{\vec{p}}_H(t) \\
					\dv{t} \hat{\vec{p}}_H(t) &= - \pdv{\hat{\vec{x}}} V \qty(\hat{\vec{x}}_H(t)) \\
				\end{aligned}
			\end{equation}

			\noindent
			Ehrenfest equations\index{Ehrenfest!Gleichungen}:
			\begin{equation}
				\begin{aligned}
					\dv{t} \langle \hat{\vec{x}} \,\rangle_{\Ket{\psi(t)}} &= \frac{1}{m} \langle \hat{\vec{p}} \,\rangle_{\Ket{\psi(t)}} \\
					\dv{t} \langle \hat{\vec{p}} \,\rangle_{\Ket{\psi(t)}} &= - \left\langle \pdv{\hat{\vec{x}}} V(\hat{\vec{x}}) \,\right\rangle_{\Ket{\psi(t)}} \\
				\end{aligned}
			\end{equation}

		\subsubsection{Conserved quantities}
			\noindent
			Symmetry transformation of a self-adjoint generator $\hat{G}$ ($\hat{T}$ becomes unitary):
			\begin{equation}
				\hat{T}(\nu) = \ex^{-\i\hat{G}\nu/\hbar}
			\end{equation}

			\noindent
			Condition of invariance \index{Invarianzbedingung} and conserved quantity (Heisenberg picture\index{Heisenberg!Bild}):
			\begin{equation}
				\hat{H} =\hat{T}^\dagger(\nu)\hat{H} \hat{T}(\nu) \iff \comm{\hat{H}}{\hat{G}} = 0
			\end{equation}

	\subsection{Bound States}
		\subsubsection{Basics}
			\noindent
			Bound states have negative interaction energy $E<0$ and discrete energy levels. For a time dependent Hamiltonian, the probability density is stationary.
			$\left|\psi(x)\right|^2=\const.$

		\subsubsection{Wave Function of an Electron in an Atom}
			\noindent
			For the hydrogen atom (Bohr radius\index{Bohr!Radius} $\rho$, normalization constant $a_0$):
			\begin{equation}
				\begin{aligned}
					\qty{-\frac{\hbar^2 \Nabla^2}{2 m_e} + V(r)}\psi(\vec{r})
					&= \qty{-\frac{\hbar^2}{2 m_e}\frac{1}{r^2}\pdv{r}\qty(r^2\pdv{r}) + \frac{1}{2 m_e r^2} \hat{L}^2 - \frac{\ex^2}{4\pi\varepsilon_0}\frac{1}{r}}\psi(\vec{r})
					= E \psi\qty(\vec{r})
					\\
					\Braket{\vec{r}|n,l,m_l,m_s} &= \psi_{n,l,m}\left(r,\theta,\phi\right)\chi(m_s)
					= R_{n,l}\left(r\right) Y_l^m\left(\theta,\phi\right)\chi(m_s) \\
					R_{n,l}\left(r\right)
					&= \left(\frac{1}{n\rho}\right)^{\frac{3}{2}}
					\sum_{j=0}^{n-l-1} a_j \left(\frac{r}{n\rho}\right)^{j+l} \exp\left({-\frac{r}{n\rho}}\right) \\
					a_j &= 2\frac{j+l-n}{j(j+2l+1)} a_{j-1} \\
				\end{aligned}
			\end{equation}

			\noindent
			Quantum numbers:
			\begin{itemize}
				\item Principal quantum number $n \in \\naturals $
				\item Orbital angular momentum quantum number $l \in \integers;\; \left|l\right| < n$
				\item Magnetic quantum number $m_l \in \integers;\; \left|m\right| \le \left|l\right|$
				\item spin quantum number $s = \frac{1}{2}$ (For fermions)
				\item spin projection $m_s = \pm \frac{1}{2}$ (For fermions)
			\end{itemize}

			\noindent
			Energy states of a single electron in an atom ($\mathrm{H}$, $\mathrm{He^{+}}$, $\mathrm{Li^{2+}}$,...) with atomic number $Z$ (Rydberg energy\index{Rydberg!Energie} $\eval{R^*=R_y}_{m_e \to \mu}$ and reduced mass $\mu$ of the atomic system):
			\begin{equation}
				\mathcal{E}_n = -R^* Z^2 \frac{1}{n^2} = -R^* \frac{\mu}{m_e} Z^2 \frac{1}{n^2} = - \frac{\mu e^4}{8 h^2 \varepsilon_0^2} Z^2 \frac{1}{n^2}
			\end{equation}

			\noindent
			Electron configuration:
			\begin{itemize}
				\item Aufbau principle\index{Aufbauprinzip} / Pauli principle\index{Pauli!Prinzip}: The total wave function of a system of electrons is totally antisymmetric with respect to permutation of two electrons.
				\item Madelung rule\index{Madelung!Energieschema}: For the ground state, orbitals with the lowest value of $n+l$ are filled first.
				\item Hund's rules\index{Hund!Regel}:
					\begin{itemize}
						\item For a given electron configuration, the term with maximum multiplicity has the lowest energy. The multiplicity is equal to $2S+1$, where $S$ is the total spin angular momentum for all electrons. The multiplicity is also equal to the number of unpaired electrons plus one. Therefore, the term with lowest energy is also the term with maximum $S$, and maximum number of unpaired electrons.
						\item For a given multiplicity, the term with the largest value of the total orbital angular momentum quantum number  $L$, has the lowest energy.
						\item For a given term, in an atom with outermost subshell half-filled or less, the level with the lowest value of the total angular momentum quantum number  $J$, (for the operator $\vec{J} = \vec{L} + \vec{S}$) lies lowest in energy. If the outermost shell is more than half-filled, the level with the highest value of  $J$, is lowest in energy.
					\end{itemize}
			\end{itemize}

			\noindent
			Selection rules\index{Auswahlregeln} for the transition of atoms by emission or absorption of a photon:
			\begin{itemize}
				\item $\Delta l = \pm 1$
				\item $\Delta m = 0, \pm 1$
				\item $\Delta S = 0$
				\item $\Delta J = 0, \pm 1$
				\item $\Delta L = \pm 1$
				\item $J=0 \nrightarrow J=0$
			\end{itemize}

			\noindent
			\href{https://www.nist.gov/pml/atomic-spectra-database}{NIST database for energy levels and spectral lines}

		\subsubsection{Quantum Mechanical Harmonic Oscillator}
			\noindent
			Hamiltonian and its construction from creation operator $\hat{a}^\dagger$ and annihilation operator $\hat{a}$\index{Leiteroperatoren}:
			\begin{equation}
				\begin{aligned}
					\hat{H} =& \frac{\hat{p}^2}{2m} + \frac{1}{2}m\omega^2 \hat{x}^2 = \hbar\omega\qty(\hat{a}^\dagger \hat{a} + \frac{1}{2}) \\
					\hat{a} :=& \sqrt{\frac{m\omega}{2\hbar}}\hat{x} + \frac{\i}{\sqrt{2m\hbar\omega}}\hat{p} \\
					\hat{a}^\dagger :=& \sqrt{\frac{m\omega}{2\hbar}}\hat{x} - \frac{\i}{\sqrt{2m\hbar\omega}}\hat{p} \\
				\end{aligned}
			\end{equation}

			\noindent
			Representation of position and momentum operators by the creation and annihilation operators:
			\begin{equation}
				\begin{aligned}
					\hat{x} &= \sqrt{\frac{\hbar}{2m\omega}}\left(\hat{a}^\dagger + \hat{a} \right) \\
					\hat{p} &= \i\sqrt{\frac{m\hbar\omega}{2}}\left(\hat{a}^\dagger - \hat{a} \right) \\
				\end{aligned}
			\end{equation}

			\noindent
			Properties of the creation and annihilation operators:
			\begin{equation}
				\begin{aligned}
					\hat{a}\Ket{n} &= \sqrt{n}\Ket{n-1} \\
					\hat{a}^\dagger\Ket{n} &= \sqrt{n+1}\Ket{n+1} \\
					\Bra{m}\hat{a}\Ket{n} &= \sqrt{n} \,\delta_{m,n-1} \\
					\Bra{m}\hat{a}^\dagger\Ket{n} &= \sqrt{n+1} \,\delta_{m,n+1} \\
				\end{aligned}
			\end{equation}

			\noindent
			Eigenstates and energy eigenvalues (With the substitution $q:=\sqrt{\frac{m\omega}{\hbar}}x$ and the Hermite polynomials\index{Hermite!Polynome} $H_n(q)$ (Eq.~\ref{Eq:HermitePolynomials})):
			\begin{equation}
				\begin{aligned}
					\Ket{n} &= \frac{1}{\sqrt{n!}}\left(\hat{a}^\dagger\right)^n\Ket{0} \\
					\Braket{x|n} &= \sqrt[4]{\frac{m\omega}{2\hbar}} \frac{1}{\sqrt{n! 2^n}} \ex^{-\frac{q^2}{2}} H_n(q) \\
				\end{aligned}
			\end{equation}

	\subsection{States in the Atom}
		\subsubsection{Zeeman Effect\index{Zeeman!Effekt}}
			\noindent
			Ordinary Zeeman effect: $\vec{s} = 0$ \\
			Anomalous Zeeman effect: $\vec{s}\ne 0$ \\
			\noindent
			Magnetic dipole moment of the electron orbit and spin respectively and energy shift in case of $\vec{j}=\vec{l}$ or $\vec{j}=\vec{s}$ respectively (Sec.~\ref{Sec:ParticleConstants}):
			\begin{equation}
				\begin{aligned}
					\vec{\mu}_l &= -\mu_\text{B} g_l \vec{l} &\hsp V_l &= m_l g_l \mu_\text{B} B\\
					\vec{\mu}_s &= -\mu_\text{B} g_s \vec{s} &\hsp V_s &= m_s g_s \mu_\text{B} B\\
				\end{aligned}
			\end{equation}

			\noindent
			Interaction Energy (see Eq.~\ref{Eq:MagneticDipoleInteraction}):
			\begin{equation}
				\begin{aligned}
					V_l &= -\vec{\mu} \cdot \vec{B} \\
				\end{aligned}
			\end{equation}

			\noindent
			Landé factor\index{Landé!Faktor} / g-factor (For $g_s\approx 2$ approximation):
			\begin{equation}
				\begin{aligned}
					\abs{\qty(\vec{\mu}_j)_j} &= \mu_\text{B}\frac{3j(j+1)+s(s+1)-l(l+1)}{2\sqrt{j(j+1)}} = \mu_\text{B} g_j \frac{\abs{\vec{j}}}{\hbar}\\
					g_j &= 1+\frac{j(j+1) + s(s+1) - l(l+1)}{2j(j+1)} \\
				\end{aligned}
			\end{equation}

			\noindent
			Paschen--Back effect\index{Paschen!--Back Effekt}\index{Back!Paschen--Back Effekt}: For strong magnetic fields, spin and orbital angular momentum decouple and precess independently around the magnetic field. In this case:
			\begin{equation}
				V_{m_s,m_l} = -\qty(\vec{\mu}_s + \vec{\mu}_l)\cdot\vec{B}
				=\qty(g_s m_s + g_l m_l) \mu_\text{B} B
			\end{equation}

	\subsection{Perturbation Theory\index{Störungstheorie}}
		\subsubsection{Periodic Pertubations}
			\noindent
			Fermi's golden rule\index{Fermi!Goldene Regel} (First order approximation for the transition rate $P_{fi}$ from the initial state $\Ket{i}$ to the final state $\Ket{f}$ with a periodic perturbation of the form $\hat{H}' = \ex^{\pm\i\omega}$):
			\begin{equation}
				P_{fi} = \lim_{t \to \infty} \frac{W_{fi}(t)}{t} = \frac{2\pi}{\hbar} \abs{\Bra{f^0}\hat{H}'\Ket{i^0}}^2 \delta\qty(E_f^0 - E_i^0 \pm \hbar\omega)
			\end{equation}

		\subsubsection{Time Dependent Perturbation Theory}
			Interaction Picture:
			\begin{equation}
				\begin{aligned}
					\hat{H} =&\, \hat{H}_0 + \hat{H}' \\
					\Ket{\psi_I(t)} :=&\, \ex^{\i \hat{H}_0 t / \hbar} \Ket{\psi(t)}
					= \hat{U}_0^\dagger \Ket{\psi(t)} \\
					\Hat{A}_I(t) :=&\, \ex^{\i \hat{H}_0 t / \hbar} \hat{A} \ex^{-\i \hat{H}_0 t / \hbar}
					= \hat{U}_0^\dagger \hat{A} \hat{U}_0
				\end{aligned}
			\end{equation}

			\noindent
			Dyson Equation\index{Dyson!Gleichung} (The time evolution operator $\hat{U} = \hat{U}_I$ is actually in the interaction picture):
			\begin{equation}
				\begin{aligned}
					\Ket{\psi_I(t)} &=: \hat{U}(t,t_0) \Ket{\psi_I(t_0)} \\
					\i \hbar \pdv{t} \hat{U}(t,t_0) &= \hat{H}_I'(t) \hat{U}(t,t_0) \\
					\hat{U}(t,t_0) &= \hat{1} - \frac{\i}{\hbar} \int_{t_0}^{t} \hat{H}'_I(t') \hat{U}(t',t_0) \dd{t'} \\
					&= T \qty[\exp\qty(-\frac{\i}{\hbar}\int_{t_0}^{t}\dd t'\,\hat{H}'_I(t'))]
				\end{aligned}
			\end{equation}

	\subsection{Scattering\index{Streuung}}
		\noindent
		Lippmann--Schwinger equation\index{Lippmann!--Schwinger Gleichung}\index{Schwinger!Lippmann--Schwinger Gleichung}
		\begin{equation}
			\Ket{\vec{p},\sigma,\mu,\pm} = \Ket{\vec{p}, \sigma, \mu} + \lim_{\eta \to 0} G_0(\epsilon_{\vec{p},\mu}\pm \i\eta) V \Ket{\vec{p},\sigma,\mu,\pm}
		\end{equation}

		\noindent
		Rutherford cross section\index{Rutherford!Streuquerschnitt} (Form factor\index{Formfaktor} $F(\vec{q}) = \int_{\reals ^3} \rho(\vec{r}) \ex^{-\i \vec{q}\cdot\vec{r}/\hbar}\dd[3]{\vec{r}}$):
		\begin{equation}
			\dv{\sigma}{\Omega} = \frac{1}{4} \qty(\frac{e}{4\pi\epsilon_0 m v^2})^2
			\frac{1}{\sin^4\frac{\theta}{2}} \abs{F}^2(\vec{q})
		\end{equation}
