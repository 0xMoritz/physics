% !TEX root = ../physics.tex
\section{Mathematical Formulary}
	\subsection{Analysis}
		\subsubsection{Exponential Function}
			\noindent
			Definition:
			\begin{equation}
				e^x=\exp(x)=\lim_{n\rightarrow \infty}\qty(1+\frac{x}{n})^n
			\end{equation}

		\subsubsection{Trigonometric Functions}
			\noindent
			Identities:
			\begin{equation}
				\begin{split}
					e^{\i x}&=\cos x+\i\sin x \\
					1&=\sin^{2}x+\cos^{2}x \\
					\cos{x}&=\frac{e^{\i x}+e^{-\i x}}{2} \\
					\sin{ x}&=\frac{e^{\i x}-e^{-\i x}}{2\i} \\
				\end{split}
			\end{equation}

			\noindent
			Addition formulae\index{Additionstheoreme}:
			\begin{equation}
				\begin{split}
					\sin\qty( x\pm y)&=\sin\qty(xt)\cos\qty(y)\pm\cos\qty(x) \sin\qty(y) \\
					\cos\qty( x\pm y)&=\cos\qty(xt)\cos\qty(y)\mp\sin\qty(x) \sin\qty(y) \\
					2 \cos(x) \cos(y) &= \cos(x + y) + \cos(x - y) \\
					2 \sin(x) \cos(y) &= \sin(x + y) + \sin(x - y) \\
					2 \sin(x) \sin(y) &= \cos(x - y) - \cos(x + y) \\
				\end{split}
			\end{equation}

			\noindent
			Composition of trigonometric functions:
			\begin{center}
				\begin{tabular}{| c || c | c | c |}
					\hline\xrowht{10pt}
					$\mathrm{trig}(\mathrm{arctrig}(x))$ & $\sin$ & $\cos$ & $\tan$ \\
					\hline
					\hline\xrowht{24pt}
					$\arcsin(x)$ & $x$ & $\sqrt{1-x^2}$ & $\dfrac{x}{\sqrt{1-x^2}}$ \\
					\hline\xrowht{24pt}
					$\arccos(x)$ & $\sqrt{1-x^2}$ & $x$ & $\dfrac{\sqrt{1-x^2}}{x}$ \\
					\hline\xrowht{24pt}
					$\arctan(x)$ & $\dfrac{x}{\sqrt{x^2+1}}$ & $\dfrac{1}{\sqrt{x^2+1}}$ & $x$ \\
					\hline
				\end{tabular}
			\end{center}

			\noindent
			Standard values for trigonometric functions:
			\begin{center}
				\begin{tabular}{| c c || l l l |}
					\hline
					Radian & Degree & Sine & Cosine & Tangens \\
					\hline
					\hline\xrowht{12pt}
					$0$ & $0^\circ$ & $\sin\left(0\right)=0$ & $\cos\left(0\right)=1$ & $\tan\left(0\right)=0$ \\
					\hline\xrowht{12pt}
					$\frac{\pi}{6}$ & $30^\circ$ & $\sin\left(\frac{\pi}{6}\right)=\frac{1}{2}$ & $\cos\left(\frac{\pi}{6}\right)=\frac{\sqrt{3}}{2}$ & $\tan\left(\frac{\pi}{6}\right)=\frac{1}{\sqrt{3}}$ \\
					\hline\xrowht{12pt}
					$\frac{\pi}{4}$ & $45^\circ$ & $\sin\left(\frac{\pi}{4}\right)=\frac{1}{\sqrt{2}}$ & $\cos\left(\frac{\pi}{4}\right)=\frac{1}{\sqrt{2}}$ & $\tan\left(\frac{\pi}{4}\right)=1$ \\
					\hline\xrowht{12pt}
					$\frac{\pi}{3}$ & $60^\circ$ & $\sin\left(\frac{\pi}{3}\right)=\frac{\sqrt{3}}{2}$ & $\cos\left(\frac{\pi}{3}\right)=\frac{1}{2}$ & $\tan\left(\frac{\pi}{3}\right)=\sqrt{3}$ \\
					\hline\xrowht{12pt}
					$\frac{\pi}{2}$ & $90^\circ$ & $\sin\left(\frac{\pi}{2}\right)=1$ & $\cos\left(\frac{\pi}{2}\right)=0$ & $\tan\left(\frac{\pi}{2}\right)=\pm\infty$ \\
					\hline
				\end{tabular}
			\end{center}

		\subsubsection{Hyperbolic Functions}
			\noindent
			Identities
			\begin{equation}
				\begin{split}
					e^{ x}&=\cosh x+\sinh x \\
					1&=\cosh^2{x}-\sinh^2{x} \\
					\cosh{ x}&=\frac{e^{ x}+e^{- x}}{2} \\
					\sinh{ x}&=\frac{e^{ x}-e^{- x}}{2} \\
					\sinh( x) &= -\i \sin(\i x)\\
					\cosh( x) &= \cos(\i x)
				\end{split}
			\end{equation}

			\noindent
			Addition formulae\index{Additionstheoreme}:
			\begin{equation}
				\begin{split}
					\sinh\left( x\pm y\right)&=\sinh\left( x\right)\cosh\left( y\right)\pm\cosh\left( x\right)\sinh\left( y\right) \\
					\cosh\left( x\pm y\right)&=\cosh\left( x\right)\cosh\left( y\right)\pm\sinh\left( x\right)\sinh\left( y\right) \\
					\sinh{2x}&=2\sinh{x}\cosh{x} \\
					 \cosh{2x}&=2\cosh^2{x}+\sinh^2{x}
				\end{split}
			\end{equation}

			\noindent
			Composition of hyperbolic functions:
			\begin{center}
				\begin{tabular}{| c || c | c | c |}
					\hline\xrowht{10pt}
					$\mathrm{trigh}(\mathrm{artrigh}(x))$ & $\sinh$ & $\cosh$ & $\tanh$ \\
					\hline
					\hline\xrowht{24pt}
					$\mathrm{arsinh}(x)$ & $x$ & $\sqrt{x^2+1}$ & $\dfrac{x}{\sqrt{x^2+1}}$ \\
					\hline\xrowht{24pt}
					$\mathrm{arcosh}(x)$ & $\sqrt{\dfrac{x-1}{x+1}}(x+1)$ & $x$ & $\sqrt{\dfrac{x-1}{x+1}}\dfrac{(x+1)}{x}$ \\
					\hline\xrowht{24pt}
					$\mathrm{artanh}(x)$ & $\dfrac{x}{\sqrt{1-x^2}}$ & $\dfrac{1}{\sqrt{1-x^2}}$ & $x$ \\
					\hline
				\end{tabular}
			\end{center}

			\noindent
			Inverse hyperbolic functions:
			\begin{equation}
				\begin{aligned}
					\mathrm{arsinh}(x) &= \ln\left(x+\sqrt{x^2+1}\right) \\
					\mathrm{arcosh}(x) &= \ln\left(x+\sqrt{x^2-1}\right) \\
					\mathrm{artanh}(x) &= \frac{1}{2}\ln\left(\frac{1+x}{1-x}\right) \\
				\end{aligned}
			\end{equation}

		\subsubsection{Differentiation}
			\noindent
			Schwarz's theorem\index{Schwartz!Satz} ($f(x,y)\in C^2$, i.e. $f$ is two times differentiable):
			\begin{equation}
				\frac{\partial^2 }{\partial x \partial y} f(x,y) = \frac{\partial^2 }{\partial y \partial x} f(x,y)
			\end{equation}

			\noindent
			For connected quantities $x(y,z), y(x,z), z(x,y)$ and an inevitable function $w(y,z)$:
			\begin{equation}
				\begin{aligned}
					\left( \pdv{x}{y} \right)_z &= \frac{1}{\left( \pdv{y}{x} \right)_z} \\
					-1 &= \left( \pdv{x}{y} \right)_z \left( \pdv{z}{x} \right)_y \left( \pdv{y}{z} \right)_x \\
					\left( \pdv{x}{w} \right)_z &= \left( \pdv{x}{y} \right)_z \left( \pdv{y}{w} \right)_z \\
					\left( \pdv{x}{y} \right)_z &= \left( \pdv{x}{y} \right)_w + \left( \pdv{x}{w} \right)_y \left( \pdv{w}{y} \right)_z \\
				\end{aligned}
			\end{equation}

			\noindent
			Derivatives of trigonometric and hyperbolic functions:
			\begin{equation}
				\begin{split}
					\frac{\dd}{\dd x}\arcsin x &= \frac{\pm 1}{\sqrt{1-x^2}} \hspace{30pt}
					&\frac{\dd}{\dd x}\mathrm{arsinh}\,x &= \frac{1}{\sqrt{1+x^2}} \\
					\frac{\dd}{\dd x}\arccos x &= \frac{\mp 1}{\sqrt{1-x^2}} \hspace{30pt}
					&\frac{\dd}{\dd x}\mathrm{arcosh}\,x &= \frac{\pm 1}{\sqrt{x^2-1}} \\
					\frac{\dd}{\dd x}\arctan x &= \frac{1}{x^2+1} \hspace{30pt}
					&\frac{\dd}{\dd x}\mathrm{artanh}\,x &= \frac{1}{1-x^2} \\
				\end{split}
			\end{equation}

			\noindent
			Jacobi's formula\index{Jacobi!Formel}:
			\begin{equation}
				\dv{t}\det A(t) = \tr \qty(A^T \dv{t} A)
			\end{equation}

			\noindent
			Definition functional differential and properties of the functional derivative (of a functional $\mathcal{F}$):
			\begin{equation}
				\begin{aligned}
					\delta \mathcal{F}\qty[\phi] &= \lim_{\varepsilon\rightarrow 0} \frac{	\mathcal{F}\qty[\phi + \delta_\varepsilon\phi] - \mathcal{F}\qty[\phi]}{\varepsilon} = \eval{\dv{\mathcal{F}\qty[\phi+\delta_\varepsilon\phi]}{\varepsilon}}_{\varepsilon=0}\\
					\frac{\delta \phi(x)}{\delta \phi(y)} &= \delta(x-y) \\
					\delta \mathcal{F}\qty[\phi(x)] &= \int_\mathbb{R} \frac{\delta \mathcal{F}\qty[\phi(x)]}{\delta \phi(y)}\delta_\varepsilon\phi(y)\;\dd y
				\end{aligned}
			\end{equation}

		\subsubsection{Integration}
			\noindent
			Leibniz integral rule\index{Leibniz!Regel für Parameterintegrale}:
			\begin{equation}
				\dv{y} \int_{\alpha(y)}^{\beta(y)}f(x,y)\;\dd x = \int_{\alpha(y)}^{\beta(y)} \pdv{f}{y}(x,y)\;\dd x + f(\beta(y),y)\dv{\beta}{y}(y)-f(\alpha(y),y)\dv{\alpha}{y}(y)
			\end{equation}

			\noindent
			Cauchy integral theorem\index{Cauchy!Integralsatz} (for a closed curve $\partial \mathcal{A}$ and a holomorphic function $f(z)$ on a connected domain $G \subseteq \mathbb{C}$):
			\begin{equation}
				\int_{\partial \mathcal{A}} f(z)\;\dd z = 0
			\end{equation}

			\noindent
			Cauchy integral formula\index{Cauchy!Integralformel} (for a positively oriented circular loop $\gamma(t)=z_0+re^{\it};\;t\in[0,2\pi]$ and a holomorphic function $f(z)$ on a domain $G \subseteq \mathbb{C}$):
			\begin{equation}
				\dv[k]{f}{z}(z_0) = \frac{k!}{2\pi \i} \int_\gamma \frac{f(z)}{(z-z_0)^{k+1}}\;\dd z
			\end{equation}

			\noindent
			Residue theorem\index{Residuensatz} (for a simple, closed piece wise continuously differentiable path (contour) $\gamma:\left[a,b\right]\rightarrow G$ and $f:G\backslash\set{a_1, a_2, .., a_n} \rightarrow \mathbb{C}$ holomorphic,
			with poles $a_1, a_2, .., a_n$ inside $\gamma$, and corresponding residues $\mathrm{Res}(f; a) = \eval{\frac{1}{(k-1)!}\dv[k-1]{g}{z}}_{z=a}$ for $f(z)=\frac{g(z)}{(z-a)^k}$ near $a$):
			\begin{equation}
				\int_\gamma f(z)\,\dd z = 2\pi \i \sum_{k=1}^{n} \mathrm{Res}(f; a_k)
			\end{equation}

			\noindent
			Trigonometric integrals:
			\begin{equation}
				\begin{split}
					\int \sin^2 x\;\dd x &= \frac{1}{2}( x-\sin  x\cos  x) \\
					\int \cos^2 x\;\dd x &= \frac{1}{2}( x+\sin  x\cos  x) \\
					\int{\sin x\cos x\;\dd x} &= -\frac{1}{2}\cos^2( x) \\
					\int \sin^3 x\;\dd x &= -\frac{1}{3}\cos{ x}\sin^2{ x}-\frac{2}{3}\cos{ x}
					= \frac{1}{12}\cos{3 x}-\frac{3}{4}\cos{ x} \\
					\int \cos^3 x\;\dd x &= \phantom{-}\frac{1}{3}\sin x \cos^2 x+\frac{2}{3}\sin  x
					= \frac{1}{12}\sin{3 x}+\frac{3}{4}\sin{ x} \\
				\end{split}
			\end{equation}

			\noindent
			Famous integrals (Gauß integral\index{Gauß!Integral}, $\sinc$-integral\index{Sinus Cardinalis}):
			\begin{equation}
				\begin{aligned}
					\int_\mathbb{R} e^{-x^2}\;\dd x &= \sqrt{\pi} \\
					\int_\mathbb{R} e^{-\frac{x^2}{\sigma}+c}\;\dd x &= \sqrt{\pi\sigma} \quad\forall c\in\mathbb{C} \\
					\int_\mathbb{R} \frac{\sin x}{x}\;\dd x &= \pi \\
				\end{aligned}
			\end{equation}

			\noindent
			Convolution\index{Faltung} ($f*g:\mathbb{R}^n \rightarrow \mathbb{C}$):
			\begin{equation}
				(f*g)(x) := \int_{\mathbb{R}^n} f(t) g(x-t)\;\dd t
			\end{equation}

		\subsubsection{Distributions}
			\noindent
			Dirac distribution / Dirac delta / Dirac function\index{Dirac!Distribution} $\delta(x)$:
			\begin{equation}
				\int_\Omega f(x)\delta(x)\;\dd x = f(0)\quad\forall\, \Omega\ni 0
			\end{equation}

			\noindent
			Dirac distribution of a function $f(x)$ zeros $x_{0,i}$:
			\begin{equation}
				\delta(f(x)) = \sum_i \frac{1}{\abs{\pdv{f}{x}(x_{0,i})}}\delta(x-x_{0,i})
			\end{equation}

			\noindent
			Heaviside function\index{Heaviside!Funktion} $\Theta(x)$ and its properties:
			\begin{equation}
				\begin{aligned}
					\Theta(x) :=& \left\{\begin{array}{ll}
						1 & x\ge 0 \\
						0 & x<0 \\
						\end{array}\right. \\
						\dv{\Theta}{x}(x) =&\; \delta(x) \\
					\int_{\mathbb{R}} f(x)\Theta(x)\;\dd x =& \int_{0}^{\infty} f(x)\;\dd x
				\end{aligned}
			\end{equation}

			\begin{equation}
				\Nabla^2\frac{1}{\abs{\vec{r} - \pvec{r}'}} = -4\pi\delta\left(\vec{r} - \pvec{r}'\right)
			\end{equation}

			\noindent
			See Eq.~\ref{Eq:FourierIdentities}

		\subsubsection{Gamma Function\index{Gamma Funktion}}
			\noindent
			Definition:
			\begin{equation}
				\Gamma(z)=\int_0^{\infty}e^{-t}t^{z-1}\,dt
			\end{equation}

			\noindent
			Properties ($\forall n\in\mathbb{N}_0$):
			\begin{equation}
				\begin{aligned}
					\Gamma(z+1)&=z\Gamma(z) \\
					\Gamma(n+1) &= n!
				\end{aligned}
				\hspace{40pt}
				\begin{aligned}
					\Gamma\left(1 \right)&=1 \\
					\Gamma\left(1/2 \right)&=\sqrt{\pi} \\
				\end{aligned}
			\end{equation}

			\noindent
			Stirling Formula\index{Stirling!Formula} (Approximation for $n\rightarrow\infty$):
			\begin{equation}
				n! = \Gamma(n+1) \approx \sqrt{2\pi n} \left( \frac{n}{e} \right)^{n}
			\end{equation}


		\subsubsection{Function space\index{Funktionenraum}}
			\noindent
			$D$ is a definition space.
			\begin{itemize}
				\item $C^p(D)$, space of $p$-times continuously differentiable functions
				\item $L^p(D)$, space of equivalence classes (with respect to lower-dimensional exceptions) of functions which $p$-Norm is Lebesque integrable\index{Lebesque!integrierbar}
				\item $S(\mathbb{R})$, space of functions that decrease faster than any polynomial function (Schwartz space\index{Schwartz!Raum})
			\end{itemize}

		\subsubsection{Sequences\index{Folge} and Series\index{Reihe}}
			\noindent
			Geometric summation formula\index{Geometrische Summenformel} ($\forall q \ne 1$):
			\begin{equation}
				\sum_{k=0}^n q^k=\frac{1-q^{n+1}}{1-q}
			\end{equation}

			\noindent
			Geometric series\index{Geometrische Reihe} ($\forall \abs{q} < 1$):
			\begin{equation}
				\sum_{k=0}^\infty q^k= \frac{1}{1-q}
			\end{equation}

			\noindent
			Binomial theorem\index{Binomischer Satz} ($n\in\mathbb{R}$):
			\begin{equation}
				(x+y)^n=\sum_{k=0}^{n}\binomkoeff{n}{k}x^k y^{n-k}
			\end{equation}

			\noindent
			Identities:
			\begin{equation}
				\lim_{n\rightarrow\infty} \sqrt[n]{n} = 1
			\end{equation}

			\noindent
			\href{https://en.wikipedia.org/wiki/Abel%27s_theorem}{Abel limit theorem}\index{Abel!Grenzwertsatz}:
			\begin{equation}
				\begin{aligned}
					\lim_{t\rightarrow+\infty} f(t) &= \lim_{\eta\rightarrow 0} \eta \int_0^\infty e^{-\eta t} f(t) \,\dd t \\
					\lim_{t\rightarrow-\infty} f(t) &= \lim_{\eta\rightarrow 0} \eta \int_{-\infty}^{0} e^{+\eta t} f(t) \,\dd t \\
				\end{aligned}
			\end{equation}

			\noindent
			\href{https://en.wikipedia.org/wiki/Euler%E2%80%93Maclaurin_formula}{Euler-Maclaurin formula}\index{Euler!-McLaurin Formel}:
			\begin{equation}
				\sum_{k=a+1}^{b} f(k) = \int_{a}^{b}f(x) \;\dd x
				+ \frac{1}{2} \eval{\dv{f}{x} }_{a}^{b}
				+ \frac{1}{12}\eval{\dv[2]{f}{x} }_{a}^{b}
				+ \frac{1}{720} \eval{\dv[3]{f}{x} }_{a}^{b}
				+ ...
			\end{equation}

		\subsubsection{Taylor Series\index{Potenzreihe}\index{Taylor!Reihe}}
			\noindent
			Polynomial sequence:
			\begin{equation}
				f(z) = \sum_{n=0}^{\infty} a_n (z-z_0)^n
			\end{equation}

			\noindent
			Radius of convergence\index{Konvergenzradius}:
			\begin{equation}
				r = \frac{1}{\limsup\limits_{n\rightarrow\infty} \sqrt[n]{\abs{a_n}}} = \lim_{n\rightarrow\infty}\abs{\frac{a_n}{a_{n+1}}}
			\end{equation}

			\noindent
			Taylor expansion\index{Taylor!Entwicklung} at $x_0$:
			\begin{equation}
				f\qty(x)=\sum_{n=0}^{\infty}\frac{1}{n!}\dv[n]{f\qty(x_0)}{x}\qty(x-x_0)^{n}
			\end{equation}

			\noindent
			Polynomial sequence of important functions:
			\begin{equation}
				\begin{aligned}
					e^x &= \sum_{k=0}^{\infty}{\frac{x^k}{k!}} \\
					\sin(x) &= \sum_{k=0}^\infty (-1)^k \frac{x^{2k+1}}{(2k+1)!} \\
					\cos(x) &= \sum_{k=0}^\infty (-1)^k \frac{x^{2k}}{(2k)!} \\
				\end{aligned}
			\end{equation}

	\subsection{Basis Functions}
		\noindent
		Complex inner product\index{Hermite!Skalarprodukt} in $R([0,2\pi],\mathbb{C})$:
		\begin{equation}
			\left<f,g\right>:=\frac{1}{2\pi}\int_0^{2\pi} f(x)g^* (x)\;\dd x
		\end{equation}

		\subsubsection{Fourier Analysis\index{Fourier!Analyse}}
			\noindent
			Complex trigonometric basis functions:
			\begin{equation}
				\hat{e}_k(x):=e^{\i kx}
			\end{equation}

			\noindent
			Discrete complex Fourier transformation\index{Fourier!Transformation}:
			\begin{equation}
				(Tf)(x)=\sum_{k\in\mathbb{Z}}c_k e^{\i kx}
			\end{equation}

			\noindent
			Complex Fourier coefficients:
			\begin{equation}
				c_k = \left<f,\hat{e}_k\right>=\frac{1}{2\pi}\int_0^{2\pi}f(x)e^{-\i kx}\;\dd x
			\end{equation}

			\noindent
			Discrete real Fourier transformation\index{Fourier!Transformation}:
			\begin{equation}
				(Tf)(x)=\frac{a_0}{2}+\sum_{k=1}^{\infty}\left[a_k \cos(kx) + b_k \sin(kx) \right]
			\end{equation}

			\noindent
			Relation between complex and real coefficients:
			\begin{equation}
				\begin{array}{clc}
					c_0 = \dfrac{a_0}{2} & a_k = c_k+c_{-k} & \phantom{_{-}}c_k = \dfrac{1}{2}\left(a_k-\i b_k\right) \\ [6pt]
					& b_k = \dfrac{1}{\i}\left(c_{-k}-c_{k}\right) & c_{-k} = \dfrac{1}{2}\left(a_k+\i b_k\right) \\
				\end{array}
			\end{equation}

			\noindent
			Continuous Fourier transformation\index{Fourier!Transformation} and inverse transformation:
			\begin{equation}
				\begin{aligned}
					(\mathcal{F}f)(\vec{k}) :=& \frac{1}{\sqrt{2\pi}^n}\int_{\mathbb{R}^n} \phantom{(\mathcal{F})}f(\vec{x})\, e^{-\i\vec{k}\cdot\vec{x}}\;\dd\vec{x} \\
					\phantom{(\mathcal{F})}f(\vec{x}) =& \frac{1}{\sqrt{2\pi}^n}\int_{\mathbb{R}^n} (\mathcal{F}f)(\vec{k}) \,e^{\i\vec{k}\cdot\vec{x}}\;\dd \vec{k} \\
				\end{aligned}
			\end{equation}

			\noindent
			Identities of the continuous Fourier transformation\index{Fourier Transformation} (where $\tilde{f}(k) := \left(\mathcal{F}f\right)(k)$)
			\begin{equation}
				\begin{aligned}
					\mathcal{F}\left(xf\right)(k) &= \phantom{-}\i \pdv{k}\tilde{f}(k) \\
					\mathcal{F}^{-1}(k\tilde{f})(x) &= -\i\pdv{x}f(x) \\
					\int_{\mathbb{R}} f^{*}(x) g(x)\;\dd x &=
					\int_{\mathbb{R}}\tilde{f}^{*}(k)\tilde{g}(k)\;\dd k \\
					\delta(x) &= \frac{1}{2\pi}\int_{\mathbb{R}}e^{\i kx}\;\dd k \\
					\label{Eq:FourierIdentities}
				\end{aligned}
			\end{equation}

			\noindent
			Further properties:
			\begin{equation}
				\begin{aligned}
					f(x)\text{ is purely real} &\Leftrightarrow \tilde{f}(-k) = \phantom{-}\tilde{f}^*(k) \\
					f(x)\text{ is purely imaginary} &\Leftrightarrow \tilde{f}(-k) = -\tilde{f}^*(k) \\
					f(x)\text{ is purely real and even} &\Leftrightarrow \tilde{f}(k)\text{ is purely real and even} \\
					f(x)\text{ is purely real and odd} &\Leftrightarrow \tilde{f}(k)\text{ is purely imaginary and odd} \\
				\end{aligned}
			\end{equation}

			\noindent
			\href{https://en.wikipedia.org/wiki/Fourier_transform#Functional_relationships,_one-dimensional}{Properties listed by wikipedia.org}

		\subsubsection{Legendre Transformation\index{Legendre!Transformation}}
			\noindent
			Legendre transformation (self-adjoint isometry between convex functions)
			\begin{equation}
				(\mathcal{L}f)(p)=\max_x\left\lbrace xp-f(x) \right\rbrace
			\end{equation}


		\subsubsection{Laurent Series\index{Laurent!Reihe}}
			\noindent
			Laurent Series\index{Laurent Reihe}:
			\begin{equation}
				f(z)=\sum_{n\in\mathbb{Z}} a_n(z-z_0)^n
			\end{equation}

			\noindent
			Coefficients:
			\begin{equation}
				a_n = \frac{1}{2 \pi \i}\int_{\abs{ z-z_0 } = r} \frac{f(z)}{(z-z_0)^{n+1}}\;\dd z
			\end{equation}

		\subsubsection{Spherical Harmonics\index{Kugelflächenfunktionen}}
			\noindent
			Expansion in spherical harmonics\index{Kugelflächenfunktionen}:
			\begin{equation}
				f(\theta, \phi) = \sum_{l=0}^{\infty} \sum_{m=-l}^{l} f_{lm} Y_{lm}(\theta,\phi)
			\end{equation}

			\noindent
			Coefficients:
			\begin{equation}
				f_{lm} = \int_0^{2\pi} \int_0^\pi Y_{lm}^{*}(\theta,\phi) f(\theta,\phi) \sin\theta\;\dd\theta\dd\phi
			\end{equation}

	\subsection{Differential Equations}
		\subsubsection{General}
			\noindent
			Cauchy-Riemann differential Equations\index{Cauchy!-Riemann Differentialgleichungen} (Always hold for complex functions $f(z)=u(z)+\i v(z)$ with $z=x+\i y$, where $x,y,u,v\in\mathbb{R}$)
			\begin{equation}
				\begin{aligned}
					\pdv{u}{x} &= \phantom{-}\pdv{v}{y} \\
					\pdv{u}{y} &= -\pdv{v}{x} \\
				\end{aligned}
			\end{equation}


		\subsubsection{Green's Functions\index{Green!Funktion}}
			\noindent
			Green's Function\index{Green!Funktion} ($\mathrm{D}$ is an arbitrary linear differential operator):
			\begin{equation}
				\mathrm{D}_{\pvec{r}}\, G\qty(\vec{r},\pvec{r}') = \delta\qty(\vec{r} - \pvec{r}')
			\end{equation}
			General solution of the differential equation $\mathrm{D}_{\pvec{r}}\, \phi = f\qty(\vec{r})$:
			\begin{equation}
				\phi\qty(\vec{r}) = \int G\qty(\vec{r},\pvec{r}')f\qty(\pvec{r}')\;\dd^3\pvec{r}'
			\end{equation}

		\subsubsection{Harmonic Oscillator\index{Harmonischer Oszillator}}
			\noindent
			Differential equation and solution:
			\begin{equation}
				\begin{aligned}
					\ddot{x}+\omega_0^2 x &= 0 \\
					x(t) &= x_0 e^{\pm \i\omega_0 t}
				\end{aligned}
			\end{equation}

		\subsubsection{Forced, Damped Harmonic Oscillator}
			\noindent
			Differential equation and solution:
			\begin{equation}
				\begin{aligned}
					m\ddot{x}+m\gamma\dot{x}+m\omega_0^2 x &= F_0 e^{-\i\omega t} \\
					x(t) &= \frac{F_0}{m} \frac{1}{\left(\omega_0^2-\omega^2\right)-\i \gamma\omega_0} e^{-\i\omega t}
				\end{aligned}
			\end{equation}

		\subsubsection{Bessel Differential Equation\index{Bessel!Differentialgleichung}}
			\noindent
			Differential equation:
			\begin{equation}
				B_\nu f = x^2\dv[2]{f}{x} + x\dv{f}{x}+\qty(x^2-\nu^2)f = 0
			\end{equation}

			\noindent
			Bessel function\index{Bessel!Funktion} of the first kind:
			\begin{equation}
				J_\nu(x) = \sum_{r=0}^\infty \frac{(-1)^r\qty(\frac{x}{2})^{2r+\nu}}{\Gamma(\nu+r+1)\,r!} = \frac{1}{2\pi}\int_{-\pi}^{\pi} e^{\i(x\sin\varphi-\nu\varphi)}\;\dd\varphi
			\end{equation}

		\subsubsection{Poisson Equation\index{Poisson!Gleichung}}
			\noindent
			Poisson equation\index{Poisson!Gleichung} of electrostatics:
			\begin{equation}
				\Nabla^2\phi = -\frac{\rho}{\epsilon_0}
			\end{equation}

			\noindent
			Sufficient conditions for the existence and uniqueness of solutions of the Poisson equation:
			\begin{description}
				\item[Dirichlet boundary conditions\index{Dirichlet!Randbedingungen}]\hfill \\
					The charge density $\rho(\vec{r})$ in $\vec{r}\in\mathcal{V}$ and the potential $\phi(\vec{r})$ on $\vec{r}\in\partial\mathcal{V}$ are known.
				\item[Neumann boundary conditions\index{Neumann!Randbedingungen}]\hfill \\
					The charge density  $\rho(\vec{r})$ in $\vec{r}\in\mathcal{V}$ and the normal gradient of the potential  $\vec{n}(\vec{r})\cdot\Nabla\phi(\vec{r})$ on $\vec{r}\in\partial\mathcal{V}$ are known.
				\item[Boundary conditions with conductors of known total charge]\hfill \\
					The charge density $\rho(\vec{r})$ in $\vec{r}\in\mathcal{V}$ and the total charge $Q_j$ of conductors restricted by $\partial\mathcal{V}$ are known.
			\end{description}

		\subsubsection{Laplace Equation}
			\noindent
			Laplace equation\index{Laplace!Gleichung}
			\begin{equation}
				\Delta\phi=\Nabla^2\phi = 0
			\end{equation}

		\subsubsection{Legendre Differential Equation}
			\noindent
			Legendre differential equation\index{Legendre!Differentialgleichung} ($l>0$):
			\begin{equation}
				\pdv{x}\left(\left(1-x^2\right)\pdv{f}{x}\right)+l\left(l+1\right)f
				= \frac{\partial^2 f}{\partial x^2} - \frac{2x}{1-x^2}\pdv{f}{x} + \frac{l(l+1)}{1-x^2}f = 0
			\end{equation}

			\noindent
			Solutions (Legendre Polynomials\index{Legendre!Polynome}) / Rodrigues Formula\index{Rodrigues!Formel}:
			\begin{equation} \label{Eq:LegendrePolynomials}
				\begin{aligned}
					P_0(x) &= 1 \\
					P_1(x) &= x \\
					P_2(x) &= \frac{1}{2}\qty(3x^2-1) \\
					P_l(x) &= \frac{1}{2^l l!}\dv[l]{x}\qty(x^2-1)^l \\
				\end{aligned}
			\end{equation}

			\noindent
			Explicit construction using the binomial coefficient:
			\begin{equation}
				P_l(x)=\frac{1}{2^l} \sum_{k=0}^{l}\binomkoeff{l}{k}^2(x+1)^k(x-1)^{l-k}
			\end{equation}

			\noindent
			Orthogonality:
			\begin{equation}
				\int_{-1}^1 P_l(x) P_k(x)\;\dd x = \delta_{lk} \frac{2}{2l+1}
			\end{equation}

		\subsubsection{Generalized Legendre Differential Equation}
			\noindent
			Differential equation ($l\in\mathbb{N}$ and $m\in\mathbb{Z}, -l\le m\le l$)
			\begin{equation}
				\pdv{x}\qty(\qty(1-x^2)\pdv{f}{x})+\qty(l\qty(l+1)-\frac{m^2}{1-x^2})f= 0
			\end{equation}

			\noindent
			Solutions (Legendre Functions\index{Legendre!Funktionen})
			\begin{equation}
				\begin{aligned}
					P_l^m(x) &= \frac{(-1)^m}{2^l l!}\sqrt{1-x^2}^m
					\dv[l+m]{x}\qty(x^2-1)^l \\
					&= (-1)^m \frac{(m+l)!}{2^l l! (l-m)!}\frac{1}{\sqrt{1-x^2}^m}
					\dv[l-m]{x}\qty(x^2-1)^l \\
				\end{aligned}
			\end{equation}

		\subsubsection{Spherical Harmonics}
			\noindent
			Differential equation:
			\begin{equation}
				-\left(
				\frac{1}{\sin\theta}\pdv{\theta}\sin\theta\pdv{\theta} + \frac{1}{\sin^2\theta}\frac{\partial^2}{\partial \phi^2}
				\right)
				Y_{lm}(\theta,\phi) = l(l+1)Y_{lm}(\theta,\phi)
			\end{equation}

			\noindent
			Solutions (Spherical Harmonics):
			\begin{equation} \label{Eq:SphericalHarmonics}
				Y_{lm}(\theta,\varphi) = \sqrt{\frac{2l+1}{4\pi}\frac{(l-m)!}{(l+m)!}} e^{\i m\varphi} P_l^m(\cos\theta)
			\end{equation}

			\noindent
			Orthogonality:
			\begin{equation}
				\int_0^{2\pi}\dd\varphi \int_0^{\pi}\dd\theta \sin\theta\, Y_{lm}(\theta,\varphi) Y^{*}_{l'm'}(\theta,\varphi) = 	\delta_{ll'}\delta_{mm'}
			\end{equation}

		\subsubsection{Hermite's Differential Equation}
			\noindent
			Differential equation\index{Hermite!Differentialgleichung} ($n\in\mathbb{N}_0$):
			\begin{equation}
				\dv[2]{x} H_n(x) -2x\dv{x} H_n+2 n H_n(x)=0
			\end{equation}

			\noindent
			Solutions (Hermite polynomials\index{Hermite!Polynome}):
			\begin{equation} \label{Eq:HermitePolynomials}
				\begin{aligned}
					H_n(x) &= (-1)^n e^{x^2} \dv[n]{x} e^{-x^2} \\
					&= e^{x^2/2}\qty(x-\dv{x})^n e^{-x^2/2} \\
				\end{aligned}
			\end{equation}

			\noindent
			Orthogonality:
			\begin{equation}
				\int_{\mathbb{R}} H_m(x) H_n(x) e^{-x^2}\;\dd x = \sqrt{\pi}\,2^n n!\delta_{nm}
			\end{equation}

		\subsubsection{Laguerre's Differential Equation}
			\noindent
			Differential Equation\index{Laguerre!Differentialgleichung} ($x>0$, $n\in\mathbb{N}_0$):
			\begin{equation}
				x\dv[2]{x}L_n(x) +(1-x)\dv{x}L_n(x) + nL_n(x) = 0
			\end{equation}

			\noindent
			Solutions (Laguerre Polynomials\index{Laguerre!Polynome}):
			\begin{equation}
				L_n(x) = \frac{e^x}{n!}\dv[n]{x}(e^{-x} x^n) = \frac{1}{n!} \qty(\dv{x}-1)^n x^n
			\end{equation}

			\noindent
			Orthogonality:
			\begin{equation}
				\int_0^\infty e^{-x} L_n(x) L_m(x)\;\dd x = \delta_{nm}
			\end{equation}

			\noindent
			Associated Laguerre Polynomials ($k\in\mathbb{N}_0$):
			\begin{equation}
				L_n^k(x) = (-1)^k\dv[k]{x} L_{n+k}(x)
			\end{equation}

	\subsection{Geometry}
		\subsubsection{General}
			\noindent
			Volume of a $n$-dimensional Sphere:
			\begin{equation}
				\Omega_n = \frac{\sqrt{\pi}^n}{\qty(\frac{n}{2})!}R^n
				= \frac{\sqrt{\pi}^n}{\Gamma\qty(\frac{n}{2}+1)}R^n
			\end{equation}

			\noindent
			Skew-symmetric matrix of a vector $\vec{v} = (v_1, v_2, v_3)^T$:
			\begin{equation}
				\tilde{\vec{v}} = \qty[\vec{v}\times] =
				\qty( \begin{matrix}
					\M 0   & -v_z   & \M v_y \\
					\M v_z & \M 0   & - v_x \\
					-v_y   & \M v_x & \M 0 \\
				\end{matrix} ) \\
			\end{equation}

		\subsubsection{Solid Angle\index{Raumwinkel}}
			\noindent
			Solid angle definition (partial Surface of the unit sphere):
			\begin{equation}
				\Omega = \frac{A}{R^2}
			\end{equation}

			\noindent
			Differential solid angle in spherical coordinates:
			\begin{equation}
				\dd \Omega = \sin\theta\;\dd \theta \,\dd \phi
			\end{equation}

			\noindent
			Solid angle of a cone with opening angle $2\theta$:
			\begin{equation}
				\Omega = 4\pi\sin^2\left(\frac{\theta}{2}\right)
			\end{equation}

		\subsubsection{Triangles}
			\noindent
			Law of sines\index{Sinussatz}:
			\begin{equation}
				\frac{a}{\sin\alpha} = \frac{b}{\sin\beta} = \frac{c}{\sin\gamma}
			\end{equation}

			\noindent
			Law of cosines\index{Kosinussatz}:
			\begin{equation}
				c^2 = a^2 + b^2 -2ab \cos\gamma
			\end{equation}


	\subsection{Vector Analysis}
		\subsubsection{Vector Spaces}
			Relation of types of Vector spaces:
			\begin{equation}
				\underbrace{\text{Hilbert Space}\;\mathcal{H}}_{\exists\; \text{scalar product}} \subset \underbrace{\text{Banach Space}\;\mathcal{B}}_{\exists\; \text{complete norm}} \subset \underbrace{\text{Normed Vector Space}\;V_{\norm{\cdot}}}_{\exists\; \text{norm} \norm{\cdot} } \subset \text{Vector space}
			\end{equation}

			\noindent
			Vector space\index{Vektorraum} axioms $V(\mathbb{K})$:
			\begin{itemize}
				\item Closure: $\alpha \cdot \vec{v} + \beta \cdot \vec{w} \in V \quad \forall \vec{v}, \vec{w} \in V; \forall \alpha, \beta \in \mathbb{K}$
				\item $\exists$ ``$+$'': Abelian group for addition
				\item $\exists$ ``$\cdot$'': associative and distributive scalar multiplication with neutral element
			\end{itemize}

			\noindent
			Dual space\index{Dualraum} $V^{*}(\mathbb{K})$ for $V(\mathbb{K})$:
			\begin{equation}
				\begin{aligned}
					&\text{Basis of $V$}: \mathcal{B} = \set{ \vec{e}_1,...,\vec{e}_n } \subset V \\
					&\text{Basis of $V^{*}$}: \mathcal{B}^{*} = \set{ \cev{e}^1,...,\cev{e}^n } \subset V^{*} \\
					&\cev{e}^i \vec{e}_i = \vec{e}_i \cev{e}^i = 1 \\
					&g_{\mu\nu} = g_{\nu\mu} = \vec{e}_\mu\cdot\vec{e}_\nu\\
					&g^{\mu\nu} = g^{\nu\mu} = \cev{e}^\mu\cdot\cev{e}^\nu \\
					&g_{\mu\lambda} g^{\lambda\nu} = g^{\nu\lambda}g_{\lambda\mu} = \delta_\mu^\nu \\
				\end{aligned}
			\end{equation}

			% \noindent 
			% $(m,n)$-Tensor $T$: 
			% \begin{equation}
			% 	T:
			% \end{equation}
			In vector analysis: vector $\rightarrow$ vector field, dual vector $\rightarrow$ differential form.

		\subsubsection{Vector Identities}
			\noindent
			Graßmann identity\index{Graßmann!Identität}\index{bac-cab Regel}:
			\begin{equation}
				\vec{a}\times\big(\vec{b}\times\pvec{c}\big) = \vec{b}\big(\vec{a}\cdot\pvec{c}\big) - \vec{c}\big(\vec{a}\cdot\pvec{b}\big)
			\end{equation}

			\noindent
			Graßmann identity\index{Graßmann!Identität} in index notation:
			\begin{equation}
				\varepsilon_{ijk}\,\varepsilon_{imn}=\delta_{jm}\delta_{kn}-\delta_{jn}\delta_{km}
			\end{equation}

			\noindent
			Triple product\index{Spatprodukt}:
			\begin{equation}
				\vec{a}\cdot\left(\vec{b}\times\vec{c}\right) = \det\left(\vec{a},\vec{b},\vec{c}\right)
			\end{equation}

			\noindent
			Jacobian matrix\index{Jacobi!Matrix} and Jacobian determinant\index{Jacobi!Determinante}:
			\begin{equation}
				D\left(\vec{f}\,\right) = \left(\frac{\partial f_j}{\partial x_k}\right)_{jk}
				= \left(\begin{matrix}
				\frac{\partial f_1}{\partial x_1} & \dotsb & \frac{\partial f_1}{\partial x_n} \\
				\vdots & \ddots & \vdots \\
				\frac{\partial f_m}{\partial x_1} & \dotsb & \frac{\partial f_m}{\partial x_n} \\
				\end{matrix}\right);\;\;J_f(\pvec{x})=\abs{\det\left(D\vec{f}(\pvec{x})\right)}
			\end{equation}

			\noindent
			Identities in connection with the gradient, divergence and rotation of vector fields:
			\begin{equation}
				\begin{aligned}
					\Nabla\cdot\left(\Nabla\times\vec{A}\right) &= 0 \\
					\Nabla\times\left(\Nabla\psi\right) &= 0 \\
					\Nabla\cdot\left(\Nabla\psi\right) &= \Nabla^2\psi \\
					\Nabla\times\left(\Nabla\times\vec{A}\right) &= \Nabla\left(\Nabla\cdot\vec{A}\right) -\Nabla^2\vec{A} \\
					\Nabla\cdot\left(\psi\vec{A}\right) &= \Nabla\psi\cdot\vec{A} + \psi\Nabla\cdot\vec{A}\\
					\Nabla\times\left(\psi\vec{A}\right) &= \Nabla\psi\times\vec{A} + \psi\Nabla\times\vec{A} \\
					\Nabla\cdot\left(\vec{A}\times\vec{B}\right) &= \vec{B}\cdot\left(\Nabla\times\vec{A}\right) - 	\vec{A}\cdot\left(\Nabla\times\vec{B}\right) \\
					\Nabla\times\left(\vec{A}\times\vec{B}\right) &= \vec{A}\left(\Nabla\cdot\vec{B}\right) - \vec{B}\left(\Nabla\cdot{A}\right) + \left(\vec{B}\cdot\Nabla\right)\vec{A} - \left(\vec{A}\cdot\Nabla\right)\vec{B} \\
				\end{aligned}
			\end{equation}

			\noindent
			Helmholtz decomposition\index{Helmholtz!Zerlegung}:
			\begin{equation}
				\forall \,\vec{v}(\pvec{r}):\; \exists\left(\phi\left(\pvec{r}\right),\vec{A}(\pvec{r})\right):\;\vec{v} = \Nabla\phi(\pvec{r}) + 	\Nabla\times\vec{A}(\pvec{r})
			\end{equation}

			\noindent
			Poincaré Lemma\index{Poincaré!Lemma} $\forall\, \vec{r}\in\mathcal{G}$ ($\mathcal{G}$ connected domain):
			\begin{equation}
				\exists U(\vec{r}): \vec{v}(\pvec{r}) = -\Nabla U(\pvec{r})
				\Leftrightarrow \vec{v}(\pvec{r}) \text{ is conservative }
				\Leftrightarrow \Nabla\times\vec{v}(\pvec{r}) = 0
			\end{equation}

		\subsubsection{Integration on Manifolds}
			\noindent
			General Stokes theorem\index{Stokes!Satz} ($\mathcal{M}$ orientable $n$-dimensional Manifold, $\omega$ continuously differentiable alternating differential form of order $n-1$):
			\begin{equation}
				\int_\mathcal{M} \dd \omega = \int_{\partial\mathcal{M}} \omega
			\end{equation}

			\noindent
			Divergence theorem / Gauß theorem\index{Gauß!Integralsatz}:
			\begin{equation}
				\oint_{\partial\mathcal{V}}\vec{f}\cdot\vec{n}\,\mathrm{d}A=\int_{\mathcal{V}}\vec{\nabla}\cdot\vec{f}\;\mathrm{d}V
			\end{equation}

			\noindent
			Green's theorem\index{Green!Integralsatz}:
			\begin{equation}
				\oint_{\partial\mathcal{A}}\vec{f}\cdot\mathrm{d}\vec{s}=\int_{\mathcal{A}}\left(\vec{\nabla}\times\pvec{f}\right)\cdot\vec{n}\;\mathrm{d}A
			\end{equation}

		\subsubsection{Coordinate Transformations}
			\noindent
			Transformation between cartesian, cylindrical and spherical coordinates:
			\begin{center}
				\begin{tabular}{| r || l | l | l |}
					\hline\xrowht{10pt}
					Coordinates & Cartesian & Cylindrical & Spherical \\
					\hline\hline\xrowht{45pt}
					Cartesian & $\begin{aligned}  x &= x \\  y &= y \\  z &= z\end{aligned}$ & $\begin{aligned}  x &= \rho \cos\varphi \\  y &= \rho \sin\varphi \\  z &= z\end{aligned}$ & $\begin{aligned}  x &= r \sin\theta \cos\varphi \\  y &= r \sin\theta \sin\varphi \\  z &= r \cos\theta\end{aligned}$ \\
					\hline\xrowht{45pt}
					Cylindrical & ${\displaystyle {\begin{aligned}\rho &={\sqrt {x^{2}+y^{2}}}\\\varphi &=\arctan \left({\frac {y}{x}}\right)\\z&=z\end{aligned}}}$ & ${\displaystyle {\begin{aligned}\rho &=\rho \\\varphi &=\varphi \\z&=z\end{aligned}}}$ & ${\displaystyle {\begin{aligned}\rho &=r\sin \theta \\\varphi &=\varphi \\z&=r\cos \theta \end{aligned}}}$ \\
					\hline\xrowht{70pt}
					Spherical & ${\displaystyle {\begin{aligned}r&={\sqrt {x^{2}+y^{2}+z^{2}}}\\\theta &=\arctan \left({\frac {\sqrt {x^{2}+y^{2}}}{z}}\right)\\\varphi &=\arctan \left({\frac {y}{x}}\right)\end{aligned}}}$ & ${\displaystyle {\begin{aligned}r&={\sqrt {\rho ^{2}+z^{2}}}\\\theta &=\arctan {\left({\frac {\rho }{z}}\right)}\\\varphi &=\varphi \end{aligned}}}$ & ${\displaystyle {\begin{aligned}r&=r\\\theta &=\theta \\\varphi &=\varphi \\\end{aligned}}}$ \\
					\hline
				\end{tabular}
			\end{center}
			\href{https://en.wikipedia.org/wiki/Vector_calculus_identities}{Vector calculus identities (en.wikipedia.org/wiki/Vector\_calculus\_identities)}


			\begin{center}
				\makebox[1\textwidth][c]{
				\begin{tabular}{| r || l | l | l |}
					\hline\xrowht{10pt}
					Unit vectors & Cartesian & Cylindrical & Spherical \\
					\hline\hline\xrowht{45pt}
					Cartesian & N/A & $\begin{aligned}  \hat{\mathbf x} &= \cos\varphi \hat{\boldsymbol \rho} - \sin\varphi \hat{\boldsymbol \varphi} \\  \hat{\mathbf y} &= \sin\varphi \hat{\boldsymbol \rho} + \cos\varphi \hat{\boldsymbol \varphi} \\  \hat{\mathbf z} &= \hat{\mathbf z}\end{aligned}$ & $\begin{aligned}  \hat{\mathbf x} &= \sin\theta \cos\varphi \hat{\mathbf r} + \cos\theta \cos\varphi \hat{\boldsymbol \theta} - \sin\varphi \hat{\boldsymbol \varphi} \\  \hat{\mathbf y} &= \sin\theta \sin\varphi \hat{\mathbf r} + \cos\theta \sin\varphi \hat{\boldsymbol \theta} + \cos\varphi \hat{\boldsymbol \varphi} \\  \hat{\mathbf z} &= \cos\theta \hat{\mathbf r} - \sin\theta \hat{\boldsymbol \theta}\end{aligned}$ \\
					\hline\xrowht{45pt}
					Cylindrical & ${\displaystyle {\begin{aligned}{\hat {\boldsymbol {\rho }}}&={\frac {x{\hat {\mathbf {x} }}+y{\hat {\mathbf {y} }}}{\sqrt {x^{2}+y^{2}}}}\\{\hat {\boldsymbol {\varphi }}}&={\frac {-y{\hat {\mathbf {x} }}+x{\hat {\mathbf {y} }}}{\sqrt {x^{2}+y^{2}}}}\\{\hat {\mathbf {z} }}&={\hat {\mathbf {z} }}\end{aligned}}}$ & N/A & ${\displaystyle {\begin{aligned}{\hat {\boldsymbol {\rho }}}&=\sin \theta {\hat {\mathbf {r} }}+\cos \theta {\hat {\boldsymbol {\theta }}}\\{\hat {\boldsymbol {\varphi }}}&={\hat {\boldsymbol {\varphi }}}\\{\hat {\mathbf {z} }}&=\cos \theta {\hat {\mathbf {r} }}-\sin \theta {\hat {\boldsymbol {\theta }}}\end{aligned}}}$ \\
					\hline\xrowht{70pt}
					Spherical & ${\displaystyle {\begin{aligned}{\hat {\mathbf {r} }}&={\frac {x{\hat {\mathbf {x} }}+y{\hat {\mathbf {y} }}+z{\hat {\mathbf {z} }}}{\sqrt {x^{2}+y^{2}+z^{2}}}}\\{\hat {\boldsymbol {\theta }}}&={\frac {\left(x{\hat {\mathbf {x} }}+y{\hat {\mathbf {y} }}\right)z-\left(x^{2}+y^{2}\right){\hat {\mathbf {z} }}}{{\sqrt {x^{2}+y^{2}+z^{2}}}{\sqrt {x^{2}+y^{2}}}}}\\{\hat {\boldsymbol {\varphi }}}&={\frac {-y{\hat {\mathbf {x} }}+x{\hat {\mathbf {y} }}}{\sqrt {x^{2}+y^{2}}}}\end{aligned}}}$ & ${\displaystyle {\begin{aligned}{\hat {\mathbf {r} }}&={\frac {\rho {\hat {\boldsymbol {\rho }}}+z{\hat {\mathbf {z} }}}{\sqrt {\rho ^{2}+z^{2}}}}\\{\hat {\boldsymbol {\theta }}}&={\frac {z{\hat {\boldsymbol {\rho }}}-\rho {\hat {\mathbf {z} }}}{\sqrt {\rho ^{2}+z^{2}}}}\\{\hat {\boldsymbol {\varphi }}}&={\hat {\boldsymbol {\varphi }}}\end{aligned}}}$ & N/A \\
					\hline
				\end{tabular}}
			\end{center}


			\noindent
			Gradient:
			\begin{equation}
				\begin{aligned}
					\Nabla f = \pdv{f}{\vec{x}} &= \frac{\partial f}{\partial x}\hat{\mathbf x} + \frac{\partial f}{\partial y}\hat{\mathbf y}+ \frac{\partial f}{\partial z}\hat{\mathbf z} \\
					&= \frac{\partial f}{\partial \rho}\hat{\boldsymbol \rho}+ \frac{1}{\rho}\frac{\partial f}{\partial \varphi}\hat{\boldsymbol \varphi}+ \frac{\partial f}{\partial z}\hat{\mathbf z} \\
					&= \frac{\partial f}{\partial r}\hat{\mathbf r}+ \frac{1}{r}\frac{\partial f}{\partial \theta}\hat{\boldsymbol \theta}+ \frac{1}{r\sin\theta}\frac{\partial f}{\partial \varphi}\hat{\boldsymbol \varphi} \\
				\end{aligned}
			\end{equation}

			\noindent
			Divergence:
			\begin{equation}
				\begin{aligned}
					\Nabla\cdot\vec{A} &= \frac{\partial A_x}{\partial x} + \frac{\partial A_y}{\partial y} + \frac{\partial A_z}{\partial z} \\
					&= \frac{1}{\rho}\frac{\partial \left( \rho A_\rho  \right)}{\partial \rho}+ \frac{1}{\rho}\frac{\partial A_\varphi}{\partial \varphi}+ \frac{\partial A_z}{\partial z} \\
					&= \frac{1}{r^2}\frac{\partial \left( r^2 A_r \right)}{\partial r}+ \frac{1}{r\sin\theta}\frac{\partial}{\partial \theta} \left(  A_\theta\sin\theta \right)+ \frac{1}{r\sin\theta}\frac{\partial A_\varphi}{\partial \varphi} \\
				\end{aligned}
			\end{equation}

			\noindent
			Rotation:
			\begin{equation}
				\begin{aligned}
					\Nabla\times\vec{A} &= \left(\frac{\partial A_z}{\partial y} - \frac{\partial A_y}{\partial z}\right) \hat{\mathbf x} + \left(\frac{\partial A_x}{\partial z} - \frac{\partial A_z}{\partial x}\right) \hat{\mathbf y} + \left(\frac{\partial A_y}{\partial x} - \frac{\partial A_x}{\partial y}\right) \hat{\mathbf z} \\
					&= {\displaystyle {\left({\frac {1}{\rho }}{\frac {\partial A_{z}}{\partial \varphi }}-{\frac {\partial A_{\varphi }}{\partial z}}\right){\hat {\boldsymbol {\rho }}}+\left({\frac {\partial A_{\rho }}{\partial z}}-{\frac {\partial A_{z}}{\partial \rho }}\right){\hat {\boldsymbol {\varphi }}}+{\frac {1}{\rho }}\left({\frac {\partial \left(\rho A_{\varphi }\right)}{\partial \rho }}-{\frac {\partial A_{\rho }}{\partial \varphi }}\right){\hat {\mathbf {z} }}}} \\
					&= {\displaystyle {{\frac {1}{r\sin \theta }}\left({\frac {\partial }{\partial \theta }}\left(A_{\varphi }\sin \theta \right)-{\frac {\partial A_{\theta }}{\partial \varphi }}\right){\hat {\mathbf {r} }}+\frac{1}{r}\left({\frac {1}{\sin \theta }}{\frac {\partial A_{r}}{\partial \varphi }}-{\frac {\partial }{\partial r}}\left(rA_{\varphi }\right)\right){\hat {\boldsymbol {\theta }}}+{\frac {1}{r}}\left({\frac {\partial }{\partial r}}\left(rA_{\theta }\right)-{\frac {\partial A_{r}}{\partial \theta }}\right){\hat {\boldsymbol {\varphi }}}}} \\
				\end{aligned}
			\end{equation}

			\noindent
			Scalar Laplace operator\index{Laplace!Operator}:
			\begin{equation}
				\begin{aligned}
					\Delta f = \Nabla^2 f &= \frac{\partial^2 f}{\partial x^2} + \frac{\partial^2 f}{\partial y^2} + \frac{\partial^2 f}{\partial z^2} \\
					&= \frac{1}{\rho} \frac{\partial}{\partial \rho}\left(\rho \frac{\partial f}{\partial \rho}\right)+ \frac{1}{\rho^2} \frac{\partial^2 f}{\partial \varphi^2}+ \frac{\partial^2 f}{\partial z^2} \\
					&= {\displaystyle \frac{1}{r^{2}} \frac{\partial }{\partial r}\!\left(r^{2}\frac{\partial f}{\partial r}\right)\!+\!\frac{1}{r^{2}\!\sin \theta } \frac{\partial }{\partial \theta }\!\left(\sin \theta \frac{\partial f}{\partial \theta }\right)\!+\!\frac{1}{r^{2}\!\sin ^{2}\theta }\frac{\partial ^{2}f}{\partial \varphi ^{2}}}
				\end{aligned}
			\end{equation}


	\subsection{Stochastic}
		\subsubsection{General}
			\noindent
			Factorial:
			\begin{equation}
				n!=\Gamma(n+1)=\prod_{k=1}^{n}k=(n)(n-1)(n-2)...
			\end{equation}

			\noindent
			Binomial coefficient ($n\ge k$):
			\begin{equation}
				\binomkoeff{n}{k} = \frac{k!}{\left(n-k\right)!\,k!}
			\end{equation}

			\noindent
			Properties of the binomial coefficient:
			\begin{equation}
				\begin{array}{cl}
					\binomkoeff{n}{0}=\binomkoeff{n}{n} = 1 & \binomkoeff{n}{k} = \binomkoeff{n}{n-k}\\ [8pt]
					\binomkoeff{n+1}{k+1}=\binomkoeff{n}{k}+\binomkoeff{n}{k+1}\\
				\end{array}
			\end{equation}

		\subsubsection{Expectation Value and Variance}
			\noindent
			Definition of the expectation value:
			\begin{equation}
				\mu = E\left[ X \right] := \int_{-\infty}^{\infty} xf(x)\;\dd x
			\end{equation}

			\noindent
			Definition of variance and standard deviation:
			\begin{equation}
				\sigma^2 = V\left[ X \right] := \int_{-\infty}^{\infty} (x-E\left[X\right])^2f(x)\;\dd x = E\left[X^2\right]-E^2\left[X\right]
			\end{equation}

			\noindent
			Arithmetic mean:
			\begin{equation}
				\bar{x}=\frac{1}{n}\sum_{j=1}^n x_i
			\end{equation}

			\noindent
			Weighed mean:
			\begin{equation}
				\begin{aligned}
					\hat{x} &= \frac{\sum_{j=1}^n \frac{x_j}{\sigma_j^2}}{\sum_{j=1}^n \frac{1}{\sigma_j^2}} \\
					\sigma_{\hat{x}}^2 &= \frac{1}{\sum_{j=1}^n \frac{1}{\sigma_j^2}}
				\end{aligned}
			\end{equation}

			\noindent
			Empirical variance:
			\begin{equation}
				s^2 = \frac{1}{n-1}\sum_{j=1}^n (x_j-\bar{x})^2
			\end{equation}

			\noindent
			Definition of variance:
			\begin{equation}
				V_{xy} = E\left[(x-\mu_x)(y-\mu_y)\right] = E\left[xy\right]-\mu_x\mu_y
			\end{equation}

			\noindent
			Correlation coefficient ($-1\le\rho_{xy}\le 1$):
			\begin{equation}
				\rho_{xy} = \frac{V_{xy}}{\sigma_x\sigma_y}
			\end{equation}

			\noindent
			Properties of expectation value and variance (The last equation is only valid for independent $X, Y$):
			\begin{equation}
				\begin{array}{rl}
					E\left[aX\right] = \phantom{^2}a E\left[X\right]
					&\hspace{20pt}
					E\left[X+Y\right] = E\left[X\right] + E\left[Y\right]
					\\
					V\left[aX\right] = a^2 V\left[X\right]
					&\hspace{20pt}
					V\left[X+Y\right] = E\left[X\right] + E\left[Y\right]
					\\
				\end{array}
			\end{equation}

		\subsubsection{Probability-Generating Function}
			\noindent
			For a random quantity $X$, the probability generating function is defined as:
			\begin{equation}
				Z(\lambda) = E\left[ \exp(\lambda X) \right]
			\end{equation}

			\noindent
			The $n$\textsuperscript{th} raw moment is defined as:
			\begin{equation}
				E\left[X^n\right] = \eval{\dv[n]{Z}{\lambda}}_{\lambda=0}
			\end{equation}

			\noindent
			From the Moment-generating function $F(\lambda)=\ln Z(\lambda)$ one can extract
			\begin{equation}
				\begin{aligned}
					F(0) &= 0 \\
					\dv{F}{\lambda}(0) &= E\qty[X] \\
					\dv[2]{F}{\lambda}(0) &= V\qty[X] \\
				\end{aligned}
			\end{equation}


		\subsubsection{Propagation of Uncertainty}
			\noindent
			Gaußian error propagation\index{Gauß!Fehlerfortpflanzungsgesetz}:
			\begin{equation}
				\sigma_y^2 = \sum_{j,k=1}^n \qty[\pdv{y}{x_j}\pdv{y}{x_k}]_{\vec{x}=\vec{\mu}} V_{jk}
			\end{equation}

			\noindent
			Gaußian error propagation\index{Gauß!Fehlerfortpflanzungsgesetz} for uncorrelated  $x_i$ (i.e. $V_{jj} = \sigma_{j}^2$ and $V_{jk} = 0 \quad\forall j\ne k$):
			\begin{equation}
				\sigma_y^2 = \sum_{j=1}^n \qty[\pdv{y}{x_j}]^2_{\vec{x}=\vec{\mu}} \sigma_j^2
			\end{equation}


		\subsubsection{Conditional Probabilities}
			\noindent
			Definition of conditional probability:
			\begin{equation}
				P(A|B) := \frac{P(A \cap B)}{P(B)}
			\end{equation}

			\noindent
			Bayes theorem\index{Bayes!Satz}:
			\begin{equation}
				P(A|B) = \frac{P(B|A)P(A)}{P(B)}
			\end{equation}

		\subsubsection{Discrete Distributions}
			\noindent
			Binomial distribution\index{Binomialverteilung} ($k$ of $n$ of a Bernoulli experiment\index{Bernoulli!Experiment}):
			\begin{equation}
				\begin{aligned}
					P(k)&=\binomkoeff{n}{k}p^k(1-p)^{n-k} \\
					\mu &= np \\
					\sigma^2 &= np(1-p)
				\end{aligned}
			\end{equation}

			\noindent
			Poisson distribution\index{Poisson!Verteilung}:
			\begin{equation}
				\begin{aligned}
					\rho(k) &= \frac{\nu^k}{k!} e^{-\nu} \\
					\mu &= \sigma^2 = \nu
				\end{aligned}
			\end{equation}

			\noindent
			Hypergeometric distribution ($k$ of $M\le N$ of $n$ draws):
			\begin{equation}
				P(x)=\frac{\binomkoeff{M}{k}\binomkoeff{N-M}{n-k}}{\binomkoeff{N}{n}}
			\end{equation}

		\subsubsection{Continuous Distributions}
			\noindent
			Central limit theorem\index{Zentraler Grenzwertsatz}: \par
				\emph{The sum of $n$ independent continuous random variables with expectation value $\mu_j$ and finite variances $\sigma_j^2$ converge in the limit $n\rightarrow \infty$ to a normal distribution with $\mu = \sum_j \mu_j$ and $\sigma^2 = \sum_j \sigma_j^2$.} \vsp

			\noindent
			Normal distribution\index{Gauß!Verteilung}:
			\begin{equation}
				\rho(x)=\frac{1}{\sigma\sqrt{2\pi}}e^{-\frac{1}{2}\left(\frac{x-\mu}{\sigma}\right)^2}
			\end{equation}

			\noindent
			Uniform distribution:
			\begin{equation}
				\begin{aligned}
					\rho(x) =& \left\{\begin{array}{ll}
					\frac{1}{\beta-\alpha} & \alpha\le x\le \beta \\
					0 & \text{else} \\
					\end{array}\right. \\
					\mu =& \frac{1}{2}(\alpha+\beta) \\
					\sigma^2 =& \frac{1}{12}(\beta-\alpha)^2 \\
				\end{aligned}
			\end{equation}

	\subsection{Group Theory}
		\subsubsection{Basics}
			\emph{Group} $(G, \diamond)$ (Where $G$ is a group and $\diamond$ is an operation):
			\begin{equation}
				\begin{aligned}
					\diamond: &G \times G \rightarrow G;\, (p,q) \mapsto p\diamond q \\
					\text{such that}\; &(p\diamond q)\diamond r = p\diamond (q\diamond r)\quad\forall{p,q,r}\subseteq G \\
					&\exists e: e\diamond p = p\diamond e = p\quad\forall p\in G\\
					&\forall p \in G: \exists p^{-1}: p\diamond p^{-1} = p^{-1}\diamond p=e
				\end{aligned}
			\end{equation}
			
			\noindent
			\emph{Homomorphism} $\phi: G_1 \rightarrow G_2$ (Groups $(G_1,\diamond)$, $(G_2,\diamond)$)
			\begin{equation}
				\begin{aligned}
					\phi(a \diamond b) = \phi(a) \diamond \phi(b) \quad \forall a,b \in G_1
				\end{aligned}
			\end{equation}
			
			\noindent
			\emph{Action} of $G\ni a, b$ on a set $M\ni m$:
			\begin{equation}
				\begin{aligned}
					a\diamond (b\diamond m) &= (a \diamond b) \diamond m \\
					e \diamond m &= m \\
				\end{aligned}
			\end{equation}
			
			\noindent
			\emph{Orbit} of $m\in M$ under the action of $G$:
			\begin{equation}
				G \diamond m = \qty{ a\diamond m \quad \forall a \in G}
			\end{equation}

			\noindent
			\emph{Isotropy group} $G_m$ of $m\in M$ with respect to $G$:
			\begin{equation}
				G_m = \qty{a\in G: a\diamond m = m} \subseteq{G}
			\end{equation}			
			
			\noindent
			Number of elements in a group $G$ is denoted by $\# G$.
			\begin{equation}
				\# G = \#(G \diamond m) \#G_m \quad \forall m \in M
			\end{equation}

			\noindent
			The (left) \emph{coset} $g \diamond H$ of $H \subseteq G$
			\begin{equation}
				g \diamond H = \qty{g\diamond h: h\in H} 
			\end{equation}

			\noindent
			A \emph{transitive} group is a group with a transitive group action $\diamond$
			\begin{equation}
				\forall g \in G, g \ne e: \quad G \diamond g = G
			\end{equation}

			\noindent
			\emph{Conjugacy class}, $b$ and $c$ belong to the same conjugacy class if
			\begin{equation}
				\exists a \in G: \quad a \diamond b \diamond a^{-1} = c
			\end{equation}

			\noindent
			Lemma on isotropy subgroups
			\begin{equation}
				G_{a m} = a G_{m} a^{-1}
			\end{equation}
			
			\noindent
			Euler's\index{Euler} description of a Rotation $R\in SO(3)$ with Euler angles $\phi, \theta, \psi$
			\begin{equation}
				\forall R \in SO(3):\quad \exists \phi, \theta, \psi:\quad R = R_\phi^z R_\theta^y R_\psi^x
			\end{equation}
			
			\begin{table}[ht]
				\begin{center}
				\begin{tabular}{ l | l | l }
					Symbol & Name & Example \\ \hline
					$C_n$ & Cyclic group of $n$ elements & $\set{[0]_n, [1]_n, ..., [n-1]_n}$, $\diamond$: Equivalence class addition \\
					$GL(n, \mathbb{K})$ & General Linear group & $\mathbb{R}^{n\times n}$, $\mathbb{C}^{n\times n}$, $\diamond$: Matrix multiplication \\
					$SL(n, \mathbb{K})$ & Special linear group & $\set{L: \det L = 1} \subseteq GL(n, \mathbb{K})$ \\
					$O(n)$ & Orthogonal group & $\set{O: OO^{T}=O^{T}O=1}\subseteq GL(n, \mathbb{R})$ \\
					$SO(n)$ & Special orthogonal group& $\set{O: \det O = 1}\subseteq O(n)$\\
					$U(n)$ & Unitary group & $\set{U: UU^{\dagger}=U^{\dagger}U=1}\subseteq GL(n, \mathbb{C})$ \\
					$SU(n)$ & Special unitary group & $\set{U: \det U = 1}\subseteq U(n)$ \\
					\end{tabular}
					\caption{List of common groups with symbol, name, example set and corresponding example operation. $\mathbb{K}$ is a field\index{Körper}}
				\end{center}
			\end{table} \vsp


		% \subsubsection{Selection of Groups Relevant for Physics}
		% 	Lorentz group $L$ ($\norm{x} = x^\mu x_\mu$, see ):
		% 	\begin{equation}
		% 		L = \left( \qty{\Lambda\in GL(4, \mathbb{R}): \norm{\Lambda x}= \norm{x} \quad\forall x \in \mathbb{R}^4}, \text{Matrix multiplication} \right)
		% 	\end{equation}
		