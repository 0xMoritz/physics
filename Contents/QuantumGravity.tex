% !TEX root = ../physics.tex
\section{Quantum Gravity}
	\subsection{Functional Renormalization}
		Effective Action\index{Effektive!Wirkung} ($\phi(x) = \Avg{\varphi(x)}_J = \fdv{W\qty[J]}{J(x)}$, where $W[J]$ is the Schwinger Functional\index{Schwinger!Functional}):
		\begin{equation}
			\Gamma(\phi) = \sup_{J} \qty( \int \dd^d x \, J(x) \phi(x) - W[J])
		\end{equation}
		From now on, the supremum of the current $J$ is assumed implicitly.

		\noindent
		Quantum equations of motion:
		\begin{equation}
			J(x)=\fdv{\Gamma\qty[\phi]}{\phi(x)}
		\end{equation}

		\noindent
		Propagator:
		\begin{equation}
			\frac{\delta^2 W[J]}{\delta J(x) \delta J(y)} = G(x,y) = \frac{1}{\frac{\delta^2\Gamma[\phi]}{\delta \phi(x) \delta \phi(y)}}
		\end{equation}

		\noindent
		Expansion of the $n$-point correlation function:
		\begin{equation}
			\Avg{\varphi(x_1) \hdots\varphi(x_n)}_J = \prod_{i=1}^{n} \qty(\int \dd^d x'_i \;G(x_i,x'_i) \fdv{\phi(x'_i)} + \phi(x_i))
		\end{equation}

		\noindent
		Functional Dyson--Schwinger Equation\index{Dyson--Schwinger Gleichung!funktional}:
		\begin{equation}
			\fdv{\Gamma[\phi]}{\phi(x)} = \fdv{S}{\varphi(x)}\qty[\varphi = \phi + \int \dd^d y\; G(x,y) \fdv{}{\phi(y)}]
		\end{equation}

		\noindent
		General Dyson--Schwinger Equation:
		\begin{equation}
			\Avg{\Psi[\varphi]} \fdv{\Gamma[\phi]}{\phi(x)} = \Avg{\Psi[\varphi] \fdv{S[\varphi]}{\varphi(x)}} - \Avg{\fdv{\Psi[\varphi]}{\varphi(x)}}
		\end{equation}

		\noindent
		Ward--Takahashi identity\index{Ward!--Takahashi Identität}\index{Takahashi!Ward--Takahashi Identität} for a gauge invariance of the effective action $\Gamma[\phi]$ with generator $G$ and the corresponding gauge fixing and ghost actions:
		\begin{equation}
			0 = G \Gamma[\phi] - \Avg{G(S_\text{g.f.} + S_\text{gh.})}_{J[\phi]}
		\end{equation}

	\subsection{Flow Equations}
		Infrared regularized generating functional with a hard infrared cutoff scale $k$:
		\begin{equation}
			Z_k[J] = \int \mathcal{D}\varphi_{p^2 \ge k^2} \exp(-S[\varphi] - \int \dd^d x \, J(x) \varphi(x))
			\hsp \mathcal{D}\varphi_{p^2 \ge k^2} = \prod_{p^2 \ge k^2} \frac{\dd \varphi_p}{(2\pi)^d}
		\end{equation}

		\noindent
		Soft cutoff integration measure
		\begin{equation}
			\mathcal{D}\varphi_{p^2 \ge k^2} := \mathcal{D} \exp(-\Delta S_k[\varphi])
			\hsp \Delta S_k[\varphi] := \frac{1}{2} \int \frac{\dd^d p}{(2\pi)^d}\, \varphi(p)R_k(p^2) \varphi(-p)
		\end{equation}
		with a Regulator function\index{Regulator Funktion} $R_k(p^2)$ that satisfies
		\begin{itemize}\itemsep -0pt
			\item $\displaystyle\lim_{p^2 \to 0} R_k(p^2) > 0$ \hfill{(Suppression of IR modes)}
			\item $\displaystyle\lim_{k \to 0} R_k(p^2) = 0$ \hfill{(Physical limit)}
			\item $\displaystyle\lim_{k \to \Lambda \to \infty} R_k(p^2) = \infty$ \hfill{(UV limit)}
		\end{itemize}
		hence, the generating functional with a soft cutoff is
		\begin{equation}
			Z_k[J] := \ex^{-\Delta S_k [\fdv{J}]}Z[J]
		\end{equation}
		Note that there are complications when dealing with fermionic regulators

		\noindent
		Example regulators ($R_k(p^2) = p^2 r(y)$ with $y=p^2/k^2$):
		\begin{itemize} \itemsep -0pt
			\item Litim regulator\index{Litim!Regulator}: $r(y) = \qty(\dfrac{1}{y}-1)\theta(1-y)$, \ie $R_k(p^2) = (k^2 - p^2) \Theta(k^2 - p^2)$
			\item Exponential regulator: $r(y) = \dfrac{cy^{b-1}}{\exp(y^b) - 1}$
			\item Sharp regulator $r(y) = \dfrac{1}{\Theta(y-1)}-1$
		\end{itemize}

		\noindent
		Renormalization group time / RG time and beta functions (using a UV cutoff $\Lambda$):
		\begin{equation}
			t = \ln \frac{k}{\Lambda}
			\hsp \beta_g := \dot{g} = \partial_t g = k \partial_k g
		\end{equation}

		\noindent
		Anomalous dimension (where $Z_\psi(k)$ is the wave function renormalization):
		\begin{equation}
			\eta_\psi = -\partial_t \ln Z_\psi
		\end{equation}

		\noindent
		Flow equation for the Schwinger Functional\index{Schwinger!Funktional}:
		\begin{equation}
			\partial_t W_k[J] = -\frac{1}{2}\int \frac{\dd^d p}{(2\pi)^d} \, \qty[W^{(2)}_k[J]+(W_k^{(1)}[J])^2](p,-p) \partial_t R_k(p^2)
		\end{equation}

		\noindent
		Flowing / Scale dependent effective action / Effective average action and modified quantum equations of motion / effective field equations ($\phi(x)= \fdv{W_k[J_k]}{J_k(x)}$):
		\begin{equation}
			\Gamma_k[\phi] = \sup_{J_k} \qty{\int \dd^d x\, J_k(x) \phi(x) - W_k[J_k] - \Delta S_k[\phi]}
			\hsp \fdv{(\Gamma_k[\phi] + \Delta S_k[\phi])}{\phi(x)} = J_k(x)
		\end{equation}

		\noindent
		The flowing effective action interpolates between the classical action $S = \Gamma_\Lambda$ where quantum fluctuations are fully suppressed and the full quantum effective action $\Gamma$ where all fluctuations are included:
		\begin{equation}
			\Gamma \;\stackrel{k \to 0}{\longleftarrow}\; \Gamma_{k} \stackrel{k \to \infty}{\longrightarrow} S
		\end{equation}

		\noindent
		Wetterich Equation\index{Wetterich!Gleichung} / flow equation for the effective action / functional renormalization group equation (Note the inclusion of a minus sign in tracing out fermionic fields, and the change of sign for the momentum of transposed field derivatives, the multi-Index $I$ denotes the combination of all occurring internal indices):
		\begin{equation}
			\begin{aligned}
				\partial_t \Gamma_k[\vec{\Phi}] 
				&= \frac{1}{2} \tr G_k \;\partial_t R_k \\
				&= \frac{1}{2} \tr \frac{1}{\Gamma^{(2)}_k + R_k} \partial_t R_k \\
				&= \frac{1}{2} \tr \int \frac{\dd^d p}{(2\pi)^d} \, \frac{1}{\Gamma^{(2)}_k[\vec{\Phi}] + R_k}(p,-p) \partial_t R_k(p^2) \\
				&= \frac{1}{2} \sum_I \pm \int \frac{\dd^d p}{(2\pi)^d} \, \frac{1}{\frac{\delta^2 \Gamma_k[\vec{\Phi}]}{\delta \Phi^I(-p) \delta \Phi_I(p)} + R_k(p^2)} \partial_t R_k(p^2) \\
				&= \frac{1}{2} \tr \eval{\partial_{\tilde{t}} \ln(\Gamma^{(2)}_k[\vec{\Phi}] + R_{\tilde{k}})}_{\tilde{k} = k}
			\end{aligned}
		\end{equation}

	\subsection{Asymptotic Safety}
		\noindent
		Definition of a fixed point $\vec{g}_{*}$:
		\begin{equation}
			\beta_{g_i}(\vec{g}_{*}) = 0 \quad \forall i
		\end{equation}

		\noindent
		Stability Matrix:
		\begin{equation}
			\mathcal{M}_{ij} = \eval{\pdv{\beta_{g_i}}{g_j}}_{\vec{g} = \vec{g_*}}
		\end{equation}
		Critical Exponent (they correspond to eigenvectors $V_I$ of the stability matrix)
		\begin{equation}
			\theta_I = - \mathrm{eig}(\mathcal{M})
			\hsp \theta_I V_I = \mathcal{M} V_I
		\end{equation}
		Fixed point expansion of the beta functions:
		\begin{equation}
			\beta_{g_i} = \sum_{j} \mathcal{M}_{ij} \qty(g_j - g_{j*}) + \mathcal{O}\qty(g_j - g_{j*})^2
		\end{equation}
		Solution (where $c_i$ are integration constants):
		\begin{equation}
			g_i(k) = g_{i*} + \sum_I c_I V_I \qty(\frac{k}{k_0})^{-\theta_I}
		\end{equation}
		\begin{itemize}
			\item $\Re(\theta_I) > 0$: relevant direction $V_I$ (IR repulsive, UV attractive), contributing a free parameter $c_I$
			\item $\Re(\theta_I) < 0$: irrelevant direction $V_I$ (IR attractive, UV repulsive), no free parameter $c_I$ insignificant
			\item $\Re(\theta_I) = 0$: marginal, look at higher order contributions
		\end{itemize}

	\subsection{Application to Quantum Gravity}
		Assumptions:
		\begin{itemize}
			\item Diffeomorphism invariance is the fundamental symmetry of spacetime.
			\item The metric carries the fundamental degrees of freedom.
		\end{itemize}

		\noindent
		Partition function:
		\begin{equation}
			Z[j] = \int \mathcal{D} g_{\mu\nu} \exp{-S_\text{gravity}[g_{\mu\nu}] + \int J^{\mu\nu}(x) g_{\mu\nu}(x) \dd[4]{x}}
		\end{equation}

		\noindent
		Effective Action (Using a background metric $g_{\mu\nu} = \bar{g}_{\mu\nu} + h_{\mu\nu}$):
		\begin{equation}
			\exp(-\Gamma_k[g_{\mu\nu}]) = \int \mathcal{D}h_{\mu\nu} \, \exp(-S[h_{\mu\nu}] - \int h_{\mu\nu} R_k(-\bar{D})^{\mu\nu\rho\sigma}h_{\rho\sigma} \dd[4]{x})
		\end{equation}

		\noindent
		Gauß--Bonnet term\index{Gauß!--Bonnet Term}:
		\begin{equation}
			E = R_{\mu\nu\rho\sigma}^2 - 4 R_{\mu\nu}^2 + R^2
			\hsp \fdv{E}{g_{\mu\nu}} = 0
		\end{equation}

		\noindent
		Lie derivative (\wrt to a vector field $\omega_\mu$; constitute generators of diffeomorphism transformation)
		\begin{equation}
			\begin{aligned}
				\mathcal{L}_\omega T_{\alpha\beta\dots} 
				&= \omega^\rho \partial_\rho T_{\alpha\beta\dots} + T_{\rho\beta\dots} \partial_\alpha \omega^\rho + T_{\alpha\rho\dots} \partial_\beta \omega^\rho + \dots \\
				&= \omega^\rho \nabla_\rho T_{\alpha\beta\dots} + T_{\rho\beta\dots} \nabla_\alpha \omega^\rho + T_{\alpha\rho\dots} \nabla_\beta \omega^\rho + \dots
			\end{aligned}
		\end{equation}

		\noindent
		Separation of background metric $\bar{g}_{\mu\nu}$ and fluctuation metric $h_{\mu\nu}$ in linear or exponential split:
		\begin{align}
			g_{\mu\nu} &= \bar{g}_{\mu\nu} + h_{\mu\nu} \\
			g_{\mu\nu} &= \bar{g}_{\mu\rho}\exp(h)^\rho{}_{\nu} \\
		\end{align}

		\noindent
		Split Symmetry:
		\begin{equation}
			g(\bar{g},h) \to g(\bar{g}+\delta \bar{g}, h + \delta h) = g(\bar{g},h)
		\end{equation}

		\noindent
		Nielsen identity\index{Nielsen!Identität} (Applying the Ward--Takahashi identity to the split symmetry):
		\begin{align}
			0 &= \fdv{\Gamma}{\bar{g}_{\mu\nu}} - \fdv{\Gamma}{h_{\mu\nu}} - \Avg{\qty(\fdv{\bar{g}_{\mu\nu}} - \fdv{h_{\mu\nu}})(S_\text{g.f.} + S_\text{gh.})} \\
			0 &= \mathrm{NI}_k - \frac{1}{2} \tr[\frac{1}{\sqrt{\bar{g}}} \fdv{\sqrt{\bar{g}}R_k}{\bar{g}_{\mu\nu}}G_k]
		\end{align}
