% !TEX root = ../physics.tex
\section{Cosmology\index{Kosmologie}}
	\subsection{Friedmann Universe}
		\emph{Cosmological Principle}
		\begin{itemize}
			\item On large enough scales, the universe is homogeneous
			\item On large enough scales, the universe is isotropic (for every observer).
		\end{itemize}

		\noindent
		\emph{Cosmological time} is the proper time measured by an observer at rest with respect to the local matter distribution.

		\subsubsection{FLRW Metric}
			Friedmann-Lemaitre-Robertson-Walker metric:
			\begin{equation}
				\dd s^2 = -\dd t^2 + a^2(t) \qty[\frac{\dd r^2}{1-kr^2} + r^2 \dd \Omega^2]
			\end{equation}
			using $(t, r, \theta, \phi)$ as coordinates and $k$ as the curvature of the universe (where $k=0$ for a flat universe, $k=1$ for a closed universe and $k=-1$ for an open universe). The scale factor $a(t)$ is a function of time only. Note that any other $k$ can always be transformed into one of these three via rescaling of $r$.
			\begin{equation}
				g_{\mu\nu} = \mathrm{diag}\qty(-1,\, \frac{a^2(t)}{1-kr^2},\, a^2(t) r^2,\, a^2(t) r^2 \sin^2 \theta)
			\end{equation}

			\noindent
			Friedmann equations:
			\begin{equation}
				\begin{aligned}
					\qty(\frac{\dot{a}}{a})^2 &= \frac{8 \pi G}{3} \rho - \frac{K c^2}{a^2} + \frac{\Lambda}{3} \\
					\frac{\ddot{a}}{a} &= -\frac{4 \pi G}{3} \qty(\rho + \frac{3p}{c^2}) + \frac{\Lambda}{3} \\
				\end{aligned}
			\end{equation}

			\noindent
			Definition Hubble Function/Hubble Parameter{\index{Hubble!Funktion}} (Hubble function today is $H_0$, also called Hubble ``constant''\index{Hubble!Konstante}):
			\begin{equation}
				H(t) := \frac{\dot{a}}{a}
			\end{equation}

			\noindent
			Conformal time:
			\begin{equation}
				\dd \tau := \frac{\dd t}{a(t)}
			\end{equation}

			\noindent
			Critical Density:
			\begin{equation}
				\rho_\text{cr.} (a) = \frac{3 H^2}{8 \pi G}
			\end{equation}

			\noindent
			Adiabatic Equation (Combination of first law of thermodynamics and Friedmann equation):
			\begin{equation}
				\dd \qty(\rho a^3 c^2) + p \dd (a^3) = 0
			\end{equation}

			\noindent
			General Friedmann Equation (Where $i$ indexes different energy density contributions with equation of state $p_i = w_i \rho_i$ and $\Omega_i = \frac{\rho_i}{\rho_\text{cr.}}$):
			\begin{equation}
				H^2 = H_0^2\sum_i \Omega_i a^{-3(1+w)}
			\end{equation}
			most prominent components are matter ($w=0$), radiation ($w=\frac{1}{3}$), dark energy/cosmological constant ($w=-1$) and the curvature term ($\Omega_{k,0}=-\frac{k c^2}{a^2 H^2}$).
			\begin{equation}
				H^2 = H_0^2 \qty(\Omega_{\text{m},0} a^{-3} + \Omega_{\text{r},0} a^{-4} + \Omega_{\Lambda,0} + \Omega_{k,0} a^{-2})				
			\end{equation}


		\subsubsection{Expanding Universe}
			Cosmological redshift (where $\lambda_0$ is the emitted wavelength and $\lambda$ is the observed wavelength):
			\begin{equation}
				z = \frac{\lambda_0 - \lambda}{\lambda}
			\end{equation}

			\noindent
			Connection between scale factor and redshift:
			\begin{equation}
				a(z) = \frac{1}{1+z}
			\end{equation}