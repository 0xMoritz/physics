% !TEX root = ../physics.tex
\section{Cosmology\index{Kosmologie}}
	\emph{Cosmological Principle}
	\begin{itemize}
		\item On large enough scales, the universe is homogeneous
		\item On large enough scales, the universe is isotropic for every observer who is at rest \wrt the local matter distribution.
	\end{itemize}

	\subsection{FLRW Universe}
		Friedmann--Lemaitre--Robertson--Walker metric\index{Friedmann Alexander!Friedmann--Lemaitre--Robertson--Walker Metrik}\index{Lemaitre, Georges!Friedmann--Lemaitre--Robertson--Walker Metrik}\index{Robertson, Howard Percy!Friedmann--Lemaitre--Robertson--Walker Metrik}\index{Walker, Arthur Geoffrey!Friedmann--Lemaitre--Robertson--Walker Metrik}\index{FLRW Metrik}:
		\begin{equation}
			\dd s^2 = -\dd t^2 + a^2(t) \qty[\frac{\dd r^2}{1-kr^2} + r^2 \dd \Omega^2]
			\hsp
			g_{\mu\nu} = \mathrm{diag}\qty(-1,\, \frac{a^2(t)}{1-kr^2},\, a^2(t) r^2,\, a^2(t) r^2 \sin^2 \theta)
		\end{equation}
		using $(t, r, \theta, \phi)$ as coordinates and $k = a^2 H^2 ( \Omega + \Omega_\Lambda - 1)$ as the curvature of the universe where
		\begin{itemize}
			\item $k=0$ for a flat universe, \ie $\Omega + \Omega_\Lambda = 1$
			\item $k=1$ for a closed universe, \ie $0 < \Omega + \Omega_\Lambda < 1$
			\item $k=-1$ for an open universe, \ie $\Omega + \Omega_\Lambda > 1$
		\end{itemize}
		The scale factor $a(t)$ is a function of time only. Note that any other $k$ can always be transformed into one of these three via rescaling of $r$. We define $a_0 \defeq 1$ for todays value.

		\noindent
		Friedmann equations (Applying the FLRW metric to the Einstein equations, using an ideal, motionless fluid $T^\mu_\nu = \text{diag}(-\rho, p, p, p)$):
		\begin{equation}
			\begin{aligned}
				\qty(\frac{\dot{a}}{a})^2 &= \frac{8 \pi G}{3} \rho - \frac{K c^2}{a^2} + \frac{\Lambda c^2}{3} \\
				\frac{\ddot{a}}{a} &= -\frac{4 \pi G}{3} \qty(\rho + \frac{3p}{c^2}) + \frac{\Lambda c^2}{3} \\
			\end{aligned}
		\end{equation}

		\noindent
		Definition Hubble Function / Hubble Parameter{\index{Hubble!Funktion}} (Hubble function today is $H_0$, also called Hubble ``constant''\index{Hubble!Konstante}):
		\begin{equation}
			H(t) \defeq \frac{\dot{a}}{a}
		\end{equation}

		\noindent
		Comoving coordinates $\vec{x}$, in relation to real coordinates $\vec{r}$ and Hubble's law\index{Hubble!Gesetz}:
		\begin{equation}
			\vec{r} = a(t) \vec{x}
			\hsp
			\dot{\vec{r}} = H \vec{r}
		\end{equation}

		\noindent
		Critical Density (\ie density leading to $k=0$):
		\begin{equation}
			\rho_\text{cr.} (a) = \frac{3 H^2}{8 \pi G}
		\end{equation}

		\noindent
		Adiabatic Equation (Combination of first law of thermodynamics and Friedmann equation):
		\begin{equation}
			\label{Eq:CosmologyAdiabaticEquation}
			\dd \qty(\rho a^3 c^2) + p \dd (a^3) = 0
		\end{equation}

		\noindent
		Fluid Equation (following form the Adiabatic Equation \ref{Eq:CosmologyAdiabaticEquation}):
		\begin{equation}
			\dot{\rho} + 3 H \qty(\rho + \frac{p}{c^2}) = 0
		\end{equation}

		\noindent
		General Friedmann Equation (Where $i$ indexes different energy density contributions with equation of state $p_i = w_i \rho_i c^2$ and $\Omega_i = \frac{\rho_i}{\rho_\text{cr.}}$ are the density parameters):
		\begin{equation}
			H^2 = H_0^2\sum_i \Omega_i a^{-3(1+w)} = H_0^2 E^2(a)
		\end{equation}
		the components are matter ($w=0$), radiation ($w=\frac{1}{3}$), dark energy / cosmological constant ($w=-1$, with $\rho_\Lambda = \frac{\Lambda}{8\pi G}$) and the curvature term ($\Omega_{k,0}=-\frac{k c^2}{H_0^2}$).
		\begin{equation}
			H^2 = H_0^2 \qty(\Omega_{\text{m},0} a^{-3} + \Omega_{\text{r},0} a^{-4} + \Omega_{\Lambda} + \Omega_{k,0} a^{-2})
		\end{equation}
		Under simplifying assumptions:
		\begin{itemize}
			\item Radiation dominated universe $a = \qty(2\sqrt{\Omega_{\text{r},0}} \, H_0 t)^{1/2}$
			\item Matter dominated universe $a = \qty(\frac{3}{2}\sqrt{\Omega_{\text{m},0}} \, H_0 t)^{2/3}$
			\item Dark energy dominated universe $a \propto \exp(\sqrt{\Omega_{\Lambda,0}} \, H_0 t)$
		\end{itemize}

		\noindent
		Cosmological redshift (where $\lambda_e$ is the emitted wavelength and $\lambda$ is the observed wavelength):
		\begin{equation}
			z = \frac{\lambda - \lambda_e}{\lambda_e}
		\end{equation}

		\noindent
		Connection between scale factor and redshift:
		\begin{equation}
			a(z) = \frac{1}{1+z}
		\end{equation}

		\noindent
		Change of wavelength due to redshift (emission of light with $\lambda_e$ at $t_e$, received with $\lambda_r$ at $t_r$) and in particular for observation today ($t_r=t_0$ \ie $a(t_r)=1$):
		\begin{equation}
			\frac{\lambda_r}{\lambda_e} = \frac{a(t_r)}{a(t_e)}
			\hsp
			\frac{\lambda_0}{\lambda_e} = 1+z
		\end{equation}

	\subsection{Big Bang Theory}
		\noindent
		\href{https://en.wikipedia.org/wiki/Chronology_of_the_universe#Tabular_summary}{History of the universe}:
		\begin{table}[ht]
			\begin{center}
				\begin{tabular}{ l | l | l | l }
					Event & Time & Redshift & Temperature \\ \hline
					Planck epoch & $t < 10^{-32}\unit{s}$ & & $T>10^{32}\unit{K}$ \\
					Inflation & $< 10^{-32}\unit{s}$ & & $>10^{22}\unit{K}$ \\
					Electroweak symmetry breaking & $\sim 10^{-12}\unit{s}$ & & $\sim10^{15}\unit{K}$ \\
					Baryogenesis & $\sim 10^{-11}\unit{s}$? & & \\
					Quark Hadron phase transition & $\sim 10^{-5}\unit{s}$ & & $\sim 10^{10}\unit{K}$ \\
					Neutrino decoupling & $\sim 1\unit{s}$ & $\sim 10^{10}$ & $\sim 10^{10}\unit{K}$ \\
					Big Bang Nucleosynthesis & $\sim 10 \unit{s}..10^3\unit{s}$ & & $\sim 10^9\unit{K}..10^7\unit{K}$ \\
					Matter radiation equivalence & $\approx 47\unit{kyr}$ & $\approx 3600$ & $\sim 10^4\unit{K}$ \\
					Recombination / Photon decoupling & $\approx 379\unit{kyr}$ & $\sim 1089$ & $\approx 3000\unit{K}$ \\
					Reionization & $\approx 150\unit{Myr}..1\unit{Gyr}$ & $\approx 20..6$ &  \\
					Dark energy starts to dominate & $\approx 9.8\unit{Gyr}$ & $\approx 0.4$ & $\sim 4\unit{K}$ \\
					Today & $\approx 13.8\unit{Gyr}$ & $0$ & $\approx 2.7\unit{K}$ \\
				\end{tabular}
				\caption{History of the universe.}
			\end{center}
		\end{table}

		\noindent
		Saha Equation\index{Saha Equation} (Equilibrium between ionized and neutral atoms, $n_i$ is the density of atoms in the $i$-th state of ionization, $n_e$ is the electron density, $g_i$ is the degeneracy of the $i$-th state of, $\lambda_\text{th}$ is the thermal de Broglie wavelength\index{De Broglie!Thermische Wellenlänge} Eq.~\ref{Eq:ThermalDeBroglieWavelength}, $\epsilon_i$ is the energy of the $i$-th state of, $T$ is the temperature):
		\begin{equation}
			\frac{n_{i+1} n_e}{n_i} = \frac{2}{\lambda_\text{th}^3} \frac{g_{i+1}}{g_i} \exp(- \frac{\epsilon_{i+1}-\epsilon_{i}}{\kB T})
		\end{equation}


		\noindent
		Important concepts:
		\begin{itemize}
			\item \emph{Comoving frame of reference}: Frame of reference in which the CMB is isotropic, apparent movement only due to Hubble expansion.
			\item \emph{Peculiar velocity}: Velocity relative to the comoving frame of reference.
			\item \emph{Cosmic time}: time measured by a clock with zero peculiar velocity, in the absence of matter anisotropies.
			\item \emph{Non-integrated Sachs--Wolfe effect\index{Sachs!--Wolfe Effekt}\index{Wolfe!Sachs--Wolfe Effekt}}: CMB photons at the surface of last scattering lose energy when climbing out of a potential well created by dark matter aggregates $\delta \hat{T} / \hat{T} = -\delta \hat{\phi} / 3c$, leading to a redshift.
			\item \emph{Integrated Sachs--Wolfe effect}: CMB photons gain or lose energy when passing through time-dependent gravitational potentials, leading to a redshift or blue shift.
			\item \emph{Sunyaev--Zeldovich effect\index{Sunyaev, Rashid!Sunyaev--Zeldovich Effekt}\index{Zeldovich, Yakov!Sunyaev--Zeldovich Effekt}\index{SZ Effekt}}: CMB photons gain energy by inverse Compton scattering\index{Compton!Streuung} when passing through hot plasma, leading to a blue shift.
			\item \emph{Recombination}: Electrons and protons combine to form neutral hydrogen, leading to the decoupling of photons from matter.
			\item \emph{Reionization}: Ionizing radiation from the first stars and galaxies ionize hydrogen and helium.
			\item \emph{Reheating}: Shortly after inflation, the temperature returns to the pre-inflationary temperature, thereby creating standard model particles in the decay of the large potential energy of the inflaton field.
			\item \emph{Baryon Acoustic Oscillations\index{BAO}}: Imprint of oscillations in the baryon-photon fluid (contracting due to gravity and expanding due to radiation pressure) left on the CMB and cosmic structures.
			\item \emph{Silk damping}: Baryon structures which are to small are damped by photon diffusion.
			\item \emph{Cosmological time} is the proper time measured by an observer at rest \wrt the local matter distribution.
		\end{itemize}

		\noindent
		Distance measurements:
		\begin{itemize}
			\item \href{https://en.wikipedia.org/wiki/Comoving_and_proper_distances}{Comoving distance} (constant for objects which are motionless in their local comoving frame; coincides with proper distance at current cosmic time; Note that for $z>0$: $d_L > x = d_c > r = d_p > d_A$ )
				\begin{equation}
					x = \int_{t_e}^{t_r} c\frac{\dd t}{a(t)}
					= \int_{a_e}^{a_r} \frac{c \dd{a}}{a^2 H(a)}
				\end{equation}
			\item Proper distance (physical distance traveled by a photon between two objects)
				\begin{equation}
					\dd r = a \dd{x} = c \dd{t}
					\hsp
					r_e - r_r = \int_{t_e}^{t_r} c \dd{t}
					= \int_{a_e}^{a_r} \frac{c\dd{a}}{a H(a)}
				\end{equation}
			\item Luminosity distance\index{Leuchtkraftentfernung} $d_L$ (apparent distance given a measured flux $F$ and presuming a luminosity $L$ and assuming inverse-square law):
				\begin{equation}
					d_L = \sqrt{\frac{L}{4 \pi F}} = x (1+z) = \frac{x}{a} = (1+z)^2 d_A
				\end{equation}
			\item Angular diameter distance\index{Winkeldurchmesserentfernung} $d_A$ (apparent distance of an object given angular size $\theta$ in comparison with known physical size $l$):
				\begin{equation}
					d_A = \frac{l}{\sin \theta} = \frac{x}{1+z} = x a
				\end{equation}
		\end{itemize}


		\noindent
		Cosmological horizons:
		\begin{itemize}
			\item \emph{Particle Horizon}\index{Beobachtungshorizont}, maximum distance from which light could have traveled to the observer in the age of the universe. It is the maximum distance from which one could possibly retrieve information at a given time and hence determines the size of the observable universe. 
				\begin{equation}
					x_p(t) = \int_{0}^{t} \frac{c \dd{t'}}{a(t')} = c \eta
				\end{equation}
			\item \emph{Event Horizon}\index{Ereignishorizont}, boundary beyond which events cannot affect an observer.
			\item \emph{Cosmic Event Horizon}\index{Kosmischer Ereignishorizont}, boundary beyond which light emitted now can never affect an observer due to the expansion of the universe.
				\begin{equation}
					r_e(t) = a(t) \int^{\infty}_{t} \frac{c \dd{t'}}{a(t')}.
				\end{equation}
			\item \emph{Hubble Horizon / Hubble sphere}\index{Hubble!Horizont}, boundary at which the Hubble flow velocity reaches $c$,
				\begin{equation}
					c = r_H H
				\end{equation}
				Note that information can still reach an observer because the Hubble sphere is expanding.
		\end{itemize}

		\noindent
		Conformal time (makes the metric look nicer):
		\begin{equation}
			\dd \eta \defeq \frac{\dd t}{a(t)}
		\end{equation}

		\subsubsection{Inflation}
			Scalar Inflaton\index{Inflaton} field:
			\begin{equation}
				\lagrangian = -\frac{1}{2}c^2 \partial_\mu \phi \partial^\mu \phi - V(\phi)
				\hsp
				T_{\mu\nu} = \frac{1}{2} c^2 \partial_\mu \phi \partial_\nu \phi + g_{\mu\nu} \lagrangian(\phi)
			\end{equation}
			using isotropy
			\begin{equation}
				\rho c^2 = T_{00} = \frac{1}{2} \dot{\phi}^2 + V(\phi)
				\hsp
				p = \frac{1}{3} T_{ii} = \frac{1}{2} \dot{\phi}^2 - V(\phi)
				\hsp
				w = \frac{p}{\rho c^2} = \frac{\dot{\phi}^2 - 2V(\phi)}{\dot{\phi}^2 + 2V(\phi)}
			\end{equation}
			field equation:
			\begin{equation}
				\ddot{\phi} + 3 H \dot{\phi} + \pdv{V}{\phi} = 0
			\end{equation}
			Slow roll approximation:
			\begin{equation}
				\dot{\phi} \ll V(\phi)
				\hsp
				\ddot{\phi} \ll \dv{V}{\phi} =: V'(\phi)
			\end{equation}
			Slow roll condition and parameters:
			\begin{equation}
				\epsilon \defeq \frac{c^2}{24\pi G} \qty(\frac{V'}{V})^2 \ll 1
				\hsp
				\eta \defeq \frac{c^2}{8\pi G} \frac{V''}{V} \ll 1
			\end{equation}

	\subsection{Structure Formation}
		Density contrast:
		\begin{equation}
			\delta  = \frac{\delta \rho}{\rho_0} = \frac{ \rho - \bar{\rho}}{\rho_0}
		\end{equation}

		\noindent
		Linearized equations of motion for the density contrast (linearized and Newtonian approximations applied and put into comoving coordinates $\vec{x}$; $\vec{u} = \delta\vec{v}/a$):
		\begin{equation}
			\begin{aligned}
				\dot{\delta} + \partial_{\vec{x}} \cdot \vec{u} &= 0
				&
				\dot{\vec{u}} + 2 H \vec{u} &= \frac{\partial_{\vec{x}} \delta p}{a^2 \rho_0} - \frac{\partial_{\vec{x}} \delta \phi}{a^2} \\
				{\partial_{\vec{x}}}^2 \delta \phi &= 4 \pi G \rho_0 a^2 \delta
				&
				\delta p &= c_s^2 \rho_0 \delta
			\end{aligned}
		\end{equation}

		\noindent
		Time evolution of the density contrast (for matter domination $n=4$ for radiation domination $n=32/3$):
		\begin{equation}
			\ddot{\delta} + 2 H \dot{\delta} = n \pi G \rho_0 \delta + \frac{c_s^2 \partial_{\vec{x}}^2 \delta}{a^2}
		\end{equation}

		\noindent
		Jeans length / Jeans instability\index{Jeans!Kriterium} (Critical extend, above which a cloud of gas will gravitationally collapse; assumes $H=0$):
		\begin{equation}
			\lambda_J = c_s \sqrt{\frac{\pi}{G \rho}}
		\end{equation}

		\noindent
		Power spectrum $P$, correlation function $\xi$ and variance $\sigma^2$ of the density perturbations $P(k)$:
		\begin{align}
			\avg{\hat{\delta}(\vec{k}) \hat{\delta}^{*}(\vec{k}')} &=: (2\pi)^3 P(k) \delta (\vec{k}-\vec{k}') \\
			\avg{\delta(\vec{x} + \vec{y}) \delta(\vec{x})} &=: \xi(y) = \int \frac{\dd[3]k}{(2\pi)^3} P(k) \ex^{\i \vec{k} \cdot \vec{y}}\\
			\sigma^2 &\defeq 4\pi\int \frac{k^2 \dd k}{(2 \pi)^3} P(k)
		\end{align}

		\noindent
		Harrison--Zeldovich--Peebles spectrum\index{Harrison!--Zeldovich--Peebles Spektrum}\index{Zeldovich!Harrison--Zeldovich--Peebles Spektrum}\index{Peebles!Harrison--Zeldovich--Peebles Spektrum} (scale-invariant CDM fluctuation spectrum; $a_{\text{eq}}$ denotes equality of radiation and matter density):
		\begin{equation}
			P(k) \sim
			\begin{cases}
				k & k < k_0\\
				k^{-3} & k > k_0
			\end{cases}
			\hsp
			k_0 = 2\pi{\frac{H_{0}}{c}}\,{\sqrt{\frac{2\Omega_{\mathrm{m0}}}{a_{\mathrm{eq}}}}}=2\pi{\frac{H_{0}}{c}}\Omega_{\mathrm{m0}}\,{\sqrt{\frac{2}{\Omega_{\mathrm{r0}}}}}
		\end{equation}

		\noindent
		Press--Schechter mass function\index{Press!--Schechter Massenfunktion}\index{Schechter!Press--Schechter Massenfunktion} (where $\delta_c \approx 1.69$ is the linear density contrast where the non-linear halo collapses, $\sigma_R$ is the variance of the filtered density contrast field $\sigma_R^2$ with some window function $\hat{W}(k)$, $D_+(a) = \delta(r,a)/\delta_0(r)$ is the growth factor of the linear density contrast, during matter domination $D_+(a) \propto a$):
		\begin{equation}
			N(M,a) \dd{M} = \sqrt{\frac{2}{\pi}} \frac{\rho_0 \delta_c}{\sigma_R D_{+}(a)} \dv{\ln\sigma_R}{M} \exp(-\frac{\delta_c^2}{2 \sigma_R^2 D_{+}^2(a)}) \frac{\dd M}{M}
		\end{equation}

		\noindent
		Navarro--Frenk--White profile\index{Navarro!--Frenk--White Profil}\index{Frenk!Navarro--Frenk--White Profil}\index{White!Navarro--Frenk--White Profil}\index{NFW-Profil} (typical density profile of dark matter halos; $x=r/r_s$ where $r_s$ is the characteristic scale radius):
		\begin{align}
			\rho(r) &= \frac{\rho_s}{x(1+x)^2} \\
			M(r) &= 4\pi \rho_s r_s^3 \qty[\ln(1+x) - \frac{x}{1+x}]
		\end{align}

		\noindent
		Variance of the power spectrum (where the window function $\hat{W}_R$ is usually a top-hat function of radius $R$, \eg $R=8h^{-1}\unit{Mpc}$ for $\sigma_8$):
		\begin{equation}
			\sigma_R^2 = \int_{0}^{\infty} \frac{k^2 \dd k}{(2\pi)^3} P(k) \hat{W}^2_R(k)
		\end{equation}

		\noindent
		Filtered density contrast:
		\begin{equation}
			\bar{\delta} \defeq \int \dd[3]{y} \delta(\vec{x}) W_R(\abs{\vec{x}-\vec{y}})
		\end{equation}

	\subsection{Observations}
		\subsubsection{Gravitational Lensing}
			Refractive index (Using the weak field approximation Eq.~\ref{eq:WeakFieldMetric})
			\begin{equation}
				n = \frac{c}{v} = 1 + \frac{2 \phi}{c^2} + \mathcal{O}\qty(\frac{\phi^2}{c^4})
			\end{equation}

			\noindent
			Lensing potential
			\begin{equation}
				\psi (\vec{\theta})  = \frac{2}{c^2} \frac{d_{ds}}{d_d d_s} \int \phi(\vec{\theta}, z) \dd{z}
			\end{equation}

			\noindent
			Reduced deflection angle
			\begin{equation}
				\hat{\vec{\alpha}}
				= \Nabla_{\vec{\theta}} \psi(\vec{\theta})
				= \int \Nabla_\perp n \dd{z}
				= \frac{2}{c^2} \int \Nabla_\perp \phi \dd{z}
			\end{equation}

			\noindent
			Convergence $\kappa$ (quantifies the )
			\begin{equation}
				\kappa = \frac{1}{2} \Nabla_{\vec{\theta}}^2 \psi(\vec{\theta})
			\end{equation}

			\noindent
			Jacobian matrix of the linearized lens equation (decomposed in magnification $(1-\kappa) 1_{2}$ and shear $\gamma =\gamma_1 + \i \gamma_2$ ):
			\begin{equation}
				A \defeq \dv{\vec{\beta}}{\vec{\theta}} = \mqty(
				1 - \kappa - \gamma_1 & -\gamma_2 \\
				-\gamma_2 & 1 - \kappa + \gamma_1)
			\end{equation}

			\noindent
			Magnification:
			\begin{equation}
				\mu  =\frac{1}{\det^2 A} = \frac{1}{(1-\kappa)^2 - \gamma^2} = \frac{\Omega}{\Omega_0}
			\end{equation}

			\noindent
			Change of solid angle (Spherical structure will become an ellipse with semi-axes $a$ and $b$):
			\begin{equation}
				\begin{aligned}
					\Omega_0 &= \pi r^2 \to \Omega = \pi a b = \mu \pi r^2 \\
					a &= \frac{r}{1-\kappa-\gamma} \hsp b = \frac{r}{1-\kappa+\gamma}
				\end{aligned}
			\end{equation}

			\noindent
			Lens equation (where $\vec{\theta}$ is the observed angle, $\vec{\beta}$ is the true angle, $\vec{\alpha}$ is the deflection angle, $\hat{\vec{\alpha}}$ is the reduced deflection angle, $d_{ds}$ is the distance between the deflector or lens and the source, $d_{os}$ is the distance between the observer and the source):
			\begin{equation}
				\vec{\beta}(\vec{\theta})
				= \vec{\theta} - \frac{d_{ds}}{d_{os}} \vec{\alpha}(\vec{\theta})
				= \vec{\theta} - \hat{\vec{\alpha}}(\vec{\theta})
			\end{equation}

			\noindent
			Time delay / Shapiro delay\index{Shapiro!Verzögerung}(integrated along the path from source to observer):
			\begin{equation}
				\Delta t = \int_{s}^{0} \frac{2}{c^3} \abs{\phi} \dd{z}
			\end{equation}