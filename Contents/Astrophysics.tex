% !TEX root = ../physics.tex
\section{Astrophysics\index{Astrophysik}}
	Conventions used in this section:
	\begin{itemize}
		\item Gaußian cgs units ($4\pi\epsilon_0 = 4 \pi / \mu_0 c = 1$)
		\item Metric signature $\eta_{\mu\nu}=\mathrm{diag}(-1,+1,+1,+1)_{\mu\nu}$
		\item Fourier transforms: the factors $(2\pi)^{-d}$ are placed in the back-transform.
	\end{itemize}

	\subsection{Radiation Processes}
		Radiation spectrum of a charge with path $\vec{x}(t)$, seen from the direction $\hat{e}$:
		\begin{equation}
			\frac{\dd^2 E}{\dd \Omega \dd \omega} = \frac{e^2 \omega^2}{8 \pi^2 c} \abs{\int_{\mathbb{R}}\dd{t'} \qty[\hat{e}\times \qty(\hat{e}\times \dot{\vec{x}}(t'))] \ex^{-\i \omega (t'-\hat{e}\cdot \vec{x}(t')/c)}}
		\end{equation}

		\noindent
		Specific Intensity (where $n_{\vec{p},\alpha}$ is the density of photons with momentum $\vec{p}$, polarization $\alpha$ and angle of incidence $\theta$; Note that $I_\omega / \omega^3$ is Lorentz invariant):
		\begin{equation}
			I_\omega = \frac{\hbar \omega^3}{(2 \pi)^3 c^2} \sum_{\alpha=1}^{2} n_{\vec{p},\alpha} = \frac{1}{\cos \theta} \frac{\dd^4 E}{\dd t \dd\omega \dd \Omega \dd A}
		\end{equation}

		\subsubsection{Larmor Formula}
			Relativistic Larmor formula\index{Larmor!Formel},
			Power emitted per solid angle and total emitted power of an accelerated charge (Emitted power as measured in the retarded time, \ie observer's time):
			\begin{equation}
				\label{eq:LarmorFormula}
				\begin{aligned}
					\pdv{P}{\Omega} &= \frac{q^2}{4\pi c\qty(1-\hat{e}\cdot\vec{\beta})^5} \left| \hat{e}\times\qty[(\hat{e}-\vec{\beta}) \times \dot{\vec{\beta}}] \right|^2 \\
					P &= \frac{2q^2}{3c}\gamma^6\qty[\dot{\vec{\beta}}^2 - (\vec{\beta}\times\dot{\vec{\beta}})^2] \\
				\end{aligned}
			\end{equation}

			\noindent
			Classical limit of the Larmor formula (Eq. \ref{eq:LarmorFormula}):
			\begin{equation}
				\begin{aligned}
					\pdv{P}{\Omega} &= \frac{q^2}{4\pi c}\dot{\vec{\beta}}_\perp^2 \\
					P &= \frac{2 q^2}{3 c}\dot{\vec{\beta}}^2\\
				\end{aligned}
			\end{equation}

		\subsubsection{Compton Scattering}
			Compton scattering\index{Compton!Streuung} for a frame with the electron at rest initially (where $\theta$ is the scattering angle, $\cos\theta = \hat{e}\cdot\hat{e}'$ between incident photon $\hat{e}$ scattered photon $\hat{e}'$ and $\lambda_{\mathrm{C}}$ is the Compton wavelength\index{Compton!Wellenlänge}):
			\begin{equation}
				\begin{aligned}
					\omega' &= \frac{\omega}{1+\frac{\hbar\omega}{mc^2}\qty(1-\cos\theta)} \\
					\lambda' - \lambda &= \lambda_\mathrm{C}(1-\cos\theta) = \frac{h}{m c}(1-\cos\theta)
				\end{aligned}
			\end{equation}

		\subsubsection{Thomson Scattering}
			Thomson scattering\index{Thomson, Joseph John!Thomson Streuung}, the elastic limit of Compton scattering\index{Compton!Streuung}, \ie $\hbar\omega \ll mc^2$; and Thomson cross section\index{Thomson, Joseph John!Thomson Wirkungsquerschnitt} (where $r_e$ is the classical electron radius, $\alpha$ is the polarization angle):
			\begin{equation}
				\begin{aligned}
					\dv{\sigma_\text{T}}{\Omega} &= r_e^2 \qty(1-\sin^2 \theta \cos^2 \alpha) \\
					\Avg{\dv{\sigma_\text{T}}{\Omega}}_{\alpha} &= \frac{r_e^2}{2} \qty(1+\cos^2 \theta) \\
					\sigma_\text{T} &= \frac{4\pi}{3}r_e^2 \\
				\end{aligned}
			\end{equation}

		\subsubsection{Synchrotron Radiation}
			\noindent
			Synchrotron Radiation\index{Synchrotron!Strahlung}, ultra relativistic limit of a charge gyrating in a magnetic field (where $u_B=\vec{B}^2/4\pi$ is the energy density of the magnetic field, $K_k$ of the spectrum are the Bessel functions of fractional order, $\xi=\frac{\omega}{3\omega_L}\qty(\frac{1}{\gamma^2}+\theta^2)^{3/2}$ is the dimensionless frequency and $\omega_\text{L}$ denotes the Larmor frequency; Describes a continuous spectrum with a high frequency cutoff at $\omega_c = 3\omega_L \gamma^3$, relativistic beaming focusses light to a cone with $\theta \lesssim 1/\gamma$):
			\begin{equation}
				\begin{aligned}
					P &= c \gamma^2 \sigma_\text{T} u_B \\
					\frac{\dd^2 E}{\dd\Omega\dd\omega} &= \frac{e^2 \omega^2}{3\pi c\omega_\text{L}^2} \qty(\frac{1}{\gamma^2} + \theta^2)^2\qty[\frac{\theta^2}{\gamma^{-2}+\theta^2} K_{1/3}^2(\xi) + K_{2/3}^2(\xi)]\\
				\end{aligned}
			\end{equation}

		\subsubsection{Bremsstrahlung}
			\noindent
			Bremsstrahlung spectrum\index{Bremsstrahlung} and the high \& low frequency approximations (Radiation produced by electrons with asymptotic velocity $v_\infty$ of density $n_e$ passing ions of density $n_i$ and charge $Ze$. $H_{\i \nu}^{(1)}$ is the Hankel function of the first kind and $\tau=a / v_\infty$ is the time scale, where $a$ is the semi-major axis; Describes a continuous spectrum with a thermal, exponential high frequency cutoff)
			\begin{equation}
				\begin{aligned}
					j(\omega) = \frac{\dd[3] E}{\dd \omega \dd t \dd V} &= \i \frac{4\pi^3 Z^2 e^6 n_i n_e}{3m^2 c^3 v_\infty}\qty(\frac{Z e^2 \omega}{m v_\infty^3}) H_{\i \nu}^{(1)}(\i\nu) H_{\i \nu}^{(1)\prime}(\i\nu) \\
					&= \frac{16\pi^3 Z^2 e^6 n_i n_e}{3m^2 c^3 v_\infty}
					\begin{cases}
						\ln\qty(\dfrac{2}{\gamma}\dfrac{m v_{\infty}^3}{Z e^2 \omega}) & \omega \ll \tau^{-1}  \\
						\dfrac{\pi}{\sqrt{3}} & \omega \gg \tau^{-1}  \\
					\end{cases}\\
				\end{aligned}
			\end{equation}

			\noindent
			Emissivity for non-relativistic thermal Bremsstrahlung (Where $\bar{g}_{ff}(\omega)$ is the mean Gaunt factor (which can often be approximated by $1$)):
			\begin{equation}
				j(\omega) = \frac{16\pi^2}{3\sqrt{3}} \frac{Z^2 e^6 n_i n_e}{m^2 c^3} \bar{g}_{\text{ff}}(\omega) \sqrt{\frac{2m}{\pi \kB T}} \exp\qty(-\frac{\hbar \omega}{\kB T})
			\end{equation}

		\subsubsection{Radiation Damping}
			Radiation damping (As a linear theory, electromagnetism cannot account for the loss of energy in a spiraling electron. This is a non-linear effect from QFT that can be corrected for by an additional force term $\vec{F}_\text{rad}$ in the classical equations of motion):
			\begin{equation}
				\vec{F}_\text{rad} = \frac{2}{3} \frac{e^2}{c^2} \ddot{\vec{\beta}}^2
			\end{equation}

			\noindent
			Power transfer between electron and EM-field (Assuming an energy density $u_\text{EM}$):
			\begin{equation}
				P = \frac{4}{3} \beta^2 \gamma^2 c u_\text{EM} \sigma_\text{T}
			\end{equation}
			characteristic path length $\lambda$ and time scale $\tau$ for such an electron:
			\begin{equation}
				\lambda = c\tau = \frac{3 m c^2}{4 u_\text{EM} \sigma_\text{T}}
			\end{equation}

		\subsubsection{Radiative Quantum Transitions}
			Transition rate between two states $\Ket{m}$ and $\Ket{n}$ ($I_\omega = \frac{c}{8 \pi^2 T} \abs*{\hat{\vec{E}}}^2$)
			\begin{equation}
				\Gamma = \frac{8\pi^{2}e^{2}}{m^{2}c}\frac{I_{\omega}T}{\omega^{2}}\abs{\mel**{m}{\mathrm{e}^{\mathrm{i}\vec{k}\cdot\vec{x}}\hat{e}\cdot\vec{\nabla}}{n}}^{2}\;\delta\left(\omega_{m n}-\omega\right)
			\end{equation}

			\noindent
			Dipole Approximation, for visible light $\vec{k}\cdot\vec{x}\sim 0$ (for unpolarized light; dipole matrix element $\vec{d}_{mn} = \mel{m}{\hat{d}}{n}$, transition frequency $\hbar \omega_{mn} = E_m - E_n$):
			\begin{equation}
				\Gamma = \frac{8\pi^{2}I_\omega T}{3 c \hbar^2}\abs{\vec{d}_{mn}}^{2}\;\delta\left(\omega_{m n}-\omega\right)
			\end{equation}

			\noindent
			Oscillator strength
			\begin{equation}
				f_{mn} = \frac{2 m \omega_{mn}}{3 e^2 \hbar} \omega_{mn} \abs{\vec{d}_{mn}}^2
			\end{equation}

			\noindent
			Cross section ($f_{mn}$ is the oscillator strength, $\phi$ is the line profile function):
			\begin{equation}
				\sigma_{mn} = 4 \pi^2 r_e c f_{mn} \phi(\omega_{mn} - \omega)
			\end{equation}

		\subsubsection{Spectral Lines}
			Natural Line-width (Heisenberg uncertainty principle, $\tau$ is the lifetime of the excited state):
			\begin{equation}
				\phi_\Gamma(\omega-\omega_{mn}) = \frac{1}{\pi} \frac{\Gamma / 2}{(\omega-\omega_{mn})^2 + \Gamma^2}
			\end{equation}

			\noindent
			Doppler broadening (Assuming thermal motion of the emitting particles, $\sigma = \sqrt{\kB T / m}$ is the standard deviation of the velocity distribution, $\bar{\omega} = \omega_0(1+v/c)$ is the centre frequency):
			\begin{equation}
				\phi(\omega-\omega_{mn}) = \frac{c}{\omega_0 \sqrt{ 2\pi} \sigma} \exp(-\frac{c^2}{2\sigma^2}\qty(\frac{\omega-\bar{\omega}}{\omega_0^2})^2)
			\end{equation}

			\noindent
			For the Voigt profile\index{Voigt!Profil}, $\phi$ will be described by a convolution of the natural line-width (Lorentzian) broadened by collisions (Lorentzian, $\Gamma \to \Gamma + \Gamma_c$) and Doppler broadening (Gaussian)

			Equivalent width ($I_0$ is the continuum intensity, $I(\omega)$ is the line intensity, $W$ quantifies the strength of the line):
			\begin{equation}
				W := \int\dd{\omega} \frac{I_0 - I(\omega)}{I_0}
			\end{equation}
			the curve of growth $W(N)$ is the equivalent width as a function of the column density $N$ of a medium.

	\subsection{Hydrodynamics}
		\subsubsection{Ideal Hydrodynamics}
			Ideal Hydrodynamics: No viscosity, mean free path $\lambda$ of particles is infinitesimally small in comparison to the characteristic length $l$, which is much smaller than the system size $L$, \ie $\lambda \ll l \ll L$.

			\noindent
			Particle current density $J^\mu$ and energy-momentum tensor $T^{\mu\nu}$:
			\begin{align}
				J^{\mu} &= c \int \frac{\dd[3] p}{E}\, f\qty(t, \vec{x}, \vec{p}) p^{\mu} = \frac{n(t, \vec{x})}{c} \mqty(c\\\avg{\dot{\vec{x}}})^\mu
				&
				\partial_\mu J^{\mu} &= 0 \\
				T^{\mu\nu} &= c^2 \int \frac{\dd[3] p}{E}\, f\qty(t, \dot{\vec{x}}, \vec{p}) p^{\mu} p^{\nu} =\rho(t, \dot{\vec{x}})\mqty(\avg{\gamma} & \avg{\gamma \dot{\vec{x}}} \\ \avg{\gamma \dot{\vec{x}}} & \avg{\gamma\dot{\vec{x}}\dot{\vec{x}}})^{\mu\nu}
				&
				\partial_\mu T^{\mu\nu} &= 0
			\end{align}

			\noindent
			Equations of motion for an ideal non-relativistic Hydrodynamics, \ie mass conservation, Euler Equation\index{Euler!Gleichung} (momentum conservation) and energy conservation (Energy density $\varepsilon=\frac{f}{2} n \kB T$, mean velocity $\vec{v}=\avg{\dot{\vec{x}}}$ and pressure $P = \frac{1}{3} T^{i}_i$):
			\begin{align}
				0 &= \partial_t \rho + \partial_i \qty(\rho v^i) \\
				0 &= \partial_t v^i + \qty(v^j \partial_j) v^i + \frac{1}{\rho} \partial^i P \\
				0 &= \partial_t \varepsilon + \partial_i \qty(\varepsilon v^i) + P \partial_i v^i
			\end{align}
			\noindent
			Sound waves (linear perturbations $\rho = \rho_0 + \delta \rho$, $\vec{v} = \delta \vec{v}$, $P = P_0 + \delta P$, with $\delta P \propto \delta \rho$ and $\delta \rho \ll \rho_0$, $\delta P \ll P_0$, using the Euler equations these can be shown to be longitudinal):
			\begin{equation}
				\frac{1}{c_s^2} \partial_t^2 \rho - \Nabla^2 \rho = 0 \hsp c_s = \sqrt{\frac{\partial P}{\partial \rho}}
			\end{equation}

			\noindent
			Kelvin's Circulation Theorem\index{Kelvin!Theorem} (valid for non-viscid fluids with polytropic equation of state; $\vec{\Omega} = \Nabla \times \vec{v}$ is the vorticity and $\Gamma = \int_{\partial \mathcal{A}} \vec{v} \cdot \dd \vec{l}$ is the circulation):
			\begin{equation}
				\partial_t \vec{\Omega} = \Nabla \times \qty(\vec{v} \times \vec{\Omega}) + \frac{1}{\rho^2} \Nabla \rho \times \Nabla P
				\iff \dv{t}\Gamma(\partial \mathcal{A}) = 0
			\end{equation}

			\noindent
			Bernoulli's constant (Valid for ideal fluids in stationary, adiabatic flow; $\phi$ is the gravitational potential, $\dd \tilde{h} = \dd P / \rho$ is the specific enthalpy):
			\begin{equation}
				\vec{v}\cdot \Nabla \qty(\frac{1}{2}\vec{v}^2 + \tilde{h} + \phi) = 0
			\end{equation}

			\noindent
			Tensor Virial Theorem\index{Virialsatz!tensorieller} ($I = 	\int \dd[3]{x} \rho \vec{x} \otimes \vec{x}$ is the inertial tensor and $\bar{T}$ is the stress energy tensor)
			\begin{equation}
				\frac{1}{2}\dv[2]{I}{t}
				= \int \dd[3]{x} \bar{T}
				= \int \dd[3]{x} \qty(\rho \vec{v}\otimes\vec{v} + 1_3 P + \bar{T}_{\text{grav.}})
			\end{equation}

			\noindent
			Polytropic equation (polytropic index $\gamma$):
			\begin{equation}
				\label{eq:PolytropicEquation}
				P V^\gamma = \const
			\end{equation}


		\subsubsection{Relativistic Hydrodynamics}
			Relativistic Hydrodynamics without Gravitational field (Energy density $\rho$, pressure $P$, $u^\mu \nabla_\mu =: \nabla_u$):
			\begin{equation}
				\begin{aligned}
					0 &= \nabla_u \qty(\rho c^2) + \qty(\rho c^2 + P) \nabla_\mu u^\mu \\
					0 &= \qty(\rho c^2 + P) \nabla_u u_\mu + \nabla_\mu P + u_\mu \nabla_u P \\
				\end{aligned}
			\end{equation}

		\subsubsection{Viscous Hydrodynamics}
			$\lambda$ becomes finite, which introduces diffusion.

			Fick's second law\index{Ficksches Gesetz} (where $D = \bar{u} \lambda/\sqrt{3}$ is the diffusion constant, $\bar{u}$ is the characteristic velocity of the particles):
			\begin{equation}
				\partial_t n = \Nabla \cdot (D \Nabla n)
			\end{equation}

			\noindent
			Diffusive part of the stress-energy tensor (where $\eta$ is the dynamic viscosity, $\zeta$ is the bulk viscosity):
			\begin{equation}
				\bar{T}_{\mathrm{d}}=-\eta\left[\left(\vec{\nabla}\otimes\vec{v}\right)+\left(\vec{\nabla}\otimes\vec{v}\right)^{T}-\frac{2}{3}\vec{\nabla}\cdot\vec{v}\,1_{3}\right]-\zeta\vec{\nabla}\cdot\vec{v}\,1_{3}
			\end{equation}

			\noindent
			Energy-conservation for a viscous fluid:
			\begin{equation}
				\partial_t \varepsilon + \Nabla \cdot (\varepsilon \vec{v}) + P \Nabla \cdot \vec{v} = \Nabla \cdot (\kappa \Nabla T) + \tr\qty(\bar{T}_d^T D \qty(\Nabla \otimes\vec{v}))
			\end{equation}

			\noindent
			Navier--Stokes equation\index{Navier--Stokes Gleichung} (where $\eta$ is the dynamic viscosity, $\zeta$ is the bulk viscosity):
			\begin{equation}
				\rho\left(\partial_{t}+\vec{v}\cdot\vec{\nabla}\right)\vec{v}+\vec{\nabla}P=\eta\vec{\nabla}^{2}\vec{v}+\left(\zeta+\frac{\eta}{3}\right)\vec{\nabla}\left(\vec{\nabla}\cdot\vec{v}\right)
			\end{equation}

			\noindent
			Reynolds Number\index{Reynolds!Zahl} (Measure of turbulence; $v$ is the characteristic velocity, $L$ is the characteristic length scale and $\nu = \eta / \rho$ is the kinematic viscosity; An ideal fluid has $\mathcal{R} \to \infty$):
			\begin{equation}
				\mathcal{R} = \frac{v L}{\nu}
			\end{equation}

			\noindent
			Lane--Emden Equation\index{Lane!--Emden Gleichung}\index{Emden!Lane--Emden Gleichung} (viscous fluid with polytropic index $n$ in hydrostatic equilibrium and spherical symmetry, \ie $\Nabla P = - \rho \Nabla \phi$, where $\theta^n = \rho/\rho_0$ and $\xi = r/r_0$ with $r_0 = \sqrt{n c_{s,0}^2 / 4\pi G \rho_0}$):
			\begin{equation}
				\frac{1}{\xi^2} \partial_\xi \qty(\xi^2 \partial_\xi \theta) = - \theta^n
			\end{equation}

		\subsubsection{Shocks}
			Discontinuity in matter, momentum and energy conservation ($\tilde{h}$ is the specific enthalpy):
			\begin{align}
				\rho_1 \vec{v_1} &= \rho_2 \vec{v_2} \\
				\qty(\frac{1}{2}\vec{v}_1^2 + \tilde{h}_1) \rho_1 \vec{v_1} &= \qty(\frac{1}{2}\vec{v}_2^2 + \tilde{h}_2) \rho_2 \vec{v_2} \\
				\rho_1 \vec{v}_1^2 + P_1^2 &= \rho_2 \vec{v}_2^2 + P_2^2
			\end{align}

			\noindent
			Mach number\index{Mach!Zahl}:
			\begin{equation}
				\mathcal{M}_i = \frac{v_i}{c_{s,i}}
			\end{equation}

			\noindent
			Rankine--Hugoniot shock jump conditions\index{Rankine!--Hugoniot Bedingungen} ($r := \rho_2/\rho_1$ and $q := P_2/P_1$, assuming a polytropic fluid Eq.~\ref{eq:PolytropicEquation}):
			\begin{equation}
				r = \frac{M_1^2(\gamma+1)}{M_1^2(\gamma-1)+2}
				\hsp
				q = \frac{2\gamma M_1^2 - \gamma + 1}{\gamma + 1}
			\end{equation}

			\noindent
			Shock velocity
			\begin{equation}
				v_{\text{shock}} = c_{s,1} \sqrt{\frac{2 r}{(\gamma+1) - (\gamma-1)r}} = c_{s,1}\sqrt{\frac{(\gamma+1)q + (\gamma-1)}{2\gamma}}
			\end{equation}

			\noindent
			Sedov Solution\index{Sedov!Lösung} (for a point explosion releasing an energy $E$ in a homogeneous medium of $\rho_0$ density)
			\begin{equation}
				R(t) = \qty(\frac{3 E}{4\pi\rho_0}t^2)^{1/5}
			\end{equation}

	\subsection{Plasma Physics}
		\subsubsection{Basics}
			Shielding of a charge pertubation $\rho=q\delta(\vec{x})$ by plasma charges (Yukawa potential\index{Yukawa!Potential}):
			\begin{equation}
				\phi(r) = \frac{q}{r} \exp(-\sqrt{Z+1}k_\text{D} r)
			\end{equation}

			\noindent
			Debye Wave number\index{Debye!Wellenzahl} and Debye length\index{Debye!Länge} (characteristic shielding length; note the lack of a factor of $2\pi$):
			\begin{equation}
				k_\text{D}^2 := 4\pi \frac{\bar{n}_e e^2}{\kB T} \hsp
				\lambda_\text{D} := \frac{1}{k_\text{D}} = \sqrt{\frac{\kB T}{4\pi \bar{n}_e e^2}}
			\end{equation}

			\noindent
			Plasma frequency $\omega_\text{p}$ and plasma reaction time $t_\text{D}$ (assuming equipartition $\avg{v_i^2} = \kB T / m_e$; characteristic time scale of particle rearrangement, incident waves with frequencies greater than $\omega_\text{p}$ can not be compensated, the plasma becomes transparent; notice the lack of a factor of $2\pi$):
			\begin{equation}
				\omega_\text{p} = \frac{1}{t_\text{D}} = \sqrt{\frac{4\pi\bar{n}_e e^2}{m_e}} \hsp
				t_\text{D} = \frac{\lambda_\text{D}}{\sqrt{\avg{v^2}}} = \sqrt{\frac{m_e}{4\pi \bar{n}_e e^2}}
			\end{equation}

			\noindent
			Displacement current:
			\begin{equation}
				\vec{D}(\omega,\vec{k})  = \hat{\varepsilon}(\omega, \vec{k}) \, \vec{E}(\omega, \vec{k})
			\end{equation}

		\subsubsection{Electromagnetic Waves in Plasma}
			Projection operators:
			\begin{equation}
				\pi_\parallel = \frac{1}{k^2}\vec{k}\otimes\vec{k}
				\hsp
				\pi_\perp = 1_3 - \pi_\parallel
			\end{equation}

			\noindent
			Dispersion relation:
			\begin{equation}
				\det(\qty(1 - \frac{\omega^2}{c^2 k^2}\hat{\varepsilon}_\perp) \pi_\perp - \frac{\omega^2}{c^2 k^2}\hat{\varepsilon}_\parallel \pi_\parallel) = 0
			\end{equation}

			\noindent
			Longitudinal and transversal dielectricities:
			\begin{equation}
				\begin{aligned}
					\hat{\varepsilon}_{\parallel} &= 1-\frac{4\pi e^{2}}{k^{2}}\int{\mathrm d^{3}}p\,\frac{1}{\vec{k}\cdot\vec{v}-\omega}\,\frac{\partial f_{0}}{\partial\vec{p}}\cdot\vec{k}\\
					\hat{\varepsilon}_{\perp} &= \frac{1}{2}\,\tr\,\pi_{\perp}\hat{\varepsilon}=1-\frac{2\pi e^{2}}{\omega}\int\mathrm{d}^{3}p\,\frac{1}{\vec{k}\cdot\vec{v}-\omega}\,\frac{\partial f_{0}}{\partial\vec{p}_{\perp}}\cdot\vec{v}_{\perp}
				\end{aligned}
			\end{equation}

			\noindent
			Landau damping\index{Landau!Dämpfung} (where $\abs{\hat{\vec{E}}^2}$ is the amplitude of the electric field oscillations and $\bar{f}$ is the equilibrium distribution function):
			\begin{equation}
				\avg{Q}
				= \frac{\omega}{8 \pi} \Im \hat{\varepsilon}_\parallel \abs{\hat{\vec{E}}^2}
				= - \abs{\hat{\vec{E}}^2} \frac{\pi}{2} \frac{m e^2 \omega}{k^2} \eval{\dv{\bar{f}}{p_x}}_{p_x = \omega m /k}
			\end{equation}

	\subsection{Magnetohydrodynamics}
		\subsubsection{Basics}
			Assuming non-relativistic velocities and a macroscopically neutral plasma, \ie locally $c\rho' \ll \abs{\vec{j}}$ and $\dot{\vec{E}}' \ll \vec{j}'$.
			Ions and electrons then have a relative drift velocity
			\begin{equation}
				\vec{v}_{\text{drift}} = \vec{v}_e - \vec{v}_i
			\end{equation}

			\noindent
			Magnetic Reynolds Number\index{Magnetic!Reynolds Number} (ratio of advection to diffusion of magnetic fields, $L$ is the characteristic length scale of the system, $v$ is the characteristic velocity and $\sigma$ is the conductivity):
			\begin{equation}
				\mathcal{R}_\text{M} = \frac{4\pi\sigma v L}{c^2}
			\end{equation}

			\noindent
			Magnetic part of the stress-energy tensor:
			\begin{equation}
				\bar{T}_m = -\frac{1}{4\pi} \qty(\vec{B} \otimes \vec{B} - \frac{1}{2}\vec{B}^2 1_3)
			\end{equation}

		\subsubsection{Equations of Motion}
			Induction equation\index{Induktionsgleichung} (where $\sigma$ is the conductivity and $\vec{v}$ is the velocity field):
			\begin{equation}
				\pdv{\vec{B}}{t} = \frac{c^2}{4\pi\sigma} \Nabla^2 \vec{B} + \Nabla\times\qty(\vec{v}\times\vec{B})
			\end{equation}

			\noindent
			Euler's equation\index{Euler!Gleichung} for an ideal fluid:
			\begin{equation}
				\rho \dv{\vec{v}}{t} = - \Nabla P + \frac{1}{4\pi}\qty(\Nabla\times\vec{B})\times\vec{B}
				= - \Nabla P + \frac{1}{4\pi} \qty(\vec{B} \cdot \Nabla) \vec{B} - \frac{1}{8\pi} \Nabla \vec{B}^2
			\end{equation}

		\subsubsection{Electromagnetic Waves in Cold Magnetized Plasma}
			Dielectricity tensor (where $w_p = \omega_p / \omega$ with the plasma frequency $\omega_p = \sqrt{4\pi e^2 n_e/m}$ and $w_L = \omega_L / \omega$ with the Larmor frequency $\omega_L = e B_0/mc$, $\pi_\parallel = \vec{b}\otimes\vec{b}$ projects parallel to the magnetic field, $\pi_\perp = 1_3 - \pi_\parallel$ orthogonal to it; $\hat{\mathcal{B}}_{ij} = \varepsilon_{ijk} b_k$ projects onto the antisymmetric contribution):
			\begin{equation}
				\varepsilon = \qty(1 - w_p^2) \pi_\parallel + \qty(1-\frac{w_p^2}{1-w_L^2})\pi_\perp + \i \frac{w_p^2 w_L}{1-w_L^2} \hat{\mathcal{B}}
			\end{equation}

			\noindent
			Faraday rotation\index{Faraday!Rotation} (Rotation of the polarization vector of light introduced by a slight difference in phase velocity for light of different helicity; $\psi$ is the rotation angle, and $\int \dd{z} n_e B_\parallel$ is called the rotation measure, where $B_\parallel$ is the magnetic field component parallel to the line of sight):
			\begin{equation}
				\psi = \frac{2\pi e^3}{m^2 c^2 \omega^2} \int \dd{z} n_e B_\parallel
			\end{equation}

		\subsubsection{Hydromagnetic Waves}
			Alfvén velocity\index{Alfvén!Geschwindigkeit} (characteristic velocity of MHD waves):
			\begin{equation}
				c_A^2 = \frac{B_0^2}{4\pi \rho_0}
			\end{equation}

			\noindent
			Dispersion relation (phase velocity $c_k = \omega/k$, $\psi$ is the angle between $\vec{k}$ and $\vec{B}$)
			\begin{equation}
				\mqty(
				c_k^2 - c_s^2 - c_A^2 \sin^2 \psi & c_A^2 \sin\psi \cos\psi & 0 \\
				c_A^2 \sin\psi \cos\psi & c_k^2 - c_A^2 \cos^2 \psi & 0 \\
				0 & 0 & c_k^2 - c_A^2 \cos^2 \psi
				) \delta \vec{v} = 0
			\end{equation}

			\noindent
			Alfvén waves\index{Alfvén!Welle} dispersion relation and group velocity (where $\vec{B} = B \hat{b}$):
			\begin{equation}
				\omega = c_A k \cos \psi
				\hsp
				\pdv{\omega}\vec{k} = c_A \hat{b}
			\end{equation}

	\subsection{Stellar Dynamics}
		\subsubsection{Basics}
			Relaxation timescale of a collection of stars\index{Relaxation!Zeit} (timescale on which the system's movement is dominated by gravity as opposed to initial velocities, where $N$ is the number of stars, $m$ is the mass of a star):
			\begin{equation}
				t_{\text{relax.}} = \frac{N}{8\ln N} \frac{R}{v} = \frac{N}{8\ln N} t_{\text{cross.}}
			\end{equation}

			Eddington limit\index{Eddington!Limit} (maximum luminosity of a star before radiation pressure overcomes gravity; $\sigma_T$ is the Thomson cross section\index{Thomson, Joseph John!Thomson Streuung}\index{Thomson, Joseph John!Thomson Wirkungsquerschnitt}, $m_p$ is the proton mass and $M$ is the mass of the star):
			\begin{equation}
				L = \frac{4 \pi G M m_p c}{\sigma_T}
			\end{equation}


		\subsubsection{Jeans Theorem}
			Jeans Equations\index{Jean!Gleichungen} (Evolution of star systems; $\sigma^2 = \qty(\vec{v}-\avg{\vec{v}})\otimes\qty(\vec{v}-\avg{\vec{v}})$ is the velocity dispersion tensor and $\phi$ is the gravitational potential):
			\begin{equation}
				\pdv{n}{t} + \Nabla\cdot\qty(n\vec{v}) = 0 \hsp
				\pdv{\vec{v}}{t} + \qty(\vec{v}\cdot\Nabla)\vec{v} = -\Nabla\phi -\frac{1}{n}\Nabla\cdot\qty(n\sigma^2) \hsp
			\end{equation}

			\noindent
			Radial Jeans Equation\index{Jean!Gleichungen!radial} (where $\beta$ is the anisotropy parameter in $\sigma_\theta^2 = \sigma_\phi^2 = (1-\beta) \sigma_r^2$)
			\begin{equation}
				\pdv{r} \qty(n\sigma_r^2) + \frac{2\beta}{r} n\sigma_r^2 = -n\pdv{\phi}{r}
			\end{equation}

