% !TEX root = ../physics.tex
\section{Classical Mechanics\index{Klassische Mechanik}}
	\subsection{Newton's Laws of Motion\index{Newton!Kraftgesetze}}
		\textbf{First Axiom} \newline
		\indent \textit{Every body continues in its state of rest, or of uniform motion in a straight line, unless it is compelled to change that state by forces impressed upon it.} \nl
		\textbf{Second Axiom} \newline
		\indent \textit{The change of motion of an object is proportional to the force impressed and is made in the direction of the straight line in which the force is impressed.}
		\begin{equation}
			\dv{\vec{p}}{t} = \vec{F}
		\end{equation}\nl
		\textbf{Third Axiom} \newline
		\indent \textit{To every action there is always opposed an equal reaction.} \nl
		\textbf{Additional}: Principle of Superposition \newline
		\indent \textit{Forces are additive.}
		\begin{equation}
			\vec{F}_{\text{tot.}} = \sum_{i=1}^{N} \vec{F}_i
		\end{equation}

	\subsection{Basics}
		\noindent
		Virial Theorem (for the average kinetic Energy $\langle T \rangle$ and the virial $\sum_j \vec{x}_j\cdot\dot{\vec{p}}_j$):
		\begin{equation}
			\langle T \rangle = - \frac{1}{2} \sum_j \left\langle \vec{x}_j\cdot\dot{\vec{p}}_j \right\rangle
		\end{equation}
		for a potential of the form $V(\vec{x}_1,\vec{x}_2) \propto |\vec{x}_1-\vec{x}_2|^n$ this implies:
		\begin{equation}
			2\Avg{T} = n\Avg{V}
		\end{equation}

		\noindent
		Lyapunov exponent\index{Ljapunov!Exponent} $\lambda$ (for a dynamical system $\vec{Z}(t)$ with two trajectories initially separated by $\delta\vec{Z}(0)$ in phase space, Lyapunov time\index{Ljapunov!Zeit} $\tau=\frac{1}{\lambda}$):
		\begin{equation}
			\begin{aligned}
				\norm{\delta\vec{Z}(t)} &\sim \ex^{\lambda t} \norm{\delta\vec{Z}(0)}\\
				\lambda &= \lim_{t\to\infty} \lim_{\norm{\delta\vec{Z}(0)}\to 0} \frac{1}{t} \ln\qty(\frac{\norm{\delta \vec{Z}(t)}}{\norm{\delta\vec{Z}(0)}}) \\
			\end{aligned}
		\end{equation}

	\subsection{Rotation}
		\noindent
		Definition of the Inertia Tensor\index{Trägheitstensor}:
		\begin{equation}
			\Theta_{ij}=\int_{\mathcal{V}} \rho \left[r^2\delta_{ij}-r_i r_j\right] \;\dd^3\vec{r}
		\end{equation}

		\noindent
		Parallel Axis Theorem\index{Steiner!Satz} ($\Theta_{ij}$ is the inertia tensor for rotation around the center of mass.
		For rotation around a point displaced by $\vec{a}$, the inertia tensor transforms to $\Theta'_{ij}$, where $M$ is the total mass.):
		\begin{equation}
			\Theta'_{ij} = \Theta_{ij} + M\qty(\vec{a}^2\delta_{ij} - a_i a_j)
		\end{equation}

		\noindent
		Angular momentum and energy in relation to the inertia tensor:
		\begin{equation}
			\begin{aligned}
				L^i &= \Theta^i_j \omega^j \\
				T_{\text{rot.}} &= \frac{1}{2}\vec{\omega}^T \Theta \vec{\omega}
			\end{aligned}
		\end{equation}

	\subsection{Moving Coordinate Systems}
		\noindent
		Accelerated system $K'$ and inertial system $K$
		\begin{equation}
			\begin{aligned}
				\vec{x}'(t) &= \vec{x}(t)-\vec{x}_0(t) &\hsp
				m\dv{\vec{p}}{t} &= \vec{F} \\
				\dv{\vec{x}'}{t} &= \dv{\vec{x}}{t}-\vec{\omega}\times\vec{x} &\hsp
				m\dv{\vec{p}'}{t} &= \vec{F} + \vec{F}_\text{T} + \vec{F}_\text{Z} + \vec{F}_\text{L} + \vec{F}_\text{C} \\
			\end{aligned}
		\end{equation}

		\noindent
		Fictitious forces\index{Scheinkraft}: displacement force\index{Scheinkraft!Translationskraft} $\vec{F}_\text{T}$, centrifugal force\index{Scheinkraft!Zentrifugalkraft} $\vec{F}_\text{Z}$, Euler force\index{Scheinkraft!Linearkraft} $\vec{F}_\text{L}$, Coriolis force\index{Scheinkraft!Corioliskraft} $\vec{F}_\text{C}$
		\begin{equation}
			\begin{aligned}
				\vec{F}_\text{T} &= -m\dv[2]{\vec{x}_0}{t} &\hsp
				\vec{F}_\text{Z} &= -m\vec{\omega}\times\qty(\vec{\omega}\times\vec{x}') \\
				\vec{F}_\text{L} &= -m\dv{\vec{\omega}}{t}\times\vec{x}' &\hsp
				\vec{F}_\text{C} &= -2m\vec{\omega}\times\dv{\vec{x}'}{t} \\
			\end{aligned}
		\end{equation}


	\subsection{Newtonian Gravitation}
		\noindent
		Newton's law of universal gravitation\index{Newton!Gravitationskraft} (Acting on mass 1):
		\begin{equation}
			\vec{F}_1 = - G m_1 m_2 \frac{\vec{r}_1-\vec{r}_2}{\abs{\vec{r}_1-\vec{r}_2}^3}
		\end{equation}

		\noindent
		Equations of motion for Newton's law of universal gravitation:
		\begin{equation}
			\vec{F}=-m\Nabla\phi
		\end{equation}

		\noindent
		Poisson Equation for gravity\index{Poisson!Gleichung-Gravitation} / field equations for the Newtonian gravitational potential\index{Newton!Gravitationspotential} $\phi$ :
		\begin{equation}
			\Nabla^2\phi = 4\pi G\rho
		\end{equation}


	\subsection{Kepler Problem\index{Kepler!Problem}}
		\noindent
		Lagrangian (Gravitational parameter\index{Gravitationsparameter} $\mu$, for central body approximation $\mu=GM$, see \ref{Sec:AstronomicalConstants}):
		\begin{equation}
			\mathcal{L}(\vec{r},\dot{\vec{r}}) = \frac{m}{2} \dot{\vec{r}}^2 + \frac{\mu}{r}
		\end{equation}

		\noindent
		Kepler's second law\index{Flächensatz} (For a particle with a central force\index{Zentralpotential}):
		\begin{equation}
			\dv{A}{t} = \frac{\abs*{\vec{L}}}{2 m} = \const
		\end{equation}

		\noindent
		Gravitational parameter \index{Gravitationsparameter} $\mu$, eccentricity\index{Exzentrizität} $\epsilon$ and $\rho_0$:
		\begin{equation}
			\begin{aligned}
				\mu &= G m &\hsp
				\epsilon &= \sqrt{1+\frac{2 E \vec{L}^2}{m \mu^2}} &\hsp
				\rho_0 &= \frac{\vec{L}^2}{m\mu} \\
			\end{aligned}
		\end{equation}

		\noindent
		Conserved Laplace--Runge--Lenz vector\index{Laplace!--Runge--Lenz Vektor}\index{Runge!Laplace--Runge--Lenz Vektor}\index{Lenz!Laplace--Runge--Lenz Vektor}:
		\begin{equation}
			\vec{A} = \dot{\vec{r}}\times\vec{L} - \frac{\mu \vec{r}}{r}
		\end{equation}

		\noindent
		Orbit parametrization (Cylindrical coordinates $\rho, \varphi$):
		\begin{equation}
			\rho = \frac{\rho_0}{1+\epsilon \cos\varphi}
		\end{equation}

		\noindent
		Vis-Viva\index{Vis-Viva Gleichung} Equation
		\begin{equation}
			v^2 = \mu\qty(\frac{2}{r} - \frac{1}{a})
		\end{equation}

		\noindent
		Rocket Equation\index{Raketengleichung} (final mass $m_\text{f}$, initial mass $m_\text{i}$, effective exhaust speed $u$, specific Impulse $I_{\text{sp.}}$, standard earth acceleration $g_0$):
		\begin{equation}
			\frac{m_\text{f}}{m_\text{i}} = \exp(-\frac{\Delta v}{u}) = \exp(-\frac{\Delta v}{g_0 I_{\text{sp.}}})
		\end{equation}

		\noindent
		Kepler's third law\index{Kepler!Drittes Gesetz}:
		\begin{equation}
			\frac{T^2}{a^3} = \frac{4\pi^2}{G(M+m)}
		\end{equation}

	\subsection{Lagrange-Formalism}
		\noindent
		Action\index{Wirkung}:
		\begin{equation}
			\mathcal{S}=\int_{t_0}^{t_1}\mathcal{L}(q_i, \dot{q_i},t)\;\mathrm{d} t
		\end{equation}

		\noindent
		Stationary-action principle / Principle of least action \index{Wirkung}:
		\begin{equation}
			\delta \mathcal{S}=0
		\end{equation}

		\noindent
		Euler--Lagrange equation\index{Euler!--Lagrange Gleichung}\index{Lagrange!Euler--Lagrange Gleichung}:
		\begin{equation}
			\frac{d}{dt} \frac{\partial \mathcal{L}(q_{i},\dot{q_{i}},t)}{\partial \dot{q_{i}}} - \frac{\partial \mathcal{L}(q_{i},\dot{q_{i}},t)}{\partial q_{i}} = 0 \quad \forall q_i
		\end{equation}

		\noindent
		Canonical momentum / conjugated momentum \index{Kanonisch-konjugierte Impulse}:
		\begin{equation}
			p_i=\frac{\partial \mathcal{L}}{\partial\dot{q_i}}
		\end{equation}

	\subsection{Hamiltonian Formalism}
		\noindent
		Hamiltonian\index{Hamilton!Funktion} (\ie the Legendre Transformation\index{Legendre!Transformation} of the Lagrangian):
		\begin{equation}
			H(q_i,p_i,t)=\sum_{j=1}^{f}p_j\dot{q_j}(p) - \mathcal{L}(q_i, \dot{q_i}(p),t)
		\end{equation}

		\noindent
		Hamilton's equations\index{Hamilton!Bewegungsgleichungen}:
		\begin{equation}
			\begin{aligned}
				\dot{q}_k &= \pdv{H}{p_k}, &&\hsp
				\dot{p}_k &= -\pdv{H}{q_k}, &&\hsp
				\pdv{H}{t} &= -\pdv{\mathcal{L}}{t}
			\end{aligned}
		\end{equation}

		\noindent
		Poisson-brackets (in Quantum Mechanics $\qty{A,B} \to -\frac{\i}{\hbar}\Avg{\comm{A}{B}}$):
		\begin{equation}
			\begin{aligned}
				\qty{A,B} &= \sum_k \left(
				\pdv{A}{q_k}\pdv{B}{p_k} - \pdv{A}{p_k}\pdv{B}{q_k}
				\right) \\
				\dv{A}{t} &= \qty{A, H} + \pdv{A}{t}
			\end{aligned}
		\end{equation}

		\noindent
		Canonical transformations\index{Kanonische Transformation} (transformations that leave the Hamilton equations invariant):
		\begin{equation}
			\begin{aligned}
				q_i &= q_i(q_i',p_i',t) &\hsp
				p_i &= p_i(q_i',p_i',t) \\
				H(q_i,p_i,t) &= H'(q_i',p_i',t) &\hsp
				\mathcal{L}(q_i,\dot{q}_i,t) &= \mathcal{L}'(q_i',\dot{q}_i',t) \\
			\end{aligned}
		\end{equation}

		Probability density function / phase space distribution function:
		\begin{equation}
			f(t,\vec{x},\vec{p}) = \frac{\dd N}{\dd^3 x \dd^3 p}
			\hsp
			n(t,\vec{x}) = \int \dd^3 p\, f(t,\vec{x},\vec{p})
			\hsp
			N(t) = \int \dd^3 x \dd^3 p\, f(t,\vec{x},\vec{p})
		\end{equation}

		\noindent
		Boltzmann Equation\index{Boltzmann!Gleichung} (where $f(\vec{q},\vec{p},t)$ is the phase space distribution function, $\mathcal{C}[f]$ is the collision term):
		\begin{equation}
			\pdv{f}{t} + \pdv{H}{\vec{p}}\cdot\pdv{f}{\vec{q}} - \pdv{H}{\vec{q}}\cdot\pdv{f}{\vec{p}} = \mathcal{C}[f]
		\end{equation}

		\noindent
		Liouville's theorem\index{Liouville!Theorem}, \ie the distribution function is constant along any trajectory in phase space. (phase space distribution function $f$, for quantum mechanical analogue see Eq.~\ref{eq:VonNeumannEquation}):
		\begin{equation}
			\label{eq:LiouvilleTheorem}
			\dv{f}{t} = \pdv{f}{t} + \qty{f, H} = 0
		\end{equation}

	\subsection{Noether's Theorem\index{Noether!Theorem}}
		\noindent
		Continuous transformation (infinitesimal $\varepsilon$):
		\begin{equation}
			\begin{aligned}
				q_i \to q_{i}^{\prime} &= q_i+\varepsilon\psi_i\left(q,\dot{q},t\right) \\
				t \to t^{\prime}\, &= t+\varepsilon\varphi\left(q,\dot{q},t\right) \\
			\end{aligned}
		\end{equation}

		\noindent
		Condition of invariance\index{Invarianzbedingung}:
		\begin{equation}
			\frac{d}{d\varepsilon}\left[\mathcal{L}\left( {\vec{q}}^{\,\prime},\frac{d {\vec{q}}^{\,\prime}}{dt'},t'\right) \frac{dt'}{dt}\,\right]_{\varepsilon=0}=\frac{df(\vec{q}, t)}{dt} \implies S \sim S'
		\end{equation}

		\noindent
		Resulting conserved quantity\index{Erhaltungsgröße}:
		\begin{equation}
			\begin{aligned}
				S &\sim S' \implies
				\dv{t} Q\left(\vec{q},\dot{\vec{q}},t\right) = 0 \\
				Q\left(\vec{q},\dot{\vec{q}},t\right) &= \sum_{i=1}^{n}\left(\frac{\partial\mathcal{L}}{\partial{\dot{q}}_i}\psi_i\right)+\left(\mathcal{L}-\sum_{i=1}^{n}{\frac{\partial\mathcal{L}}{\partial{\dot{q}}_i}{\dot{q}}_i}\right)\varphi - f\left(\vec{q},t\right).
			\end{aligned}
		\end{equation}