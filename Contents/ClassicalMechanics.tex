% !TEX root = ../physics.tex
\section{Classical Mechanics\index{Klassische Mechanik}}
	\subsection{Newton's Laws of Motion\index{Newton!Kraftgesetze}}
		\textbf{First Axiom / lex prima} \newline%Definition des Inertialsystems / Trägheitsprinzip / Inertialgesetz \newline
			\indent \textit{Every body continues in its state of rest, or of uniform motion in a straight line, unless it is compelled to change that state by forces impressed upon it.} \nl%\textit{Ein Körper verharrt im Zustand der Ruhe oder der gleichförmig geradlinigen Bewegung, sofern jener nicht durch einwirkende Kräfte zur Änderung seines Zustands gezwungen wird.} \nl
		\textbf{Second Axiom / lex secunda}%Kinematisches Grundgesetz / Aktionsprinzip / Impulssatz
			\begin{equation}
				\tder{\vec{p}}{t} = \vec{F}
			\end{equation}
			\indent \textit{The change of motion of an object is proportional to the force impressed; and is made in the direction of the straight line in which the force is impressed.}\nl%\textit{Die Änderung der Bewegung ist der Einwirkung der bewegenden Kraft proportional und geschieht nach der Richtung derjenigen geraden Linie, nach welcher jene Kraft wirkt.} \nl
		\textbf{Third Axiom / lex tertia} \newline %Actio und Reactio / Reaktionsprinzip \newline
			\indent \textit{To every action there is always opposed an equal reaction; or, the mutual actions of two bodies upon each other are always equal, and directed to contrary parts.} \nl%\textit{Kräfte treten immer paarweise auf. Übt ein Körper $A$ auf einen anderen Körper $B$ eine Kraft aus (actio), so wirkt eine gleich große, aber entgegen gerichtete Kraft von Körper $B$ auf Körper $A$ (reactio).} \nl
		\textbf{Additional}: Principle of Superposition \newline%Superpositionsprinzip \newline
			\indent \textit{Forces are additive.}%\textit{Kräfte sind additiv.}
			\begin{equation}
				\vec{F}_{\mathrm{tot}} = \sum_{i=1}^{N} \vec{F}_i
			\end{equation}

	\subsection{Basics}
		\noindent
		Definition virial:
		\begin{equation}
			\sum_j \vec{x}_j\cdot\vec{p}_j
		\end{equation}

		\noindent
		Derivative of angular momentum\index{Drallsatz} (Torque\index{Drehmoment} $\vec{M}$):
		\begin{equation}
			\tder{\vec{L}}{t} = \vec{M}
		\end{equation}

		\noindent
		Kepler's second law\index{Flächensatz} (For a particle with a central force\index{Zentralpotential}):%(Für Teilchen im Zentralpotential):
		\begin{equation}
			\tder{A}{t} = \frac{\vbr{\vec{L}}}{2 m} = \const
		\end{equation}

	\subsection{Rotation}
		\noindent
		Definition of the Inertia Tensor\index{Trägheitstensor}:
		\begin{equation}
			\Theta_{ij}=\int_{\mathcal{V}} \rho \left[r^2\delta_{ij}-r_i r_j\right] \;\dd^3\vec{r}
		\end{equation}

		\noindent
		Parallel Axis Theorem\index{Steiner!Satz} ($\Theta_{ij}$ is the inertia tensor for rotation around the center of mass.
		For rotation around a point displaced by $\vec{a}$, the inertia tensor transforms to $\Theta'_{ij}$, where $M$ is the total mass.):%($\Theta_{ij}$ ist der Trägheitstensor für eine Drehung mit Drehachse im Schwerpunkt. Verläuft die Drehachse stattdessen durch einen um $\vec{a}$ verschobenen Punkt transformiert sich der Trägheitstensor zu $\Theta'_{ij}$, wobei $M$ die Gesamtmasse ist):
		\begin{equation}
			\Theta'_{ij} = \Theta_{ij} + M\Br{\vec{a}^2\delta_{ij} - a_i a_j}
		\end{equation}

		\noindent
		Angular momentum and energy in relation to the inertia tensor:% von Drehimpuls und Energie mit dem	Trägheitstensor:
		\begin{equation}
			\begin{aligned}
				L^i &= \Theta^i_j \omega^j \\
				T_{\mathrm{rot}} &= \frac{1}{2}\vec{\omega}^T \Theta \vec{\omega}
			\end{aligned}
		\end{equation}

	\subsection{Moving Coordinate Systems}%Bewegte Bezugssysteme}
		\noindent
		Accelerated system $K'$ and inertial system $K$%Beschleunigtes Bezugssystemn $K'$ und Inertialsystem $K$:
		\begin{equation}
			\begin{aligned}
				\pvec{x}'(t) &= \vec{x}(t)-\vec{x}_0(t) &\hspace{30pt}
				m\tder{\vec{p}}{t} &= \vec{F} \\
				\tder{\vec{x}'}{t} &= \tder{\vec{x}}{t}-\vec{\omega}\times\vec{x} &\hspace{30pt}
				m\tder{\pvec{p}'}{t} &= \vec{F} + \vec{F}_T + \vec{F}_Z + \vec{F}_L + \vec{F}_C \\
			\end{aligned}
		\end{equation}

		\noindent
		Fictitious forces\index{Scheinkraft}: displacement force\index{Scheinkraft!Translationskraft} $\vec{F}_T$, centrifugal force\index{Scheinkraft!Zentrifugalkraft} $\vec{F}_Z$, Euler force\index{Scheinkraft!Linearkraft} $\vec{F}_L$, Coriolis force\index{Scheinkraft!Corioliskraft} $\vec{F}_C$
		\begin{equation}
			\begin{aligned}
				\vec{F}_T &= -m\frac{\dd^2 \vec{x}_0}{\dd t^2} &\hspace{30pt}
				\vec{F}_Z &= -m\vec{\omega}\times\Br{\vec{\omega}\times\pvec{x}'} \\
				\vec{F}_L &= -m\tder{\vec{\omega}}{t}\times\pvec{x}' &\hspace{30pt}
				\vec{F}_C &= -2m\vec{\omega}\times\tder{\pvec{x}'}{t} \\
			\end{aligned}
		\end{equation}


	\subsection{Gravitation}
		\noindent
		Newton's law of universal gravitation\index{Newton!Gravitationskraft} (Acting on mass 1):
		\begin{equation}
			\vec{F}_1 = - G m_1 m_2 \frac{\vec{r}_1-\vec{r}_2}{\left|\vec{r}_1-\vec{r}_2\right|^3}
		\end{equation}

		\noindent
		Equations of motion for Newton's law of universal gravitation:
		\begin{equation}
			\vec{F}=-m\Nabla\phi
		\end{equation}

		\noindent
		Field equations for the Newtonian Gravitational potential\index{Newton!Gravitationspotential} $\phi$:
		\begin{equation}
			\Nabla^2\phi=-4\pi G\rho
		\end{equation}

	\subsection{Lagrange-Formalism}
		\subsubsection{Basics}
			\noindent
			Action\index{Wirkung}:
			\begin{equation}
				\mathcal{S}=\int_{t_0}^{t_1}\mathcal{L}(q_i, \dot{q_i},t)\;\mathrm{d} t
			\end{equation}

			\noindent
			Stationary-action principle / Principle of least action \index{Wirkung}:%Hamilton'sches Prinzip / Prinzip der kleinsten (stationären) Wirkung:
			\begin{equation}
				\delta \mathcal{S}=0
			\end{equation}

			\noindent
			Euler-Lagrange equation\index{Euler!-Lagrange Gleichung}:
			\begin{equation}
				 \frac{d}{dt} \frac{\partial \mathcal{L}(q_{i},\dot{q_{i}},t)}{\partial \dot{q_{i}}} - \frac{\partial \mathcal{L}(q_{i},\dot{q_{i}},t)}{\partial q_{i}} = 0
			\end{equation}

			\noindent
			Canonical momentum / conjugated momentum: \index{Kanonisch-konjugierte Impulse}:
			\begin{equation}
				p_i=\frac{\partial \mathcal{L}}{\partial\dot{q_i}}
			\end{equation}

			\noindent
			Hamiltonian\index{Hamilton!Funktion}:
			\begin{equation}
				\mathcal{H}(q_i,p_i,t)=\sum_{j=1}^{f}p_j\dot{q_j}(p) - \mathcal{L}(q_i, \dot{q_i}(p),t)
			\end{equation}

			\noindent
			Hamilton's equations\index{Hamilton!Bewegungsgleichungen}:
			\begin{equation}
				\begin{aligned}
					\dot{q}_k &= \phantom{-}\pder{\mathcal{H}}{p_k}, &&\hspace{30pt}
					\dot{p}_k &= -\pder{\mathcal{H}}{q_k}, &&\hspace{30pt}
					\pder{\mathcal{H}}{t} &= -\pder{\mathcal{L}}{t}
				\end{aligned}
			\end{equation}

			\noindent
			Poisson-Brackets:
			\begin{equation}
				\begin{aligned}
					\lbrace A, B \rbrace &= \sum_k \left(
						\pder{A}{q_k}\pder{B}{p_k} - \pder{A}{p_k}\pder{B}{q_k}
					\right) \\
					\tder{A}{t} &= \cBr{A, H} + \pder{A}{t}
				\end{aligned}
			\end{equation}

		\subsubsection{Noether's Theorem\index{Noether!Theorem}}
			\noindent
			Continuous transformation (infinitesimal $\varepsilon$): %Kontinuierliche Transformation (mit infinitesimalem $\varepsilon$):
			\begin{equation}
				\begin{aligned}
				x_i \rightarrow x_{i}^{\prime} &= x_i+\varepsilon\psi_i\left(x,\dot{x},t\right) \\
					t\rightarrow t^{\prime}\, &= t+\varepsilon\varphi\left(x,\dot{x},t\right) \\
				\end{aligned}
			\end{equation}

			\noindent
			Condition of invariance\index{Invarianzbedingung}:
			\begin{equation}
				\frac{d}{d\varepsilon}\left[\mathcal{L}\left( {\vec{x}}^{\,\prime},\frac{d {\vec{x}}^{\,\prime}}{dt'},t'\right) \frac{dt'}{dt}\,\right]_{\varepsilon=0}=\frac{df(\vec{x}, t)}{dt}
			\end{equation}

			\noindent
			Resulting conserved quantity\index{Erhaltungsgröße}:
			\begin{equation}
				\begin{aligned}
					S &\sim S' \;\Rightarrow\;
					\tder{}{t} Q\left(\vec{x},\dot{\vec{x}},t\right) = 0 \\
					Q\left(\vec{x},\dot{\vec{x}},t\right) &= \sum_{i=1}^{n}\left(\frac{\partial\mathcal{L}}{\partial{\dot{x}}_i}\psi_i\right)+\left(\mathcal{L}-\sum_{i=1}^{n}{\frac{\partial\mathcal{L}}{\partial{\dot{x}}_i}{\dot{x}}_i}\right)\varphi - f\left(\vec{x},t\right).
				\end{aligned}
			\end{equation}

		\subsubsection{Kepler Problem\index{Kepler!Problem}}
			\noindent
			Gravitational parameter \index{Gravitationsparameter} $\mu$, eccentricity\index{Exzentrizität} $\epsilon$ and $\rho_0$:
			\begin{equation}
				\begin{aligned}
					\mu &= G m &\hspace{30pt}
					\epsilon &= \sqrt{1+\frac{2 E L^2}{m \mu^2}} &\hspace{30pt}
					\rho_0 &= \frac{L^2}{m\mu} \\
				\end{aligned}
			\end{equation}

			\noindent
			Lagrangian (Gravitational parameter\index{Gravitationsparameter} $\mu$, for central body approximation $\mu=GM$, see \ref{Sec:AstronomicalConstants}):
			\begin{equation}
				\mathcal{L}(\vec{r},\dot{\vec{r}}) = \frac{m}{2} \dot{\pvec{r}}^2 + \frac{\mu}{r}
			\end{equation}

			\noindent
			Conserved Laplace-Runge-Lenz vector:%Erhaltener Runge Lenz-Vektor:
			\begin{equation}
				\vec{A} = \dot{\vec{r}}\times\vec{L} - \frac{\mu \vec{r}}{r}
			\end{equation}

			\noindent
			Orbit parametrization (Cylindrical coordinates $\rho, \varphi$):
			\begin{equation}
				\rho = \frac{\rho_0}{1+\epsilon \cos\varphi}
			\end{equation}

			\noindent
			Semi-major axis\index{Große Halbachse} $a$:
			\begin{equation}
				a = \frac{p}{1-\epsilon^2}
			\end{equation}

			\noindent
			Vis-Viva\index{Vis-Viva Gleichung} Equation
			\begin{equation}
				v^2 = \mu\Br{\frac{2}{r} - \frac{1}{a}}
			\end{equation}
