% !TEX root = ../physics.tex
\section{Thermodynamics}
	\subsection{Definitions}
		At the \emph{Thermodynamical equilibrium}, macroscopic properties of a system are constant in time.
		Thermodynamic equilibria are stable owing to the \emph{principle of Le Chatelier}\index{Chatelier!Prinzip} where a counteracting force forms upon fluctuations.\vsp

		The \emph{internal energy} $U$ is the energy of a many particle system, which is restricted to a volume $V$ by an external potential. Thus, it has a vanishing total momentum $\vec{P}=0$ and total angular momentum $\vec{L} = 0$, such that all kinetic energy is based on internal properties. \vsp

		Definition of the inexact differential ($\delta$) for \emph{Work} by the controlled work parameters\index{Arbeitskoordinaten} $Z_\alpha$ and equilibrium parameters $g_\alpha$.
		The choice of work parameters (i.e. the access to the system) defines work as usable form of energy, as well as properties like entropy and temperature. The ambiguity of this choice is known as \emph{relative objectivity}.
		\begin{equation}
			\delta W = \sum_\alpha g_\alpha \dd Z_\alpha
		\end{equation} \vsp

		The state function\index{Zustandsgrößen} $\vec{Z}=\left(U, Z_1,... \right)$ are macroscopically observable, i.e. accessible quantities that uniquely define an equilibrium state. \vsp

		\emph{Intensive Variables}: homogeneous of order $0$ ($T$, $p$, ...). \vsp

		\emph{Extensive Variables}: homogeneous of order $1$ ($S$, $U$, $V$, ...; i.e. additive quantities). \vsp

		\begin{table}[ht]
			\begin{center}
				\begin{tabular}{ l | l l l l l l }
					$g_\alpha$ & $-p$ & $\mu$ & $\sigma$ & $E_0$ & $B_0$ & $\Phi$ \\ \hline
					$\dd Z_\alpha$ & $\dd V$ & $\dd N$ & $\dd A$ & $\dd \mathcal{P}$ & $\dd \mathcal{M}$ & $\dd Q$ \\
				\end{tabular}
				\caption{work parameters and corresponding equilibrium parameters. $\sigma$: Surface tension, $A$: Surface, $B_0$: Magnetic field, $\mathcal{M}$: Total magnetization, $\mu$ Chemical Potential, $N$: Particle count...}
				\label{tab:WorkParametersAndEquilibriumParameters}
			\end{center}
		\end{table} \vsp

		\emph{Thermodynamical limit}: \newline Deterministic behavior for $U,V,N \to \infty$ while $U/N,V/N=\const$. \vsp

		\noindent
		Thermal de Broglie wavelength\index{De Broglie!Thermische Wellenlänge}:
		\begin{equation}
			\label{Eq:ThermalDeBroglieWavelength}
			\lambda_T = \frac{h}{\sqrt{2\pi m \kB T}}
		\end{equation} \vsp

		\noindent
		Fugacity\index{Fugazität}
		\begin{equation}
			z = \ex^{\beta\mu}
		\end{equation} \vsp

	\subsection{Entropy}
		\noindent
		Thermodynamic definition (additionally $\lim_{T \to 0} S := 0$, Eq.~\ref{Eq:ThirdLawOfThermodynamics}):
		\begin{equation}
			\dd S := \frac{\delta Q_\text{rev.}}{T}
		\end{equation}

		\noindent
		Properties of entropy (resulting from the second law):
		\begin{itemize}\itemsep -0pt	% reduce space between items
			\item $\Delta S = S(t>t_0)-S(t_0) \ge 0$ \hfill{(Entropy increases in isolated systems)}
			\item $S(\vec{Z}_1 + \vec{Z}_2) \ge S(\vec{Z}_1) + S(\vec{Z}_2)$ \hfill{(Superadditivity)}
			\item $\partial^2 S \le 0 \wedge \dd{S} = 0$ \hfill{(Maximum at equilibrium)}
		\end{itemize}

		\subsubsection{Cyclic Processes}
			\noindent
			Idealized processes:
			\begin{itemize}
				\item Isothermic: $\dd T = 0$
				\item Isochoric: $\dd V = 0$
				\item Isobaric: $\dd p = 0$
				\item Adiabatic: $\delta Q = 0$
				\item Isentropic: $\dd S = 0$ (reversible adiabatic process)
			\end{itemize}

			\noindent
			Cyclic process\index{Kreisprozess}\index{Wärmekraftmaschine} (Equation for reversible, inequation for irreversible processes):
			\begin{equation}
				\oint \frac{\delta Q_\text{rev.}}{T} \le 0
			\end{equation}

			\noindent
			Carnot Process\index{Carnot!Prozess}:
			Only possible reversible cyclic process consisting of two adiabatic processes and two isothermic processes between the temperatures $T_1$, $T_2$ with $T_1 < T_2$. Every realizable cyclic process has efficiency that is smaller than the Carnot efficiency\index{Carnot!Wirkungsgrad}:
			\begin{equation}
				\eta_\text{C} = 1-\frac{T_1}{T_2}
			\end{equation}

		\subsubsection{Information Theory}
			Landauer principle\index{Landauer!Prinzip}

			\indent \emph{The deletion of one bit of information requires the dissipation of $\Delta W = \kB T \ln 2$ of energy.}\nl
			\noindent
			Bekenstein bound\index{Bekenstein!Grenze} \\
			\indent \emph{The entropy of a physical system is bounded by the area of the system's event horizon.}
			\begin{equation}
				S \le \frac{2\pi \kB E R}{\hbar c}
			\end{equation}
			($R$ is the event horizon radius and $E$ is the total energy)

	\subsection{Theory}
		\subsubsection{Laws of Thermodynamics}
			\textbf{Zeroth Law}: Existence and transitivity of equilibrium states\newline
			\indent \emph{The thermodynamic equilibrium is a equivalence relations between thermodynamic systems (Equilibrium between $A \sim B$ and $B \sim C$ implies $A \sim C$.)} \nl
			\textbf{First Law}: Energy
			\begin{equation}
				\label{Eq:FirstLawOfThermodynamics}
				\dd U = \delta Q + \delta W
			\end{equation}
			\indent Inner Energy $\dd U$, added heat $\delta Q$, Work applied to the system $\delta W$. \nl
			\textbf{Second Law}: Entropy
			\begin{equation}
				\label{Eq:SecondLawOfThermodynamics}
				\Delta S	\ge 0
			\end{equation}
			\indent Kelvin\index{Kelvin!Wärmesatz}: \emph{It is impossible to generate work in a cycle by solely cooling a heat sink.}\nl
			\textbf{Third Law}: Nernst's law of heat\index{Nernst!Wärmesatz}\newline
			\indent Nernst: \emph{The Entropy of every system converges to a finite value $S_0:=0$ independently of the work parameters $Z_\alpha$ at low temperatures.
				\begin{equation}
					S \stackrel{T \to 0}{\longrightarrow} S_0
					\label{Eq:ThirdLawOfThermodynamics}
				\end{equation}}

		\subsubsection{Fundamental Thermodynamic Relation}
			\noindent
			Gibbs fundamental thermodynamic relation\index{Gibbs!Fundamentalgleichung} (result from first and second laws, Eq.~\ref{Eq:FirstLawOfThermodynamics}, \ref{Eq:SecondLawOfThermodynamics})
			\begin{equation}
				\label{Eq:GibbsFundamental}
				\dd S = \frac{1}{T} \dd U - \frac{1}{T} \sum_{\alpha} g_\alpha \dd Z_\alpha
			\end{equation}

			\noindent
			Euler's homogeneity relation\index{Euler!Homogenitätsrelation} (Resulting from the homogeneity of $S(\vec{Z})$):
			\begin{equation}
				\label{Eq:EulerHomogenity}
				S(\vec{Z}) = U\pdv{S}{U} + \sum_\alpha Z_\alpha\pdv{S}{Z_\alpha}
				= \frac{1}{T} \left( U-\sum_\alpha g_\alpha Z_\alpha \right)
			\end{equation}

			\noindent
			Gibbs--Duhem equation\index{Gibbs!--Duhem Beziehung}\index{Duhem!Gibbs--Duhem Beziehung} (Resulting from Eq.~\ref{Eq:GibbsFundamental}, \ref{Eq:EulerHomogenity}):
			\begin{equation}
				0 = S \dd T + \sum_\alpha Z_\alpha \dd g_\alpha
			\end{equation}

			\noindent
			Connection between thermodynamic state equation and caloric state equation\index{Thermische Zustandsgleichung}\index{Kalorische Zustandsgleichung}:
			\begin{equation}
				\left(\pdv{U}{V}\right)_T = T\left(\pdv{p}{T}\right)_V - p
			\end{equation}

			\noindent
			$T \dd S$-Equation
			\begin{equation}
				\begin{aligned}
					T \dd S &= C_V \dd T + \frac{\alpha T}{\kappa_T} \dd V \\
					T \dd S &= C_p \dd T - \alpha T V \dd p
				\end{aligned}
			\end{equation}


		\subsubsection{Thermodynamic Potentials}
			\noindent
			Guggenheim Quadrat:
			Unheimlich Viele Forscher Trinken Gerne pils Hinterm Schreibtisch. % cspell:disable-line

			\begin{table}[ht]
				\begin{center}
					\makebox[1\textwidth][c]{
						\begin{tabular}{ r | l | l | l }
							Name & Relation & Differential & Maxwell relation\index{Maxwell!Beziehung} \\ \hline \xrowht{26pt}
							Entropy & $S(U,V)$ & $\dd S = \frac{1}{T} \dd U + \frac{p}{T} \dd V$ & $\left(\dpder{}{V} \dfrac{1}{T}\right)_U = \left(\dpder{}{U}\dfrac{p}{T}\right)_V$ \\ \hline \xrowht{26pt}
							Internal Energy & $U(S,V)$ & $\dd U = T \dd S - p \dd V$ & $\left( \dpder{T}{V} \right)_S =	- \left( \dpder{p}{S} \right)_V$ \\ \hline \xrowht{26pt}
							Free Energy & $F(V,T) = U - TS$ & $\dd F = -S \dd T - p \dd V$ & \phantom{-} $\left(\dpder{S}{V}\right)_T = \left(\dpder{p}{T}\right)_V$ \\ \hline \xrowht{26pt}
							Gibbs Energy & $G(T,p) = U - TS + pV$ & $\dd G = -S \dd T - V \dd p$ & $-\left(\dpder{S}{p}\right)_T = \left(\dpder{V}{T}\right)_p$ \\ \hline \xrowht{26pt}
							Enthalpy & $H(p,S) = U + pV$ & $\dd H = T \dd S + V \dd p$ & \phantom{-} $\left(\dpder{V}{S}\right)_p = \left(\dpder{T}{p}\right)_S$ \\ \hline \xrowht{26pt}
							Grand Potential & $\Omega(T,V,\mu) = U - T S - \mu N$ & $\dd \Omega = -S \dd T - p \dd V -N \dd \mu$ &	\\ \hline
						\end{tabular}}
					\caption{List of commonly used potentials}
					\label{tab:ThermodynamischePotentiale}
				\end{center}
			\end{table}

			\noindent
			For homogeneous systems:
			\begin{equation}
				\begin{aligned}
					G(T,p) &= \phantom{-}N\mu(T,p) \\
					\Omega(T,V,\mu) &= -V p(T,\mu) \\
				\end{aligned}
			\end{equation}

	\subsection{Phenomena}
		\subsubsection{Response Functions}
			\noindent
			Definitions of response functions\index{Antwortgrößen} (Heat capacity / thermal capacity at constant volume $C_V$ and at constant pressure $C_p$, specific heat capacities $c_V$ and $c_p$, expansion coefficient $\alpha$, tension coefficient $\beta$, isothermal compressibility $\kappa_T$, adiabatic compressibility $\kappa_S$):
			\begin{equation}
				\begin{aligned}
					N c_V = C_V &= \left( \pdv{Q}{T} \right)_V = T \left( \pdv{S}{T} \right)_V \\
					N c_p = C_p &= \left( \pdv{Q}{T} \right)_p = T \left( \pdv{S}{T} \right)_p \\
					\alpha &= \frac{1}{V} \left( \pdv{V}{T} \right)_p \\
					\beta &= \frac{1}{p} \left( \pdv{p}{T} \right)_V \\
					\kappa_T &= -\frac{1}{V} \left( \pdv{V}{p} \right)_T \\
					\kappa_S &= -\frac{1}{V} \left( \pdv{V}{p} \right)_S \\
				\end{aligned}
			\end{equation}

			\noindent
			Mayer Relation and other Relations \index{Mayer!Beziehung} (In particular: $c_p > c_V$ and $\kappa_T > \kappa_S$):
			\begin{equation}
				\begin{aligned}
					C_p - C_V &= \frac{TV\alpha^2}{\kappa_T} \\
					\beta &= \frac{\alpha}{p \kappa_T} \\
					\frac{c_p}{c_V} &= \frac{\kappa_T}{\kappa_S} \\
				\end{aligned}
			\end{equation}

			\noindent
			Joule--Thomson Coefficient\index{Joule, James Prescott!Joule--Thomson Koeffizient}\index{Thomson, William!Joule--Thomson Koeffizient}\index{Kelvin!Joule--Thomson Koeffizient}:
			\begin{equation}
				\mu_{JT} = \left( \pdv{T}{p} \right)_H = \frac{V}{C_p}\left(\alpha T - 1\right)
			\end{equation}

			\noindent
			For magnetic systems, the analogue of compressibility is the magnetic susceptibility\index{Magnetische Suszeptibilität} $\chi_{T/S}$:
			\begin{equation}
				\chi_{S/T} = \left(\pdv{\mathcal{M}}{B_0}\right)_{S/T}
			\end{equation}

		\subsubsection{Chemical Equilibria}
			\noindent
			$s$ distinct chemical reactions of $r$ different substances with chemical symbols $S_i$, characterized by the stoichiometric coefficients\index{Stöchiometrischer Koeffizient} $\nu_i^k\in\integers$, $\forall k\in\lbrace1,...,s\rbrace$:
			\begin{equation}
				\sum_{i=1}^r \nu_i^k S_i = 0
			\end{equation}

			\noindent
			Chemical conversion with arbitrary conversion variable $\dd \lambda_k$:
			\begin{equation}
				\dd N_i = \sum_{k=1}^s \nu_i^k \dd \lambda_k
			\end{equation}

			\noindent
			At thermodynamic equilibrium, Gibbs Energy\index{Gibbs!Energie} is minimal:
			\begin{equation}
				0 = \dd G = \sum_{i=1}^r \mu_i \dd N_i \implies \sum_{i=1}^r \nu_i^k \mu_i = 0 \quad\forall k\in\lbrace1,...,s\rbrace
			\end{equation}

			\noindent
			Law of mass action\index{Massenwirkungsgesetz} (With particle concentration $c_i = N_i/N$ mass action constant $K$):
			\begin{equation}
				\prod_{i=1}^r	c_i^{\nu_i} = \exp\left( -\frac{1}{\kB T}\sum_{i=1}^r \nu_i\mu_i^0(T,p) \right) =: K(T,p)
			\end{equation}

		\subsubsection{Phase Transitions}
			\noindent
			Clausius--Clapeyron Relation\index{Clausius!--Clapeyron Beziehung}\index{Clapeyron!Clausius--Clapeyron Beziehung}:
			\begin{equation}
				\dv{p}{T} = \frac{\Delta S}{\Delta V} = \frac{S_g - S_{fl}}{V_g - V_{fl}}
			\end{equation}

			\noindent
			Latent heat per particle:
			\begin{equation}
				l = \frac{\Delta H}{N} = T\frac{\Delta S}{N}
			\end{equation}

			\noindent
			\href{https://en.wikipedia.org/wiki/Phase_rule}{Gibbs phase law}\index{Gibbs!Phasenregel} (for the degrees of freedom $f$ at $r$ components distributed in $\nu$ phases):
			\begin{equation}
				f = 2 + r - \nu
			\end{equation}

			\noindent
			Critical exponents ($d$ is the dimensionality of the system, $C$ is the heat capacity, $\Psi$ is the order parameter \eg magnetization, $J$ is the source field \eg magnetic field, $\chi$ is the susceptibility, $\avg{\psi(0)\psi(r)}$ is the correlation function, $\xi$ is the correlation length and $\tau = \frac{T-T_c}{T_c}$ the reduced temperature):
			\begin{equation}
				\begin{aligned}
					&C \propto \tau^{-\alpha}
					\hsp \Psi \propto (-\tau)^{-\beta}
					\hsp \chi \propto \tau^{-\gamma} \\
					&J \propto \tau^{-\delta}
					\hsp \avg{\psi(0)\psi(r)} \propto r^{d+2-\eta}
				\end{aligned}
			\end{equation}

			\noindent
			Scaling relations and hyperscaling relations (valid if correlation function and Gibbs energy are homogenous functions):
			\begin{equation}
				\begin{aligned}
					&2 - \alpha = 2\beta + \gamma
					\hsp 2 - \alpha = \beta(\delta + 1) \\
					&2 - \alpha = d\nu
					\hsp \gamma = (2 - \eta)\nu \\
				\end{aligned}
			\end{equation}

			\noindent
			Redundancy (only two \dof for the critical exponents, note that $d$ must be smaller than the certain upper limit, dependent on the universality class):
			\begin{equation}
				\begin{aligned}
					&\alpha = 2 - d\nu
					\hsp \beta = \frac{\nu}{2} (d-2+\eta) \\
					& \gamma = \nu(2-\eta)
					\hsp \delta = \frac{d+2-\eta}{d-2+\eta}
				\end{aligned}
			\end{equation}

	\subsection{Non-Equilibrium}
		\noindent
		Assumption: Sufficiently fast processes create local equilibria, such that local equilibrium properties $T(t,\vec{r})$, $p(t,\vec{r})$,... exist. \\
		Continuity Equation\index{Kontinuitätsgleichung!Thermodynamik} for the energy density $u(t,\vec{r})$ and energy current density $\vec{j}^q$:
		\begin{equation}
			\pdv{u}{t} + \Nabla\cdot\vec{j}^q = 0
		\end{equation}

		\noindent
		Fourier's law\index{Fourier!Gesetz} / law of heat conduction draws a connection between heat current density with the heat conductivity\index{Wärmeleitfähigkeit} $\kappa >0$ and the temperature gradient:
		\begin{equation}
			\vec{j}^q = -\kappa \Nabla T
		\end{equation}

		\noindent
		Heat equation\index{Wärmeleitungsgleichung} (for $\dd u = nc_V \dd T$ (\eg in solids); diffusion constant $D=\kappa/nc_V$):
		\begin{equation}
			\pdv{T}{t} = D\Nabla^2 T
		\end{equation}

		\noindent
		Heat conduction is irreversible. For the entropy density $s(t,\vec{r})$ and entropy current density $\vec{j}^s (t,\vec{r}) = \vec{j}^q/T$ the following holds (Entropy creation rate $\dot{s}$):
		\begin{equation}
			\dot{s} = \pdv{s}{t} + \Nabla\cdot\vec{j}^s = - \frac{\vec{j}^q}{T^2}\cdot\Nabla T \ge 0
		\end{equation}

		\noindent
		\href{https://en.wikipedia.org/wiki/Onsager_reciprocal_relations}{Onsager reciprocal relations}\index{Onsager!Reziprozitätsbeziehung}:
		\newline\indent\emph{Aus der mikroskopischen Reversibilität folgen Dinge für die Entropieerzeugungsrate.} % cspell:disable-line

		\hfill ---\,Fabian Hassler\vsp

		\noindent
		Thermoelectric effects: in systems with charge carriers, the chemical potential should be replaced with the electro-chemical potential $\mu_{EC}$, because a change in particle count will imply a change of charges in the potential, such that (Peltier coefficient\index{Peltier!Koeffizient} $\Pi$, Seebeck coefficient\index{Seebeck!Koeffizient} $\epsilon$, electric current $\vec{j} = q\vec{j}^n$ and specific heat conductivity $\sigma$, the last equation follows from the reciprocity condition):
		\begin{equation}
			\begin{aligned}
				\delta W &= \mu \dd N + \Phi \dd Q = (\mu+q\Phi)\dd N = \mu_{EC} \dd N \\
				\Nabla \mu_{EC} &= -q\vec{\mathcal{E}} \\
				\vec{j}^q &= - \kappa \Nabla T + \Pi \vec{j} \\
				\vec{\mathcal{E}} &= \vec{E} - \frac{1}{q} \Nabla \mu = \frac{1}{\sigma} \vec{j} + \epsilon \Nabla T \\
				\Pi &= \epsilon T \\
			\end{aligned}
		\end{equation}

	\subsection{Model Systems}
		\subsubsection{Classical Ideal Gas}
			\noindent
			An ideal gas is a system of $N$ identical particles in a volume $V$, that do not interact with one another.\vsp

			\noindent
			Ideal Gas Law\index{Thermische Zustandsgleichung!Ideales Gas} $p(T, N, V)$:
			\begin{equation}
				pV = N \kB T
			\end{equation}

			\noindent
			Caloric state equation\index{Kalorische Zustandsgleichung!Ideales Gas} $U(T, N, V)$:
			\begin{equation}
				U = \frac{3}{2} N \kB T
			\end{equation}

			\noindent
			Entropy:
			\begin{equation}
				S(T,p) = N \kB \ln{\left[ \frac{V}{N}\left( \frac{4\pi m	U}{3 h^2 N} \right)^{\frac{3}{2}} \right]} + \frac{5}{2} N \kB
			\end{equation}

			\noindent
			Adiabatic equation (holds for constant heat capacities with adiabatic coefficient $\gamma = \frac{c_p}{c_V} = \frac{c_V + \kB}{c_V}$)
			\begin{equation}
				p V^\gamma = \const
			\end{equation}

		\subsubsection{Ideal Quantum Gas}
			\noindent
			Model: indistinguishable, non-interacting particles (Fermions or Bosons) with spin degeneracy. \vsp

			\noindent
			Grand canonical partition function (With $\varepsilon_\alpha = \frac{\hbar^2 \vec{k}^2}{2m}$ being the entropy for the state $\alpha=(\vec{k},\sigma)$):
			\begin{equation}
				\mathcal{Z} = \prod_{\alpha}
				\begin{cases}
					\left( 1-\ex^{\beta(\mu-\varepsilon_\alpha)}\right)^{-1} & \text{(Bosons)} \\
					\phantom{\big(} 1+\ex^{\beta(\mu-\varepsilon_\alpha)} & \text{(Fermions)}
				\end{cases}
			\end{equation}

			\noindent
			Mean occupation number\index{Besetzungszahl} (Bose--Einstein statistics\index{Bose!--Einstein Statistik}\index{Einstein!Bose--Einstein Statistik}, Fermi--Dirac statistics\index{Fermi!--Dirac Statistik}\index{Dirac!Fermi--Dirac Statistik}):
			\begin{equation}
				\overline{n_\alpha} =
				\begin{cases}
					\dfrac{1}{\ex^{\beta(\varepsilon_\alpha-\mu)} - 1} & \text{(Bosons)} \\
					\dfrac{1}{\ex^{\beta(\varepsilon_\alpha-\mu)} + 1} & \text{(Fermions)}
				\end{cases} \\
			\end{equation}

			\noindent
			Entropy:
			\begin{equation}
				S= -\kB\sum_\alpha
				\begin{cases}
					\overline{n_\alpha} \ln \overline{n_\alpha} - (1+\overline{n_\alpha}) \ln (1+\overline{n_\alpha}) & \text{(Bosons)} \\
					\overline{n_\alpha} \ln \overline{n_\alpha} + (1-\overline{n_\alpha}) \ln (1-\overline{n_\alpha}) & \text{(Fermions)}
				\end{cases} \\
			\end{equation}

			\noindent
			Internal energy, particle number:
			\begin{equation}
				\begin{aligned}
					U &= \sum_\alpha \overline{n_\alpha} \varepsilon_\alpha \\
					N &= \sum_\alpha \overline{n_\alpha}
				\end{aligned}
			\end{equation}

			\noindent
			Caloric state equation\index{Kalorische Zustandsgleichung!Quantenmechanisches ideales Gas}:
			\begin{equation}
				U=\frac{3}{2}pV
			\end{equation}

			\noindent
			Thermic state equation\index{Thermische Zustandsgleichung!Quantenmechanisches ideales Gas}:
			\begin{equation}
				\frac{pV}{N \kB T} = \frac{\sum_{l=1}^{\infty}\frac{z^l}{l^{5/2}}}{\dv{\ln z}\sum_{l=1}^{\infty}\frac{z^l}{l^{5/2}}}
			\end{equation}

		\subsubsection{Van der Waals Gas}
			\noindent
			Van der Waals Equation\index{Zustandsgleichung!Van der Waals Gas}\index{Van der Waals Gas} (Covolume $b \ge 0$, cohesion pressure parameter $a \ge 0$, specific volume $v=V/N$):
			\begin{equation}
				\left( p+\frac{a}{v^2} \right) \left( v-b \right) = \kB T
			\end{equation}

		\subsubsection{Phonon Gas}
			\label{Sec:PhononGas}
			\noindent
			Hamilton function of a solid with $N$ atoms:
			\begin{equation}
				H=\sum_i \frac{\vec{p}_i^2}{2m} + \frac{1}{2}m\sum_{\stackrel{i,j}{\alpha,\beta}} D_{\alpha\beta}(\vec{r}_i - \vec{r}_j)u_{i,\alpha}u_{j,\beta}
			\end{equation}

			\noindent
			The Hamilton function is separable, such that $3N$ eigenmodes are described by harmonic oscillator potentials with $\omega_\nu$. For the ensemble, mean occupation number $\overline{n}_\nu$ (Bose--Einstein statistic\index{Einstein, Albert!Bose--Einstein Statistik}\index{Bose, Satyendra Nath!Bose--Einstein Statistik}) and internal Energy $U$, it follows:
			\begin{equation}
				\begin{aligned}
					Z_\nu &= \sum_{n_\nu=0}^\infty \ex^{-\beta E_{n_\nu}} \\
					Z &= \prod_\nu Z_\nu = \frac{1}{2\sinh{\left( \frac{\hbar\omega_\nu}{2 \kB T} \right)}} \\
					\overline{n}_\nu &= \frac{1}{\ex^{\hbar\omega_\nu/\kB T} - 1} \\
					U &= \sum_\nu \hbar\omega_\nu \left(\overline{n}_\nu + \frac{1}{2}\right)
				\end{aligned}
			\end{equation}


			\noindent
			Einstein absorption and emission model\index{Einstein!Koeffizienten}
			\begin{equation}
				\begin{aligned}
					\frac{\dd N_1}{\dd t} &= -B_{12}N_1 \frac{\dd u}{\dd \omega}(\omega_{12}) \\
					\frac{\dd N_2}{\dd t} &= -B_{21}N_2 \frac{\dd u}{\dd \omega}(\omega_{21}) \\
					\frac{\dd N_2}{\dd t} &= -A_{21}N_2\\
				\end{aligned}
			\end{equation}

			\noindent
			Where
			\begin{itemize}
				\setlength\itemsep{0pt}
				\item[] $N_1$: Occupation number of the ground state,
				\item[] $N_2$: Occupation number of the excited state,
				\item[] $B_{12}$: Einstein's B-coefficient for absorption,
				\item[] $B_{21} = B_{12}$: Einstein's B-coefficient for stimulated emission,
				\item[] $A_{21}$: Einstein's A-coefficient for spontaneous emission,
				\item[] $\tau_{R}=A_{21}^{-1}$: radiating life time,
				\item[] $\frac{\dd u}{\dd \omega}(\omega)$: spectral energy density.
			\end{itemize}

		\subsubsection{Black Body}
			\noindent
			Similar to the phonon gas of Sec.~\ref{Sec:PhononGas}, Planck found a description of a photonic gas for a black body\index{Planck!Strahler}. \vsp

			\noindent
			Planck's law\index{Planck!Strahlungsgesetz} (For the spectral energy density $\pdv{u}{\omega}$):
			\begin{equation}
				\begin{aligned}
					\pdv{u}{\omega}(\omega,T) &= \frac{\hbar}{\pi^2 c^3} \frac{\omega^3}{\ex^{\hbar \omega / \kB T} - 1} \\
				\end{aligned}
			\end{equation}

			\noindent
			Planck's law\index{Planck!Strahlungsgesetz} for the specific intensity / spectral radiance (Energy per time per area per solid angle):
			\begin{equation}
				I_\omega = \frac{c}{4\pi} \pdv{u}{\omega}
				= \frac{\hbar}{4 \pi^3 c^2} \frac{\omega^3}{\ex^{\hbar \omega / \kB T} - 1}
			\end{equation}

			\noindent
			Wien's displacement law\index{Wien!Verschiebungsgesetz} (For the wavelength or frequency of the $\lambda_{\max}$ emission maximum; $x_\lambda=4.965\,511\dots$ solves $(x_\lambda-5)\ex^{x_\lambda}+5=0$, and $x_f\approx2.821\,439\dots$ solves $(3-x_f)\ex^{x_f}+3=0$):
			\begin{equation}
				\begin{aligned}
					\lambda_{\max} \kB T &= \frac{h c}{x_\lambda} = \const \\
					h f_{\max} &= \kB T x_f = \const \\
				\end{aligned}
			\end{equation}

			\noindent
			Stefan--Boltzmann law\index{Stefan!--Boltzmann Gesetz}\index{Boltzmann!Stefan--Boltzmann Gesetz} (Intensity / emission power per surface area $I$, Stefan--Boltzmann constant $\sigma$ and emission coefficient $\epsilon$; $\epsilon=1$ in case of a black body):
			\begin{equation}
				I = \epsilon\sigma T^4
			\end{equation}

			\noindent
			Internal energy, Entropy, particle number and pressure of a black body photon gas (Stefan--Boltzmann constant\index{Stefan!--Boltzmann Konstante}\index{Boltzmann!Stefan--Boltzmann Konstante} $\sigma$):
			\begin{align}
				U &= \frac{4\sigma}{c} V T^4 \\
				S &= \frac{4 U}{3 T} = \frac{16 \sigma}{3 c} V T^3 \\
				N &= \frac{2 k^3 \zeta(3)}{\pi^2 c^3 \hbar^3} V T^3 \\
				P &= \frac{U}{3V} = \frac{4\sigma}{3c} T^4
			\end{align}

			\noindent
			State Equation for the photon gas:
			\begin{equation}
				p(T) = \frac{U}{3V}
			\end{equation}

			\noindent
			Photon density (where $\zeta$ is the Riemann zeta function\index{Riemann!Zeta Funktion}):
			\begin{equation}
				n_\gamma(T) = \frac{2\zeta(3)}{\pi^2} \qty(\frac{\kB T}{\hbar c})^3
			\end{equation}

		\subsubsection{Ideal Paramagnetism}
			\noindent
			Zeemann energy\index{Zeemann!Energie} in an external field in an external field $\vec{B}_0$:
			\begin{equation}
				H_B = -\sum_{i=1}^N \vec{m}_i \cdot \vec{B}_0
			\end{equation}

			\noindent
			For Ising spins\index{Ising!Spin} ($s_i\in\lbrace +1, -1\rbrace$ with $|\vec{m}_i| = \mu$), the following partition Gibbs energy and magnetization result:
			\begin{equation}
				\begin{aligned}
					Z(T,B_0) &= 2^N \cosh^N (\beta\mu B_0) \\
					G(T,B_0) &= -\kB T \ln Z \\
					\mathcal{M}(T,B_0) &= -\left(\pdv{G}{B_0}\right)_T = N\mu \tanh(\mu B_0 /\kB T)
				\end{aligned}
			\end{equation}

			\noindent
			Curie law\index{Curie!Gesetz} ($\kB T \gg \mu B_0$):
			\begin{equation}
				\mathcal{M} \propto \frac{1}{T}
			\end{equation}

		\subsubsection{Ising Model\index{Ising!Modell}}
			\noindent
			Interaction energy (Usually, just the interaction to the directly neighboring spins are considered: $J_{ij} = J \delta_{i,j\pm 1}$):
			\begin{equation}
				H = -\frac{1}{2}\sum_{i,j} J_{ij} s_i s_j -\mu B_0 \sum_i s_i
			\end{equation}

			\noindent
			Mean-field theory\index{Molekularfeldnäherung}: \newline
			Replace $s_i s_j \to s_i m$, where $m = \overline{s}$ is the mean field, while disregarding correlations $\overline{(s_i-m)(s_j-m)}\ll m^2$. \nl
			Mean field approximation and self consistency equation (in $D$ dimensions, $h:=\mu B_0$):
			\begin{equation}
				\begin{aligned}
					H_\text{MF} &= -(h + 2 D J m) \sum_i s_i + D N J m^2 \\
					m &= -\frac{1}{N}\pdv{G}{h} = \tanh\left({\beta(h + 2 D J m)}\right)
				\end{aligned}
			\end{equation}
