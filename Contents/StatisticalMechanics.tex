% !TEX root = ../physics.tex
\section{Statistische Physik}
	\subsection{Statistische Mechanik}
		\subsubsection{Allgemeines}
			\noindent
			Äquipartitionstheorem $(z_1,...,z_{2f}) = (q_1, p_1, ..., q_f, p_f)$:
			\begin{equation}
				\overline{ z_i \pder{H}{z_j}} = k_B T \delta_{ij}
			\end{equation}

		\subsubsection{Definition der Entropie}
			\noindent
			Boltzmann Entropie (mit mikrokanonischer Zustandssumme $\Gamma$ und Zustandsvektor $\vec{Z}=(U,V,N,...)$)
			\begin{equation}
				S_{B}(\vec{Z}) = k_B \ln\Gamma(\vec{Z})
			\end{equation}

			\noindent
			Gibbs Entropie:
			\begin{equation}
				S_{G} = -k_B \sum_i p_i \ln p_i
			\end{equation}

			\noindent
			Shannon Entropie:
			\begin{equation}
				S_{Sh} = -\sum_i p_i\log_2{p_i} \ge 0
			\end{equation}

			\noindent
			Von Neumann Entropie:
			\begin{equation}
				S_{vN} = -\mathrm{tr}(\hat{\rho}\log_2\hat{\rho})
			\end{equation}

			\noindent
			Äquivalenz der Entropien:
			\begin{equation}
				S = -k_B\,\mathrm{tr}(\hat{\rho} \ln \hat{\rho}) = S_B = S_G = k_B\ln(2) S_{Sh} = k_B\ln(2) S_{vN}
			\end{equation}

		\subsubsection{Dichtematrix}
			\noindent
			Definition der Dichtematrix / des Dichteoperators (Wobei $p_i$ klassische, d.h. Laplace'sche Wahrscheinlichkeiten für den Zustand $\Ket{\psi_i}$ sind):
			\begin{equation}
				\hat{\rho} = \sum_i p_i \Ket{\psi_i}\Bra{\psi_i}
			\end{equation}

			\noindent
			Axiome der Dichtematrix:
			\begin{itemize}\itemsep -0pt	% reduce space between items
				\item $\hat{\rho} = \hat{\rho}^\dagger$ \hfill{(hermitesch)}
				\item $\hat{\rho} \ge 0 $ \hfill{(positiv semidefinit)}
				\item $\mathrm{tr} \hat{\rho} = 1$ \hfill{(normiert)}
			\end{itemize}

			\noindent
			Erwartungswert einer Observablen $\mathcal{A}$:
			\begin{equation}
				\overline{A} = \sum_i p_i \Bra{\psi_i}\hat{A}\Ket{\psi_i} = \mathrm{tr} (\hat{\rho}\hat{A})
			\end{equation}

			\noindent
			Von-Neumann-Gleichung:
			\begin{equation}
				\tder{}{t}\hat{\rho}(t) = -\frac{i}{\hbar} \com{\hat{H}}{\hat{\rho}(t)}
			\end{equation}

			\noindent
			Liouville-Gleichung (Analogon der Von-Neumann-Gleichung für klassische Systeme):
			\begin{equation}
				\tder{\rho}{t} = \pder{\rho}{t}+\cBr{\rho,H}
			\end{equation}

		\subsubsection{Mikrokanonisches Ensemble}
			\noindent
			Axiome für die mikrokanonische Zustandssumme $\Gamma$:
			\begin{itemize}\itemsep -0pt	% reduce space between items
				\item $\pder{}{t}\Gamma(\vec{Z}) = 0$ \hfill{(stationär)}
				\item $\Gamma(\vec{Z}_1,\vec{Z}_2) = \Gamma(\vec{Z}_1)\Gamma(\vec{Z}_2) $ \hfill{(multiplikativ)}
				\item $\ln\Gamma(\vec{Z}) \propto N$ \hfill{(extensiv)}
			\end{itemize}
			Wobei der dritte Punkt für Thermodynamik jedoch nicht für statistische Mechanik im Allgemeinen notwendig wird.

			\noindent
			Berechnung von $\Gamma$ ($\Delta$ wird im thermodynamischen Grenzfall irrelevant. Das $\Gamma(\vec{Z})$ stationär bleibt ist durch die unitäre Zeitentwicklung im quantenmechanischen Fall bzw. durch den Satz von Liouville im klassischen Fall garantiert.):
			\begin{equation}
				\begin{aligned}
					\Gamma(\vec{Z}) = \langle \delta_\Delta (U-H) \rangle
						&= \begin{cases}
								\SumInt_n \delta_\Delta(U-E_n) & \text{quantenmechanisch} \\
								\int \diff \Gamma_N \delta_\Delta(U-E_n) & \text{klassisch}
							\end{cases} \\
					\delta_\Delta(x) &= \Theta(x) - \Theta(x-\Delta)\\
					\diff \Gamma_N &= \frac{\diff^{3N}q \diff^{3N}p }{h^{3N} N!} \\
					\hat{\rho} &= \frac{1}{\Gamma} \SumInt \Ket{\psi}\Bra{\psi}\,\diff\psi
				\end{aligned}
			\end{equation}

		\subsubsection{Kanonisches Ensemble}
			\noindent
			Kanonische Zustandssumme ($\beta = \frac{1}{k_B T}$):
			\begin{equation}
				Z_N=\langle e^{-\beta H} \rangle
					= \begin{cases}
							\SumInt_n e^{-\beta E_n} & \text{quantenmechanisch} \\
							\int \diff \Gamma_N e^{-\beta H} & \text{klassisch}
						\end{cases} \\
			\end{equation}

			\noindent
			Zusammenhang zur Freien Energie $F$:
			\begin{equation}
				F(T, V, N) = -k_B T \ln{Z_N(T)}
			\end{equation}

			% Bose Einstein Verteilung (Mittlere Besetzungszahl bei harmonischen Oszillatoren):
			% \begin{equation}
			%		 \overline{n}_\nu = \frac{1}{e^{\hbar \omega_\nu / k_B T} - 1}
			% \end{equation}

			\noindent
			Innere Energie im kanonischen Ensemble:
			\begin{equation}
				U = \overline{H} = \frac{1}{Z_N}\langle He^{-\beta H}\rangle = -\frac{1}{Z_N}\pder{Z_N}{\beta} = -\pder{\ln Z_N}{\beta}
			\end{equation}

		\subsubsection{Großkanonisches Ensemble}
			\noindent
			Großkanonische Zustandssumme (Die Spur im quantenmechanischen Fall wird nun über den Fock-Raum gebildet):
			\begin{equation}
				\mathcal{Z} = \sum_N z^N Z_N(\beta) = \langle e^{-\beta(H-\mu N)} \rangle
			\end{equation}

			\noindent
			Zusammenhang zum großen Potential $\Omega$:
			\begin{equation}
				\Omega(\beta,z) = -k_B T \ln \mathcal{Z}(\beta,z)
			\end{equation}


		\subsubsection{Gibbssches Variationsprinzip}
			\noindent
			Wahrscheinlichkeiten $p_i$ sind so verteilt, dass die Entropie im thermodynamischen Gleichgewicht unter den gegebenen Nebenbedingungen $g_k(p_i) = 0$ maximiert wird.
			\begin{equation}
				0 = \diff \big( S - \sum_k \lambda_k g_k(p_i) \big)
			\end{equation}

			\noindent
			Für mikrokanonisches Ensemble folgt die Gleichverteilung (Nebenbedingung $\sum_i p_i = 1$):
			\begin{equation}
				\begin{aligned}
					0 &= \diff \big( S - \lambda \sum_i p_i \big) \\
					p_i &= e^{-1-\lambda} = \konst \\
					\hat{\rho} &= \frac{\hat{1}}{\Gamma} \\ %\frac{\delta_\Delta(U-\hat{H})}{\Gamma} \\
				\end{aligned}
			\end{equation}

			\noindent
			Für kanonisches Ensemble folgt die Boltzmannverteilung (Nebenbedingungen $\sum_i p_i = 1$; $\sum_i p_i H_i = \overline{H} = U$):
			\begin{equation}
				\begin{aligned}
					0 &= \diff \big( S - \lambda \sum_i p_i - \beta \sum_i p_i H_i \big) \\
					p_i &= e^{-1-\lambda-\beta H_i} \propto e^{-\beta H_i} \\
					\hat{\rho} &= \frac{e^{-\beta \hat{H}}}{Z_N} \\
				\end{aligned}
			\end{equation}

			\noindent
			Großkanonisches Ensemble (Nebenbedingungen $\sum_i p_i = 1$; $\sum_i p_i H_i = U$; $\sum_i p_i N_i = \overline{N} = N$, Für den Teilchenzahloperator $\hat{N}$ muss die Spur über den Fockraum gebildet werden):
			\begin{equation}
				\begin{aligned}
					0 &= \diff \big( S - \lambda \sum_i p_i - \beta \sum_i p_i H_i - \beta\mu \sum_i p_i N_i \big) \\
					p_i &= e^{-1-\lambda-\beta H_i-\beta\mu N} \propto e^{-\beta H_i - \beta\mu N} \\
					\hat{\rho} &= \frac{e^{-\beta (\hat{H} - \mu \hat{N})}}{\mathcal{Z}} \\
				\end{aligned}
			\end{equation}

			\noindent
			Äquivalenz der Ensembles (Laplace Transformation zwischen Ensembles entspricht Legendre Transformation zwischen zugehörigen Thermodynamischen Potentialen):
			\begin{equation}
				\int \frac{\Gamma(U)}{\Delta} e^{-\beta U} \diff U = Z_N(\beta)
			\end{equation}


	\subsection{Thermodynamik}
		\subsubsection{Definitionen}
			Im \emph{Thermodynamisches Gleichgewicht} sind makroskopische Eigenschaften eines Systems zeitlich konstant. Thermodynamische Gleichgewichte verdanken ihre Stabilität gemäß dem \emph{Prinzip von le Chatelier} einer rückstellenden Kraft, die sich bei Fluktuationen ausbildet. \vsp

			Die \emph{Innere Energie} $U$ ist die Energie eines Vielteilchesystems, welches durch ein äußeres Potential auf ein Volumen $V$ begrenzt wird und damit einen verschwindenden makroskopischen Gesamtimpuls $\vec{P}=0$ und Gesamtdrehimpuls $\vec{L} = 0$ besitzt, sodass die gesamte kinetische Energie in internen Eigenschaften liegt. \vsp

			Definition des unvollständigen Differentials ($\delta$) der \emph{Arbeit} über kontrollierte \emph{Arbeitskoordinaten} $Z_\alpha$ und \emph{Gleichgewichtsgrößen} $g_\alpha$. Die Wahl der Arbeitskoordinaten (d.h. des Zugriffs über das System) legt die Arbeit als nutzbare Energieform fest. Und legt Eigenschaften wie Entropie und Temperatur fest, die nicht-Eindeutigkeit dieser Wahl wird als \emph{relative Objektivität} bezeichnet.
			\begin{equation}
				\delta W = \sum_\alpha g_\alpha \diff Z_\alpha
			\end{equation} \vsp

			Die \emph{Zustandsgrößen} $\vec{Z}=\left(U, Z_1,... \right)$ sind die makroskopisch beobachtbaren bzw. zugreifbaren Größen und legen einen Gleichgewichtszustand eindeutig fest. \vsp

			\emph{Intensive Variablen}: Homogen vom Grad $0$ ($T$, $p$, ...). \vsp

			\emph{Extensive Variablen}: Homogen vom Grad $1$ ($S$, $U$, $V$, ...). Diese sind insbesondere Additiv. \vsp

			\begin{table}[h]
				\begin{center}
				\begin{tabular}{ l | l l l l l l }
					$g_\alpha$ & $-p$ & $\mu$ & $\sigma$ & $E_0$ & $B_0$ & $\Phi$ \\ \hline
					$\diff Z_\alpha$ & $\diff V$ & $\diff N$ & $\diff A$ & $\diff \mathcal{P}$ & $\diff \mathcal{M}$ & $\diff Q$ \\
					\end{tabular}
				\caption{Arbeitskoordinaten und zugehörige Gleichgewichtsgrößen. $\sigma$: Oberflächenspannung, $A$: Oberfläche, $B_0$: magnetische Feld, $\mathcal{M}$: Gesamtmagnetisierung, $\mu$ chemisches Potential (Arbeit, die geleistet werden muss, um dem System ein Teilchen hinzuzufügen), $N$: Teilchenzahl...}
				\label{tab:ArbeitskoordinatenUndGleichgewichtsgroessen}
				\end{center}
			\end{table} \vsp

			\noindent
			\emph{Thermodynamischer Grenzfall}: \newline Deterministisches Verhalten bei $U,V,N\rightarrow\infty$ unter $U/N,V/N=\konst$. \vsp


			\noindent
			\emph{Thermische Wellenlänge}
			\begin{equation}
				\lambda_T = \frac{h}{\sqrt{2\pi m k_B T}}
			\end{equation} \vsp

			\noindent
			\emph{Fugazität}
			\begin{equation}
				z = e^{\beta\mu}
			\end{equation} \vsp

		\subsubsection{Hauptsätze}
			\textbf{Nullter Hauptsatz}: Existenz und Transitivität des Gleichgewichtzustandes\newline
				\indent \emph{Das Thermodynamische Gleichgewicht ist eine Äquivalenzrelation unter thermodynamischen Systemen (Gleichgewicht zwischen $A \sim B$ und $A \sim C$ impliziert $ B \sim C$).} \nl
			\textbf{Erster Hauptsatz}: Energiesatz
				\begin{equation}
					\label{eq:FirstLawOfThermodynamics}
					\diff U = \delta Q + \delta W
				\end{equation}
				\indent Innere Energie $\diff U$, zugeführte Wärmemenge $\delta Q$, Arbeit am System $\delta W$. \nl
			\textbf{Zweiter Hauptsatz}: Entropiesatz
				\begin{equation}
					\label{eq:SecondLawOfThermodynamics}
					\Delta S	\ge 0
				\end{equation}
				\indent Kelvin: \emph{Es ist unmöglich, durch bloße Abkühlung eines einzelnen Wärmebades zyklisch Arbeit zu erzeugen. }\nl
			\textbf{Dritter Hauptsatz}: Nernstscher Wärmesatz\newline
				\indent Nernst: \emph{Für jedes System strebt die Entropie für $T \rightarrow 0$ gegen einen von den Arbeitskoordinaten $Z_\alpha$ unabhängigen endlichen Wert $S_0:=0$.}

		\subsubsection{Gibbssche Fundamentalgleichung}
			\noindent
			Gibbssche Fundamentalgleichung / Gibbssche Fundamentalform (folgt aus 1. und 2. Hauptsatz, Gl. \ref{eq:FirstLawOfThermodynamics} und Gl. \ref{eq:SecondLawOfThermodynamics})
			\begin{equation}
				\label{eq:GibbsFundamental}
				\diff S = \frac{1}{T} \diff U - \frac{1}{T} \sum_{\alpha} g_\alpha \diff Z_\alpha
			\end{equation}

			\noindent
			Eulersche Homogenitätsrelation (folgt aus Homogenität von $S(\vec{Z})$):
			\begin{equation}
				\label{eq:EulerHomogenity}
				S(\vec{Z}) = U\pder{S}{U} + \sum_\alpha Z_\alpha\pder{S}{Z_\alpha}
				= \frac{1}{T} \left( U-\sum_\alpha g_\alpha Z_\alpha \right)
			\end{equation}

			\noindent
			Gibbs-Duhen Beziehung (folgt aus Gl. \ref{eq:GibbsFundamental} und Gl. \ref{eq:EulerHomogenity}):
			\begin{equation}
				0 = S \diff T + \sum_\alpha Z_\alpha \diff g_\alpha
			\end{equation}

			\noindent
			Zusammenhang zwischen thermischer und kalorischer Zustandsgleichung:
			\begin{equation}
				\left(\pder{U}{V}\right)_T = T\left(\pder{p}{T}\right)_V - p
			\end{equation}

			\noindent
			$T \diff S$-Gleichungen
			\begin{equation}
				\begin{aligned}
					T \diff S &= C_V \diff T + \frac{\alpha T}{\kappa_T} \diff V \\
					T \diff S &= C_p \diff T - \alpha T V \diff p
				\end{aligned}
			\end{equation}

		\subsubsection{Entropie}
			\noindent
			Thermodynamische Definition (Zusätzlich $\lim_{T\rightarrow 0} S := 0$):
			\begin{equation}
				\diff S := \frac{\delta Q_{rev}}{T}
			\end{equation}

			\noindent
			Eigenschaften der Entropie (Folgen des zweiten Hauptsatzes):
			\begin{itemize}\itemsep -0pt	% reduce space between items
				\item $\Delta S = S(t>t_0)-S(t_0) \ge 0$ \hfill{(Entropiezunahme in isolierten Systemen)}
				\item $S(\vec{Z}_1 + \vec{Z}_2) \ge S(\vec{Z}_1) + S(\vec{Z}_2)$ \hfill{(Superadditivität)}
				\item $\partial^2 S \le 0,\;\diff S = 0$ \hfill{(Maximum im Gleichgewicht)}
			\end{itemize}

		\subsubsection{Kreisprozesse}
			\noindent
			Idealisierte Prozesse:
			\begin{itemize}
				\item Isotherme: $\diff T = 0$
				\item Isochore: $\diff V = 0$
				\item Isobare: $\diff p = 0$
				\item Adiabate: $\delta Q = 0$
				\item Isentrope: $\diff S = 0$ (adiabatischer Prozess, der zusätzlich reversibel ist)
			\end{itemize}

			\noindent
			Kreisprozess (Gleichheit für reversible, Ungleichheit für irreversible Prozesse):
			\begin{equation}
				\oint \frac{\delta Q_{rev}}{T} \le 0
			\end{equation}

			\noindent
			Carnot Prozess: Einzig möglicher reversibler Kreisprozess, er besteht aus zwei Adiabaten und zwei Isothermen, zwischen den Temperaturen $T_1$, $T_2$ mit $T_1 < T_2$. Jede reale Wärmekraftmaschine hat einen Wirkungsgrad, der kleiner ist als der Carnot Wirkungsgrad:
			\begin{equation}
				\eta_C = 1-\frac{T_1}{T_2}
			\end{equation}

		\subsubsection{Antwortgrößen}
			\noindent
			Definition der Antwortgrößen (Wärmekapazität bei konstantem Volumen $C_V$ und bei konstantem Druck $C_p$, sowie spezifische Wärmekapazitäten $c_V$ und $c_p$, Ausdehnungskoeffizient $\alpha$, Spannungskoeffizient $\beta$, isothermische Kompressibilität $\kappa_T$, adiabatische Kompressibilität $\kappa_S$):
			\begin{equation}
				\begin{aligned}
					N c_V = C_V &= \left( \pder{Q}{T} \right)_V = T \left( \pder{S}{T} \right)_V \\
					N c_p = C_p &= \left( \pder{Q}{T} \right)_p = T \left( \pder{S}{T} \right)_p \\
					\alpha &= \frac{1}{V} \left( \pder{V}{T} \right)_p \\
					\beta &= \frac{1}{p} \left( \pder{p}{T} \right)_V \\
					\kappa_T &= -\frac{1}{V} \left( \pder{V}{p} \right)_T \\
					\kappa_S &= -\frac{1}{V} \left( \pder{V}{p} \right)_S \\
				\end{aligned}
			\end{equation}

			\noindent
			Beziehungen (Es folgt also insbesondere $c_p > c_V$ und
			$\kappa_T > \kappa_S$):
			\begin{equation}
				\begin{aligned}
					\beta &= \frac{\alpha}{p \kappa_T} \\
					C_p - C_V &= \frac{TV\alpha^2}{\kappa_T} \\
					\frac{c_p}{c_V} &= \frac{\kappa_T}{\kappa_S} \\
				\end{aligned}
			\end{equation}

			\noindent
			Joule-Thomsom-Koeffizient:
			\begin{equation}
				\mu_{JT} = \left( \pder{T}{p} \right)_H = \frac{V}{C_p}\left(\alpha T - 1\right)
			\end{equation}

			\noindent
			Für Magnetismus ist das Analogon zur Kompressibilität die magnetische Suszeptibilität $\chi_{T/S}$:
			\begin{equation}
				\chi_{S/T} = \left(\pder{\mathcal{M}}{B_0}\right)_{S/T}
			\end{equation}

		\subsubsection{Thermodynamische Potentiale}
			\noindent
			Guggenheim Quadrat:
			Unheimlich Viele Forscher Trinken Gerne pils Hinterm Schreibtisch.

			\begin{table}[h]
				\begin{center}
				\makebox[1\textwidth][c]{
				\begin{tabular}{ r | l | l | l }
					Name & Relation & Differential & Maxwellbeziehung \\ \hline \xrowht{26pt}
					Entropie & $S(U,V)$ & $\diff S = \frac{1}{T} \diff U + \frac{p}{T} \diff V$ & $\left(\dpder{}{V} \dfrac{1}{T}\right)_U = \left(\dpder{}{U}\dfrac{p}{T}\right)_V$ \\ \hline \xrowht{26pt}
					Innere Energie & $U(S,V)$ & $\diff U = T \diff S - p \diff V$ & $\left( \dpder{T}{V} \right)_S =	- \left( \dpder{p}{S} \right)_V$ \\ \hline \xrowht{26pt}
					Freie Energie & $F(V,T) = U - TS$ & $\diff F = -S \diff T - p \diff V$ & \phantom{-} $\left(\dpder{S}{V}\right)_T = \left(\dpder{p}{T}\right)_V$ \\ \hline \xrowht{26pt}
					Gibbs-Energie & $G(T,p) = U - TS + pV$ & $\diff G = -S \diff T - V \diff p$ & $-\left(\dpder{S}{p}\right)_T = \left(\dpder{V}{T}\right)_p$ \\ \hline \xrowht{26pt}
					Enthalpie & $H(p,S) = U + pV$ & $\diff H = T \diff S + V \diff p$ & \phantom{-} $\left(\dpder{V}{S}\right)_p = \left(\dpder{T}{p}\right)_S$ \\ \hline \xrowht{26pt}
					Großes Potential & $\Omega(T,V,\mu) = U - T S - \mu N$ & $\diff \Omega = -S \diff T - p \diff V -N \diff \mu$ &	\\ \hline
					\end{tabular}}
				\caption{Liste nützlicher thermodynamischer Potentiale}
				\label{tab:ThermodynamischePotentiale}
				\end{center}
			\end{table}

			\noindent
			Für homogene Systeme:
			\begin{equation}
				\begin{aligned}
					G(T,p) &= \phantom{-}N\mu(T,p) \\
					\Omega(T,V,\mu) &= -V p(T,\mu) \\
				\end{aligned}
			\end{equation}

		\subsubsection{Chemische Gleichgewichte}
			\noindent
			$s$ verschiedene chemische Reaktion von $r$ verschiedenen Stoffen mit chemischen Symbolen $S_i$, charaktierisiert durch die stöchiometrischen Koeffizienten $\nu_i^k\in\mathbb{Z}$, $\forall k\in\lbrace1,...,s\rbrace$:
			\begin{equation}
					\sum_{i=1}^r \nu_i^k S_i = 0
			\end{equation}

			\noindent
			Chemische Umwandlung mit beliebiger Umsatzvariable $\diff \lambda_k$:
			\begin{equation}
				\diff N_i = \sum_{k=1}^s \nu_i^k \diff \lambda_k
			\end{equation}

			\noindent
			Im thermodynamischen Gleichgewicht wird die Gibbs-Energie minimal:
			\begin{equation}
					0 = \diff G = \sum_{i=1}^r \mu_i \diff N_i \;\Rightarrow\; \sum_{i=1}^r \nu_i^k \mu_i = 0 \;\forall k\in\lbrace1,...,s\rbrace
			\end{equation}

			\noindent
			Massenwirkungsgesetz (mit Teilchenkonzentrationen $c_i = N_i/N$ und Massenwirkungskonstante $K$):
			\begin{equation}
				\prod_{i=1}^r	c_i^{\nu_i} = \exp\left( -\frac{1}{k_B T}\sum_{i=1}^r \nu_i\mu_i^0(T,p) \right) =: K(T,p)
			\end{equation}

		\subsubsection{Phasenübergänge}
			\noindent
			Clausius-Clapeyron-Beziehung:
			\begin{equation}
				\tder{p}{T} = \frac{\Delta S}{\Delta V} = \frac{S_g - S_{fl}}{V_g - V_{fl}}
			\end{equation}

			\noindent
			Latente Wärme pro Teilchen:
			\begin{equation}
				l = \frac{\Delta H}{N} = T\frac{\Delta S}{N}
			\end{equation}

			\noindent
			Gibbssche Phasenregel (für die Freiheitsgrade $f$ bei $r$ Komponenten verteilt in $\nu$ Phasen):
			\begin{equation}
				f = 2 + r - \nu
			\end{equation}

		\subsubsection{Nichtgleichgewicht}
			\noindent
			Annahme: Hinreichend schnelle Prozesse erzeugen zu jedem Punkt eine \emph{lokales Gleichgewicht}, sodass lokale Gleichgewichtsgrößen $T(t,\vec{r})$, $p(t,\vec{r})$,... existieren. \\
			Kontinuitätsgleichung für die Energiedichte $u(t,\vec{r})$ und die Wärmestromdichte $\vec{j}^q$:
			\begin{equation}
				\pder{u}{t} + \Nabla\cdot\vec{j}^q = 0
			\end{equation}

			\noindent
			Das \emph{Fourier Gesetz} bringt die Wärmestromdichte über die Wärmeleitfähigkeit $\kappa >0$ mit dem Temperaturgradienten in Verbindung:
			\begin{equation}
				\vec{j}^q = -\kappa \Nabla T
			\end{equation}

			\noindent
			Für $\diff u = nc_V \diff T$ (also z.B. in guter Näherung in Festkörpern) gilt die \emph{Wärmeleitungsgleichung} (Mit der Diffusionskonstanten $D=\kappa/nc_V$):
			\begin{equation}
				\pder{T}{t} = D\Nabla^2 T
			\end{equation}

			\noindent
			Der Wärmetransport ist irreversibel, für die Entropiedichte $s(t,\vec{r})$ und die Entropiestromdichte \linebreak $\vec{j}^s (t,\vec{r}) = \vec{j}^q/T$ gilt (Mit der \emph{Entropieerzeugungsrate} $\dot{s}$):
			\begin{equation}
				\dot{s} = \pder{s}{t} + \Nabla\cdot\vec{j}^s = - \frac{\vec{j}^q}{T^2}\cdot\Nabla T \ge 0
			\end{equation}

			\noindent
			Onsager Reziprozitätsbeziehung: Aus der mikroskopischen Reversibilität folgen Dinge für die Entropieerzeugungsrate.\vsp

			\noindent
			Thermoelektrische Effekte: In Systemen mit Ladungsträgern ist es sinnvoll ein elektrochemisches Potential $\mu_{EC}$ einzuführen, da eine Veränderung der Teilchenzahl auch eine Veränderung der Ladung im Potential impliziert, sodass (mit dem \emph{Peltier-Koeffizienten} $\Pi$, der \emph{Thermokraft} $\epsilon$, dem elektrischen Strom $\vec{j} = q\vec{j}^n$ und der spezifischen Leitfähigkeit $\sigma$, die letzte Gleichung folgt aus der Reziprozitätsbeziehung):
			\begin{equation}
				\begin{aligned}
					\delta W &= \mu \diff N + \Phi \diff Q = (\mu+q\Phi)\diff N = \mu_{EC} \diff N \\
					\Nabla \mu_{EC} &= -q\vec{\mathcal{E}} \\
					\vec{j}^q &= - \kappa \Nabla T + \Pi \vec{j} \\
					\vec{\mathcal{E}} &= \vec{E} - \frac{1}{q} \Nabla \mu = \frac{1}{\sigma} \vec{j} + \epsilon \Nabla T \\
					\Pi &= \epsilon T \\
				\end{aligned}
			\end{equation}


	\subsection{Modelle}
		\subsubsection{Klassisches ideales Gas}
			\noindent
			Ein \emph{ideales Gas} ist ein System von $N$ identischen Teilchen im Volumen $V$, die nicht untereinander wechselwirken. \vsp

			\noindent
			Thermische Zustandsgleichung $p(T, N, V)$:
			\begin{equation}
				pV = N k_B T
			\end{equation}

			\noindent
			Kalorische Zustandsgleichung $U(T, N, V)$:
			\begin{equation}
				U = \frac{3}{2} N k_B T
			\end{equation}

			\noindent
			Entropie:
			\begin{equation}
				S(T,p) = N k_B \ln{\left[ \frac{V}{N}\left( \frac{4\pi m	U}{3 h^2 N} \right)^{\frac{3}{2}} \right]} + \frac{5}{2} N k_B
			\end{equation}

			\noindent
			Adiabatengleichung \newline(gilt bei konstanten Wärmekapazitäten mit Adiabatenkoeffizient $\gamma = \frac{c_p}{c_V} = \frac{c_V + k_B}{c_V}$)
			\begin{equation}
				p V^\gamma = \konst
			\end{equation}

		\subsubsection{Quantenmechanisches ideales Gas}
			\noindent
			Modell: Ununterscheidbare, wechselwirkungsfreie Teilchen (Fermionen oder Bosonen) mit Spinentartung. \vsp

			\noindent
			Großkanonische Zustandssumme (Mit $\varepsilon_\alpha = \frac{\hbar^2 \vec{k}^2}{2m}$ der Energie im Zustand $\alpha=(\vec{k},\sigma)$):
			\begin{equation}
				\mathcal{Z} = \prod_{\alpha}
					\begin{cases}
						\left( 1-e^{\beta(\mu-\varepsilon_\alpha)}\right)^{-1} & \text{(Bosonen)} \\
						\phantom{\big(} 1+e^{\beta(\mu-\varepsilon_\alpha)} & \text{(Fermionen)}
					\end{cases}
			\end{equation}

			\noindent
			Mittlere Besetzungszahl (Bose-Einstein Statistik, Fermi-Dirac Statistik):
			\begin{equation}
				\overline{n_\alpha} =
					\begin{cases}
						\dfrac{1}{e^{\beta(\varepsilon_\alpha-\mu)} - 1} & \text{(Bosonen)} \\
						\dfrac{1}{e^{\beta(\varepsilon_\alpha-\mu)} + 1} & \text{(Fermionen)}
					\end{cases} \\
			\end{equation}

			\noindent
			Entropie:
			\begin{equation}
				S= -k_B\sum_\alpha
					\begin{cases}
						\overline{n_\alpha} \ln \overline{n_\alpha} - (1+\overline{n_\alpha}) \ln (1+\overline{n_\alpha}) & \text{(Bosonen)} \\
						\overline{n_\alpha} \ln \overline{n_\alpha} + (1-\overline{n_\alpha}) \ln (1-\overline{n_\alpha}) & \text{(Fermionen)}
					\end{cases} \\
			\end{equation}

			\noindent
			Innere Energie, Teilchenzahl:
			\begin{equation}
				\begin{aligned}
					U &= \sum_\alpha \overline{n_\alpha} \varepsilon_\alpha \\
					N &= \sum_\alpha \overline{n_\alpha}
				\end{aligned}
			\end{equation}

			\noindent
			Kalorische Zustandsgleichung:
			\begin{equation}
				U=\frac{3}{2}pV
			\end{equation}

			\noindent
			Thermische Zustandsgleichung:
			\begin{equation}
				\frac{pV}{N k_B T} = \frac{\sum_{l=1}^{\infty}\frac{z^l}{l^{5/2}}}{\tder{}{\ln z}\sum_{l=1}^{\infty}\frac{z^l}{l^{5/2}}}
			\end{equation}

		\subsubsection{Van-der-Waals-Gas}
			\noindent
			Zustandsgleichung des Van-der-Waals-Gas (Mit Kovolumen $b \ge 0$, Kohäsionsdruckparameter $a \ge 0$):
			\begin{equation}
				\left( p+\frac{a}{v^2} \right) \left( v-b \right) = k_B T
			\end{equation}

		\subsubsection{Phononengas}
			\noindent
			Hamiltonfunktion eines Festkörpers mit $N$ Atomen:
			\begin{equation}
				H=\sum_i \frac{\pvec{p}_i^2}{2m} + \frac{1}{2}m\sum_{\stackrel{i,j}{\alpha,\beta}} D_{\alpha\beta}(\vec{r}_i - \vec{r}_j)u_{i,\alpha}u_{j,\beta}
			\end{equation}

			\noindent
			Die Hamiltonfunktion lässt sich separieren, sodass $3N$ Eigenmoden durch ein harmonisches Oszillatorpotential mit $\omega_\nu$ beschrieben werden. Für die Zustandssumme, mittlere Besetzungszahl $\overline{n}_\nu$ (Bose-Einstein-Verteilung) und innere Energie $U$ folgt:
			\begin{equation}
				\begin{aligned}
					Z_\nu &= \sum_{n_\nu=0}^\infty e^{-\beta E_{n_\nu}} \\
					Z &= \prod_\nu Z_\nu = \frac{1}{2\sinh{\left( \frac{\hbar\omega_\nu}{2 k_B T} \right)}} \\
					\overline{n}_\nu &= \frac{1}{e^{\hbar\omega_\nu/k_B T} - 1} \\
					U &= \sum_\nu \hbar\omega_\nu \left(\overline{n}_\nu + \frac{1}{2}\right)
				\end{aligned}
			\end{equation}


			\noindent
			Emission und Absorption nach Modell von Einstein:
			\begin{equation}
				\begin{aligned}
					\frac{\diff N_1}{\diff t} &= -B_{12}N_1 \frac{\diff w}{\diff \omega}(\omega_{12}) \\
					\frac{\diff N_2}{\diff t} &= -B_{21}N_2 \frac{\diff w}{\diff \omega}(\omega_{21}) \\
					\frac{\diff N_2}{\diff t} &= -A_{21}N_2\\
				\end{aligned}
			\end{equation}
			Wobei \\
			\indent $N_1$: Besetzungszahl des Grundzustandes,\\
			\indent $N_2$: Besetzungszahl des angeregten Zustands,\\
			\indent $B_{12}$: Einstein'scher B-Koeffizient für Absorption,\\
			\indent $B_{21} = B_{12}$: Einstein'scher B-Koeffizient für stimulierte Emission,\\
			\indent $A_{21}$: Einstein'scher A-Koeffizient für die spontane Emission,\\
			\indent $\tau_{R}=A_{21}^{-1}$: Strahlende Lebensdauer,\\
			\indent $\frac{\diff w}{\diff \omega}(\omega)$: spektrale Energiedichte.\\

		\subsubsection{Hohlraumstrahlung}
			\noindent
			Ähnlich dem Phononengas existiert in Planck's Beschreibung des schwarzen Strahlers ein Photonengas. \vsp

			\noindent
			Innere Energie (Mit Stefan-Boltzmann-Konstante $\sigma$):
			\begin{equation}
				U = \frac{4\sigma}{c} V T^4
			\end{equation}

			\noindent
			Zustandsgleichung des Photonengas:
			\begin{equation}
				p(T) = \frac{U}{3V}
			\end{equation}

			\noindent
			Planck'sches Strahlungsgesetz (Für die spektrale Energiedichte $\pder{u}{\omega}$):
			\begin{equation}
					\begin{aligned}
							\pder{u}{\omega}(\omega,T) &= \frac{\hbar \omega^3}{\pi^2 c^3} \frac{1}{e^{\hbar \omega / k_B T} - 1} \\
							\pder{u}{\nu} (\nu, T) &= \frac{8 \pi h \nu^3}{c^3} \frac{1}{e^{h \nu / k_B T} - 1} \\
					\end{aligned}
			\end{equation}

			\noindent
			Stefan-Boltzmann-Gesetz (Intensität / Strahlungsleistung pro Fläche $I$, Stefan-Boltzmann-Konstante $\sigma$ und Emissionskoeffizient $\epsilon$; $\epsilon=1$ für Schwarzen Strahler):
			\begin{equation}
				I = \epsilon\sigma T^4
			\end{equation}

			\noindent
			Wien'sches Verschiebungsgesetz (Für die Wellenlänge $\lambda_{\max}$ des Strahlungsmaximums):
			\begin{equation}
				\lambda_{\max} T = \mathrm{konst}.
			\end{equation}

		\subsubsection{Idealer Paramagnetismus}
			\noindent
			Zeemann-Energie im äußeren Feld $\vec{B}_0$:
			\begin{equation}
				H_B = -\sum_{i=1}^N \vec{m}_i \cdot \vec{B}_0
			\end{equation}

			\noindent
			Für Ising Spins ($s_i\in\lbrace +1, -1\rbrace$ mit $|\vec{m}_i| = \mu$) folgen Zustandssumme, Gibbs-Energie und Magnetisierung:
			\begin{equation}
				\begin{aligned}
					Z(T,B_0) &= 2^N \cosh^N (\beta\mu B_0) \\
					G(T,B_0) &= -k_B T \ln Z \\
					\mathcal{M}(T,B_0) &= -\left(\pder{G}{B_0}\right)_T = N\mu \tanh(\mu B_0 /k_B T)
				\end{aligned}
			\end{equation}

			\noindent
			Curie-Gesetz ($k_B T \gg \mu B_0$):
			\begin{equation}
				\mathcal{M} \propto \frac{1}{T}
			\end{equation}

		\subsubsection{Ising-Modell}
			\noindent
			Wechselwirkungsenergie (Häufig nur mit Wechselwirkung direkter Nachbarn $J_{ij} = J \delta_{i,j\pm 1}$):
			\begin{equation}
				H = -\frac{1}{2}\sum_{i,j} J_{ij} s_i s_j -\mu B_0 \sum_i s_i
			\end{equation}

			\noindent
			Molekularfeldnäherung: \newline Ersetzung von $s_i s_j \rightarrow s_i m$ wobei $m = \overline{s}$ das Molekularfeld ist. Vernachlässigung von Korrelationen $\overline{(s_i-m)(s_j-m)}\ll m^2$. \nl
			Molekularfeldnäherung und Selbstkonsistenzgleichung (in $D$ Dimensionen, $h:=\mu B_0$):
			\begin{equation}
				\begin{aligned}
					H_{MF} &= -(h + 2 D J m) \sum_i s_i + D N J m^2 \\
					m &= -\frac{1}{N}\pder{G}{h} = \tanh\left({\beta(h + 2 D J m)}\right)
				\end{aligned}
			\end{equation}
