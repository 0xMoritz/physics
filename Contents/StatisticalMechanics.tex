% !TEX root = ../physics.tex
\section{Statistical Mechanics}
	\subsection{General}
		\noindent
		Equipartition theorem\index{Äquipartitionstheorem} $(z_1,...,z_{2f}) = (q_1, p_1, ..., q_f, p_f)$:
		\begin{equation}
			\overline{ z_i \pdv{H}{z_j}} = \kB T \delta_{ij}
		\end{equation}

		\subsubsection{Definition of Entropy}
			\noindent
			Boltzmann Entropy\index{Boltzmann!Entropie} (number of microstates\index{Mikrokanonische Zustandssumme} $\Gamma$ and state vector $\vec{Z}=(U,V,N,...)$)
			\begin{equation}
				S_\text{B}(\vec{Z}) = \kB \ln\Gamma(\vec{Z})
			\end{equation}

			\noindent
			Gibbs Entropy\index{Gibbs!Entropie}:
			\begin{equation}
				S_\text{G} = -\kB \sum_i p_i \ln p_i
			\end{equation}

			\noindent
			Shannon Entropy\index{Shannon!Entropie}:
			\begin{equation}
				S_\text{Sh} = -\sum_i p_i\log_2{p_i} \ge 0
			\end{equation}

			\noindent
			Von Neumann Entropy\index{Von Neumann!Entropie}:
			\begin{equation}
				S_\text{vN} = -\tr(\hat{\rho}\log_2\hat{\rho})
			\end{equation}

			\noindent
			Equivalence of these entropies:
			\begin{equation}
				S = -\kB\,\tr(\hat{\rho} \ln \hat{\rho}) = S_\text{B} = S_\text{G} = \kB\ln(2) S_\text{Sh} = \kB\ln(2) S_\text{vN}
			\end{equation}

		\subsubsection{Density Matrix\index{Dichtematrix}}
			\noindent
			Definition of the density matrix / density operator (Where $p_i$ are classical, i.e. Laplacian probability / Bayesian probability interpretations\index{Bayes!Wahrscheinlichkeitsbegriff}\index{Laplace!Wahrscheinlichkeitsbegriff} for a state $\Ket{\psi_i}$):
			\begin{equation}
				\hat{\rho} = \sum_i p_i \Ket{\psi_i}\Bra{\psi_i}
			\end{equation}

			\noindent
			Axioms of the density operator:
			\begin{itemize}\itemsep -0pt	% reduce space between items
				\item $\hat{\rho} = \hat{\rho}^\dagger$ \hfill{(self-adjoint)}
				\item $\hat{\rho} \ge 0 $ \hfill{(positive semidefinite)}
				\item $\tr \hat{\rho} = 1$ \hfill{(normed)}
			\end{itemize}

			\noindent
			Expectation value of an observable $\mathcal{A}$:
			\begin{equation}
				\overline{A} = \sum_i p_i \Bra{\psi_i}\hat{A}\Ket{\psi_i} = \tr (\hat{\rho}\hat{A}) = \avg{\hat{\rho}\hat{A}}
			\end{equation}

			\noindent
			Von Neumann equation\index{Von Neumann!Gleichung} (for classical analogue see Eq.~\ref{eq:LiouvilleTheorem}):
			\begin{equation}
				\label{eq:VonNeumannEquation}
				\dv{t}\hat{\rho}(t) = \pdv{\rho}{t} + \frac{\i}{\hbar} \comm{\hat{H}}{\hat{\rho}(t)} = 0
			\end{equation}

	\subsection{Thermodynamic Ensembles}
		\subsubsection{Microcanonical Ensemble\index{Mikrokanonisches Ensemble}}
			\noindent
			Axioms for the microcanonical partition function $\Gamma$:
			\begin{itemize}\itemsep -0pt	% reduce space between items
				\item $\pdv{t}\Gamma(\vec{Z}) = 0$ \hfill{(stationary)}
				\item $\Gamma(\vec{Z}_1,\vec{Z}_2) = \Gamma(\vec{Z}_1)\Gamma(\vec{Z}_2) $ \hfill{(multiplicative)}
				\item $\ln\Gamma(\vec{Z}) \propto N$ \hfill{(extensive)}
			\end{itemize}
			Where the third condition is needed for thermodynamics but not statistical mechanics in general.

			\noindent
			Calculating $\Gamma$ ($\Delta$ is irrelevant in the thermodynamic limit. A stationary $\Gamma(\vec{Z})$ is guaranteed by the unitary time evolution for quantum systems and the Liouville theorem for classical systems respectively):
			\begin{equation}
				\begin{aligned}
					\Gamma(\vec{Z}) = \langle \delta_\Delta (U-H) \rangle
					&= \begin{cases}
						\SumInt_n \delta_\Delta(U-E_n) & \text{quantum mechanical} \\
						\int \dd \Gamma_N \, \delta_\Delta(U-E_n) & \text{classical}
					\end{cases} \\
					\delta_\Delta(x) &= \Theta(x) - \Theta(x-\Delta)\\
					\dd \Gamma_N &= \frac{\dd^{3N}q \dd^{3N}p }{h^{3N} N!} \\
					\hat{\rho} &= \frac{1}{\Gamma} \SumInt \Ket{\psi}\Bra{\psi}\dd{\psi}
				\end{aligned}
			\end{equation}

		\subsubsection{Canonical Ensemble}
			A canonical ensemble is the statistical ensemble that represents the possible states of a mechanical system in thermal equilibrium with a heat bath at a fixed temperature.

			Density matrix in the canonical ensemble:
			\begin{equation}
				\hat{\rho} = \frac{1}{Z_N} \ex^{-\beta \hat{H}} = \ex^{-\beta (F-\hat{H})}
			\end{equation}

			\noindent
			Canonical partition function ($\beta = \frac{1}{\kB T}$):
			\begin{equation}
				Z_N=\langle \ex^{-\beta H} \rangle
				= \begin{cases}
					\SumInt_n \ex^{-\beta E_n} & \text{quantum mechanical} \\
					\int \dd \Gamma_N \, \ex^{-\beta H} & \text{classical}
				\end{cases} \\
			\end{equation}

			\noindent
			Connection to the Helmholtz free energy\index{Helmholtz!Freie Energie} $F$:
			\begin{equation}
				F(T, V, N) = -\kB T \ln{Z_N(T)}
			\end{equation}

			\noindent
			Internal energy in the canonical ensemble:
			\begin{equation}
				U = \overline{H} = \frac{1}{Z_N}\langle H\ex^{-\beta H}\rangle = -\frac{1}{Z_N}\pdv{Z_N}{\beta} = -\pdv{\ln Z_N}{\beta}
			\end{equation}

		\subsubsection{Grand Canonical Ensemble}
			The grand canonical ensemble is the statistical ensemble that represents the possible states of a mechanical system in thermal and chemical equilibrium with a heat bath and particle reservoir.

			Density matrix for the grand canonical ensemble
			\begin{equation}
				\hat{\rho} = \frac{1}{\mathcal{Z}} \ex^{-\beta (\hat{H} - \mu \hat{N})}
				= \ex^{-\beta (\Omega + \hat{H} - \mu \hat{N})}
			\end{equation}

			\noindent
			Grand canonical partition function (The trace has to be calculated in Fock space\index{Fock!Raum} for the quantum mechanical case):
			\begin{equation}
				\mathcal{Z} = \sum_N z^N Z_N(\beta) = \langle \ex^{-\beta(H-\mu N)} \rangle
			\end{equation}

			\noindent
			Connection to the grand potential $\Omega$:
			\begin{equation}
				\Omega(\beta,z) = -\kB T \ln \mathcal{Z}(\beta,z)
			\end{equation}


		\subsection{Gibbs Variation Principle\index{Gibbs!Variationsprinzip}}
			\noindent
			In thermodynamic equilibrium, given the side conditions $g_k(p_i) = 0$, the probabilities $p_i$ are distributed such that the entropy is maximal.
			\begin{equation}
				0 = \dd \big( S - \sum_k \lambda_k g_k(p_i) \big)
			\end{equation}

			\noindent
			This leads to a uniform distribution for the microcanonical ensemble (side condition $\sum_i p_i = 1$):
			\begin{equation}
				\begin{aligned}
					0 &= \dd \big( S - \lambda \sum_i p_i \big) \\
					p_i &= \ex^{-1-\lambda} = \const \\
					\hat{\rho} &= \frac{\hat{1}}{\Gamma} \\
				\end{aligned}
			\end{equation}

			\noindent
			This leads to the Boltzmann distribution\index{Boltzmann!Verteilung} for the canonical ensemble (side condition $\sum_i p_i = 1$; $\sum_i p_i H_i = \overline{H} = U$):
			\begin{equation}
				\begin{aligned}
					0 &= \dd \big( S - \lambda \sum_i p_i - \beta \sum_i p_i H_i \big) \\
					p_i &= \ex^{-1-\lambda-\beta H_i} \propto \ex^{-\beta H_i} \\
					\hat{\rho} &= \frac{\ex^{-\beta \hat{H}}}{Z_N} \\
				\end{aligned}
			\end{equation}

			\noindent
			Grand canonical ensemble (side condition $\sum_i p_i = 1$; $\sum_i p_i H_i = U$; $\sum_i p_i N_i = \overline{N} = N$, For the particle count operator $\hat{N}$ the trace must be calculated in Fock space\index{Fock!Raum}):
			\begin{equation}
				\begin{aligned}
					0 &= \dd \big( S - \lambda \sum_i p_i - \beta \sum_i p_i H_i - \beta\mu \sum_i p_i N_i \big) \\
					p_i &= \ex^{-1-\lambda-\beta H_i-\beta\mu N} \propto \ex^{-\beta H_i - \beta\mu N} \\
					\hat{\rho} &= \frac{\ex^{-\beta (\hat{H} - \mu \hat{N})}}{\mathcal{Z}} \\
				\end{aligned}
			\end{equation}

			\noindent
			Equivalence of the ensembles (Laplace transformation\index{Laplace!Transformation} between ensembles is equivalent to a Legendre transformation\index{Legendre!Transformation} of the corresponding potentials):
			\begin{equation}
				\int \frac{\Gamma(U)}{\Delta} \ex^{-\beta U} \dd U = Z_N(\beta)
			\end{equation}


