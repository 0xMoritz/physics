% !TEX root = ../physics.tex
\section{Classical field theory\index{Klassische Feldtheorie}}
	\subsection{Lagrange Formalism}
		\subsubsection{Scalar Field Theory}
			\noindent 
			Action for a scalar field $\phi(x)$ (\ie $\phi(x)\rightarrow P(\phi(x))=\phi(x)$, where $P\in\mathcal{P}$ is a Poincaré Transform, $\mathcal{L}$ is the Lagrange density):
			\begin{equation}
				\mathcal{S}\qty[\phi(x)] = \int_{\mathbb{R}} L \;\dd t = \int_{\mathbb{R}^4} \mathcal{L}(\phi(x),\partial_\mu \phi(x)) \;\dd^4 x				
			\end{equation}
			
			\noindent
			Euler-Lagrange equation\index{Euler!-Lagrange Gleichung}\index{Lagrange!Gleichung}:
			\begin{equation}
				\delta\mathcal{S} = 0
				\;\Rightarrow\; \partial_\mu \qty(\pdv{\mathcal{L}}{\left(\partial_\mu \phi\right)}) - \pdv{\mathcal{L}}{\phi} = 0
			\end{equation}
			
			\noindent
			Classical Klein-Gordon Equation\index{Klein!-Gordon Gleichung} for a free Field $\mathcal{L}(\phi(x)) = \frac{1}{2} \partial_\mu\phi \partial^\mu\phi - \frac{1}{2}m^2\phi^2$:
			\begin{equation}
				\delta \mathcal{S} = 0 \quad \Rightarrow \quad \qty(\partial_\mu \partial^\mu + m^2)\phi = 0
			\end{equation}


		\subsubsection{Hamilton Formalism}
			\noindent
			Transformation to Hamilton function $H$ and Hamiltonian density $\mathcal{H}$:
			\begin{equation}
				H = \int_{\mathbb{R}} \mathcal{H}(\phi(x),\pi(x)) \;\dd^3 x = \int_{\mathbb{R}^3} \qty[\pi(x)\dot{\phi}(x) - \mathcal{L}] \;\dd x^3
			\end{equation}

			% \noindent
			% Action\index{Wirkung} of a vector? Field Theory:
			% \begin{equation}
			% 	\mathcal{S} = \int_{\mathbb{R}^4} \mathcal{L}(A^\nu(x),\partial_\mu A^\nu(x)) \;\dd^4 x
			% \end{equation}

			% \noindent
			% Euler-Lagrange equation\index{Euler!-Lagrange Gleichung}\index{Lagrange!Gleichung}:
			% \begin{equation}
			% 	\delta\mathcal{S} = 0
			% 	\;\Rightarrow\; \partial_\mu \qty(\pdv{\mathcal{L}}{\left(\partial_\mu A^\nu\right)}) - \pdv{\mathcal{L}}{A^\nu} = 0 \quad\forall \nu
			% \end{equation}
		\subsubsection{Noether's Theorem\index{Noether!Theorem}}
			\noindent
			Continuous transformation (with a continuous infinitesimal variation $\delta_\varepsilon \phi$, characterized by an infinitesimal variable $\varepsilon$):
			\begin{equation}
				\phi(x)\rightarrow\phi'(x) = \phi(x) + \delta_\varepsilon \phi(x)
			\end{equation}
			
			\noindent
			Condition of invariance\index{Invarianzbedingung}:
			\begin{equation}
				\delta\mathcal{L} = \partial_\mu F^\mu(\phi) 
				\quad \Rightarrow \quad \mathcal{S} \sim \mathcal{S}'
			\end{equation}

			\noindent
			Resulting conserved current/Noether current\index{Noether!-Strom} $j^{\mu}$ and conserved charge $Q$ (Where $\Delta \phi = \eval{\pdv{\delta_\varepsilon \phi}{\varepsilon}}_{\varepsilon=0}$ is the generator of that transformation):
			\begin{equation}
				\begin{aligned}
					\mathcal{S} \sim \mathcal{S}' \quad\Rightarrow\quad 
					\partial_\mu j^\mu &= 0;\quad
					j^\mu \qty( \phi(x), \partial_\nu\phi(x) ) = \pdv{\mathcal{L}}{\left(\partial_\mu \phi\right)}\Delta\phi - F^\mu(\phi) \\
					\dv{Q}{t} &= 0; \quad Q = \int_{\mathbb{R}^3} j^0\;\dd^3\vec{x} \\
				\end{aligned}
			\end{equation}
			Note: The conserved current is only defined up to divergence $j^\mu \sim j^\mu + K^\mu$ for $\partial_\mu K^\mu = 0$.