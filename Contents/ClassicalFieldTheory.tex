% !TEX root = ../physics.tex
\section{Klassische Feldtheorie}
	\subsection{Lagrange-Formalismus}
		\subsubsection{Allgemeines}
			\noindent
			Wirkung und Lagrangedichte $\mathcal{L}$:
			\begin{equation}
				\mathcal{S} = \int_{\mathbb{R}^4} \mathcal{L}(A^\nu(x),\partial_\mu A^\nu(x)) \;\diff^4 x
			\end{equation}

			\noindent
			Euler-Lagrange-Gleichung:
			\begin{equation}
				\delta\mathcal{S} = 0
				\;\Rightarrow\; \partial_\mu \left(\pder{\mathcal{L}}{\left(\partial_\mu A^\nu\right)}\right) - \pder{\mathcal{L}}{A^\nu} = 0 \;\;\forall \nu
			\end{equation}

		\subsubsection{Noether-Theorem}
			\noindent
			Kontinuierliche Transformation:
			\begin{equation}
				\phi(x)\rightarrow\phi'(x) = \phi(x) + \varepsilon \delta \phi(x)
			\end{equation}

			\noindent
			Invarianzbedingung:
			\begin{equation}
				\delta\mathcal{L} = \partial_\mu F^\mu(\phi)
			\end{equation}

			\noindent
			Resultierender erhaltener Strom und Erhaltene Ladung:
			\begin{equation}
				\begin{aligned}
					S \sim S' \;\Rightarrow\;\partial_\mu j^\mu = 0;&\; j^\mu \left( \phi(x), \partial_\nu\phi(x) \right) = \pder{\mathcal{L}}{\left(\partial_\mu \phi\right)}\delta\phi - F^\mu(\phi) \\
					\tder{Q}{t} = 0;&\; Q = \int_{\mathbb{R}^3} j^0\;\diff^3\vec{x}
				\end{aligned}
			\end{equation}
