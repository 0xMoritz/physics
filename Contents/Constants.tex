% !TEX root = ../physics.tex
\section{Constants}
	\href{https://en.wikipedia.org/wiki/List_of_physical_constants}{Wikipedia list of physical constants (en.wikipedia.org/wiki/List\_of\_physical\_constants)}\\
	\href{https://physics.nist.gov/cuu/Constants/}{CODATA List of physical constants (physics.nist.gov/cuu/Constants/)}
	\subsection{Fundamental Constants}
		\begin{center}
			\begin{tabular}{| L{.35\textwidth} L{.15\textwidth} L{.4\textwidth} |}
				\hline Name & Symbol & Value \\ \hline \hline
				\href{https://en.wikipedia.org/wiki/Speed_of_light}{Speed of light in a vacuum} & $c$ & $299\,792\,458\;\frac{\mathrm{m}}{\mathrm{s}}$ \exact \\ \hline
				\href{https://en.wikipedia.org/wiki/Planck_constant}{Planck's Constant}\index{Planck!Wirkungsquantum} & $h$ & $6.626\,070\,15\e{-34}\unit{J\,s}$ \exact \\ \hline
				\href{https://en.wikipedia.org/wiki/Gravitational_constant}{Gravitational Constant}\index{Newton!Gravitationskonstante} & $G$ & $6.674\,30(15)\e{-11}\;\frac{\mathrm{m^3}}{\mathrm{kg\,s^2}} $ \\ \hline
				\href{https://en.wikipedia.org/wiki/Elementary_charge}{Elementary charge}\index{Elementarladung} & $e$ & $1.602\,176\,634\e{-19}\unit{C}$ \exact \\ \hline
				\href{https://en.wikipedia.org/wiki/Vacuum_permittivity}{Vacuum permittivity} \index{Vakuum!Permittivität} & $\varepsilon_0$ & $8.854\,187\,812\,8(13)\e{-12}\;\frac{\mathrm{A\,s}}{\mathrm{V\,m}}$ \\ \hline
				\href{https://en.wikipedia.org/wiki/Vacuum_permeability}{Vacuum permeability} \index{Vakuum!Permeabilität} & $\mu_0$ & $1.256\,637\,062\,12(19)\e{-6}\;\frac{\mathrm{N}}{\mathrm{A^2}}$ \\ \hline
				\href{https://en.wikipedia.org/wiki/Boltzmann_constant}{Boltzmann constant}\index{Boltzmann!Konstante} & $\kB$ & $1.380\,649\e{-23}\;\frac{\mathrm{J}}{\mathrm{K}}$ \exact \\ \hline
				\href{https://en.wikipedia.org/wiki/Weinberg_angle}{Weinberg angle}\index{Weinberg!Winkel} & $\theta_W$ & $\sin^2(\theta_W) = 0.223\,05(23)$ \\ \hline
			\end{tabular}
		\end{center}

	\subsection{Particle Constants}
		\label{Sec:ParticleConstants}
		\begin{center}
			\begin{tabular}{| L{.35\textwidth} L{.15\textwidth} L{.4\textwidth} |}
				\hline Name & Symbol & $\phantom{-}$Value \\ \hline \hline
				\href{https://en.wikipedia.org/wiki/Electron_mass}{Electron mass} & $m_e$ & $\phantom{-}9.109\,383\,701\,5(28)\e{-31}\unit{kg} = 0.510\,998\,950\,00(15)\unit{MeV/c^2}$ \\ \hline
				\href{https://en.wikipedia.org/wiki/Proton}{Proton mass} & $m_p$ & $\phantom{-}1.672\,621\,923\,69(51)\e{-27}\unit{kg} = 938.272\,088\,16(29)\unit{MeV/c^2}$ \\ \hline
				\href{https://en.wikipedia.org/wiki/Neutron}{Neutron mass} & $m_n$ & $\phantom{-}1.674\,927\,498\, 04(95)\e{-27}\unit{kg} = 939.565\,420\,52(54)\unit{MeV/c^2}$ \\ \hline
				Landé-Factor of the Electron & $g_e$ & $\phantom{-}2.002\,319\,304\,362\,56(35)$ \\ \hline
				Landé-Factor of the Proton & $g_p$ & $\phantom{-}5.585\,694\,689\,3(1\,6)$ \\ \hline
				Landé-Factor of the Neutron & $g_n$ & $- 3.826\,085\,45(90)$ \\ \hline
				\href{https://en.wikipedia.org/wiki/W_and_Z_bosons#W_bosons}{$W$-Boson mass} & $m_W$ & $\phantom{-}80.377(12)\unit{GeV/c^2}$ \\ \hline
				\href{https://en.wikipedia.org/wiki/W_and_Z_bosons#W_bosons}{$Z$-Boson mass} & $m_Z$ & $\phantom{-}91.1876(21)\unit{GeV/c^2}$ \\ \hline
				\href{https://en.wikipedia.org/wiki/Higgs_boson}{Higgs-Boson mass}\index{Higgs!Boson} & $m_H$ & $\phantom{-}125.11(11)\unit{GeV/c^2}$ \\ \hline
			\end{tabular}
		\end{center}

		\href{https://pdg.lbl.gov/}{Particle Data group (pdg.lbl.gov)}



	\subsection{Fermion Quantum Numbers}
		\label{Sec:FermionConstants}
		$Y$ is the weak hypercharge, $T_3$ is the weak isospin, $Q$ is the electric charge:
		\begin{center}
			\begin{tabular}{| L{.45\textwidth} | L{.45\textwidth} |}
				\hline
				Left-handed quarks: $\doublet{u}{d}_L, \doublet{c}{s}_L, \doublet{t}{b}_L$

				$Y=\doublet{1/6}{1/6}$, $T_3=\doublet{+1/2}{-1/2}$, $Q=\doublet{+2/3}{-1/3}$
				&
				Right-handed quarks: $\doublet{u}{d}_R, \doublet{c}{s}_R, \doublet{t}{b}_R$

				$Y=\doublet{+2/3}{-1/3}$, $T_3=\doublet{0}{0}$, $Q=\doublet{+2/3}{-1/3}$
				\\
				\hline
				Left-handed leptons: $\doublet{\nu_e}{e}_L, \doublet{\nu_\mu}{\mu}_L, \doublet{\nu_\tau}{\tau}_L$

				$Y=\doublet{-1/2}{+1/2}$, $T_3=\doublet{+1/2}{-1/2}$, $Q=\doublet{0}{-1}$
				&
				Right-handed leptons: $\doublet{u}{d}_R, \doublet{c}{s}_R, \doublet{t}{b}_R$

				$Y=\doublet{0}{0}$, $T_3=\doublet{-1}{0}$, $Q=\doublet{0}{-1}$
				\\
				\hline
			\end{tabular}
		\end{center}

		\renewcommand{\arraystretch}{2.0}
	\subsection{Composite Constants}
		\begin{center}
			\begin{tabular}{| L{.35\textwidth} L{.19 \textwidth} L{.36\textwidth} |}
				\hline Name & Definition & Value \\ \hline \hline
				\href{https://en.wikipedia.org/wiki/Planck_constant#Reduced_Planck_constant}{Reduced Planck's Constant} & $\hbar:=\dfrac{h}{2\pi}$ & $1.054\,571\,817...\e{-34}\unit{J\,s}$ \\[3pt] \hline
				\href{https://en.wikipedia.org/wiki/Stefan%E2%80%93Boltzmann_constant}{Stefan-Boltzmann Constant}\index{Stefan!-Boltzmann Konstante} & $\sigma:=\dfrac{2\pi^5 \kB^4}{15h^3c^2}$ & $5.670\,374\,419...\e{-8}\;\frac{\mathrm{W}}{\mathrm{m^2\,K^4}}$ \\[3pt] \hline
				\href{https://en.wikipedia.org/wiki/Rydberg_constant}{Rydberg constant}\index{Rydberg!Konstante} & $R_\infty := \dfrac{m_e e^4}{8 c h^3 \varepsilon_0^2}$ & $1.097\,373\,156\,816\,0(2\,1)\e{7}\;\frac{1}{\mathrm{m}}$ \\[3pt] \hline
				\href{https://en.wikipedia.org/wiki/Rydberg_constant#Rydberg_unit_of_energy}{Rydberg-Energy}\index{Rydberg!Energie} & $R_y := \dfrac{m_e e^4}{8 h^2 \varepsilon_0^2}$ & $2.179\,872\,361\,103\,5(4\,2)\e{-18}\unit{J}$ \\[3pt] \hline
				\href{https://en.wikipedia.org/wiki/Bohr_radius}{Bohr radius}\index{Bohr!Radius} & $\rho := \dfrac{4\pi\hbar^2\varepsilon_0}{m_e e^2}$ & $5.291\,772\,109\,03(80) \e{-11}\unit{m}$ \\[3pt] \hline
				\href{https://en.wikipedia.org/wiki/Fine-structure_constant}{Fine structure constant}\index{Feinstrukturkonstante} & $\alpha := \dfrac{e^2}{4\pi\varepsilon_0\hbar c}$ & $7.297\,352\,537\,6(5\,0) \e{-3} \approx\dfrac{1}{137}$ \\[3pt] \hline
				Fermi coupling constant\index{Fermi!Kopplungskonstante} & $G_F := \dfrac{\sqrt{2} g^2}{8 m_W^2 c^4}$ & $1.166\,378\,7(6)\unit{GeV^{-2}}$ \\[3pt] \hline
				\href{https://en.wikipedia.org/wiki/Bohr_magneton}{Bohr magneton}\index{Bohr!Magneton} & $\mu_\text{B} := \dfrac{e \hbar}{2 m_e}$ & $9.274\,009\,994\,(57)\e{-24}\;\frac{\mathrm{J}}{\mathrm{T}}$ \\[3pt] \hline
				\href{https://en.wikipedia.org/wiki/Nuclear_magneton}{Nuclear magneton}\index{Kern Magneton} & $\mu_\text{N} := \dfrac{e \hbar}{2 m_p}$ & $5.050\,783\,746\,1(1\,5)\e{-27}\;\frac{\mathrm{J}}{\mathrm{T}}$ \\[3pt] \hline
				\href{https://en.wikipedia.org/wiki/Classical_electron_radius}{Classical electron radius}\index{Klassischer Elektronenradius} & $r_e := \dfrac{e^2}{4\pi\varepsilon_0 m_e c^2}$ & $2.817\,940\,322\,7(1\,9)\e{-15}\;\mathrm{m}$ \\[3pt] \hline
			\end{tabular}
		\end{center}
		\renewcommand{\arraystretch}{1.4}

	\subsection{Astronomical Constants}
		\label{Sec:AstronomicalConstants}
		\begin{center}
			\begin{tabular}{| L{.4\textwidth} L{.15\textwidth} L{.35\textwidth} |}
				\hline Name & Symbol & Value \\ \hline \hline
				\href{https://en.wikipedia.org/wiki/Solar_mass}{Solar mass} & $M_\odot$ & $1.988\,92(25)\e{30}\unit{kg}$ \\ \hline
				\href{https://en.wikipedia.org/wiki/Earth_mass}{Earth mass} & $M_\oplus$ & $5.972\,2(6) \e{24}\unit{kg}$ \\ \hline
				\href{https://en.wikipedia.org/wiki/Earth_radius}{Mean earth radius} & $R_\oplus$ & $6.3781 \e{6}\unit{m}$ \\ \hline
				\href{https://en.wikipedia.org/wiki/Solar_constant}{Solar constant}\index{Solarkonstante} & $E_0$ & $1361 \unit{\frac{W}{m^2}}$ \\ \hline
				\href{https://en.wikipedia.org/wiki/Hubble%27s_law}{Current Hubble's constant}\index{Hubble!Konstante} & $H_0$ & $2.33 \e{-18} \unit{\frac{1}{s}}$ \\ \hline
				\href{https://en.wikipedia.org/wiki/Hubble%27s_law}{Cosmological constant} & $\Lambda$ & $1.088(30)\e{-52} \unit{\frac{1}{m^2}}$ \\ \hline
				\href{https://en.wikipedia.org/wiki/Solar_luminosity}{Nominal solar luminosity} & $L_\odot$ & $3.828\e{26} \unit{W}$ \\ \hline
				\href{https://en.wikipedia.org/wiki/Cosmic_microwave_background#cite_note-apj707_2_916-6}{Current temperature of the CMB} & $T_\text{CMB}$ & $2.725\,48 (57) \unit{K}$ \\ \hline
			\end{tabular}
		\end{center}

		\noindent
		\href{https://en.wikipedia.org/wiki/Astronomical_constant}{List of astronomical constants (en.wikipedia.org/wiki/Astronomical\_constant)}

		\noindent
		\href{https://en.wikipedia.org/wiki/Standard_gravitational_parameter}{List of standard gravitational parameters (en.wikipedia.org/wiki/Standard\_gravitational\_parameter)}\index{Gravitationsparameter}

	\subsection{Units}
		\begin{center}
			\begin{tabular}{| L{.35\textwidth} L{.15\textwidth} L{.4\textwidth} |}
				\hline
				Name & Symbol & Definition \\ \hline \hline
				Year & $\mathrm{yr}$ & $31\,557\,600\,\unit{s}$ \\ \hline
				\href{https://en.wikipedia.org/wiki/Light-year}{Light-year}\index{Lichtjahr} & $\mathrm{ly}$ & $9\,460\,730\,472\,580\,800\unit{m}$ \\ \hline
				\href{https://en.wikipedia.org/wiki/Astronomical_unit}{Astronomical Unit} & $\mathrm{AU}$ & $149\,597\,870\,700\unit{m}$ \\ \hline
				\href{https://en.wikipedia.org/wiki/Parsec}{Parsec} & $\mathrm{pc}$ & $\frac{648000}{\pi}\,\mathrm{AU} = 3.085\,677\,581...\e{16}\unit{m}$ \\ \hline
				\href{https://en.wikipedia.org/wiki/Angstrom}{\r{A}ngström} & ${\mbox{\normalfont\AA}}$ & $10^{-10}\unit{m}$ \\ \hline
				\href{https://en.wikipedia.org/wiki/Barn_(unit)}{Barn} & $\mathrm{b}$ & $10^{-18}\unit{m^2}$ \\ \hline
				\href{https://en.wikipedia.org/wiki/Wavenumber#In_spectroscopy}{Kayser} & $\mathrm{K}$ & $10^2\unit{m^{-1}}$ \\ \hline
				\href{https://en.wikipedia.org/wiki/Dalton_(unit)}{Dalton / Atomic mass unit}\index{Atomare Masseneinheit} & $\mathrm{u}$, $\mathrm{Da}$ & $\frac{1}{12}m(\prescript{12}{6}{\mathbf{C}}) = 1.660\,539\,066\,60(50)\e{-27}\unit{kg}$ \\ \hline
				\href{https://en.wikipedia.org/wiki/Erg}{Erg} & $\mathrm{erg}$ & $10^{-7}\,\mathrm{J}$ \\ \hline
				\href{https://en.wikipedia.org/wiki/Calorie}{Calorie\index{Calorie!Einheit}} & $\mathrm{cal}$ & $4\,184\,\mathrm{J}$ \\ \hline
				\href{https://en.wikipedia.org/wiki/Bar_(unit)}{Bar} & $\mathrm{bar}$ & $10^5\unit{Pa}$ \\ \hline
				\href{https://en.wikipedia.org/wiki/Standard_atmosphere_(unit)}{Standard atmosphere} & $\mathrm{atm}$ & $101\,325\unit{Pa}$ \\ \hline
				\href{https://en.wikipedia.org/wiki/Celsius}{Degree Celsius} & $\mathrm{^\circ C}$ & $(x)\mathrm{^\circ C}=(x+273.15)\mathrm{K}$ \\ \hline
				\href{https://en.wikipedia.org/wiki/Gauss_(unit)}{Gauß}\index{Gauß!Einheit} & $\mathrm{Gs}$ & $10^{-4}\unit{T}$ \\ \hline
				\href{https://en.wikipedia.org/wiki/Mole_(unit)}{Mole} & $\mathrm{mol}$ & $6.022\,140\,76\e{23}$ \\ \hline
			\end{tabular}
		\end{center}

		\subsubsection{Logarithmic Units}
			\noindent
			Bel\index{Bel} and Decibel\index{Dezibel} ($L_P$ is the power ratio and $P_0$ is the reference power):
			\begin{equation}
				L_P = \log_{10}\qty(\frac{P}{P_0})\unit{B}
				=10\log_{10}\qty(\frac{P}{P_0})\unit{dB}
			\end{equation}

			\noindent
			Apparent Magnitude ($I_0$ is the reference intensity, originally Vega):
			\begin{equation}
				m = -2.5 \log_{10} \qty( \frac{I}{I_0} )
			\end{equation}

			\noindent
			Absolute Magnitude (Apparent magnitude at a distance of $10\unit{pc}$):
			\begin{equation}
				M = m - 5 \log_{10} \qty( \frac{I(10\unit{pc})}{I_0} )
			\end{equation}

			\noindent
			Distance modulus:
			\begin{equation}
				m - M = 5 \log_{10} \qty( \frac{d}{10\unit{pc}} )
			\end{equation}

		\subsubsection{Other Units}
			\href{https://en.wikipedia.org/wiki/Sievert}{Sievert\index{Sievert!Einheit} $\mathrm{Sv}$} \href{https://en.wikipedia.org/wiki/Equivalent_dose}{(Equivalent dose $H_T$)}:
			\begin{equation}
				\frac{H_T}{\mathrm{Sv}} = \frac{1}{\mathrm{J}}\sum_R W_R D_{T,R}
			\end{equation}
			where $D_{T,R}$ is the absorbed energy dose and $W_R$ is the weighting factor:
			\begin{equation}
				\begin{aligned}
					W_R =
					\begin{cases}
						\gamma,\beta,\mu: & 1 \\
						p, \pi^\pm: & 2 \\
						\alpha, \prescript{A}{Z}{\mathbf{X}}\quad\forall Z>1: & 20 \\
						n: &\begin{cases}
							E<1\unit{MeV}: & \qty(2.5+18.2 \,\ex^{\ln^2(E)/6}) \\
							1\unit{MeV} <E<50\unit{MeV}: & \qty(5.0+17.0 \,\ex^{\ln^2(2E)/6}) \\
							50\unit{MeV}<E: & \qty(2.5+3.25 \,\ex^{\ln^2(0.04E)/6}) \\
						\end{cases}
					\end{cases}\\
				\end{aligned}
			\end{equation}

	\subsection{Natural Units}
		Using $c=\hbar=\kB=1$, where $\qty{x}=x/\qty[x]$ is the numerical value of a quantity in SI units.
		\begin{center}
			\begin{tabular}{| L{.15\textwidth} L{.1\textwidth} L{.1\textwidth} L{.5\textwidth} |}
				\hline
				Dimension & Natural & SI & Conversion \\ \hline \hline
				Mass & $1\unit{eV}$ & $1\unit{eV} / c^2$ & $1\unit{eV}\doteq \qty{e}\qty{c}^{-2}\unit{kg}=1.782\,661\,921\e{-36}\unit{kg}$ \\ \hline
				Length & $1\unit{eV^{-1}}$ & $c \hbar / 1 \unit{eV}$ & $1\unit{eV^{-1}}\doteq \qty{c}\qty{\hbar}\qty{e}^{-1}\unit{m}=1.602\,176\,634\e{-7}\unit{m}$ \\ \hline
				Time & $1\unit{eV^{-1}}$ & $\hbar / 1 \unit{eV}$ & $1\unit{eV^{-1}}\doteq \qty{\hbar}\qty{e}^{-1}\unit{s}=6.582\,119\,569\e{-16}\unit{s}$ \\ \hline
				Temperature & $1\unit{eV}$ & $1\unit{eV} / \kB$ & $1\unit{eV}\doteq \qty{e}\qty{\kB}^{-1}\unit{K}=1.160\,451\,812\e{4}\unit{K}$ \\ \hline
			\end{tabular}
		\end{center}