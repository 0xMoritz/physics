% !TEX root = ../physics.tex
\section{Relativity}
	Conventions used in this section:
	\begin{itemize}
		\item Metric signature: $\eta_{\mu\nu} = \mathrm{diag}(-,+,+,+)_{\mu\nu}$
	\end{itemize}
	\subsection{Basics}
		\subsubsection{Symmetries of Flat Spacetime}
			\begin{itemize}
				\item Homogeneity of spacetime
				\item Isotropy of spacetime
				\item Lorentz invariance
			\end{itemize}

			\noindent
			Inertial frame of reference\index{Inertialsystem} \\
			\indent \textit{Frame of reference in which spacetime is homogeneous and isotropic.}

		\subsubsection{Einstein Postulates}
			\textbf{First Postulate}: Principle of Relativity\index{Relativitätsprinzip} \newline
			\indent \textit{The laws of nature are form invariant in all inertial frames of reference.}  \nl
			\textbf{Second Postulate}: Constancy of the Speed of Light\index{Konstanz der Lichtgeschwindigkeit} \newline
			\indent \textit{The maximum propagation speed of information is identical in all frames of reference.\index{Lichtgeschwindigkeit}}
			\label{post:c}\nl
			\textbf{Third Postulate}: Strong Equivalence Principle\index{Starkes Äquivalenzprinzip} \newline
			\indent \textit{Spacetime is locally flat, \ie the laws of special relativity hold in local inertial frames of reference. It is therefore always possible to construct a coordinate system with a locally Minkowskian metric for one point.}  \vsp

			\noindent
			Metric compatibility:
			\begin{equation}
				\nabla_\lambda g_{\mu\nu} = 0
			\end{equation}

			\noindent
			Torsion-free connection assumption / condition:
			\begin{equation}
				\Gamma^{\lambda}_{\qty[\mu\nu]} = 0
			\end{equation}

		\subsubsection{Metric Tensor\index{Metrischer Tensor}}
			Metric Tensor and Lorentz invariant spacetime interval:
			\begin{equation}
				\dd s^2 = c^2 \dd \tau^2 = \dd x^\mu \dd x_\mu = g_{\mu\nu} \dd x^\mu \dd x^\nu
			\end{equation}
			Signature used in this subsection: $(-1,+1,+1,+1)$.

	\subsection{General Relativity\index{Allgemeine Relativitätstheorie}}
		\subsubsection{Definitions}
			\noindent
			Partial derivatives:
			\begin{equation}
				\begin{aligned}
					\pdv{\phi}{x^\mu} &= \partial_\mu \phi = \phi_{,\mu} \\
					\partial_\mu &= \qty(\pdv{x^0}, \pdv{x^1}, \pdv{x^2}, \pdv{x^3}) \\
				\end{aligned}
			\end{equation}

			\noindent
			Christoffel symbols\index{Christoffel!Symbole} (local inertial frame with coordinates $\xi^\alpha$):
			\begin{equation}
				\Gamma_{\mu\nu}^{\kappa} := \frac{\partial x^\kappa}{\partial \xi^\alpha}\frac{\partial^2 \xi^\alpha}{\partial x^\mu\partial x^\nu}
				=\frac{1}{2}g^{\kappa\lambda}\qty(\partial_\mu g_{\nu\lambda} + \partial_\nu g_{\mu\lambda} - \partial_\lambda g_{\nu\mu})
			\end{equation}
			The Christoffel symbols are symmetric in the lower indices $\Gamma_{\mu\nu}^{\kappa} = \Gamma_{\nu\mu}^{\kappa}$. In local frames of reference $\partial^\mu g_{\mu\nu} = 0$ and the Christoffel symbols vanish.
			\begin{equation}
				\Gamma^\mu_{\mu\nu} = \frac{1}{\sqrt{-g}}\nabla_\nu \sqrt{-g}
			\end{equation}

			\noindent
			Proper time\index{Eigenzeit}:
			\begin{equation}
				\Delta\tau := \frac{1}{c}\int_A^B\sqrt{-g_{\mu\nu}\dd x^\mu \dd x^\nu}
			\end{equation}

			\noindent
			Covariant derivative\index{Kovariante Ableitung}:
			\begin{equation}
				\nabla_\lambda T^{\alpha \dots}_{\beta\dots} =
				T^{\alpha \dots}_{\beta\dots;\lambda} = T^{\alpha\dots}_{\beta \dots,\lambda}
				+ \Gamma^\alpha_{\lambda\alpha'} T^{\alpha'\dots}_{\beta\phantom{\prime}\dots} + \dots
				-\Gamma^{\beta'}_{\lambda\beta} T^{\alpha\phantom{\prime}\dots}_{\beta'\dots} - \dots
			\end{equation}
			The term additional to the partial derivative vanish in locally free falling frames of reference.

			\noindent
			Riemann curvature tensor / Riemann tensor\index{Riemann!Krümmungstensor}:
			\begin{align}
				R^{\lambda}{}_{\sigma\mu\nu} &=
				\partial_{\mu}\Gamma^{\lambda}{}_{\nu\sigma} -
				\partial_{\nu}\Gamma^{\lambda}{}_{\mu\sigma} +
				\Gamma^{\lambda}{}_{\mu\kappa}\Gamma^{\kappa}{}_{\nu\sigma} -
				\Gamma^{\lambda}{}_{\nu\kappa}\Gamma^{\kappa}{}_{\mu\sigma} \\
				R_{\lambda\sigma\mu\nu} &= \frac{1}{2}\qty(
				\partial_\sigma \partial_\mu g_{\lambda\nu} - \partial_\sigma \partial_\nu g_{\lambda\mu} + \partial_\lambda \partial_\nu g_{\sigma\mu} - \partial_\lambda \partial_\mu g_{\sigma\nu})
				+ g_{\alpha\beta} \qty(
				\Gamma^{\alpha}_{\mu\sigma} \Gamma^{\beta}_{\nu\lambda} - \Gamma^{\alpha}_{\nu\sigma} \Gamma^{\beta}_{\mu\lambda})
				\\
				R^{\lambda}{}_{\sigma\mu\nu}v^\sigma &= \qty(\nabla_\mu \nabla_\nu - \nabla_\nu \nabla_\mu) v^\lambda = \nabla_{\left[\mu\right.}\nabla_{\left.\nu\right]} v^\lambda
			\end{align}

			\noindent
			Symmetries of the curvature tensor:
			\begin{align}
				R_{\lambda\sigma\mu\nu} &= - R_{\lambda\sigma\nu\mu} \\
				R_{\lambda\sigma\mu\nu} &= \phantom{-} R_{\mu\nu\lambda\sigma} \\
				R_{\lambda\sigma\mu\nu} &= - R_{\sigma\lambda\mu\nu}
			\end{align}

			\noindent
			Bianchi Identities\index{Bianchi!Identitäten}:
			\begin{align}
				0 &= R^{\lambda}_{\phantom{\lambda}\alpha\beta\gamma} + R^{\lambda}_{\phantom{\lambda}\beta\gamma\alpha} + R^{\lambda}_{\phantom{\lambda}\gamma\alpha\beta} \\
				0 &= \nabla_\gamma R^{\lambda}_{\phantom{\lambda}\mu\alpha\beta} + \nabla_\alpha R^{\lambda}_{\phantom{\lambda}\mu\beta\gamma} + \nabla_\beta R^{\lambda}_{\phantom{\lambda}\mu\gamma\alpha}
			\end{align}

			\noindent
			Ricci-Tensor\index{Ricci!Tensor}:
			\begin{equation}
				R_{\mu\nu} = R^\lambda_{\phantom{\lambda}\mu\lambda\nu}
			\end{equation}

			\noindent
			Curvature scalar\index{Krümmungsskalar}\index{Ricci!Skalar}:
			\begin{equation}
				R = R^\mu_{\phantom{\mu}\mu}
			\end{equation}

			\noindent
			Decomposition of the Riemann tensor\index{Riemman!Krümmungstensor} into Weyl tensor\index{Weyl!Tensor} (traceless part), Ricci tensor and Ricci scalar:
			\begin{align}				
				\begin{split}
					R_{\mu\nu\rho\sigma} =&\; C_{\mu\nu\rho\sigma} - \frac{1}{2} \qty(R_{\mu\sigma} g_{\nu\rho} - R_{\mu\rho} g_{\nu\sigma} + R_{\nu\rho} g_{\mu\sigma} - R_{\nu\sigma} g_{\mu\rho}) \\ 
					&- \frac{1}{6} R \qty(g_{\mu\rho} g_{\nu\sigma} + g_{\mu\sigma} g_{\nu\rho})
				\end{split} \\
				C^{\lambda}{}_{\mu\lambda\nu} =&\; 0
			\end{align}

		\subsubsection{Field Equations\index{Feldgleichungen}}
			\noindent
			Einstein--Hilbert Action\index{Einstein!--Hilbert Wirkung}\index{Hilbert!Einstein--Hilbert Wirkung} ($g=\det(g_{\mu\nu}(x))$; Lovelock's theorem\index{Lovelock!Theorem} guaranties that the this is the unique second order action for a metric theory of gravity in four dimensions):
			\begin{equation}
				\mathcal{S}_H = \frac{c^4}{16\pi G} \int\dd^4 x\; \sqrt{-g} \qty(R - 2\Lambda)
			\end{equation}

			\noindent
			Einstein Tensor:
			\begin{equation}
				G_{\mu\nu} = R_{\mu\nu} - \frac{1}{2} R g_{\mu\nu}
			\end{equation}

			\noindent
			Einstein field equations and equivalent formulations\index{Einstein!Feldgleichungen}:
			\begin{align}
				G_{\mu\nu} + \Lambda g_{\mu\nu} &= \frac{8\pi G}{c^4} T_{\mu\nu} \\
				R_{\mu\nu} - \frac{1}{2} R g_{\mu\nu} + \Lambda g_{\mu\nu} &= \frac{8\pi G}{c^4} T_{\mu\nu} \\
				R_{\mu\nu} - \Lambda g_{\mu\nu}&= \frac{8\pi G}{c^4} \qty( T_{\mu\nu} - \frac{1}{2} T g_{\mu\nu} )
			\end{align}

			\noindent
			Energy-momentum tensor (Where $S_M$ is the matter action):
			\begin{equation}
				T_{\mu\nu} = -\frac{2}{\sqrt{-g}} \fdv{S_M}{g^{\mu\nu}}
			\end{equation}


		\subsubsection{Tetrads\index{Vielbein}}
			Tetrad field $e_a^\mu(x)$ and inverse tetrad field $e_\mu^a(x)$ (Tensor field for transforming into a local Minkowskian frame of reference, denoted with latin indices):
			\begin{equation}
				\begin{aligned}
					g_{\mu\nu} e_a^\mu e_b^\nu &= \eta_{ab}
					\hsp g_{\mu\nu} = e^a_\mu e^b_\nu \eta_{ab} \\
					\partial_a &= e_a^\mu \partial_\mu
					\hsp \dd{x}^a = e^a_\mu \dd{x}^\mu \\
					\partial_\mu &= e_\mu^a \partial_a
					\hsp \dd{x}^\mu = e^\mu_a \dd{x}^a
				\end{aligned}
			\end{equation}

			\noindent
			Covariant derivative in terms of the tetrad field:
			\begin{equation}
				\nabla_\mu X^a_b = \partial_\mu X^a_b + \omega_\mu{}^a{}_c X^c_b - \omega_\mu{}^c{}_b X^a_c,
			\end{equation}
			where $\omega_\mu{}^a{}_b$ is the spin connection\index{Spin-Verbindung} (its relation to the Christoffel symbols is equivalent to $\nabla_\mu e^a_\nu = 0$):
			\begin{equation}
				\omega_\mu{}^a{}_b = e^a_\nu e^\sigma_b \Gamma^\nu_{\mu\sigma} - e^\rho_b \partial_\mu e^a_\rho
			\end{equation}


		\subsubsection{Tensors}
			Torsion Tensor\index{Torsionstensor}:
			\begin{equation}
				T_{\mu\nu}{}^\lambda = \Gamma_{\mu\nu}^\lambda - \Gamma_{\nu\mu}^\lambda
			\end{equation}

			\noindent
			Maurer--Cartan structure Equations\index{Maurer--Cartan Strukturgleichungen}\index{Cartan!Maurer--Cartan Strukturgleichungen} (For the torsion tensor $T$ and the curvature tensor $R$):
			\begin{equation}
				\begin{aligned}
					T^a &= \dd e^a + \omega^a{}_b \wedge e^b \\
					R^a_b &= \dd \omega^a{}_b + \omega^a_{}c \wedge \omega^c{}_b
				\end{aligned}
			\end{equation}

		\subsubsection{Gauge Freedom}
			General invariance / diffeomorphism covariance:
			\begin{equation}
				x^\mu \to x^\mu + \xi^\mu
			\end{equation}

			\noindent
			Harmonic gauge\index{Harmonische Eichung} / De Donder gauge\index{De Donder Eichung} (all formulations are equivalent):
			\begin{equation}
				\nabla_\mu \nabla^\mu x^\nu = 0
				\hsp g^{\mu\nu} \Gamma^\rho_{\mu\nu} = 0
				\hsp \partial_\nu \qty(g^{\mu\nu}\sqrt{-g}) = 0
			\end{equation}

		\subsubsection{Geodesics\index{Geodäten}}
			\noindent
			Action of a free particle:
			\begin{equation}
				\mathcal{S}
				= -\int_A^B m u^\mu u_\mu \dd{\tau}
				= -\int_A^B mc^2\dd{\tau}
				= -\int_A^B mc\sqrt{-g_{\mu\nu} \dv{x^\mu}{\lambda} \dv{x^\nu}{\lambda}} \dd{\lambda}
			\end{equation}

			\noindent
			Equations of motion for a free particle (time-like geodesic\index{Zeitartige Geodäte}):
			\begin{equation}
				\frac{\mathrm{d}^2 x^\kappa}{\mathrm{d}\tau^2}=-\Gamma_{\mu\nu}^{\kappa}\frac{\mathrm{d}x^\mu}{\mathrm{d}\tau}\frac{\mathrm{d}x^\nu}{\mathrm{d}\tau}
			\end{equation}

			\noindent
			Four-velocity\index{Vierergeschwindigkeit}:
			\begin{equation}
				u^\mu = \dv{x^\mu}{\tau} = \gamma \mqty(c \\ \vec{v})^\mu
			\end{equation}

			\noindent
			Killing's Equation\index{Killing!Gleichung}:
			\begin{equation}
				\begin{aligned}
					\nabla_{\left(\mu\right.} V_{\left.\nu\right)} &= 0
					\implies u^\nu \nabla_\nu (V_\mu u^\mu) &= 0
				\end{aligned}
			\end{equation}

			\noindent
			Geodesic deviation equation (deviation vector $S^\mu = \dv{x^\mu}{s}$, tangent vector $T^\mu = \dv{x^\mu}{t}$):
			\begin{equation}
				\frac{\mathrm{D}^2}{\dd \tau^2} S^\mu
				= T^\alpha \nabla_\alpha T^\beta \nabla_\beta S^\mu
				= R^\mu{}_{\nu\rho\sigma} T^\nu T^\rho S^\sigma
			\end{equation}


		\subsubsection{Solutions}
			Weak energy condition (equivalent to $T_{00} \ge 0$ for a locally inertial frame):
			\begin{equation}
				\label{Eq:WeakEnergyCondition}
				T_{\mu\nu} V^\mu V^\nu \ge 0 \quad \forall V^\mu\;\text{timelike}
			\end{equation}

			\noindent
			Energy-momentum tensor for an ideal fluid:
			\begin{equation}
				T_{\mu\nu} = (\rho + p)u_\mu u_\nu + p g_{\mu\nu}
			\end{equation}

		\subsubsection{Black Holes}
			Schwarzschild Metric\index{Schwarzschild!Metrik} ($\dd \Omega^2 = \dd{\theta^2} + \sin^2 \theta \dd{\phi^2}$, the Schwarzschild radius\index{Schwarzschild!Radius} is $r_s = 2 G M/c^2$):
			\begin{equation}
				\dd s^2 = - \qty(1-\frac{r_s}{r}) \dd t^2 + \qty(1 - \frac{r_s}{r})^{-1} \dd r^2 + r^2 \dd \Omega^2
			\end{equation}
			in Eddington--Finkelstein coordinates\index{Edington!--Finkelstein Koordinaten}\index{Finkelstein!Edington-Finkelstein Koordinaten}:
			\begin{equation}
				\begin{aligned}
					\xi &= t + r + 2 G M \ln(\frac{r}{2 G M} - 1) \\
					\dd s^2 &= -\qty(1-\frac{2 G M}{r}) \dd \xi^2 + (\dd \xi \dd r + \dd r \dd \xi) + r^2\dd \Omega^2
				\end{aligned}
			\end{equation}
			in Kruskal coordinates\index{Kruskal!Koordinaten}:
			\begin{equation}
				\begin{aligned}
					v &= \sqrt{\frac{r}{2 G M} - 1} \exp(\frac{r}{4 G M}) \cosh(\frac{t}{4 G M}) \\
					u &= \sqrt{\frac{r}{2 G M} - 1} \exp(\frac{r}{4 G M}) \sinh(\frac{t}{4 G M}) \\
					\dd s^2 &= -\frac{32 G^3 M^3}{r} \exp(-\frac{r}{2 G M}) (-\dd v^2 + \dd u^2) + r^2 \dd \Omega^2
				\end{aligned}
			\end{equation}

			\noindent
			Kerr Metric\index{Kerr!Metrik} (in Boyer--Lindquist coordinates\index{Boyer!--Lindquist Koordinaten}\index{Lindquist!Boyer--Lindquist Koordinaten}, where $\Delta:=r^2 - 2Mr + a^2$ and $\rho^2:=r^2+a^2\cos^2\theta$):
			\begin{equation}
				\begin{aligned}
					\dd s^2 =& - \frac{\Delta - a^2 \sin^2\theta}{\rho^2} \dd t^2 - 2a\frac{2Mr\sin^2\theta}{\rho^2}\dd t \dd \varphi \\
					&+ \frac{(r^2+a^2)^2 - a^2\Delta \sin^2\theta}{\rho^2}\sin^2\theta\dd\varphi^2 + \frac{\rho^2}{\Delta}\dd r^2 + \rho^2\dd\theta^2
				\end{aligned}
			\end{equation}

			\noindent
			Reissner--Nordström metric\index{Reissner!--Nordström Metrik}\index{Nordström!Reissner--Nordström Metrik} (where $r_Q = \frac{Q^2 G}{4\pi \varepsilon_0 c^4}$ is the charge radius):
			\begin{equation}
				\dd s^{2}=c^{2}\dd\tau ^{2}=\left(1-{\frac {r_s}{r}}+{\frac {r_Q^{2}}{r^{2}}}\right)c^{2}\dd t^{2}-\left(1-{\frac {r_s}{r}}+{\frac {r_Q^{2}}{r^{2}}}\right)^{-1}\dd r^{2}-r^{2}\dd\theta ^{2}-r^{2}\sin ^{2}\theta \dd\varphi ^{2}
			\end{equation}

			\noindent
			Black Hole Entropy\index{Schwarzes Loch!Entropie} / Bekenstein--Hawking formula\index{Bekenstein!--Hawking Formel}\index{Hawking!Bekenstein--Hawking Formel} ($A$ is the event horizon area, \eg a Schwarzschild black hole has $A=4\pi r_s^2$):
			\begin{equation}
				S_\mathrm{BH} = \frac{\kB A}{4 l_P^2}
				= \frac{\kB c^3 A}{4 G \hbar}
			\end{equation}

			\noindent
			Hawking radiation temperature\index{Hawking!Temperatur}:
			\begin{equation}
				T = \frac{\hbar c^3}{8\pi \kB G M}
			\end{equation}

			\noindent
			Laws of Black Hole Mechanics\index{Schwarzes Loch!Mechanik}:
			\begin{itemize}
				\item \emph{Zeroth Law}: The surface gravity $g$ is constant over the event horizon.
				\item \emph{First Law}: For perturbations of stationary black holes, the change of energy $E$ is related to change of area $A$, angular momentum $J$, and electric charge $Q$ by
					\begin{equation}
						\dd E = \frac{g}{8\pi} \dd A + \Omega \dd J + \Phi \dd Q,
					\end{equation}
					where $\Phi$ is the electrostatic potential and $\Omega$ is the angular velocity.
				\item \emph{Second Law}: Assuming the Weak energy condition (Eq.~\ref{Eq:WeakEnergyCondition}), the area of the event horizon never decreases $\dd A \ge 0$.
				\item \emph{Third Law}: It is impossible to reach a state of zero surface gravity $g=0$.
			\end{itemize}

		\subsubsection{Weak Field Limit}
			Weak field limit:
			\begin{equation}
				g_{\mu\nu} = \eta_{\mu\nu} + h_{\mu\nu} \hsp \abs{h_{\mu\nu}} \ll 1 \quad \forall \mu,\nu
			\end{equation}
			direct consequences:
			\begin{equation}
				\begin{aligned}
					g^{\mu\nu} &= \eta^{\mu\nu} - h^{\mu\nu} \\
					h_{\mu'\nu'} &= \Lambda_{\mu'}^{\phantom{\mu'}\mu} \Lambda_{\nu'}^{\phantom{\nu'}\nu} h_{\mu\nu} \\
					R_{\mu\nu} &= \frac{1}{2}\qty(\partial_\rho \partial_\mu h_{\nu}^{\rho} + \partial_\rho \partial_\nu h_{\mu}^{\rho} - \partial_\mu \partial_\nu h - \partial^2 h_{\mu\nu}) \\
				\end{aligned}
			\end{equation}

			\noindent
			Residual gauge freedom:
			\begin{equation}
				h_{\mu\nu} \implies h_{\mu\nu} + \partial_\mu \xi_\nu + \partial_\nu \xi_\mu
			\end{equation}

			\noindent
			Harmonic gauge in the weak field limit:
			\begin{equation}
				\partial_\mu h^\mu{}_\lambda - \frac{1}{2}\partial_\lambda h = 0
			\end{equation}

			\noindent
			Trace-reversed metric pertubation:
			\begin{equation}
				\begin{aligned}
					\bar{h}_{\mu\nu} :=&\, h_{\mu\nu} - \frac{1}{2} \eta_{\mu\nu} h \\
					\partial^2 \bar{h}_{\mu\nu} =&\, -16\pi G T_{\mu\nu}
				\end{aligned}
			\end{equation}

			\noindent
			Connection to the Newtonian Potential\index{Newtonisches Potential}:
			\begin{equation}
				h_{00} = -2\phi = \frac{2GM}{r}
			\end{equation}

			\noindent
			Weak field approximation (\eg for the metric of a star / planet; $\phi$ is the Newtonian potential\index{Newtonisches Potential}):
			\begin{equation}
				\label{eq:WeakFieldMetric}
				\dd{s}^2 = -\qty(1+\frac{2\phi}{c^2}) c^2 \dd{t}^2 + \qty(1-\frac{2\phi}{c^2}) \dd{\vec{x}}^2
			\end{equation}

		\subsubsection{Gravitational Radiation}
			Solutions to the vacuum field equations in the weak field limit (EOM: $\partial^2 \bar{h}_{\mu\nu} = 0$):
			\begin{equation}
				\bar{h}_{\mu\nu} = C_{\mu\nu} \exp(i k_\lambda x^\lambda) \hsp k_\lambda k^\lambda = 0
			\end{equation}

			\noindent
			In transverse traceless gauge / radiation gauge / TT gauge (subgauge of the harmonic gauge):
			\begin{equation}
				\begin{aligned}
					k^\lambda C_{\lambda\mu} &= 0
					\hsp C^\mu_\mu = 0
					\hsp C_{0\mu} = 0
					\hsp \bar{h}_{\mu\nu}^{TT} = {h}_{\mu\nu}^{TT}
					\hsp R_{\mu 0 0 \nu} = \frac{1}{2} \partial_0 \partial_0 h_{\mu\nu} \\
				\end{aligned}
			\end{equation}
			polarizations (all other components vanish):
			\begin{equation}
				C_{11} = -C_{22} =: C_+
				\hsp C_{12} = C_{21} =: C_\times
			\end{equation}
			circular polarizations:
			\begin{equation}
				C_R = \frac{1}{\sqrt{2}} (C_+ + \i C_\times)
				\hsp C_L = \frac{1}{\sqrt{2}} (C_+ - \i C_\times)
			\end{equation}

			\noindent
			Quadrupole expansion of the gravitational wave (Assuming the movement of the source is negligible with regard to the distance $\delta R \ll R=\abs{\vec{x}}$ and that other components of the energy-momentum tensor are unimportant. $q_{ij}$ is the quadrupole, $T^{00}$ is the energy density):
			\begin{equation}
				\begin{aligned}
					\tilde{\bar{h}}_{ij}(\omega,\vec{x}) &= -\frac{2G\omega^2}{3}\frac{\ex^{\i\omega R}}{R} \tilde{q}_{ij}(\omega) \\
					q_{ij}(t) &= 3\int \dd[3]{y} y^i y^j T^{00}(t,\vec{y}) \\
					\tilde{\bar{h}}^{0\nu} &= \frac{\i}{\omega}\partial_i \tilde{\bar{h}}^{i\nu}
				\end{aligned}
			\end{equation}

			\noindent
			Energy of a gravitational wave radiated by a quadrupole source (where $Q$ is the reduced quadrupole moment $Q_{ij} = q_{ij} - \frac{1}{3}\delta_{ij}\delta^{kl}q_{kl}$):
			\begin{equation}
				P = \frac{G}{45} \eval{\dv[3]{\tilde{Q}_{ij}}{t}\dv[3]{\tilde{Q}^{ij}}{t}}_{t_r}
			\end{equation}


	\subsection{Special Relativity\index{Spezielle Relativitätstheorie}}
		\noindent
		Special Relativity\index{Spezielle Relativitätstheorie} is a limiting case of General Relativity for flat spacetime, it can be realized by $G \to 0$. In Special Relativity the Metric\index{Metrischer Tensor} is constant.

		\noindent
		Minkowski-Metric\index{Minkowski!Metrik}:
		\begin{equation}
			g_{\mu\nu} = \eta_{\mu\nu}
			= \left( \begin{matrix}
					\pm1 & 0    & 0    & 0    \\
					0    & \mp1 & 0    & 0    \\
					0    & 0    & \mp1 & 0    \\
					0    & 0    & 0    & \mp1 \\
				\end{matrix} \right)
		\end{equation}
		Signature used in this subsection: $(+1,-1,-1,-1)$.

		\noindent
		Classical Limit\index{Klassischer Grenzfall}:
		\begin{equation}
			c \to \infty
		\end{equation}

		\noindent
		Ultrarelativistic limit\index{Ultrarelativistischer Grenzfall}:
		\begin{equation}
			v \to c
		\end{equation}

		\subsubsection{Definitions}
			\noindent
			Lorentz-factor\index{Lorentz!Faktor} ($\beta = \frac{v}{c}$):
			\begin{equation}
				\gamma = \frac{1}{\sqrt{1-\beta^2}}
			\end{equation}

			\noindent
			Four-momentum\index{Viererimpuls}:
			\begin{equation}
				P^\mu =
				\mqty(
				E \\ \vec{p}
				)^\mu
				= \mqty(
				m\gamma c \\ m\gamma\vec{v}
				)^\mu
				= m u^\mu
			\end{equation}

		\subsubsection{Lorentz Transformations}
			\noindent
			Transformation rules of Lorentz tensors (defining property of Lorentz tensors\index{Lorentz!Tensor}\index{Tensor}):
			\begin{equation}
				\begin{aligned}
					{x'}^\mu &= \Lambda^{\mu}_{\nu}(\vec{v}) x^\nu \\
					{x'}_\mu &=  x_\nu \overline{\Lambda}_\mu^{\nu}(\vec{v}) \\
					{T'}_{\mu_1,\mu_2,\dots}^{\nu_1,\nu_2,\dots} &=  \Lambda^{\nu_1}_{\beta_1}(\vec{v}) \Lambda^{\nu_2}_{\beta_2}(\vec{v}) \dots T_{\alpha_1,\alpha_2,\dots}^{\beta_1,\beta_2,\dots} \overline{\Lambda}_{\mu_1}^{\alpha_1}(\vec{v}) \overline{\Lambda}_{\mu_2}^{\alpha_2}(\vec{v}) \dots \\
				\end{aligned}
			\end{equation}

			\noindent
			Properties of the homogeneous Lorentz transformation\index{Lorentz!Transformation}:
			\begin{equation}
				\Lambda^{\mu}_{\nu}(\vec{v}) = \Lambda^{\mu}_{\phantom{\mu}\nu}(\vec{v}) = \Lambda^{\phantom{\nu}\mu}_{\nu}(\vec{v}) = \overline{\Lambda}^{\mu}_{\nu}(-\vec{v}) \\
			\end{equation}

			\noindent
			Poincaré transformation\index{Poincaré!Transformation} $(\Lambda, b) \in\mathcal{P} \supseteq \mathcal{L}$ (The Poincaré group\index{Poincaré!Gruppe} is the general inhomogeneous Lorentz transformation, where the Lorentz group\index{Lorentz!Gruppe} $\mathcal{L}$ is the homogeneous special case):
			\begin{equation}
				x'^\mu = \Lambda^\mu_{\nu} x^\nu + b^\mu \\
			\end{equation}

			\noindent
			Condition for the Lorentz transformation\index{Lorentz!Transformation} following from the constancy of the speed of light \ref{post:c}:
			\begin{equation}
				\Lambda^{\alpha}_{\mu} \eta_{\alpha\beta} \Lambda^{\beta}_{\nu} = \eta_{\mu\nu}
				\iff \Lambda^\trp \eta \Lambda = \eta
			\end{equation}

			\noindent
			Classification of the homogeneous Lorentz transformations\index{Lorentz!Transformation} $\Lambda \in \mathcal{L} = \mathcal{L}^\uparrow_+ \cup \mathcal{L}^\uparrow_- \cup \mathcal{L}^\downarrow_+ \cup \mathcal{L}^\downarrow_-$:
			\begin{itemize}
				\item Chronus Lorentz Transformation $\mathcal{L}^\uparrow$: $\Lambda^0_0 > 1$
				\item Antichronus Lorentz Transformation $\mathcal{L}^\downarrow$: $\Lambda^0_0 < 1$
				\item Proper Lorentz Transformation $\mathcal{L}_+$: $\det\Lambda = 1$
				\item Improper Lorentz Transformation $\mathcal{L}_-$: $\det\Lambda = -1$
			\end{itemize}

			\noindent
			Lorentz Boost\index{Lorentz!Boost} ($\Lambda\in\mathcal{L}^\uparrow_+$) between two Inertial frames of reference with parallel coordinate axes:
			\begin{equation}
				\begin{aligned}
					\Lambda^{\mu}_{\nu}(\vec{v})
					=& \left( \begin{matrix}
						\gamma                    & -\gamma\dfrac{\vec{v}^\trp}{c}                \\[6pt]
						-\gamma\dfrac{\vec{v}}{c} & \delta_{ij}+\dfrac{v_i v_j(\gamma-1)}{v^2}
					\end{matrix} \right)^{\mu}_{\nu} \\
					=&
					\mqty(
					\gamma & -\gamma \beta_1 & -\gamma \beta_2 & -\gamma \beta_3 \\
					-\gamma \beta_1 & 1+(\gamma -1){\dfrac {\beta_1^{2}}{\beta^{2}}} & (\gamma -1){\dfrac {\beta_1 \beta_2}{\beta^{2}}}&(\gamma -1){\dfrac {\beta_1\beta_3}{\beta^{2}}} \\
					-\gamma \beta_2 & (\gamma -1){\dfrac {\beta_2\beta_1}{\beta^{2}}} & 1+(\gamma -1){\dfrac {\beta_2^{2}}{\beta^{2}}}&(\gamma -1){\dfrac {\beta_2 \beta_3}{\beta^{2}}} \\
					-\gamma \beta_3 &(\gamma -1){\dfrac {\beta_3\beta_1}{\beta^{2}}}&(\gamma -1){\dfrac {\beta_3\beta_2}{\beta^{2}}}&1+(\gamma -1){\dfrac {\beta_3^{2}}{\beta^{2}}}
					)^{\mu}_{\nu}
				\end{aligned}
			\end{equation}

		\subsubsection{Implications of Special Relativity}
			\noindent
			Proper time\index{Eigenzeit} in special Relativity:
			\begin{equation}
				\Delta\tau = \int_A^B \frac{\dd t}{\gamma}
			\end{equation}

			\noindent
			Relativistic Doppler effect\index{Doppler!Effekt} (Signals of frequency $\omega_0$ are sent with angle $\vartheta$ as seen from the emitter):
			\begin{equation}
				\omega = \omega_0 \gamma \qty(1-\beta\cos(\theta))
			\end{equation}

			\noindent
			Composition of velocities\index{Geschwindigkeitsaddition} (Rapidity\index{Rapidität} $\psi = \mathrm{artanh}\left(\frac{v}{c}\right)$):
			\begin{equation}
				\begin{aligned}
					\psi_{\text{tot.}} &= \psi_1+\psi_2 \\
					\vec{v}_{\text{tot.}} &= \frac{\vec{v}_1+\vec{v}_{2\parallel}+\vec{v}_{2\perp}\sqrt{1-\dfrac{\vec{v}_1^2}{c^2}}}{1+\dfrac{\vec{v}_1\cdot\vec{v}_2}{c^2}} \\
					\vec{v}_1\parallel\vec{v}_2 \implies v_{\text{tot.}} &= \frac{v_1+v_2}{1+\dfrac{v_1 v_2}{c^2}}
				\end{aligned}
			\end{equation}

			\noindent
			Relativistic aberration\index{Relativistische Aberration} (Observed Angle $\vartheta'$ for a relative velocity $\beta$ and an inclination of $\vartheta$ as measured in the observer's reference frame; Both formulas are equivalent):
			\begin{equation}
				\begin{aligned}
					\tan\left(\frac{\theta}{2}\right) &= \sqrt{\frac{1-\beta}{1+\beta}}\tan\left(\frac{\theta'}{2}\right)\\
					\cos\vartheta' &= \frac{\cos\vartheta+\beta}{1+\beta\cos\vartheta}
				\end{aligned}
			\end{equation}

			\noindent
			Energy-momentum relation\index{Energie-Impuls-Relation}:
			\begin{equation}
				\begin{aligned}
					P^\mu P_\mu &= m^2 c^2\\
					E^2 &= p^2 c^2 + m^2 c^4 \\
				\end{aligned}
			\end{equation}
			\newpage
