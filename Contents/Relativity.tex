% !TEX root = ../physics.tex
\section{Relativity}
	\subsection{Basics}
		\subsubsection{Symmetries of Flat Spacetime}
			\begin{itemize}
				\item Homogeneity of spacetime
				\item Isotropy of spacetime
				\item Lorentz invariance
			\end{itemize}

			\noindent
			Inertial frame of reference\index{Inertialsystem} \\
				\indent \textit{Frame of reference in which spacetime is homogeneous and isotropic.}

		\subsubsection{Einstein Postulates}
			\textbf{First Postulate}: Principle of Relativity\index{Relativitätsprinzip} \newline
				\indent \textit{The laws of nature are forminvariant in all inertial frames of reference.}  \nl
			\textbf{Second Postulate}: Constancy of the Speed of Light\index{Konstanz der Lichtgeschwindigkeit} \newline
				\indent \textit{The maximum propagation speed of information is equal to $c$ in all frames of reference.\index{Lichtgeschwindigkeit}}
				\label{post:c}\nl
			\textbf{Third Postulate}: Strong Equivalence Principle\index{Starkes Äquivalenzprinzip} \newline
				\indent \textit{Spacetime is locally flat, \ie the laws of special relativity hold in local inertial frames of reference.}  \vsp

	% \begin{description}
	%   \item[Relativitätsprinzip]\hfill \\
	%	 Die Gesetze der Physik sind in allen Inertialsystemen gleich.
	%   \item[Konstanz der Lichtgeschwindigkeit]\hfill \\
	%	 Die maximale Ausbreitungsgeschwindigkeit von Informationen ist in allen Bezugssystemen gleich $c$.
	%   \item[Starkes Äquivalenzprinzip]\hfill \\
	%	 Die Raumzeit ist lokal flach, d.h. in lokalen Inertialsystemen gelten die Gesetze der Speziellen Relativitätstheorie.
	% \end{description}

	\subsubsection{Metric Tensor\index{Metrischer Tensor}}
		Metric Tensor and Lorentz invariant spacetime interval:
		\begin{equation}
			\dd s^2 = c^2 \dd \tau^2 = \dd x^\mu \dd x_\mu = g_{\mu\nu} \dd x^\mu \dd x^\nu
		\end{equation}

	\subsection{General Relativity\index{Allgemeine Relativitätstheorie}}
		\subsubsection{Definitions}
			\noindent
			Partial derivatives:
			\begin{equation}
				\pder{\phi}{x^\mu} = \partial_\mu \phi = \phi_{,\mu}
			\end{equation}

			\noindent
			Christoffel symbols\index{Christoffel!Symbole} (local inertial frame with coordinates $\xi^\alpha$):
			\begin{equation}
				\Gamma_{\mu\nu}^{\kappa} := \frac{\partial x^\kappa}{\partial \xi^\alpha}\frac{\partial^2 \xi^\alpha}{\partial x^\mu\partial x^\nu}
				=\frac{1}{2}g^{\kappa\lambda}\left(\frac{\partial g_{\nu\lambda}}{\partial x^\mu}+\frac{\partial g_{\mu\lambda}}{\partial x^\nu}-\frac{\partial g_{\nu\mu}}{\partial x^\lambda}\right)
			\end{equation}
			The Christoffel symbols are symmetric in the lower indices $\Gamma_{\mu\nu}^{\kappa} = \Gamma_{\nu\mu}^{\kappa}$. They are zero in all free falling local frames of reference.

			\noindent
			Proper time\index{Eigenzeit}:
			\begin{equation}
				\Delta\tau := \frac{1}{c}\int_A^B\sqrt{g_{\mu\nu}\dd x^\mu \dd x^\nu}
			\end{equation}

			\noindent
			Covariant derivative\index{Kovariante Ableitung}:
			\begin{equation}
				D_\lambda T^{\alpha ...}_{\beta...} =
				T^{\alpha ...}_{\beta...;\lambda} = T^{\alpha...}_{\beta ...,\lambda}
				+ \Gamma^\alpha_{\lambda\alpha'} T^{\alpha'...}_{\beta\phantom{\prime}...} + ...
				-\Gamma^{\beta'}_{\lambda\beta} T^{\alpha\phantom{\prime}...}_{\beta'...} - ...
			\end{equation}
			The second term vanishes in all free falling frames of reference and the covariant derivative becomes a partial derivative, %In frei fallenden Bezugssystemen verschwindet der zweite Term und die kovariante Ableitung wird zur partiellen Ableitung.

			\noindent
			Riemann curvature tensor\index{Riemann!Krümmungstensor}:
			\begin{equation}
				\begin{aligned}
					R^{\lambda}_{\phantom{\lambda}\sigma\mu\nu} &= \Gamma^{\lambda}_{\phantom{\lambda}\nu\sigma,\mu}
					+ \Gamma^{\lambda}_{\phantom{\lambda}\mu\kappa}\Gamma^{\kappa}_{\phantom{\kappa}\nu\sigma}
					-
					\Gamma^{\lambda}_{\phantom{\lambda}\mu\sigma,\nu}
					+ \Gamma^{\lambda}_{\phantom{\lambda}\nu\kappa}\Gamma^{\kappa}_{\phantom{\kappa}\mu\sigma} \\
					R_{\lambda\sigma\mu\nu} &= \frac{1}{2}\left(
					g_{\lambda\nu,\sigma,\mu} - g_{\lambda\mu,\sigma,\nu} + g_{\sigma\mu,\lambda,\nu} -	 g_{\sigma\nu,\lambda,\mu}
					\right)
					+ g_{\alpha\beta} \left(
					\Gamma^{\alpha}_{\mu\sigma} \Gamma^{\beta}_{\nu\lambda} - \Gamma^{\alpha}_{\nu\sigma} \Gamma^{\beta}_{\mu\lambda}
					\right)
					\\
					R^{\lambda}_{\phantom{\lambda}\sigma\mu\nu}v^\sigma &= \left(D_\mu D_\nu - D_\nu D_\mu \right) v^\lambda
				\end{aligned}
			\end{equation}

			\noindent
			Symmetries of the curvature tensor:%Symmetrien des Krümmungstensors:
			\begin{equation}
				\begin{aligned}
					R_{\lambda\sigma\mu\nu} &= - R_{\lambda\sigma\nu\mu} \\
					R_{\lambda\sigma\mu\nu} &= \phantom{-} R_{\mu\nu\lambda\sigma} \\
					R_{\lambda\sigma\mu\nu} &= - R_{\sigma\lambda\mu\nu} \\
				\end{aligned}
			\end{equation}

			\noindent
			Bianchi Identities\index{Bianchi!Identitäten}:
			\begin{equation}
				\begin{aligned}
					R^{\lambda}_{\phantom{\lambda}\alpha\beta\gamma} + R^{\lambda}_{\phantom{\lambda}\beta\gamma\alpha} + R^{\lambda}_{\phantom{\lambda}\gamma\alpha\beta} &= 0 \\
					R^{\lambda}_{\phantom{\lambda}\mu\alpha\beta;\gamma} + R^{\lambda}_{\phantom{\lambda}\mu\beta\gamma;\alpha} + R^{\lambda}_{\phantom{\lambda}\mu\gamma\alpha;\beta} &= 0 \\
				\end{aligned}
			\end{equation}

			\noindent
			Ricci-Tensor\index{Ricci!Tensor}:
			\begin{equation}
				R_{\mu\nu} = R^\lambda_{\phantom{\lambda}\mu\lambda\nu}
			\end{equation}

			\noindent
			Curvature scalar\index{Krümmungsskalar}:
			\begin{equation}
				R = R^\mu_{\phantom{\mu}\mu}
			\end{equation}

		\subsubsection{Field Equations\index{Feldgleichungen}}
			\noindent
			Einstein-Hilbert Action\index{Einstein!-Hilbert Wirkung}:
			\begin{equation}
				\mathcal{S} = \frac{c^4}{16\pi G} \int \sqrt{\left|\det{g_{\mu\nu}(x)}\right|} R(g_{\mu\nu}(x))\;\dd^4 x
			\end{equation}

			\noindent
			Einstein field equations\index{Einstein!Feldgleichungen}:
			\begin{equation}
				R_{\mu\nu} - \frac{1}{2} R g_{\mu\nu} = \frac{8\pi G}{c^4} T_{\mu\nu}
			\end{equation}

			\noindent
			Alternative Formulation of the field equations:
			\begin{equation}
				R_{\mu\nu} = \frac{8\pi G}{c^4} \left( T_{\mu\nu} - \frac{1}{2} T g_{\mu\nu} \right)
			\end{equation}


		\subsubsection{Geodesics\index{Geodäten}}
			\noindent
			Action of a free particle%Wirkung eines freien Teilchens:
			\begin{equation}
				\mathcal{S} = -\int_A^B mc^2\;\dd \tau = -\int_A^B mc\sqrt{g_{\mu\nu} \tder{x^\mu}{\tau} \tder{x^\nu}{\tau}} \;\dd \tau
			\end{equation}

			\noindent
			Equations of motion for a free particle (time-like geodesic\index{Zeitartige Geodäte}):%Bewegungsgleichungen eines freien Teilchens (Zeitartige Geodäte):
			\begin{equation}
				\frac{\mathrm{d}^2 x^\kappa}{\mathrm{d}\tau^2}=-\Gamma_{\mu\nu}^{\kappa}\frac{\mathrm{d}x^\mu}{\mathrm{d}\tau}\frac{\mathrm{d}x^\nu}{\mathrm{d}\tau}
			\end{equation}

	\subsection{Special Relativity\index{Spezielle Relativitätstheorie}}
		\noindent
		Special Relativity\index{Spezielle Relativitätstheorie} is a limiting case of General Relativity for flat spacetime, it can be realized by $G \rightarrow 0$. In Special Relativity the Metric\index{Metrischer Tensor}is constant.%Die Spezielle Relativität ist der Spezialfall der flachen bzw. ungekrümmten Raumzeit $G\rightarrow 0$, in ihr wird der metrische Tensor konstant.

		\noindent
		Minkowski-Metric\index{Minkowski!Metrik} (Sign is dependent on convention):
		\begin{equation}
			g_{\mu\nu} = \eta_{\mu\nu}
			= \left( \begin{matrix}
				\pm1 & 0		& 0		& 0		\\
				0		& \mp1 & 0		& 0		\\
				0		& 0		& \mp1 & 0		\\
				0		& 0		& 0		& \mp1 \\
			\end{matrix} \right)
		\end{equation}

		\noindent
		Classical Limit\index{Klassischer Grenzfall}:
		\begin{equation}
			c \rightarrow \infty
		\end{equation}

		\noindent
		Ultrarelativistic limit\index{Ultrarelativistischer Grenzfall}:
		\begin{equation}
			v\rightarrow c
		\end{equation}

		\subsubsection{Definitions}
			\noindent
			Lorentz-factor\index{Lorentz!Faktor} ($\beta = \frac{v}{c}$):
			\begin{equation}
				\gamma = \frac{1}{\sqrt{1-\beta^2}}
			\end{equation}

			\noindent
			Four-momentum\index{Viererimpuls}:
			\begin{equation}
				P^\mu =
				\left(\begin{matrix}
					E \\ \vec{p}
				\end{matrix}\right)
				= \left(\begin{matrix}
					m\gamma c \\ m\gamma\vec{v}
				\end{matrix}\right)
				= \left(\begin{matrix}
					\frac{mc}{\sqrt{1-\beta^2}} \\ \frac{m\vec{v}}{\sqrt{1-\beta^2}}
				\end{matrix}\right)
				= m u^\mu
			\end{equation}

		\subsubsection{Lorentz Transformations}
			\noindent
			Homogeneous Lorentz transformation\index{Lorentz!Transformation} and back-transformation of tensors (defining property of Lorentz Tensors\index{Lorentz!Tensor}\index{Tensor}):%Homogene Lorentztransformation und Rücktransformation von Tensoren (definierende Eigenschaft eines Tensors):
			\begin{equation}
				\begin{aligned}
					x'^\mu &= \Lambda^{\mu}_{\nu}(\vec{v}) x^\mu \\
					x'_\mu &= \overline{\Lambda}_\mu^{\nu}(\vec{v}) x_\nu \\
					\Lambda^{\mu}_{\nu}(\vec{v}) = \Lambda^{\mu}_{\phantom{\mu}\nu}(\vec{v}) &= \Lambda^{\phantom{\nu}\mu}_{\nu}(\vec{v}) = \overline{\Lambda}^{\mu}_{\nu}(-\vec{v})
				\end{aligned}
			\end{equation}

			\noindent
			General (inhomogeneous) Lorentz transformation\index{Lorentz!Transformation} (Poincaré group\index{Poincaré!Gruppe} $\mathcal{L}$%Allgemeine (inhomogene) Lorentz-transformation (Mit Poincaré Gruppe $\mathcal{L}$):
			\begin{equation}
				\begin{aligned}
					x'^\mu &= \Lambda^\mu_{\nu} x^\nu + b^\mu \\
					\Lambda \in \mathcal{L} &= \mathcal{L}^\uparrow_+ \cup \mathcal{L}^\uparrow_- \cup \mathcal{L}^\downarrow_+ \cup \mathcal{L}^\downarrow_-
				\end{aligned}
			\end{equation}

			\noindent
			Condition for the Lorentz transformation\index{Lorentz!Transformation} following from the constancy of the speed of light \ref{post:c}:%Bedingung an die Lorentztransformation als Folge der Konstanz der Lichtgeschwindigkeit:
			\begin{equation}
				\Lambda^{\alpha}_{\mu} \eta_{\alpha\beta} \Lambda^{\beta}_{\nu} = \eta_{\mu\nu}
				 \;\Leftrightarrow\; \Lambda^T \eta \Lambda = \eta
			\end{equation}

			\noindent
			Classification of the homogeneous Lorentz transformations\index{Lorentz!Transformation}:%Klassifikation der homogenen Lorentztransformation:
			\begin{itemize}
				\item Chronus Lorentz Transformation $\mathcal{L}^\uparrow$: $\Lambda^0_0 > 1$
				\item Antichronus Lorentz Transformation $\mathcal{L}^\downarrow$: $\Lambda^0_0 < 1$
				\item Proper Lorentz Transformation $\mathcal{L}_+$: $\det\Lambda = 1$
				\item Improper Lorentz Transformation $\mathcal{L}_-$: $\det\Lambda = -1$
			\end{itemize}

			\noindent
			Lorentz Boost\index{Lorentz!Boost} ($\Lambda\in\mathcal{L}^\uparrow_+$) between two Inertial frames of reference with parallel coordinate axes ($\vec{v}=c\vec{\beta}$):%Lorentz Boost ($\Lambda\in\mathcal{L}^\uparrow_+$) zwischen zwei Inertialsystemen mit parallelen Koordinatenachsen ($\vec{v}=c\vec{\beta}$):
			\begin{equation}
				\Lambda(\vec{v}) = \left( \begin{matrix}
					\gamma & -\gamma\dfrac{\pvec{v}^T}{c} \\[6pt]
					-\gamma\dfrac{\vec{v}}{c} & \delta_{ij}+\dfrac{v_i v_j(\gamma-1)}{v^2}
					\end{matrix} \right)
					=
					\left(\begin{matrix}
						\gamma & -\gamma \beta_1 & -\gamma \beta_2 & -\gamma \beta_3 \\
						-\gamma \beta_1 & 1+(\gamma -1){\dfrac {\beta_1^{2}}{\beta^{2}}} & (\gamma -1){\dfrac {\beta_1 \beta_2}{\beta^{2}}}&(\gamma -1){\dfrac {\beta_1\beta_3}{\beta^{2}}} \\
						-\gamma \beta_2 & (\gamma -1){\dfrac {\beta_2\beta_1}{\beta^{2}}} & 1+(\gamma -1){\dfrac {\beta_2^{2}}{\beta^{2}}}&(\gamma -1){\dfrac {\beta_2 \beta_3}{\beta^{2}}} \\
						-\gamma \beta_3 &(\gamma -1){\dfrac {\beta_3\beta_1}{\beta^{2}}}&(\gamma -1){\dfrac {\beta_3\beta_2}{\beta^{2}}}&1+(\gamma -1){\dfrac {\beta_3^{2}}{\beta^{2}}}
					\end{matrix}\right)
			\end{equation}

		\subsubsection{Implications of Special Relativity}
			\noindent
			Proper time\index{Eigenzeit} in special Relativity:%Eigenzeit in der speziellen Relativitätstheorie:
			\begin{equation}
				\Delta\tau = \int_A^B \frac{\dd t}{\gamma}
			\end{equation}

			\noindent
			Relativistic Doppler effect\index{Doppler!Effekt} (Signals are sent with angle $\vartheta$ as seen from the emitter):%Relativistischer Doppler-effekt (Signale werden von Betrachter aus gemessen mit Winkel $\theta$ ausgesendet):
			\begin{equation}
				\omega = \omega_0\frac{\sqrt{1-\beta^2}}{1+\beta\cos\vartheta}
			\end{equation}

			\noindent
			Addition of velocites\index{Geschwindigkeitsaddition} (Rapidity\index{Rapidität} $\psi = \mathrm{artanh}\left(\frac{v}{c}\right)$):
			\begin{equation}
				\begin{aligned}
					\psi_{tot} &= \psi_1+\psi_2 \\
					\vec{v}_{tot} &= \frac{\vec{v}_1+\vec{v}_{2\parallel}+\vec{v}_{2\perp}\sqrt{1-\dfrac{\vec{v}_1^2}{c^2}}}{1+\dfrac{\vec{v}_1\cdot\vec{v}_2}{c^2}} \\
					\vec{v}_1\parallel\vec{v}_2 \;\Rightarrow\; v_{tot} &= \frac{v_1+v_2}{1+\dfrac{v_1 v_2}{c^2}}
				\end{aligned}
			\end{equation}

			\noindent
			Relativistic aberration\index{Relativistische Aberration} (Observed Angle $\vartheta'$ for a relative velocity $\beta$ and an inclination of $\vartheta$ as measured in the observer's reference frame; Both formulas are equivalent):
			\begin{equation}
				\begin{aligned}
					\tan\left(\frac{\theta}{2}\right) = \sqrt{\frac{1-\beta}{1+\beta}}\tan\left(\frac{\theta'}{2}\right)\\
					\cos\vartheta' = \frac{\cos\vartheta+\beta}{1+\beta\cos\vartheta}
				\end{aligned}
			\end{equation}

			\noindent
			Energy Momentum Relation\index{Energie-Impuls-Relation}:
			\begin{equation}
				\begin{aligned}
					P^\mu P_\mu &= m^2 c^2\\
					E^2 &= p^2 c^2 + m^2 c^4 \\
				\end{aligned}
			\end{equation}
			\newpage
