% !TEX root = ../physics.tex
\section{Elektromagnetismus}
	\subsection{Bewegungsgleichungen}
		\noindent
		Wirkung eines freien Teilchens in einem Elektromagnetischen Feld:
		\begin{equation}
			\mathcal{S}=-\int_{a}^{b}\left(mc\sqrt{g_{\mu\nu}\frac{\diff x^\mu}{\diff \tau}\frac{\diff x^\nu}{\diff \tau}}
	+ q\frac{\diff x^\mu}{\diff \tau}A_\mu(x)\right)\;\diff\tau
		\end{equation}

		\noindent
		Lagrange Funktion mit Parametrisierung $\tau$ (Zum Beispiel durch die Eigenzeit):
		\begin{equation}
			\mathcal{L} (\tau,x,u) =-\left(mc\sqrt{g_{\mu\nu}\frac{\diff x^\mu}{\diff \tau}\frac{\diff x^\nu}{\diff \tau}}
			+ q\frac{\diff x^\mu}{\diff \tau}A_\mu(x)\right)
		\end{equation}

		\noindent
		Nicht-relativistische Lagrange Funktion:
		\begin{equation}
			\mathcal{L}(t,\vec{x},\dot{\vec{x}}) = \frac{1}{2}m\dot{\vec{x}}^2 - q\phi(t,\vec{x}) - \dot{\vec{x}}\cdot\vec{A}(t,\vec{x})
		\end{equation}

		\noindent
		Lorentzkraft (relativistische Minkowski-Kraft und Newton'sche Kraft):
		\begin{equation}
			\begin{aligned}
				\frac{\diff P^\mu}{\diff \tau} &= q F^{\mu\nu}\frac{\diff x_\nu}{\diff \tau} \\
				\vec{F} &= q\left(\vec{E}+\vec{v}\times\vec{B}\right) \\
			\end{aligned}
		\end{equation}

	\subsection{Feldgleichungen}
		\noindent
		Lagrangedichte:
		\begin{equation}
			\mathcal{L} = -\frac{1}{4\mu_0}F^{\mu\nu} F_{\mu\nu} - A_\mu J^\mu
		\end{equation}

		\noindent
		Allgemeine (ungeeichte) Entwicklungsgleichung:
		\begin{equation}
			\Box A^\mu-\partial^\mu\left(\partial_\nu A^\nu\right) = \partial_\nu F^{\nu\mu} =  \mu_0 J^\mu
		\end{equation}

		\noindent
		Maxwell-Gleichungen im Vakuum:
		\begin{equation}
		\begin{array}{rl}
			\vec{\nabla}\times \vec{E} + \cfrac{\partial\vec{B}}{\partial t} = 0 \phantom{\mu_0}
			&\hspace{30pt} \vec{\nabla}\cdot\vec{E} = \cfrac{\rho}{\varepsilon_0} \\ \xrowht{40pt}
			\vec{\nabla}\times\vec{B} - \mu_0 \epsilon_0 \cfrac{\partial \vec{E}}{\partial t} = \mu_o\vec{j}
			&\hspace{30pt} \vec{\nabla}\cdot\vec{B} = 0 \\
		\end{array}
		\end{equation}

		\noindent
		Maxwell-Gleichungen in Materie:
		\begin{equation}
		\begin{array}{rl}
			\vec{\nabla}\times \vec{E} + \cfrac{\partial\vec{B}}{\partial t} = 0 \phantom{_f}
			&\hspace{20pt} \vec{\nabla}\cdot\vec{D} = \rho_f\\ \xrowht{40pt}
			\vec{\nabla}\times\vec{H} - \cfrac{\partial \vec{D}}{\partial t} = \vec{j}_f
			&\hspace{20pt} \vec{\nabla}\cdot\vec{B} = 0 \\
		\end{array}
		\end{equation}

		\noindent
		Elektrische Flussdichte und Magnetische Feldstärke:
		\begin{equation} \label{eq:flussdichte-feldstärke}
		\begin{array}{cc}
			\vec{H} := \cfrac{1}{\mu_0}\Vec{B} - \vec{M}\left(\vec{B}\right)
			&\hspace{20pt} \vec{D} := \varepsilon_0\vec{E} + \vec{P}\left(\vec{E}\right)
		\end{array}
		\end{equation}


		\noindent
		Globale Maxwell-Gleichungen (Mit elektrischer Spannung $U(\Gamma)=\int_\Gamma \vec{E}\cdot\diff\vec{l}$, Elektrischem Fluss $\Psi(\mathcal{A})=\int_\mathcal{A}\vec{D}\cdot\vec{n}\;\diff A$, magnetischer Spannung $V(\Gamma)=\int_\Gamma \vec{H}\cdot\diff\vec{l}$ und magnetischem Fluss $\Phi(\mathcal{A})=\int_\mathcal{A}\vec{B}\cdot\vec{n}\;\diff A$):
		\begin{equation}
		\begin{array}{rl}
			U(\partial \mathcal{A}) + \cfrac{\diff\Phi}{\diff t}(\mathcal{A}) = 0\phantom{(\mathcal{A})}
			&\hspace{20pt} \Psi(\partial\mathcal{V}) = Q_f(\mathcal{V}) \\ \xrowht{40pt}
			V(\partial \mathcal{A}) - \cfrac{\diff \Psi}{\diff t}(\mathcal{A}) = I_f(\mathcal{A})
			&\hspace{20pt} \Phi(\partial\mathcal{V}) = 0 \\
		\end{array}
		\end{equation}

		\subsubsection{Diskontinuitätsgleichungen}
			\noindent
			Randbedingungen im Vakuum (Oberflächenladungsdichte $\sigma$, Oberflächenladungsstromdichte $\vec{K}$):
		\begin{equation}
		\begin{array}{cc}
			\vec{E}^2 - \vec{E}^1 = \cfrac{\sigma}{\varepsilon_0}\Vec{n}^2
			&\hspace{20pt} \vec{B}^2 - \vec{B}^1 = \mu_0\vec{K}\times\Vec{n}^2
		\end{array}
		\end{equation}

			\noindent
			Randbedingungen in Materie:
		\begin{equation}
		\begin{array}{cc}
			D_\perp^2 - D_\perp^1 = \sigma_f
			&\hspace{20pt} \vec{H}_\parallel^2 - \vec{H}_\parallel^1 = \vec{K}_f\times\Vec{n}^2
		\end{array}
		\end{equation}

	\subsection{Elektrodynamik}
		\noindent
		Kontinuitätsgleichung (direkte Folge der Maxwell-Glechungen):
		\begin{equation}
			\partial_\mu J^\mu = \pder{\rho}{t} + \Nabla\cdot\vec{j} = 0
		\end{equation}

		\noindent
		In integrierter Form:
		\begin{equation}
			\dot{Q}(\mathcal{V}) = I(\partial\mathcal{V})
		\end{equation}

		\noindent
		Lorentz-Invarianten des Elektromagnetische Feldes:
		\begin{equation}
			\begin{aligned}
				F_{\mu\nu} F^{\mu\nu} = \mathrm{invariant} &\Rightarrow \vec{E}^2 - \vec{B}^2 = \mathrm{invariant} \\
				F_{\mu\nu} F^{\nu}_{\;\gamma} F^{\gamma}_{\;\rho} F^{\rho\mu} = \mathrm{invariant} &\Rightarrow \vec{E}\cdot \vec{B} = \mathrm{invariant} \\
			\end{aligned}
		\end{equation}

		\subsubsection{Elektromagnetische Energie und Impuls}
			\noindent
			Energiedichte:
			\begin{equation}
				u_{EM}=\frac{1}{2}\epsilon_0 \vec{E}^2+\frac{1}{2\mu_0}\vec{B}^2
			\end{equation}

			\noindent
			Poynting Vektor / Energiestromdichte:
			\begin{equation}
				\vec{S} = \frac{1}{\mu_0}\vec{E}\times\vec{B}
			\end{equation}

			\noindent
			Poynting Theorem / Energie-Kontinuität:
			\begin{equation}
				-\Nabla\cdot\vec{S}
				= \pder{u_{\mathrm{mech}}}{t} + \pder{u_{EM}}{t}
				= \vec{j}\cdot\vec{E} + \frac{1}{2}\pder{}{t}\Br{\varepsilon_0\vec{E}^2 + \frac{1}{\mu_0}\vec{B}^2}
			\end{equation}

			\noindent
			Impulsdichte:
			\begin{equation}
				\vec{\pi} = \tder{\vec{p}_{EM}}{V} = \frac{1}{c^2}\vec{S}
			\end{equation}

			\noindent
			Impulserhaltung (Maxwellscher Spannungstensor $T$):
			\begin{equation}
				\pder{\vec{p}_{EM}}{t} - \Nabla\cdot T + \rho \vec{E} + \vec{J}\times\vec{B} = 0
			\end{equation}

			\noindent
			Energieerhaltung und Impulserhaltung (Mit Kraft $F$, Leistung $P$, Poynting-Vektor $S_i$, Spannungstensor $T_{ij}$, Impulsdichte $\pi_i$):
			\begin{equation}
				\begin{aligned}
					\pder{u_{EM}}{t} &= \tder{P}{V} - \pder{S_j}{x_j} \\
					\pder{\pi_i}{t} &=	\tder{F_i}{V} - \pder{T_{ij}}{x_j}
				\end{aligned}
			\end{equation}

		\subsubsection{Elektromagnetischer Energie-Impuls-Tensor}
			\noindent
			Energie-Impuls-Tensor (Wobei $T_{ij}$ der Maxwell'sche Spannungstensor ist):
			\begin{equation}
				T^{\mu\nu} = \frac{1}{\mu_0}\left(g^{\mu\alpha} F_{\alpha\beta} F^{\beta\nu} +\frac{1}{4}g^{\mu\nu} F_{\alpha\beta} F^{\alpha\beta} \right)
				= \left( \begin{matrix}
					u_{EM} & S_1/c & S_2/c & S_3/c \\
					S_1/c & -T_{11} & -T_{12} & -T_{13} \\
					S_2/c	& -T_{21} & -T_{22} & -T_{23} \\
					S_3/c & -T_{31} & -T_{32} & -T_{33}
				\end{matrix} \right)
			\end{equation}

			\noindent
			Maxwell'scher Spannungstensor:
			\begin{equation}
				T_{ij} = \varepsilon_0 \Br{ E_i E_j + c^2 B_i B_j - \frac{\delta_{ij}}{2}\Br{ \vec{E}^2 + c^2 \vec{B}^2 } }
			\end{equation}

			\noindent
			Eigenschaften:
			\begin{equation}
				\begin{aligned}
					T^{\mu\nu} &= T^{\nu\mu} &\hspace{20pt}
					T^\mu_{\phantom{\mu}\mu} &= 0 \\
				\end{aligned}
			\end{equation}

			\noindent
			Energie und Impulserhaltung:
			\begin{equation}
				\begin{aligned}
					\partial_\mu T^{\mu\nu} + F^{\nu\lambda} J_\lambda &= 0
					%\partial_\mu T^{\mu\nu} &= 0
				\end{aligned}
			\end{equation}


		\subsubsection{Elektromagnetische Wellen im Vakuum}
			\noindent
			Wellengleichung im Vakuum:
			\begin{equation}
				\Box\psi = \frac{1}{c^2} \frac{\partial^2 \psi}{\partial t^2} - \Nabla^2 \psi = 0
			\end{equation}

			\noindent
			Vektorwelle (In der Elektrodynamik $\vec{\psi}=\vec{E}$ und $ \vec{\psi}=\vec{B}$):
			\begin{equation}
				\vec{\psi}(t,\vec{r}) = \int_{\mathbb{R}^3} \tilde{\vec{\psi}} (\vec{k}) e^{\i\left(\vec{k}\cdot\vec{r} - \omega(\vec{k})t \right)}\; \diff^3\vec{k}
			\end{equation}

			\noindent
			Orthogonalität des Wellenvektors (Elektromagnetische Wellen im Vakuum sind immer transversal):
			\begin{equation}
				\tilde{\vec{B}} = \frac{1}{c}\hat{\vec{k}}\times\tilde{\vec{E}} = \frac{1}{\omega}\vec{k}\times\tilde{\vec{E}}
			\end{equation}

			\noindent
			Energiedichte einer linear polarisierten Welle:
			\begin{equation}
				u_{EM} = \frac{1}{2}\epsilon_0\vec{E}^2
			\end{equation}

			\noindent
			Energiedichte einer zirkular polarisierten Welle:
			\begin{equation}
				u_{EM} = \epsilon_0\vec{E}^2
			\end{equation}

			\noindent
			Energiestromdichte einer elektromagnetischen Welle:
			\begin{equation}
				\vec{S} = u_{EM}c\hat{\vec{k}}
			\end{equation}

			\noindent
			Intensität einer linear polarisierten Welle:
			\begin{equation}
				I = \langle|\vec{S}|\rangle = \frac{1}{2}c\epsilon_0\vec{E}^2
			\end{equation}

			\noindent
			Intensität einer zirkular polarisierten Welle:
			\begin{equation}
				I = \langle|\vec{S}|\rangle = c\epsilon_0\vec{E}^2
			\end{equation}

			\noindent
			Abgestrahlte Welle eines oszillierenden Dipols $\vec{p} = \vec{p}_0 e^{-\i\omega t}$ (Mit retardierter Zeit $\tilde{t} = t-\frac{\left|\vec{r}-\pvec{r}'\right|}{c}$):
			\begin{equation}
				\begin{aligned}
					\vec{E}(t,\vec{r}) &= -\frac{\mu_0}{4\pi r c}	\left(\vec{n}\times\ddot{\vec{p}}(\tilde{t})\right) \times \vec{n} \\
					\vec{B}(t,\vec{r}) &= -\frac{\mu_0}{4\pi r c} \phantom{\Big(}\vec{n}\times\ddot{\vec{p}}(\tilde{t}) \\
				\end{aligned}
			\end{equation}

			\noindent
			Larmor'sche Formel (für die von einem nicht-relativistischen Teilchen mit retardierter Beschleunigung $a(\tilde{t}$ abgestrahlte Energie):
			\begin{equation}
				P(t) = \frac{q^2 \mu_0}{6\pi c}a^2(\tilde{t})
			\end{equation}

		\subsubsection{Elektromagnetische Wellen in Medien}
			\noindent
			Definition des Brechungsindex (Mit Phasengeschwindigkeit $v_{PH}$):
			\begin{equation}
				n(k) := \frac{c}{v_{PH}(k)} = \sqrt{\frac{\epsilon\mu}{\epsilon_0\mu_0}}
			\end{equation}

			\noindent
			Snellius'sches Brechungsgesetz:
			\begin{equation}
				\frac{\sin\alpha_1}{\sin\alpha_2} = \frac{n_2}{n_1}
			\end{equation}



		\subsubsection{Elektrotechnik}
			\noindent
			Ohm'sches Gesetz (Mit spezifischer Leitfähigkeit $\sigma$ und Widerstand $R$, die beide als linear angenommen werden):
			\begin{equation}
				\begin{aligned}
					\vec{j} &= \sigma\vec{E} \\
					U &= R I
				\end{aligned}
			\end{equation}

			\noindent
			Kirchhoff'sche Regeln:
			\begin{description}
				\item[1. Knotenregel]\hfill \\
					An jedem Knoten ist die Summe der einfließenden elektrischen und ausfließenden elektrischen Ströme gleich null (Kontinuitätsgleichung).
				\item[2. Maschenregel]\hfill \\
					Alle Teilspannungen eines Umlaufs bzw. einer Masche addieren sich zu null (Elektrostatik)
			\end{description}

			\noindent
			Joule'sches Wärmegesetz (Wärmeerzeugung in einem ohm'schen Leiter):
			\begin{equation}
				P = UI = RI^2 = \frac{U^2}{R}
			\end{equation}



	\subsection{Definitionen für die Elektrodynamik}
		\noindent
		Lichtgeschwindigkeit:
		\begin{equation}
			c=\frac{1}{\sqrt{\epsilon_0 \mu_0}}
		\end{equation}

		\noindent
		D'Alembert'scher Operator:
		\begin{equation}
			\Box = \partial^\mu \partial_\mu = \frac{1}{c^2}\frac{\partial^2}{\partial t^2} - \vec{\nabla}^2
		\end{equation}

		\noindent
		Viererstromdichte:
		\begin{equation}
			J^\mu = \binomkoeff{c\rho}{\vec{j}} = \binomkoeff{c\tder{q}{V}}{\vec{v}\tder{q}{V}}
		\end{equation}

		\noindent
		Viererpotential:
		\begin{equation}
			A^\mu = \binomkoeff{\phi/c}{\vec{A}}
		\end{equation}

		\noindent
		Faraday-Tensor / Feldstärken tensor:
		\begin{equation}
			F^{\mu\nu} = \partial^\mu A^\nu - \partial^\nu A^\mu
			= \left( \begin{matrix}
				0 & -E_x/c & -E_y/c & -E_z/c \\
				E_x/c & 0 & -B_z & B_y \\
				E_y/c	& B_z & 0 & -B_x \\
				E_z/c & -B_y & B_x & 0
			\end{matrix} \right)
		\end{equation}

		\noindent
		Dualer Feldstärkentensor:
		\begin{equation}
			\tilde{F}^{\mu\nu} = \frac{1}{2}\varepsilon^{\mu\nu\alpha\beta}F_{\alpha\beta}
		\end{equation}

		\noindent
		Elektrische Feldstärke:
		\begin{equation}
			\vec{E} = -\Nabla\phi-\pder{\vec{A}}{t}
		\end{equation}

		\noindent
		Magnetische Flussdichte:
		\begin{equation}
			\vec{B} = \Nabla\times\vec{A}
		\end{equation}

	\subsection{Eichtransformation}
		\noindent
		Eichtransformation:
		\begin{equation}
			A^\mu \rightarrow A^\mu-\partial^\mu \lambda
		\end{equation}

		\subsubsection{Lorenz-Eichung}
			\noindent
			Lorenz-Eichung:
			\begin{equation}
				\partial_\mu A^\mu = \frac{1}{c}\frac{\partial \phi}{\partial t} + \vec{\nabla}\cdot\vec{A} = 0
			\end{equation}

			\noindent
			Maxwellgleichungen in der Lorentz-Eichung:
			\begin{equation}
				\Box A^\mu = \mu_0 J^\mu \;\Leftrightarrow\;
				\Box \phi = \dfrac{\rho}{\epsilon_0} \;\wedge\;
				\Box \vec{A} = \mu_0 \vec{j}
			\end{equation}

		\subsubsection{Coulomb-Eichung}
			\noindent
			Coulomb-Eichung (nicht eindeutig):
			\begin{equation}
				\Nabla\cdot\vec{A}(t,\vec{r})=0
			\end{equation}

	\subsection{Spezielle Lösungen der Feldgleichungen}

		\subsubsection{Retardierte Potentiale}
			\noindent
			Retardierte Potentiale (Allgemeine Lösung der nicht-statischen Maxwellgleichungen mit Lorenz-Eichung für Strom- und Ladungsverteilung die zu jedem Zeitpunkt bekannt sind):
			\begin{equation}
				\begin{aligned}
					\phi\left(t,\vec{r}\right)
					= \frac{1}{4\pi\epsilon_0} \int_{\mathbb{R}^3} \frac{\rho(\tilde{t},\pvec{r}')}{\left|\vec{r}-\pvec{r}'\right|}\;\diff^3 \pvec{r}'
					&=	\frac{1}{4\pi\epsilon_0} \int_{\mathbb{R}^4} \frac{\rho(t',\pvec{r}')}{\left|\vec{r}-\pvec{r}'\right|}\delta\left( c(t-t')-\left|\vec{r}-\pvec{r}'\right|\right)\,\diff^3 \pvec{r}'\diff t' \\
					\vec{A}\left(t,\vec{r}\right)
					= \phantom{\epsilon_0} \frac{\mu_0}{4\pi} \int_{\mathbb{R}^3} \frac{\vec{j}(\tilde{t},\pvec{r}')}{\left|\vec{r}-\pvec{r}'\right|}\,\diff^3 \pvec{r}'
					&=	\phantom{\epsilon_0} \frac{\mu_0}{4\pi} \int_{\mathbb{R}^4} \frac{\vec{j}(t',\pvec{r}')}{\left|\vec{r}-\pvec{r}'\right|}\delta\left( c(t-t')-\left|\vec{r}-\pvec{r}'\right|\right)\;\diff^3 \pvec{r}'\diff t' \\
					\tilde{t}&=t-\frac{\left|\vec{r}-\pvec{r}'\right|}{c}
				\end{aligned}
			\end{equation}

		\subsubsection{Liénard-Wiechert-Potentiale}

			\noindent
			Lösung der Feldgleichungen für eine bewegte Punktladung mit Ort $\vec{u}(\tilde{t})$ und Geschwindigkeit $\vec{v}(\tilde{t})$ mit der retardierten Zeit $\tilde{t}$. Liénard-Wiechert-Potentiale:
			\begin{equation}
				\begin{aligned}
					\phi(t,\vec{r}) & =\frac{q c}{4\pi \epsilon_0}\frac{1}{c \left|\vec{r}-\vec{u}(\tilde{t})\right|-\vec{v}(\tilde{t})\cdot\left(\vec{r}-\vec{u}(\tilde{t})\right)} \\
					\vec{A}(t,\vec{r}) &= \frac{1}{c^2}\vec{v}(\tilde{t})\phi(t,\vec{r}) \\
				\end{aligned}
			\end{equation}

	\subsection{Elektrostatik}
		\subsubsection{Allgemeines}

			\noindent
			Coulomb-Kraft (Die von Ladung 2 auf Ladung 1 wirkt):
			\begin{equation}
				\vec{F}_1 = \frac{q_1 q_2}{4\pi\varepsilon_0}\frac{\vec{r}_1-\vec{r}_2}{\left|\vec{r}_1-\vec{r}_2\right|^3}
			\end{equation}

			\noindent
			Poisson-Gleichung der Elektrostatik:
			\begin{equation}
				\Nabla^2\phi = -\frac{\rho}{\epsilon_0}
			\end{equation}

			\noindent
			Elektrisches Potential in der Elektrostatik (folgt aus zeitunabhängigem retardiertem Potential):
			\begin{equation}
				\phi\left(t,\vec{r}\right)
				= \frac{1}{4\pi\epsilon_0} \int_{\mathbb{R}^3} \frac{\rho(\pvec{r}')}{\left|\vec{r}-\pvec{r}'\right|}\;\diff^3 \pvec{r}'
			\end{equation}

		\subsubsection{Lösen der Poisson-Gleichung}
			\noindent
			Dirichlet-Green'sche Funktion $G_D$:
			\begin{equation}
				\forall\pvec{r}\in\mathcal{V}': \left\{\begin{array}{ll}
						\forall\vec{r}\in\mathcal{V}\phantom\partial:
						\Nabla^2 G_D(\vec{r},\pvec{r}') = \delta(\vec{r}-\pvec{r}') \\
						\forall\vec{r}\in\partial\mathcal{V}:
						\phantom{\Nabla^2}G_D(\vec{r},\pvec{r}') = 0
					\end{array}\right.
			\end{equation}

		\subsubsection{Multipolentwicklung}
			\noindent
			Multipolentwicklung in kartesischen Koordinaten (Entwicklung der allgemeinen Lösung in $\frac{r'}{r}=0$):
			\begin{equation}
				\begin{aligned}
					\phi(\pvec{r}) = \frac{1}{4\pi\epsilon_0}\int\frac{\rho(\pvec{r}')}{\left|\vec{r}-\pvec{r}'\right|}\;\diff \pvec{r}'
					&= \frac{1}{4\pi\epsilon_0}\left(\frac{Q}{r} + \frac{\vec{r}\cdot\vec{p}}{r^3} + \frac{1}{2}\frac{r_i r_j Q_{ij}}{r^5} + \mathcal{O} \left(\frac{1}{r^4}\right)\right) \\
					&= \frac{1}{4\pi\epsilon_0}\frac{1}{r}\int\rho(\pvec{r}')\;\diff \pvec{r}' \\
					&+ \frac{1}{4\pi\epsilon_0}\frac{1}{r}\int\rho(\pvec{r}')\left(\frac{r'}{r}\right)\cos\theta\;\diff \pvec{r}' \\
					&+ \frac{1}{4\pi\epsilon_0}\frac{1}{r}\int\rho(\pvec{r}')\left(\frac{r'}{r}\right)^2\frac{3\cos^2\theta-1}{2}\;\diff \pvec{r}' \\
					&+ \mathcal{O}\left(\;\;\frac{1}{r}\int\rho(\pvec{r}')\left(\frac{r'}{r}\right)^3 \;\diff \pvec{r}'\right) \\
				\end{aligned}
			\end{equation}
			Mit Monopol $Q$, Dipol $\vec{p}$ und Quadropol $Q_{ij}$:
			\begin{equation}
				\begin{aligned}
					Q &= \int \rho(\pvec{r}') \;\diff^3 \pvec{r}' \\
					\vec{p} &= \int \rho(\pvec{r}')\pvec{r}' \;\diff^3 \pvec{r}' \\
					Q_{ij} &= \int \rho(\pvec{r}')\left(3r'_i r'_j - \delta_{ij} r'^2 \right)	\;\diff^3 \pvec{r}' \\
				\end{aligned}
			\end{equation}

			\noindent
			Entwicklung des Abstandes in Kugelflächenfunktionen:
			\begin{equation}
				\frac{1}{\left|\vec{r}-\pvec{r}'\right|}=\sum_{l=0}^{\infty}\sum_{m=-l}^{l} \frac{4\pi}{2l+1}\frac{r'^l}{r^{l+1}} Y^{*}_{lm}(\theta',\varphi')Y_{lm}(\theta,\varphi)
			\end{equation}

			\noindent
			Multipolentwicklung in sphärischen Koordinaten:
			\begin{equation}
				\begin{aligned}
					\phi(r,\theta,\varphi) &= \sum_{l=0}^{\infty}\sum_{m=-l}^{l} \frac{b_{lm}}{r^{l+1}}Y_{lm}(\theta,\varphi) \\
					b_{lm} &= \frac{1}{(2l+1)\epsilon_0}\int\rho(\pvec{r}')r'^lY^{*}_{lm}(\theta',\varphi')\;\diff^3\pvec{r}'
				\end{aligned}
			\end{equation}

			\noindent
			Elektrisches Dipolmoment zweier Punktladung mit Abstandsvektor $\vec{d}$ von der negativen zur positiven Ladung:
			\begin{equation}
				\vec{p}=q\vec{d}
			\end{equation}

			\noindent
			Kraft, Drehmoment und potentielle Energie eines Dipols (Drehmoment in Bezug auf den Massenschwerpunkt $\vec{r}_m$):
			\begin{equation}
				\begin{aligned}
					\vec{F} &= \Nabla\left(\vec{p}\cdot\vec{E}\right) \\
					\vec{D}(\vec{r}_m) &= \vec{p}\times\vec{E} \\
					\mathcal{E}_{pot} &= -\vec{p}\cdot\vec{E} \\
				\end{aligned}
			\end{equation}

		\subsubsection{Lösungen der Laplace-Gleichung mit sphärischer Symmetrie}
			\noindent
			Lösungen der Laplace-Gleichung mit Axialsymmetrie (Mit Legendre-Polynomen $P_l$ (\ref{eq:legendre-polynome})):
			\begin{equation}
				\phi(r,\theta)=\sum_{l=0}^\infty \left(A_l r^l + \frac{B_l}{r^{l+1}}\right)P_l(\cos\theta)
			\end{equation}

			\noindent
			Lösungen der Laplace-Gleichung ohne Axialsymmetrie (Mit Kugelflächenfunktionen $Y_{lm}$ (\ref{eq:kugelflächenfunktionen})):
			\begin{equation}
				\phi(r,\theta,\varphi) = \sum_{l=0}^{\infty}\sum_{m=-l}^{m=l} \left(a_{lm} r^l + \frac{b_{lm}}{r^{l+1}}\right) Y_{lm}(\theta, \varphi)
			\end{equation}

		\subsubsection{Kapazität}
			\noindent
			Kapazität:
			\begin{equation}
				C=\frac{Q}{U}
			\end{equation}

			\noindent
			Kapazität eines Plattenkondensators mit Fläche $A$, Plattenabstand $d$ und Dielektrikum mit Permittivität $\epsilon$:
			\begin{equation}
				C=\epsilon \frac{A}{d}
			\end{equation}

			\noindent
			Energie eines Kondensators:
			\begin{equation}
				\mathcal{E} = \frac{1}{2}\frac{Q^2}{C} = \frac{1}{2}CU^2
			\end{equation}

		\subsubsection{Dielektrika}
			\noindent
			In Materialien gilt allgemein (\ref{eq:flussdichte-feldstärke}), wobei $\vec{P}$ von $\vec{E}$, $\vec{r}$, $T$, ... abhängen kann. \vspace{10pt}

			\noindent
			Gebundene Ladungsdichte und gebundene Oberflächenladungsdichte:
			\begin{equation}
				\begin{aligned}
					\rho_P(\vec{r}) &= -\Nabla \cdot\vec{P}(\vec{r}) \\
					\sigma_P(\vec{r}) &= \vec{n}(\vec{r})\cdot P(\vec{r})
				\end{aligned}
			\end{equation}

			\noindent
			In homogenen, isotropen, linearen Dielektrika (mit elektrischer Suszeptibilität $\chi_e$) gilt für die Polarisationsdichte $P$:
			\begin{equation}
				\begin{aligned}
					\vec{P} &= \epsilon_0 \chi_e \vec{E} \\
					\epsilon &= \epsilon_0 \epsilon_r = \epsilon_0(1+\chi_e) \\
					\vec{D} &= \epsilon \vec{E} \\
				\end{aligned}
			\end{equation}


	\subsection{Magnetostatik}
		\subsubsection{Allgemeines}
			\noindent
			Biot-Savart'sches Gesetz:
			\begin{equation}
				\vec{B}(\vec{r}) = \frac{\mu_0}{4\pi}\int_{\mathcal{V}} \vec{j}(\pvec{r}')\times\frac{\vec{r}-\pvec{r}'}{\left|\vec{r}-\pvec{r}'\right|^3}\;\diff^3\pvec{r}'
			\end{equation}

			\noindent
			Statische Potentialgleichung
			\begin{equation}
				\Nabla^2 \vec{A} = - \mu_0 \vec{j}
			\end{equation}

			\noindent
			Statisches Vektorpotential (folgt aus zeitunabhängigem retardiertem Vektorpotential):
			\begin{equation}
				\vec{A}(\vec{r}) = \frac{\mu_0}{4\pi} \int_{\mathcal{V}} \frac{\vec{j}(\pvec{r}')}{\left|\vec{r}-\pvec{r}'\right|} \;\diff^3\pvec{r}'
			\end{equation}

			\noindent
			Geladene Teilchen im homogenen magnetischen Feld bewegen sich auf Kreisen mit dem Larmor-Radius $r_L$ und der Zyklotronfrequenz / Gyrationsfrequenz $\omega$:
			\begin{equation}
				\begin{aligned}
					r_L = \frac{p}{qB} &&\hspace{30pt} % horizontal space
					\omega = \frac{|q| B}{m}
				\end{aligned}
			\end{equation}

		\subsubsection{Multipolentwicklung}
			\noindent
			Multipolentwicklung (Entwicklung der allgemeinen Lösung in $\frac{r'}{r}=0$):
			\begin{equation}
				\begin{aligned}
					\vec{A}(\vec{r}) = \int_{\mathcal{V}} \frac{\vec{j}(\pvec{r}')}{\left|\vec{r}-\pvec{r}'\right|} \;\diff^3\pvec{r}'
					=& \frac{\mu_0}{4\pi} \left(\frac{\vec{m}\times\vec{r}}{r^3} + \mathcal{O}\left(\frac{1}{r^3}\right)\right) \\
					=& \frac{\mu_0}{4\pi}\frac{1}{r}\int\vec{j}(\pvec{r}')\;\diff \pvec{r}' \\
					&+ \frac{\mu_0}{4\pi}\frac{1}{r}\int\vec{j}(\pvec{r}')\left(\frac{r'}{r}\right)\cos\theta\;\diff \pvec{r}' \\
					&+ \frac{\mu_0}{4\pi}\frac{1}{r}\int\vec{j}(\pvec{r}')\left(\frac{r'}{r}\right)^2\frac{3\cos^2\theta-1}{2}\;\diff \pvec{r}' \\
					&+ \mathcal{O}\left(\;\frac{1}{r}\int\vec{j}(\pvec{r}')\left(\frac{r'}{r}\right)^3 \;\diff \pvec{r}'\right) \\
				\end{aligned}
			\end{equation}

			\noindent
			Magnetisches Dipolmoment $\vec{m}$:
			\begin{equation}
				\begin{aligned}
					\vec{m} = \frac{1}{2}\int \pvec{r}'\times\vec{j}(\pvec{r}')\;\diff^3\pvec{r}'
				\end{aligned}
			\end{equation}

			\noindent
			Magnetisches Dipolmoment einer Stromschleife mit Flächeninhalt $A$ in der ein Strom $I$ fließt:
			\begin{equation}
				\vec{m} = IA\vec{n}
			\end{equation}

			\noindent
			Kraft, Drehmoment und potentielle Energie eines Dipols (Drehmoment in Bezug auf den Massenschwerpunkt $\vec{r}_m$):
			\begin{equation}
				\begin{aligned}
					\vec{F} &= \Nabla\left(\vec{m}\cdot\vec{B}\right) \\
					\vec{D}(\vec{r}_m) &= \vec{m}\times\vec{B} \\
					\mathcal{E}_{pot} &= -\vec{m}\cdot\vec{B} \\
				\end{aligned}
			\end{equation}

		\subsubsection{Magnete}
			\noindent
			Gebunde Stromdichte und gebundene Oberflächenstromdichte:
			\begin{equation}
				\begin{aligned}
					\vec{j}_m(\vec{r}) &= \Nabla\times\vec{M}(\vec{r}) \\
					\vec{K}_m(\vec{r}) &= \vec{M}(\vec{r})\times \vec{n}(\vec{r}) \\
				\end{aligned}
			\end{equation}

		\subsubsection{lineare Medien}
			\noindent
			In linearen, homogenen, isotropen, magnetischen Medien (Mit magnetischer Suszeptibilität $\chi_m$, Permabilität $\mu$) gilt für die magnetisierungsdichte $M$:
			\begin{equation}
				\begin{aligned}
					\vec{M} &= \chi_m\vec{H} \\
					\mu &= \mu_0 \mu_r = \mu_0(1+\chi_m) \\
					\vec{B} &= \mu \vec{H} \\
				\end{aligned}
			\end{equation}
