% !TEX root = ../physics.tex
\section{Electromagnetism\index{Elektromagnetismus}}
	Conventions used in this section:
	\begin{itemize}
		\item Metric signature: $\eta_{\mu\nu} = \mathrm{diag}(+,-,-,-)_{\mu\nu}$
	\end{itemize}


	\subsection{Definitions}
		\noindent
		Speed of light\index{Lichtgeschwindigkeit}:
		\begin{equation}
			c=\frac{1}{\sqrt{\varepsilon_0 \mu_0}}
		\end{equation}

		\noindent
		D'Alembert operator\index{D'Alembert!Operator}:
		\begin{equation}
			\Box = \partial^\mu \partial_\mu = \frac{1}{c^2}\frac{\partial^2}{\partial t^2} - \vec{\nabla}^2
		\end{equation}

		\noindent
		Four-current\index{Viererstromdichte}:
		\begin{equation}
			J^\mu
			= \mqty(c\rho\\\vec{j})^\mu
			= \mqty(c\dv{q}{V}\\\vec{v}\dv{q}{V})^\mu
		\end{equation}

		\noindent
		Four-potential\index{Viererpotential}:
		\begin{equation}
			A^\mu
			= \mqty(\phi/c\\\vec{A})^\mu
		\end{equation}

		\noindent
		Faraday tensor\index{Faraday!tensor} / electromagnetic tensor / field strength tensor\index{Faraday!Feldstärkentensor}:
		\begin{equation}
			F^{\mu\nu} = \partial^\mu A^\nu - \partial^\nu A^\mu
			= \left( \begin{matrix}
				0     & -E_x/c & -E_y/c & -E_z/c \\
				E_x/c & 0      & -B_z   & B_y    \\
				E_y/c & B_z    & 0      & -B_x   \\
				E_z/c & -B_y   & B_x    & 0
			\end{matrix} \right)^{\mu\nu}
		\end{equation}

		\noindent
		Hodge dual of the Faraday Tensor\index{Faraday!Dualer Feldstärkentensor}:
		\begin{equation}
			\tilde{F}^{\mu\nu} = \frac{1}{2}\varepsilon^{\mu\nu\alpha\beta}F_{\alpha\beta}
		\end{equation}

		\noindent
		Electric field strength:
		\begin{equation}
			\vec{E} = -\Nabla\phi-\pdv{\vec{A}}{t}
		\end{equation}

		\noindent
		Magnetic flux density:
		\begin{equation}
			\vec{B} = \Nabla\times\vec{A}
		\end{equation}


	\subsection{Field Equations}
		\noindent
		Lagrange density (Note that in Gaußian cgs units, the dimensions of quantities change due to the choice $\mathcal{L} = -\frac{1}{16\pi}F^{\mu\nu} F_{\mu\nu} - \frac{1}{c} A_\mu J^\mu$.):
		\begin{equation}
			\mathcal{L} = -\frac{1}{4\mu_0}F^{\mu\nu} F_{\mu\nu} - A_\mu J^\mu
		\end{equation}

		\noindent
		General field equations without gauge \index{Entwicklungsgleichung}:
		\begin{equation}
			\Box A^\mu-\partial^\mu\left(\partial_\nu A^\nu\right) = \partial_\nu F^{\nu\mu} =  \mu_0 J^\mu
		\end{equation}

		\noindent
		Maxwell's equations\index{Maxwell!Gleichungen} in a vacuum:
		\begin{equation}
			\begin{array}{rl}
				\vec{\nabla}\times \vec{E} + \cfrac{\partial\vec{B}}{\partial t} = 0 \phantom{\mu_0}
				& \hsp \vec{\nabla}\cdot\vec{E} = \cfrac{\rho}{\varepsilon_0} \\ \xrowht{40pt}
				\vec{\nabla}\times\vec{B} - \mu_0 \varepsilon_0 \cfrac{\partial \vec{E}}{\partial t} = \mu_o\vec{j}
				& \hsp \vec{\nabla}\cdot\vec{B} = 0
			\end{array}
		\end{equation}

		\noindent
		Maxwell's equations\index{Maxwell!Gleichungen} in matter:
		\begin{equation}
			\begin{array}{rl}
				\vec{\nabla}\times \vec{E} + \cfrac{\partial\vec{B}}{\partial t} = 0 \phantom{_\text{f}}
				& \hsp \vec{\nabla}\cdot\vec{D} = \rho_\text{f} \\ \xrowht{40pt}
				\vec{\nabla}\times\vec{H} - \cfrac{\partial \vec{D}}{\partial t} = \vec{j}_\text{f}
				& \hsp \vec{\nabla}\cdot\vec{B} = 0
			\end{array}
		\end{equation}

		\noindent
		Electric flux density and magnetic field strength dependence on magnetization and polarization:
		\begin{equation} \label{Eq:FluxDensityFieldStrength}
			\begin{array}{cc}
				\vec{H} := \cfrac{1}{\mu_0}\Vec{B} - \vec{M}\left(\vec{B}\right)
				& \hsp \vec{D} := \varepsilon_0\vec{E} + \vec{P}\left(\vec{E}\right)
			\end{array}
		\end{equation}


		\noindent
		Maxwell's equations\index{Maxwell!Gleichungen} in their integrated form (Electric voltage $U(\Gamma)=\int_\Gamma \vec{E}\cdot\dd\vec{l}$, electric flux $\Psi(\mathcal{A})=\int_\mathcal{A}\vec{D}\cdot\vec{n}\;\dd A$, magnetic voltage $V(\Gamma)=\int_\Gamma \vec{H}\cdot\dd\vec{l}$ and magnetic flux $\Phi(\mathcal{A})=\int_\mathcal{A}\vec{B}\cdot\vec{n}\;\dd A$):
		\begin{equation}
			\begin{array}{rl}
				U(\partial \mathcal{A}) + \cfrac{\dd\Phi}{\dd t}(\mathcal{A}) = 0\phantom{(\mathcal{A})}
				& \hsp \Psi(\partial\mathcal{V}) = Q_\text{f}(\mathcal{V}) \\ \xrowht{40pt}
				V(\partial \mathcal{A}) - \cfrac{\dd \Psi}{\dd t}(\mathcal{A}) = I_\text{f}(\mathcal{A})
				& \hsp \Phi(\partial\mathcal{V}) = 0
			\end{array}
		\end{equation}

		\subsubsection{Discontinuity Equations\index{Diskontinuitätsgleichungen}}
			\noindent
			Boundary conditions in a vacuum (Surface charge density $\sigma$, surface charge current density $\vec{K}$):
			\begin{equation}
				\begin{array}{cc}
					\vec{E}^2 - \vec{E}^1 = \cfrac{\sigma}{\varepsilon_0}\Vec{n}^2
					& \hsp \vec{B}^2 - \vec{B}^1 = \mu_0\vec{K}\times\Vec{n}^2
				\end{array}
			\end{equation}

			\noindent
			Boundary conditions in matter:
			\begin{equation}
				\begin{array}{cc}
					D_\perp^2 - D_\perp^1 = \sigma_\text{f}
					& \hsp \vec{H}_\parallel^2 - \vec{H}_\parallel^1 = \vec{K}_\text{f}\times\Vec{n}^2
				\end{array}
			\end{equation}

	\subsection{Electrodynamics\index{Elektrodynamik}}
		\noindent
		Continuity equation\index{Kontinuitätsgleichung!Elektromagnetismus}:
		\begin{equation}
			\partial_\mu J^\mu = \pdv{\rho}{t} + \Nabla\cdot\vec{j} = 0
		\end{equation}

		\noindent
		Integrated form:
		\begin{equation}
			\dot{Q}(\mathcal{V}) = I(\partial\mathcal{V})
		\end{equation}

		\noindent
		Lorentz invariants of the electromagnetic field:
		\begin{equation}
			\begin{aligned}
				F_{\mu\nu} F^{\mu\nu} = \vec{E}^2 - \vec{B}^2 = \mathrm{invariant} \\
				\sqrt{ F_{\mu\nu} F^{\nu}_{\;\gamma} F^{\gamma}_{\;\rho} F^{\rho\mu}} = \vec{E}\cdot \vec{B} = \mathrm{invariant} \\
			\end{aligned}
		\end{equation}

		\subsubsection{Electromagnetic Energy and Momentum}
			\noindent
			Energy density:
			\begin{equation}
				u_{\text{EM}}=\frac{1}{2}\varepsilon_0 \vec{E}^2+\frac{1}{2\mu_0}\vec{B}^2
			\end{equation}

			\noindent
			Poynting vector\index{Poynting!Vektor} / Energy flux density:
			\begin{equation}
				\vec{S} = \frac{1}{\mu_0}\vec{E}\times\vec{B}
			\end{equation}

			\noindent
			Poynting theorem\index{Poynting!Theorem} / Energy continuity:
			\begin{equation}
				-\Nabla\cdot\vec{S}
				= \pdv{u_{\mathrm{mech}}}{t} + \pdv{u_{\text{EM}}}{t}
				= \vec{j}\cdot\vec{E} + \frac{1}{2}\pdv{t}\qty(\varepsilon_0\vec{E}^2 + \frac{1}{\mu_0}\vec{B}^2)
			\end{equation}

			\noindent
			Momentum density:
			\begin{equation}
				\vec{\pi} = \dv{\vec{p}_{\text{EM}}}{V} = \frac{1}{c^2}\vec{S}
			\end{equation}

			\noindent
			Conservation of Momentum (Maxwell stress tensor\index{Maxwell!Spannungstensor} $T$):
			\begin{equation}
				\pdv{\vec{p}_{\text{EM}}}{t} - \Nabla\cdot T + \rho \vec{E} + \vec{J}\times\vec{B} = 0
			\end{equation}

			\noindent
			Conservation of energy and momentum (Force $F$, power $P$, Poynting vector $S_i$, stress tensor $T_{ij}$, momentum density $\pi_i$):
			\begin{equation}
				\begin{aligned}
					\pdv{u_{\text{EM}}}{t} &= \dv{P}{V} - \pdv{S_j}{x_j} \\
					\pdv{\pi_i}{t} &=	\dv{F_i}{V} - \pdv{T_{ij}}{x_j}
				\end{aligned}
			\end{equation}

		\subsubsection{Electromagnetic Energy-Momentum Tensor\index{Energie-Impuls-Tensor}}
			\noindent
			Energy-momentum tensor (Where $\vec{S}$ is the Poynting vector\index{Poynting!Vektor} and $T_{ij}$ the Maxwell stress tensor):
			\begin{equation}
				T^{\mu\nu} = \frac{1}{\mu_0}\left(g^{\mu\alpha} F_{\alpha\beta} F^{\beta\nu} +\frac{1}{4}g^{\mu\nu} F_{\alpha\beta} F^{\alpha\beta} \right)
				= \left( \begin{matrix}
						u_{\text{EM}} & S_1/c   & S_2/c   & S_3/c   \\
						S_1/c         & -T_{11} & -T_{12} & -T_{13} \\
						S_2/c         & -T_{21} & -T_{22} & -T_{23} \\
						S_3/c         & -T_{31} & -T_{32} & -T_{33}
					\end{matrix} \right) \\
			\end{equation}
			
			\noindent
			Maxwell stress tensor\index{Maxwell!Spannungstensor}:
			\begin{equation}
				T_{ij} = \varepsilon_0 \qty( E_i E_j + c^2 B_i B_j - \frac{\delta_{ij}}{2}\qty(\vec{E}^2 + c^2 \vec{B}^2 ) )
			\end{equation}
			
			\noindent
			Properties of the electromagnetic energy-momentum tensor:
			\begin{equation}
				T^{\mu\nu} = T^{\nu\mu}
				\hsp
				T^\mu_{\phantom{\mu}\mu} = 0
				\hsp
				T^{00} \ge 0
			\end{equation}

			\noindent
			Conservation of energy and momentum:
			\begin{equation}
				\begin{aligned}
					\partial_\mu T^{\mu\nu} + F^{\nu\lambda} J_\lambda &= 0
				\end{aligned}
			\end{equation}


		\subsubsection{Electromagnetic Waves in a Vacuum}
			\noindent
			Wave equation in a vacuum\index{Wellengleichung}:
			\begin{equation}
				\Box\psi = \frac{1}{c^2} \frac{\partial^2 \psi}{\partial t^2} - \Nabla^2 \psi = 0
			\end{equation}

			\noindent
			General vector wave (e.g. $\vec{\psi}=\vec{E}$ or $ \vec{\psi}=\vec{B}$):
			\begin{equation}
				\vec{\psi}(t,\vec{r}) = \int_{\mathbb{R}^3} \tilde{\vec{\psi}} (\vec{k}) \ex^{\i\left(\vec{k}\cdot\vec{r} - \omega(\vec{k})t \right)}\;\dd^3\vec{k}
			\end{equation}

			\noindent
			Orthogonality of the wave vector (Electromagnetic waves in a vacuum are always transversal):
			\begin{equation}
				\tilde{\vec{B}} = \frac{1}{c}\hat{\vec{k}}\times\tilde{\vec{E}} = \frac{1}{\omega}\vec{k}\times\tilde{\vec{E}}
			\end{equation}

			\noindent
			Dispersion relation\index{Dispersionsrelation} (reduces to $v=c$ in vacuum)
			\begin{equation}
				v = \dv{\omega}{k}
			\end{equation}

			\noindent
			Energy flux density of an electromagnetic wave:
			\begin{equation}
				\vec{S} = u_{\text{EM}}c\hat{\vec{k}}
			\end{equation}

			\noindent
			Energy density and intensity of a linearly polarized wave:
			\begin{equation}
				\begin{aligned}
					u_{\text{EM}} &= \frac{1}{2}\varepsilon_0\vec{E}^2 &\hsp
					I &= \langle|\vec{S}|\rangle = \frac{1}{2}c\varepsilon_0\vec{E}^2
				\end{aligned}
			\end{equation}

			\noindent
			Energy density and intensity of a circularly polarized wave:
			\begin{equation}
				\begin{aligned}
					u_{\text{EM}} &= \varepsilon_0\vec{E}^2 &\hsp
					I &= \langle|\vec{S}|\rangle = c\varepsilon_0\vec{E}^2
				\end{aligned}
			\end{equation}

			\noindent
			Emitted wave of an oscillating dipole\index{Dipol!Strahlung} $\vec{p} = \vec{p}_0 \ex^{-\i\omega t}$ (Retarded time $\tilde{t} = t-\frac{\left|\vec{r}-\vec{r}'\right|}{c}$):
			\begin{equation}
				\begin{aligned}
					\vec{E}(t,\vec{r}) &= -\frac{\mu_0}{4\pi r c}	\left(\vec{n}\times\ddot{\vec{p}}(\tilde{t})\right) \times \vec{n} \\
					\vec{B}(t,\vec{r}) &= -\frac{\mu_0}{4\pi r c} \phantom{\Big(}\vec{n}\times\ddot{\vec{p}}(\tilde{t}) \\
				\end{aligned}
			\end{equation}

			\noindent
			Relativistic Larmor formulas\index{Larmor!Formel} for the power emitted per solid angle and total emitted power (Emitted energy of a non-relativistic particle measured in the retarded time, retarded acceleration $a(\tilde{t}$):
			\begin{equation}
				\begin{aligned}
					\pdv{P}{\Omega} &= \frac{q^2 \varepsilon_0}{\qty(1-\hat{e}\cdot\vec{\beta})^5} \left| \hat{e}\times\qty[(\hat{e}-\vec{\beta}) \times \dot{\vec{\beta}}] \right|^2 \\
				\end{aligned}
			\end{equation}

			\noindent
			Classical limit:
			\begin{equation}
				\begin{aligned}
					P(t) &= \frac{q^2 \mu_0}{6\pi c}a^2(\tilde{t}) \\
				\end{aligned}
			\end{equation}


		\subsubsection{Electromagnetic Waves in Matter}
			\noindent
			Definition index of refraction\index{Brechungsindex} (phase velocity $v_\text{ph.}$):
			\begin{equation}
				n(k) := \frac{c}{v_\text{ph.}(k)} = \sqrt{\frac{\varepsilon\mu}{\varepsilon_0\mu_0}}
			\end{equation}

			\noindent
			Snell's law\index{Snellius!Brechungsgesetz}:
			\begin{equation}
				\frac{\sin\alpha_1}{\sin\alpha_2} = \frac{n_2}{n_1}
			\end{equation}



		\subsubsection{Charged Particles}
			\noindent
			Action of a free particle in an electromagnetic field:
			\begin{equation}
				\mathcal{S}= -\int_{a}^{b}\left(m u_\mu u^\mu + q A_\nu u^\nu\right)\;\dd\tau
			\end{equation}

			\noindent
			Non-relativistic Lagrangian:
			\begin{equation}
				\mathcal{L}(t,\vec{x},\dot{\vec{x}}) = \frac{1}{2}m\dot{\vec{x}}^2 - q\phi(t,\vec{x}) - \dot{\vec{x}}\cdot\vec{A}(t,\vec{x})
			\end{equation}

			\noindent
			Lorentz force\index{Lorentz!Kraft} (relativistic Minkowski force\index{Minkowski!Kraft} and Newtonian force in the classical limit):
			\begin{equation}
				\begin{aligned}
					\frac{\dd P^\mu}{\dd \tau} &= q F^{\mu\nu}\frac{\dd x_\nu}{\dd \tau} \\
					\vec{F} &= q\left(\vec{E}+\vec{v}\times\vec{B}\right) \\
				\end{aligned}
			\end{equation}

		\subsubsection{Electrical Engineering}
			\noindent
			Ohm's law\index{Ohm!Gesetz} (Specific conductivity $\sigma$, resistance $R$, both assumed linear):
			\begin{equation}
				\begin{aligned}
					\vec{j} &= \sigma\vec{E} \\
					U &= R I
				\end{aligned}
			\end{equation}

			\noindent
			Kirchhoff's circuit laws\index{Kirchhoff!Regeln}:
			\begin{description}
				\item[1. Kirchhoff's current law\index{Kirchhoff!Knotenregel}] \hfill \\
					{\textit{The algebraic sum of currents in a network of conductors meeting at a point is zero.} \\(special case of continuity equation)}
				\item[2. Kirchhoff's voltage law\index{Kirchhoff!Maschenregel}] \hfill \\
					{\textit{The directed sum of the potential differences (voltages) around any closed loop is zero.} \\(electrostatics)}
			\end{description}

			\noindent
			Joule heating\index{Joule!Wärmegesetz} (Heat generation in an ohmic conductor):
			\begin{equation}
				P = UI = RI^2 = \frac{U^2}{R}
			\end{equation}

	\subsection{Gauge Transformation\index{Eichtransformation}}
		\noindent
		Gauge transformation\index{Eichtransformation}:
		\begin{equation}
			A^\mu \to A^\mu-\partial^\mu \lambda
		\end{equation}

		\subsubsection{Lorenz Gauge\index{Lorenz!Eichung}}
			\noindent
			Lorenz gauge\index{Lorenz!Eichung}:
			\begin{equation}
				\partial_\mu A^\mu = \frac{1}{c}\frac{\partial \phi}{\partial t} + \vec{\nabla}\cdot\vec{A} = 0
			\end{equation}

			\noindent
			Maxwell equations using Lorenz gauge \index{Maxwell!Gleichungen}:
			\begin{equation}
				\Box A^\mu = \mu_0 J^\mu \iff
				\Box \phi = \dfrac{\rho}{\varepsilon_0} \;\wedge\;
				\Box \vec{A} = \mu_0 \vec{j}
			\end{equation}

		\subsubsection{Coulomb Gauge \index{Coulomb!Eichung}}
			\noindent
			Coulomb gauge (not unique):
			\begin{equation}
				\Nabla\cdot\vec{A}(t,\vec{r})=0
			\end{equation}

	\subsection{Special Solutions of the Field Equations}
		\subsubsection{Retarded Potentials\index{Retardierte Potentiale}}
			\label{Sec:RetardedPotentials}
			\noindent
			Retarded potentials (general solutions of the non-static Maxwell's equations with Lorenz gauge for known charge and current distributions for all times):
			\begin{equation}
				\begin{aligned}
					\phi\left(t,\vec{r}\right)
					= \frac{1}{4\pi\varepsilon_0} \int_{\mathbb{R}^3} \frac{\rho(\tilde{t},\vec{r}')}{\left|\vec{r}-\vec{r}'\right|}\;\dd^3 \vec{r}'
					&=	\frac{1}{4\pi\varepsilon_0} \int_{\mathbb{R}^4} \frac{\rho(t',\vec{r}')}{\left|\vec{r}-\vec{r}'\right|}\delta\left( c(t-t')-\left|\vec{r}-\vec{r}'\right|\right)\;\dd^3 \vec{r}'\dd t' \\
					\vec{A}\left(t,\vec{r}\right)
					= \phantom{\varepsilon_0} \frac{\mu_0}{4\pi} \int_{\mathbb{R}^3} \frac{\vec{j}(\tilde{t},\vec{r}')}{\left|\vec{r}-\vec{r}'\right|}\;\dd^3 \vec{r}'
					&=	\phantom{\varepsilon_0} \frac{\mu_0}{4\pi} \int_{\mathbb{R}^4} \frac{\vec{j}(t',\vec{r}')}{\left|\vec{r}-\vec{r}'\right|}\delta\left( c(t-t')-\left|\vec{r}-\vec{r}'\right|\right)\;\dd^3 \vec{r}'\dd t' \\
					\tilde{t}&=t-\frac{\left|\vec{r}-\vec{r}'\right|}{c}
				\end{aligned}
			\end{equation}

		\subsubsection{Liénard-Wiechert Potentials\index{Liénard!-Wiechert Potentiale}}
			\noindent
			Liénard-Wiechert potentials\index{Liénard!-Wiechert Potentiale} (Solutions of the field equations for a moving point charge at $\vec{r}_\text{c}(\tilde{t})$ and velocity $\vec{v}(\tilde{t})$; retarded time $\tilde{t}$, $\vec{R}= \vec{r}-\vec{r}_\text{c}(\tilde{t})$):
			\begin{equation}
				A^\mu = \frac{1}{4\pi\varepsilon_0} \frac{q u^\mu}{c R \qty(1 - \vec{\beta}\cdot\vec{R}/R) }
			\end{equation}
			Equivalently
			\begin{equation}
				\begin{aligned}
					\phi(t,\vec{r}) & =\frac{q c}{4\pi \varepsilon_0}\frac{1}{c \left|\vec{r}-\vec{r}_\text{c}(\tilde{t})\right|-\vec{v}(\tilde{t})\cdot\left(\vec{r}-\vec{r}_\text{c}(\tilde{t})\right)} \\
					\vec{A}(t,\vec{r}) &= \frac{1}{c^2}\vec{v}(\tilde{t})\phi(t,\vec{r}) \\
				\end{aligned}
			\end{equation}

	\subsection{Electrostatics\index{Elektrostatik}}
		\subsubsection{Basics}
			\noindent
			Coulomb force\index{Coulomb!Kraft} (Acting on charge 1):
			\begin{equation}
				\vec{F}_1 = \frac{q_1 q_2}{4\pi\varepsilon_0}\frac{\vec{r}_1-\vec{r}_2}{\left|\vec{r}_1-\vec{r}_2\right|^3}
			\end{equation}

			\noindent
			Poisson equation\index{Poisson!Gleichung} of electrostatics:
			\begin{equation}
				\Nabla^2\phi = -\frac{\rho}{\varepsilon_0}
				\label{Eq:Poisson}
			\end{equation}

			\noindent
			Electrical potential for static systems (Resulting from the retarded potentials Sec.~\ref{Sec:RetardedPotentials}):
			\begin{equation}
				\phi\left(t,\vec{r}\right)
				= \frac{1}{4\pi\varepsilon_0} \int_{\mathbb{R}^3} \frac{\rho(\vec{r}')}{\left|\vec{r}-\vec{r}'\right|}\;\dd^3 \vec{r}'
			\end{equation}

		\subsubsection{Solving the Poisson Equation\index{Poisson!Gleichung}}
			\noindent
			Dirichlet-Green function\index{Dirichlet!-Green Function} $G_D$:
			\begin{equation}
				\forall\vec{r}\in\mathcal{V}': \left\{\begin{array}{ll}
					\forall\vec{r}\in\mathcal{V}\phantom\partial:
					\Nabla^2 G_D(\vec{r},\vec{r}') = \delta(\vec{r}-\vec{r}') \\
					\forall\vec{r}\in\partial\mathcal{V}:
					\phantom{\Nabla^2}G_D(\vec{r},\vec{r}') = 0
				\end{array}\right.
			\end{equation}

		\subsubsection{Multipole Expansion\index{Multipolentwicklung}}
			\noindent
			Multipole expansion\index{Multipolentwicklung} in cartesian coordinates (expansion of the general solution in $r'/r \to 0$):
			\begin{equation}
				\begin{aligned}
					\phi(\vec{r}) = \frac{1}{4\pi\varepsilon_0}\int\frac{\rho(\vec{r}')}{\left|\vec{r}-\vec{r}'\right|}\;\dd \vec{r}'
					=& \frac{1}{4\pi\varepsilon_0}\left(\frac{Q}{r} + \frac{\vec{r}\cdot\vec{p}}{r^3} + \frac{1}{2}\frac{r_i r_j Q_{ij}}{r^5} + \mathcal{O} \left(\frac{1}{r^4}\right)\right) \\
					=& \frac{1}{4\pi\varepsilon_0}\frac{1}{r}\int\rho(\vec{r}')\;\dd \vec{r}' \\
					&+ \frac{1}{4\pi\varepsilon_0}\frac{1}{r}\int\rho(\vec{r}')\left(\frac{r'}{r}\right)\cos\theta\;\dd \vec{r}' \\
					&+ \frac{1}{4\pi\varepsilon_0}\frac{1}{r}\int\rho(\vec{r}')\left(\frac{r'}{r}\right)^2\frac{3\cos^2\theta-1}{2}\;\dd \vec{r}' \\
					&+ \mathcal{O}\left(\;\;\frac{1}{r}\int\rho(\vec{r}')\left(\frac{r'}{r}\right)^3 \;\dd \vec{r}'\right) \\
				\end{aligned}
			\end{equation}
			Monopole $Q$, dipole $\vec{p}$, quadrupole $Q_{ij}$:
			\begin{equation}
				\begin{aligned}
					Q &= \int \rho(\vec{r}') \;\dd^3 \vec{r}' \\
					\vec{p} &= \int \rho(\vec{r}')\vec{r}' \;\dd^3 \vec{r}' \\
					Q_{ij} &= \int \rho(\vec{r}')\left(3r'_i r'_j - \delta_{ij} r'^2 \right)	\;\dd^3 \vec{r}' \\
				\end{aligned}
			\end{equation}

			\noindent
			Expansion of the distance in spherical harmonics\index{Kugelflächenfunktionen}:
			\begin{equation}
				\frac{1}{\left|\vec{r}-\vec{r}'\right|}=\sum_{l=0}^{\infty}\sum_{m=-l}^{l} \frac{4\pi}{2l+1}\frac{r'^l}{r^{l+1}} Y^{*}_{lm}(\theta',\varphi')Y_{lm}(\theta,\varphi)
			\end{equation}

			\noindent
			Multipole expansion in spherical harmonics:
			\begin{equation}
				\begin{aligned}
					\phi(r,\theta,\varphi) &= \sum_{l=0}^{\infty}\sum_{m=-l}^{l} \frac{b_{lm}}{r^{l+1}}Y_{lm}(\theta,\varphi) \\
					b_{lm} &= \frac{1}{(2l+1)\varepsilon_0}\int\rho(\vec{r}')r'^lY^{*}_{lm}(\theta',\varphi')\;\dd^3\vec{r}'
				\end{aligned}
			\end{equation}

			\noindent
			Electric dipole moment\index{Elektrisches Dipolmoment} of two point charges with a distance vector $\vec{d}$ spanning from the negative to the positive charge:
			\begin{equation}
				\vec{p}=q\vec{d}
			\end{equation}

			\noindent
			Force $\vec{F}$, torque $\vec{\tau}$ (with respect to the center of mass) and potential energy $\mathcal{E}_\text{pot.}$ of an electric dipole\index{Elektrisches Dipolmoment}:
			\begin{equation}
				\begin{aligned}
					\vec{F} &= \Nabla\left(\vec{p}\cdot\vec{E}\right) \\
					\vec{\tau}(\vec{r}_m) &= \vec{p}\times\vec{E} \\
					\mathcal{E}_\text{pot.} &= -\vec{p}\cdot\vec{E} \\
				\end{aligned}
			\end{equation}

		\subsubsection{Solving the Laplace Equation} \label{Sec:SolvingTheLaplaceEquation}
			\noindent
			Laplace equation\index{Laplace!Gleichung} (harmonic Poisson equation\index{Poisson!Gleichung} \ref{Eq:Poisson})
			\begin{equation}
				\Nabla^2 \phi = 0
			\end{equation}

			\noindent
			Solution of the Laplace equation\index{Laplace!Gleichung} in systems with spherical and axial symmetry (Legendre polynomials\index{Legendre!Polynome} $P_l$): \ref{Eq:LegendrePolynomials}):
			\begin{equation}
				\phi(r,\theta)=\sum_{l=0}^\infty \left(A_l r^l + \frac{B_l}{r^{l+1}}\right)P_l(\cos\theta)
			\end{equation}

			\noindent
			Solutions of the Laplace equation\index{Laplace!Gleichung} with spherical but without axial symmetry (spherical harmonics\index{Kugelflächenfunktionen} $Y_{lm}$ (\ref{Eq:SphericalHarmonics})):
			\begin{equation}
				\phi(r,\theta,\varphi) = \sum_{l=0}^{\infty}\sum_{m=-l}^{m=l} \left(a_{lm} r^l + \frac{b_{lm}}{r^{l+1}}\right) Y_{lm}(\theta, \varphi)
			\end{equation}

		\subsubsection{Capacitance\index{Kapazität}}
			\noindent
			Capacitance\index{Kapazität}:
			\begin{equation}
				C=\frac{Q}{U}
			\end{equation}

			\noindent
			Capacity\index{Kapazität} between two plates with surface area $A$, distance $d$ and dielectric\index{Dielektrikum} with permittivity \index{Permittivität} $\varepsilon$:
			\begin{equation}
				C=\varepsilon \frac{A}{d}
			\end{equation}

			\noindent
			Energy of a capacitor:
			\begin{equation}
				\mathcal{E} = \frac{1}{2}\frac{Q^2}{C} = \frac{1}{2}CU^2
			\end{equation}

		\subsubsection{Dielectric\index{Dielektrikum}}
			\noindent
			Generally, Eq.~\ref{Eq:FluxDensityFieldStrength} holds, where $\vec{P}$ depends on $\vec{E}$, $\vec{r}$, $T$, ... \vspace{10pt}

			\noindent
			Bound charge density and bound surface charge density:
			\begin{equation}
				\begin{aligned}
					\rho_P(\vec{r}) &= -\Nabla \cdot\vec{P}(\vec{r}) \\
					\sigma_P(\vec{r}) &= \vec{n}(\vec{r})\cdot P(\vec{r})
				\end{aligned}
			\end{equation}

			\noindent
			Polarization density in homogeneous, isotropic, linear dielectrics (Electric susceptibility\index{Elektrische Suszeptibilität} $\chi_\text{e}$):
			\begin{equation}
				\begin{aligned}
					\vec{P} &= \varepsilon_0 \chi_\text{e} \vec{E} \\
					\varepsilon &= \varepsilon_0 \varepsilon = \varepsilon_0(1+\chi_\text{e}) \\
					\vec{D} &= \varepsilon \vec{E} \\
				\end{aligned}
			\end{equation}


	\subsection{Magnetostatics\index{Magnetostatik}}
		\subsubsection{Basics}
			\noindent
			Biot-Savart law\index{Biot!-Savart Gesetz}:
			\begin{equation}
				\vec{B}(\vec{r}) = \frac{\mu_0}{4\pi}\int_{\mathcal{V}} \vec{j}(\vec{r}')\times\frac{\vec{r}-\vec{r}'}{\left|\vec{r}-\vec{r}'\right|^3}\;\dd^3\vec{r}'
			\end{equation}

			\noindent
			Static potential equation:
			\begin{equation}
				\Nabla^2 \vec{A} = - \mu_0 \vec{j}
			\end{equation}

			\noindent
			Vector potential for static systems (Resulting from the retarded potentials Sec.~\ref{Sec:RetardedPotentials}):
			\begin{equation}
				\vec{A}(\vec{r}) = \frac{\mu_0}{4\pi} \int_{\mathcal{V}} \frac{\vec{j}(\vec{r}')}{\left|\vec{r}-\vec{r}'\right|} \;\dd^3\vec{r}'
			\end{equation}

			\noindent
			Charged particles in a homogeneous magnetic field move on circles with Gyroradius / Larmor radius\index{Larmor!Radius} $r_\text{L}$ and gyration frequency\index{Zyklotronfrequenz} $\omega$:
			\begin{equation}
				\begin{aligned}
					r_\text{L} = \frac{p}{qB} &&\hsp
					\omega = \frac{|q| B}{m}
				\end{aligned}
			\end{equation}

		\subsubsection{Multipole Expansion}
			\noindent
			Multipole expansion\index{Multipolentwicklung} (Expansion of the general solution in $r'/r \to 0$):
			\begin{equation}
				\begin{aligned}
					\vec{A}(\vec{r}) = \int_{\mathcal{V}} \frac{\vec{j}(\vec{r}')}{\left|\vec{r}-\vec{r}'\right|} \;\dd^3\vec{r}'
					=& \frac{\mu_0}{4\pi} \left(\frac{\vec{m}\times\vec{r}}{r^3} + \mathcal{O}\left(\frac{1}{r^3}\right)\right) \\
					=& \frac{\mu_0}{4\pi}\frac{1}{r}\int\vec{j}(\vec{r}')\;\dd \vec{r}' \\
					&+ \frac{\mu_0}{4\pi}\frac{1}{r}\int\vec{j}(\vec{r}')\left(\frac{r'}{r}\right)\cos\theta\;\dd \vec{r}' \\
					&+ \frac{\mu_0}{4\pi}\frac{1}{r}\int\vec{j}(\vec{r}')\left(\frac{r'}{r}\right)^2\frac{3\cos^2\theta-1}{2}\;\dd \vec{r}' \\
					&+ \mathcal{O}\left(\;\frac{1}{r}\int\vec{j}(\vec{r}')\left(\frac{r'}{r}\right)^3 \;\dd \vec{r}'\right) \\
				\end{aligned}
			\end{equation}

			\noindent
			Magnetic dipole moment\index{Magnetisches Dipolmoment} $\vec{m}$:
			\begin{equation}
				\begin{aligned}
					\vec{m} = \frac{1}{2}\int \vec{r}'\times\vec{j}(\vec{r}')\;\dd^3\vec{r}'
				\end{aligned}
			\end{equation}

			\noindent
			Magnetic dipole moment\index{Magnetisches Dipolmoment} of a current $I$ encircling a surface $A$:
			\begin{equation}
				\vec{m} = IA\vec{n}
			\end{equation}


			\noindent
			Force $\vec{F}$, torque $\vec{\tau}$ (with respect to the center of mass) and potential energy $\mathcal{E}_\text{pot.}$ of an magnetic dipole\index{Magnetisches Dipolmoment}:
			\begin{equation}
				\begin{aligned}
					\vec{F} &= \Nabla\left(\vec{m}\cdot\vec{B}\right) \\
					\vec{\tau}(\vec{r}_m) &= \vec{m}\times\vec{B} \\
					\mathcal{E}_\text{pot.} &= -\vec{m}\cdot\vec{B} \\
				\end{aligned}
				\label{Eq:MagneticDipoleInteraction}
			\end{equation}

		\subsubsection{Magnets}
			\noindent
			Bound current density and bound surface current density:
			\begin{equation}
				\begin{aligned}
					\vec{j}_M(\vec{r}) &= \Nabla\times\vec{M}(\vec{r}) \\
					\vec{K}_M(\vec{r}) &= \vec{M}(\vec{r})\times \vec{n}(\vec{r}) \\
				\end{aligned}
			\end{equation}

			\noindent
			Magnetization density $M$ in linear, homogeneous, isotropic, magnetic materials (Magnetic susceptibility\index{Magnetische Suszeptibilität} $\chi_\text{m}$, permeability $\mu$):
			\begin{equation}
				\begin{aligned}
					\vec{M} &= \chi_\text{m}\vec{H} \\
					\mu &= \mu_0 \mu = \mu_0(1+\chi_\text{m}) \\
					\vec{B} &= \mu \vec{H} \\
				\end{aligned}
			\end{equation}
